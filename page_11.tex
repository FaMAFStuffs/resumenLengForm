\subsubsection{Macros asociados a funciones \(\Sigma \)-computables}

Dada una funcion \(f:S\subseteq \omega ^{n}\times \Sigma ^{\ast m}\rightarrow \omega \), con

\(\displaystyle \left[ \mathrm{V}\overline{n+1}\leftarrow f(\mathrm{V}1,...,\mathrm{V}\bar{n} ,\mathrm{W}1,...,\mathrm{W}\bar{m})\right] \)

denotaremos un macro el cual cumpla lo siguiente. Si reemplazamos sus variables y labels auxiliares por variables y labels concretos (distintos de a dos), y reemplazamos las variables
\(\displaystyle \mathrm{V}1,...,\mathrm{V}\bar{n},\mathrm{V}\overline{n+1},\mathrm{W}1,..., \mathrm{W}\bar{m} \)

por variables
\(\displaystyle \mathrm{N}\overline{k_{1}},...,\mathrm{N}\overline{k_{n}},\mathrm{N} \overline{k_{n+1}},\mathrm{P}\overline{j_{1}},...,\mathrm{P}\overline{j_{m}} \)

ninguna de las cuales es de las auxiliares antes seleccionadas, entonces la palabra obtenida es un programa que denotaremos con
\(\displaystyle \left[ \mathrm{N}\overline{k_{n+1}}\leftarrow f(\mathrm{N}\overline{k_{1}} ,...,\mathrm{N}\overline{k_{n}},\mathrm{P}\overline{j_{1}},...,\mathrm{P} \overline{j_{m}})\right] \)

el cual debe tener la siguiente propiedad:
- Si hacemos correr \(\left[ \mathrm{N}\overline{k_{n+1}}\leftarrow f( \mathrm{N}\overline{k_{1}},...,\mathrm{N}\overline{k_{n}},\mathrm{P} \overline{j_{1}},...,\mathrm{P}\overline{j_{m}})\right] \) partiendo de cualquier estado que le asigne a las variables \(\mathrm{N}\overline{k_{1}} ,...,\mathrm{N}\overline{k_{n}},\mathrm{P}\overline{j_{1}},...,\mathrm{P} \overline{j_{m}}\) valores \(x_{1},...,x_{n},\alpha _{1},...,\alpha _{m}\), entonces \(\left[ \mathrm{N}\overline{k_{n+1}}\leftarrow f(\mathrm{N} \overline{k_{1}},...,\mathrm{N}\overline{k_{n}},\mathrm{P}\overline{j_{1}} ,...,\mathrm{P}\overline{j_{m}})\right] \) termina si y solo si \( (x_{1},...,x_{n},\alpha _{1},...,\alpha _{m})\in D_{f}\) y en caso de terminacion se llega a un estado el cual le asigna a la variable \(\mathrm{N} \overline{k_{n+1}}\) el valor \(f(x_{1},...,x_{n},\alpha _{1},...,\alpha _{m})\) y a los contenidos de las variables \(\mathrm{N}\overline{k_{1}},...,\mathrm{N }\overline{k_{n}},\mathrm{P}\overline{j_{1}},...,\mathrm{P}\overline{j_{m}}\) no los modifica, salvo en el caso de que \(\mathrm{N}\overline{k_{i}}=\mathrm{ N}\overline{k_{n+1}}\), situacion en la cual el valor final de la variable \( \mathrm{N}\overline{k_{i}}\) sera \(f(x_{1},...,x_{n},\alpha _{1},...,\alpha _{m})\).
El programa \(\left[ \mathrm{N}\overline{k_{n+1}}\leftarrow f(\mathrm{N} \overline{k_{1}},...,\mathrm{N}\overline{k_{n}},\mathrm{P}\overline{j_{1}} ,...,\mathrm{P}\overline{j_{m}})\right] \) es comunmente llamado la expansion del macro con respecto a la eleccion de variables y labels realizada



\textbf{Proposición 54}

(a) Sea \(f:S\subseteq \omega ^{n}\times \Sigma ^{\ast m}\rightarrow \omega \) una funcion \(\Sigma \)-computable. Entonces hay un macro
\(\displaystyle \left[ \mathrm{V}\overline{n+1}\leftarrow f(\mathrm{V}1,...,\mathrm{V}\bar{n} ,\mathrm{W}1,...,\mathrm{W}\bar{m})\right] \)
(b) Sea \(f:S\subseteq \omega ^{n}\times \Sigma ^{\ast m}\rightarrow \Sigma ^{\ast }\) una funcion \(\Sigma \)-computable. Entonces hay un macro
\(\displaystyle \left[ \mathrm{W}\overline{m+1}\leftarrow f(\mathrm{V}1,...,\mathrm{V}\bar{n} ,\mathrm{W}1,...,\mathrm{W}\bar{m})\right] \)
Prueba: (b) Sea \(\mathcal{P}\) un programa que compute a \(f\). Tomemos un \(k\) tal que \( k\geq n,m\) y tal que todas las variables y labels de \(\mathcal{P}\) estan en el conjunto

\(\displaystyle \{\mathrm{N}1,...,\mathrm{N}\bar{k},\mathrm{P}1,...,\mathrm{P}\bar{k}, \mathrm{L}1,...,\mathrm{L}\bar{k}\}\text{.} \)

Sea \(\mathcal{P}^{\prime }\) la palabra que resulta de reemplazar en \( \mathcal{P}\):
- la variable \(\mathrm{N}\overline{j}\) por \(\mathrm{V}\overline{n+j}\) , para cada \(j=1,...,k\)
- la variable \(\mathrm{P}\overline{j}\) por \(\mathrm{W}\overline{m+j}\) , para cada \(j=1,...,k\)
- el label \(\mathrm{L}\overline{j}\) por \(\mathrm{A}\overline{j}\), para cada \(j=1,...,k\)
Notese que

\(\displaystyle \begin{array}{l} \mathrm{V}\overline{n+1}\leftarrow \mathrm{V}1 \\ \ \ \ \ \ \ \ \ \ \vdots \\ \mathrm{V}\overline{n+n}\leftarrow \mathrm{V}\overline{n} \\ \mathrm{V}\overline{n+n+1}\leftarrow 0 \\ \ \ \ \ \ \ \ \ \ \vdots \\ \mathrm{V}\overline{n+k}\leftarrow 0 \\ \mathrm{W}\overline{m+1}\leftarrow \mathrm{W}1 \\ \ \ \ \ \ \ \ \ \ \vdots \\ \mathrm{W}\overline{m+m}\leftarrow \mathrm{W}\overline{m} \\ \mathrm{W}\overline{m+m+1}\leftarrow \varepsilon \\ \ \ \ \ \ \ \ \ \ \vdots \\ \mathrm{W}\overline{m+k}\leftarrow \varepsilon \\ \mathcal{P}^{\prime } \end{array} \)

es el macro buscado, el cual tendra sus variables auxiliares y labels en la lista
\(\displaystyle \mathrm{V}\overline{n+1},...,\mathrm{V}\overline{n+k},\mathrm{W}\overline{m+2 },...,\mathrm{V}\overline{m+k},\mathrm{A}1,...,\mathrm{A}\overline{k}. \)

\(\Box\)




Proposición 55 Sea \(P:S\subseteq \omega ^{n}\times \Sigma ^{\ast m}\rightarrow \omega \) un predicado \(\Sigma \)-computable. Entonces hay un macro
\(\displaystyle \left[ \mathrm{IF}\;P(\mathrm{V}1,...,\mathrm{V}\bar{n},\mathrm{W}1,..., \mathrm{W}\bar{m})\;\mathrm{GOTO}\;\mathrm{A}1\right] \)
Usando macros podemos ahora probar el siguiente importante teorema.




\textbf{Teorema 56} Si \(h\) es \(\Sigma \)-recursiva, entonces \(h\) es \(\Sigma \) -computable.
Prueba: Probaremos por induccion en \(k\) que

(*) Si \(h\in \mathrm{R}_{k}^{\Sigma }\), entonces \(h\) es \(\Sigma \) -computable.
El caso \(k=0\) es dejado al lector. Supongamos (*) vale para \(k\), veremos que vale para \(k+1\). Sea \(h\in \mathrm{R}_{k+1}^{\Sigma }-\mathrm{R} _{k}^{\Sigma }.\) Hay varios casos

Caso 1. Supongamos \(h=M(P)\), con \(P:\omega \times \omega ^{n}\times \Sigma ^{\ast m}\rightarrow \omega \), un predicado perteneciente a \(\mathrm{R} _{k}^{\Sigma }\). Por hipotesis inductiva, \(P\) es \(\Sigma \)- computable y por lo tanto tenemos un macro

\(\displaystyle \left[ \mathrm{IF}\;P(\mathrm{V}1,...,\mathrm{V}\overline{n+1},\mathrm{W} 1,...,\mathrm{W}\bar{m})\;\mathrm{GOTO}\;\mathrm{A}1\right] \)

lo cual nos permite realizar el siguiente programa
\(\displaystyle \begin{array}{ll} \mathrm{L}2 & \mathrm{IF}\;P(\mathrm{N}\overline{n+1},\mathrm{N}1,..., \mathrm{N}\bar{n},\mathrm{P}1,...,\mathrm{P}\bar{m})\text{\ }\mathrm{GOTO}\; \mathrm{L}1 \\ & \mathrm{N}\overline{n+1}\leftarrow \mathrm{N}\overline{n+1}+1 \\ & \mathrm{GOTO}\;\mathrm{L}2 \\ \mathrm{L}1 & \mathrm{N}1\leftarrow \mathrm{N}\overline{n+1} \end{array} \)

Es facil chequear que este programa computa \(h.\)
Caso 2. Supongamos \(h=R(f,\mathcal{G})\), con

\(\displaystyle \begin{array}{rcl} f & :& S_{1}\times ...\times S_{n}\times L_{1}\times ...\times L_{m}\rightarrow \Sigma ^{\ast } \\ \mathcal{G}_{a} & :& S_{1}\times ...\times S_{n}\times L_{1}\times ...\times L_{m}\times \Sigma ^{\ast }\times \Sigma ^{\ast }\rightarrow \Sigma ^{\ast } \text{, }a\in \Sigma \end{array} \)

elementos de \(\mathrm{R}_{k}^{\Sigma }\). Sea \(\Sigma =\{a_{1},...,a_{r}\}.\) Por hipotesis inductiva, las funciones \(f\), \(\mathcal{G}_{a}\), \(a\in \Sigma \) , son \(\Sigma \)-computables y por lo tanto podemos hacer el siguiente programa via el uso de macros
\(\displaystyle \begin{array}{rl} & \left[ \mathrm{P}\overline{m+3}\leftarrow f(\mathrm{N}1,...,\mathrm{N}\bar{ n},\mathrm{P}1,...,\mathrm{P}\bar{m})\right] \\ \mathrm{L}\overline{r+1} & \mathrm{IF}\;\mathrm{P}\overline{m+1}\ \text{ {B}}\mathrm{EGINS\ }a_{1}\text{ }\mathrm{GOTO}\;\mathrm{L}1 \\ & \ \ \ \ \ \ \ \ \ \ \ \ \vdots \\ & \mathrm{IF}\;\mathrm{P}\overline{m+1}\ \mathrm{BEGINS\ }a_{r}\text{ } \mathrm{GOTO}\;\mathrm{L}\bar{r} \\ & \mathrm{GOTO}\;\mathrm{L}\overline{r+2} \\ \mathrm{L}1 & \mathrm{P}\overline{m+1}\leftarrow \text{ }^{\curvearrowright } \mathrm{P}\overline{m+1} \\ & \left[ \mathrm{P}\overline{m+3}\leftarrow \mathcal{G}_{a_{1}}(\mathrm{N} 1,...,\mathrm{N}\bar{n},\mathrm{P}1,...,\mathrm{P}\bar{m},\mathrm{P} \overline{m+2},\mathrm{P}\overline{m+3})\right] \\ & \mathrm{P}\overline{m+2}\leftarrow \mathrm{P}\overline{m+2}a_{1} \\ & \mathrm{GOTO}\;\mathrm{L}\overline{r+1} \\ & \ \ \ \ \ \ \ \ \ \ \ \ \vdots \\ \mathrm{L}\bar{r} & \mathrm{P}\overline{m+1}\leftarrow \text{ } ^{\curvearrowright }\mathrm{P}\overline{m+1} \\ & \mathrm{P}\overline{m+3}\leftarrow \mathcal{G}_{a_{r}}(\mathrm{N}1,..., \mathrm{N}\bar{n},\mathrm{P}1,...,\mathrm{P}\bar{m},\mathrm{P}\overline{m+2}, \mathrm{P}\overline{m+3}) \\ & \mathrm{P}\overline{m+2}\leftarrow \mathrm{P}\overline{m+2}a_{r} \\ & \mathrm{GOTO}\;\mathrm{L}\overline{r+1} \\ \mathrm{L}\overline{r+2} & \mathrm{P}1\leftarrow \mathrm{P}\overline{m+3} \end{array} \)

Es facil chequear que este programa computa \(h.\)
El resto de los casos son dejados al lector. \(\Box\)


\subsubsection{Analisis de la recursividad de \(\mathcal{S}^{\Sigma }\)}

Primero probaremos dos lemas que muestran que la sintaxis de \(\mathcal{S} ^{\Sigma }\) es \((\Sigma \cup \Sigma _{p})\)-recursiva primitiva. Recordemos que \(S:Num^{\ast }\rightarrow Num^{\ast }\) fue definida de la siguiente manera

\(\displaystyle \begin{array}{rcl} S(\varepsilon ) & =& 1 \\ S(\alpha 0) & =& \alpha 1 \\ S(\alpha 1) & =& \alpha 2 \\ S(\alpha 2) & =& \alpha 3 \\ S(\alpha 3) & =& \alpha 4 \\ S(\alpha 4) & =& \alpha 5 \\ S(\alpha 5) & =& \alpha 6 \\ S(\alpha 6) & =& \alpha 7 \\ S(\alpha 7) & =& \alpha 8 \\ S(\alpha 8) & =& \alpha 9 \\ S(\alpha 9) & =& S(\alpha )0 \end{array} \)

Tambien \(\overline{\ \;}:\omega \rightarrow Num^{\ast }\) fue definida de la siguiente manera
\(\displaystyle \begin{array}{rcl} \bar{0} & =& \varepsilon \\ \overline{n+1} & =& S(\bar{n}) \end{array} \)

Es obvio de las definiciones que ambas funciones son \(Num\)-p.r.. Mas aun tenemos



\textbf{Lema 57} Sea \(\Sigma \) un alfabeto cualquiera. Las funciones \(S\) y \(\overline{\ \;}\) son \((\Sigma \cup \Sigma _{p})\)-p.r..
Prueba: Use el Teorema 51. \(\Box\)

Recordemos que \(Bas:\mathrm{Ins}^{\Sigma }\rightarrow (\Sigma \cup \Sigma _{p})^{\ast }\), fue definida por

\(\displaystyle Bas(I)=\left\{ \begin{array}{ccl} J & & \text{si }I\text{ es de la forma }\mathrm{L}\bar{k}J\text{ con }J\in \mathrm{Ins}^{\Sigma } \\ I & & \text{caso contrario} \end{array} \right. \)

Definamos \(Lab:\mathrm{Ins}^{\Sigma }\rightarrow (\Sigma \cup \Sigma _{p})^{\ast }\) de la siguiente manera
\(\displaystyle Lab(I)=\left\{ \begin{array}{lll} \mathrm{L}\bar{k} & & \text{si }I\text{ es de la forma }\mathrm{L}\bar{k}J \text{ con }J\in \mathrm{Ins}^{\Sigma } \\ \varepsilon & & \text{caso contrario} \end{array} \right. \)



\textbf{Lema 58} Para cada \(n,x\in \omega \), tenemos que \( \left\vert \bar{n}\right\vert \leq x\) si y solo si \(n\leq 10^{x}-1\)



\textbf{Lema 59} \(\mathrm{Ins}^{\Sigma }\) es un conjunto \((\Sigma \cup \Sigma _{p})\)-p.r..
Prueba: Para simplificar la prueba asumiremos que \(\Sigma =\{@,\& \}\). Ya que \( \mathrm{Ins}^{\Sigma }\) es union de los siguientes conjuntos

\(\displaystyle \begin{array}{rcl} L_{1} & =& \left\{ \mathrm{N}\bar{k}\leftarrow \mathrm{N}\bar{k}+1:k\in \mathbf{N}\right\} \\ L_{2} & =& \left\{ \mathrm{N}\bar{k}\leftarrow \mathrm{N}\bar{k}\dot{-}1:k\in \mathbf{N}\right\} \\ L_{3} & =& \left\{ \mathrm{N}\bar{k}\leftarrow \mathrm{N}\bar{n}:k,n\in \mathbf{N}\right\} \\ L_{4} & =& \left\{ \mathrm{N}\bar{k}\leftarrow 0:k\in \mathbf{N}\right\} \\ L_{5} & =& \left\{ \mathrm{IF}\;\mathrm{N}\bar{k}\neq 0\;\mathrm{GOTO}\; \mathrm{L}\bar{m}:k,m\in \mathbf{N}\right\} \\ L_{6} & =& \left\{ \mathrm{P}\bar{k}\leftarrow \mathrm{P}\bar{k}.@:k\in \mathbf{N}\right\} \\ L_{7} & =& \left\{ \mathrm{P}\bar{k}\leftarrow \mathrm{P}\bar{k}.\& :k\in \mathbf{N}\right\} \\ L_{8} & =& \left\{ \mathrm{P}\bar{k}\leftarrow \text{ }^{\curvearrowright } \mathrm{P}\bar{k}:k\in \mathbf{N}\right\} \\ L_{9} & =& \left\{ \mathrm{P}\bar{k}\leftarrow \mathrm{P}\bar{n}:k,n\in \mathbf{N}\right\} \\ L_{9} & =& \left\{ \mathrm{P}\bar{k}\leftarrow \varepsilon :k\in \mathbf{N} \right\} \\ L_{10} & =& \left\{ \mathrm{IF}\;\mathrm{P}\bar{k}\;\mathrm{BEGINS}\;@\; \mathrm{GOTO}\;\mathrm{L}\bar{m}:k,m\in \mathbf{N}\right\} \\ L_{11} & =& \left\{ \mathrm{IF}\;\mathrm{P}\bar{k}\;\mathrm{BEGINS}\;\& \; \mathrm{GOTO}\;\mathrm{L}\bar{m}:k,m\in \mathbf{N}\right\} \\ L_{12} & =& \left\{ \mathrm{GOTO}\;\mathrm{L}\bar{m}:m\in \mathbf{N}\right\} \\ L_{13} & =& \left\{ \mathrm{SKIP}\right\} \\ L_{14} & =& \left\{ \mathrm{L}\bar{k}\alpha :k\in \mathbf{N\;}\text{y }\alpha \in L_{1}\cup ...\cup L_{13}\right\} \end{array} \)

solo debemos probar que \(L_{1},...,L_{14}\) son \((\Sigma \cup \Sigma _{p})\) -p.r.. Veremos primero por ejemplo que
\(\displaystyle L_{10}=\left\{ \mathrm{IFP}\bar{k}\mathrm{BEGINS}@\mathrm{GOTOL}\bar{m} :k,m\in \mathbf{N}\right\} \)

es \((\Sigma \cup \Sigma _{p})\)-p.r.. Primero notese que \(\alpha \in L_{10}\) si y solo si existen \(k,m\in \mathbf{N}\) tales que
\(\displaystyle \alpha =\mathrm{IFP}\bar{k}\mathrm{BEGINS}@\mathrm{GOTOL}\bar{m} \)

Mas formalmente tenemos que \(\alpha \in L_{10}\) si y solo si
\(\displaystyle (\exists k\in \mathbf{N})(\exists m\in \mathbf{N})\;\alpha =\mathrm{IFP}\bar{ k}\mathrm{BEGINS}@\mathrm{GOTOL}\bar{m} \)

Ya que cuando existen tales \(k,m\) tenemos que \(\bar{k}\) y \(\bar{m}\) son subpalabras de \(\alpha \), el lema anterior nos dice que \(\alpha \in L_{10}\) si y solo si
\(\displaystyle (\exists k\in \mathbf{N})_{k\leq 10^{\left\vert \alpha \right\vert }}(\exists m\in \mathbf{N})_{m\leq 10^{\left\vert \alpha \right\vert }}\;\alpha =\mathrm{IFP}\bar{k}\mathrm{BEGINS}@\mathrm{GOTOL}\bar{m} \)

Sea
\(\displaystyle P=\lambda mk\alpha \left[ \alpha =\mathrm{IFP}\bar{k}\mathrm{BEGINS}@\mathrm{ GOTOL}\bar{m}\right] \)

Ya que \(D_{\lambda k\left[ \bar{k}\right] }=\omega \), tenemos que \( D_{P}=\omega \times (\Sigma \cup \Sigma _{p})^{\ast }\times (\Sigma \cup \Sigma _{p})^{\ast }\). Notese que
\(\displaystyle P=\lambda \alpha \beta \left[ \alpha =\beta \right] \circ \left( p_{3}^{2,1},f\right) \)

donde
\(\displaystyle f=\lambda \alpha _{1}\alpha _{2}\alpha _{3}\alpha _{4}\left[ \alpha _{1}\alpha _{2}\alpha _{3}\alpha _{4}\right] \circ \left( C_{\mathrm{IFP} }^{2,1},\lambda k\left[ \bar{k}\right] \circ p_{2}^{2,1},C_{\mathrm{BEGINS}@ \mathrm{GOTOL}}^{2,1},\lambda k\left[ \bar{k}\right] \circ p_{1}^{2,1}\right) \)

lo cual nos dice que \(P\) es \((\Sigma \cup \Sigma _{p})\)-p.r..
Notese que

\(\displaystyle \chi _{L_{10}}=\lambda \alpha \left[ (\exists k\in \mathbf{N})_{k\leq 10^{\left\vert \alpha \right\vert }}(\exists m\in \mathbf{N})_{m\leq 10^{\left\vert \alpha \right\vert }}\;P(m,k,\alpha )\right] \)

Esto nos dice que podemos usar dos veces el Lema 39 para ver que \(\chi _{L_{10}}\) es \((\Sigma \cup \Sigma _{p})\)-p.r.. Veamos como. Sea
\(\displaystyle Q=\lambda k\alpha \left[ (\exists m\in \mathbf{N})_{m\leq 10^{\left\vert \alpha \right\vert }}\;P(m,k,\alpha )\right] \)

Por el Lema 39 tenemos que
\(\displaystyle \lambda xk\alpha \left[ (\exists m\in \mathbf{N})_{m\leq x}\;P(m,k,\alpha ) \right] \)

es \((\Sigma \cup \Sigma _{p})\)-p.r. lo cual nos dice que
\(\displaystyle Q=\lambda xk\alpha \left[ (\exists m\in \mathbf{N})_{m\leq x}\;P(m,k,\alpha ) \right] \circ (\lambda \alpha \left[ 10^{\left\vert \alpha \right\vert } \right] \circ p_{2}^{1,1},p_{1}^{1,1},p_{2}^{1,1}) \)

lo es. Ya que
\(\displaystyle \chi _{L_{10}}=\lambda \alpha \left[ (\exists k\in \mathbf{N})_{k\leq 10^{\left\vert \alpha \right\vert }}\;Q(k,\alpha )\right] \)

podemos en forma similar aplicar el Lema 39 y obtener finalmente que \(\chi _{L_{10}}\) es \((\Sigma \cup \Sigma _{p})\)-p.r..
En forma similar podemos probar que \(L_{1},...,L_{13}\) son \((\Sigma \cup \Sigma _{p})\)-p.r.. Esto nos dice que \(L_{1}\cup ...\cup L_{13}\) es \((\Sigma \cup \Sigma _{p})\)-p.r.. Notese que \(L_{1}\cup ...\cup L_{13}\) es el conjunto de las instrucciones basicas de \(\mathcal{S}^{\Sigma }\). Llamemos \( \mathrm{InsBas}^{\Sigma }\) a dicho conjunto. Para ver que \(L_{14}\) es \( (\Sigma \cup \Sigma _{p})\)-p.r. notemos que

\(\displaystyle \chi _{L_{14}}=\lambda \alpha \left[ (\exists k\in \mathbf{N})_{k\leq 10^{\left\vert \alpha \right\vert }}(\exists \beta \in \mathrm{InsBas} ^{\Sigma })_{\left\vert \beta \right\vert \leq \left\vert \alpha \right\vert }\;\alpha =\mathrm{L}\bar{k}\beta \right] \)

lo cual nos dice que aplicando dos veces el Lema 39 obtenemos que \(\chi _{L_{14}}\) es \((\Sigma \cup \Sigma _{p})\)-p.r.. Ya que \( \mathrm{Ins}^{\Sigma }=\mathrm{InsBas}^{\Sigma }\cup L_{14}\) tenemos que \( \mathrm{Ins}^{\Sigma }\) es \((\Sigma \cup \Sigma _{p})\)-p.r.. \(\Box\)
