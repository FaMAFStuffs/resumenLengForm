\section{Sintaxis de $\mathcal{S}^{\Sigma}$}

\textbf{\underline{Lemma 52:}} Se tiene que:
(a) Si \(I_{1}...I_{n}=J_{1}...J_{m}\), con \( I_{1},...,I_{n},J_{1},...,J_{m}\in \mathrm{Ins}^{\Sigma }\), entonces \(n=m\) y \(I_{j}=J_{j}\) para cada \(j\geq 1\).
(b) Si \(\mathcal{P}\in \mathrm{Pro}^{\Sigma }\), entonces existe una unica sucesion de instrucciones \(I_{1},...,I_{n}\) tal que \(\mathcal{P} =I_{1}...I_{n}\)


\textbf{\underline{Proof:}} (a) Supongamos \(I_{n}\) es un tramo final propio de \(J_{m}.\) Notar que entonces \(n >1\). Es facil ver que entonces ya sea \(J_{m}=\mathrm{L}\bar{u} I_{n}\) para algun \(u\in \mathbf{N}\), o \(I_{n}\) es de la forma \(\mathrm{GOTO} \;\mathrm{L}\bar{n}\) y \(J_{m}\) es de la forma \(w\mathrm{IF}\;\mathrm{P}\bar{k }\;\mathrm{BEGINS}\;a\;\mathrm{GOTO}\;\mathrm{L}\bar{n}\) donde \(w\in \{ \mathrm{L}\bar{n}:n\in \mathbf{N}\}\cup \{\varepsilon \}\). El segundo caso no puede darse porque entonces el anteultimo simbolo de \(I_{n-1}\) deberia ser \(\mathrm{S}\) lo cual no sucede para ninguna instruccion. O sea que

\(\displaystyle I_{1}...I_{n}=J_{1}...J_{m-1}\mathrm{L}\bar{u}I_{n} \)

lo cual dice que
(*) \(I_{1}...I_{n-1}=J_{1}...J_{m-1}\mathrm{L}\bar{u}.\)
Es decir que \(\mathrm{L}\bar{u}\) es tramo final de \(I_{n-1}\) y por lo tanto \(\mathrm{GOTO}\;\mathrm{L}\bar{u}\) es tramo final de \(I_{n-1}.\) Por (*), \(\mathrm{GOTO}\) es tramo final de \(J_{1}...J_{m-1}\), lo cual es impossible. Hemos llegado a una contradiccion lo cual nos dice que \(I_{n}\) no es un tramo final propio de \(J_{m}.\) Por simetria tenemos que \( I_{n}=J_{m} \), lo cual usando un razonamiento inductivo nos dice que \(n=m\) y \(I_{j}=J_{j} \) para cada \(j\geq 1\).

(b) Es consecuencia directa de (a). \(\Box\)
