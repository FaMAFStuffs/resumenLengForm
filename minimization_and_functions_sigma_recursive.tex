\section{Minimización y funciones $\Sigma$-recursivas}

\textbf{\underline{Lemma 41:}} Si \(P:D_{P}\subseteq \omega \times \omega ^{n}\times \Sigma ^{\ast m}\rightarrow \omega \) es un predicado \(\Sigma \)-efectivamente computable y \( D_{P}\) es \(\Sigma \)-efectivamente computable, entonces la funcion \(M(P)\) es \( \Sigma \)-efectivamente computable.

\textbf{\underline{Proof:}} Ejercicio \(\Box\)

\textbf{\underline{Theorem 42:}} Si \(f\in \mathrm{R} ^{\Sigma }\), entonces \(f\) es \(\Sigma \)-efectivamente computable.

\textbf{\underline{Proof:}} Dejamos al lector la prueba por induccion en \(k\) de que si \(f\in \mathrm{R} _{k}^{\Sigma }\), entonces \(f\) es \(\Sigma \)-efectivamente computable. \(\Box\)

\textbf{\underline{Lemma 43:}} Sean \(n,m\geq 0\). Sea \(P:D_{P}\subseteq \omega \times \omega ^{n}\times \Sigma ^{\ast m}\rightarrow \omega \) un predicado \(\Sigma \) -p.r.. Entonces
(a) \(M(P)\) es \(\Sigma \)-recursiva.
(b) Si hay una funcion \(\Sigma \)-p.r. \(f:\omega ^{n}\times \Sigma ^{\ast m}\rightarrow \omega \) tal que
\(\displaystyle M(P)(\vec{x},\vec{\alpha})=\min_{t}P(t,\vec{x},\vec{\alpha})\leq f(\vec{x}, \vec{\alpha})\text{, para cada }(\vec{x},\vec{\alpha})\in D_{M(P)}\text{,} \)
entonces \(M(P)\) es \(\Sigma \)-p.r..

\textbf{\underline{Proof:}} (a) Sea \(\bar{P}=P\mid _{D_{P}}\cup C_{0}^{n+1,m}\mid _{(\omega ^{n+1}\times \Sigma ^{\ast m})-D_{P}}\). Dejamos al lector verificar cuidadosamente que \( M(P)=M(\bar{P})\). Veremos entonces que \(M(\bar{P})\) es \(\Sigma \)-recursiva. Note que \(\bar{P}\) es \(\Sigma \)-p.r. (por que?). Sea \(k\) tal que \(\bar{P}\in \mathrm{PR}_{k}^{\Sigma }\). Ya que \(\bar{P}\) es \(\Sigma \)-total y \(\bar{P} \in \mathrm{PR}_{k}^{\Sigma }\subseteq \mathrm{R}_{k}^{\Sigma }\), tenemos que \(M(\bar{P})\in \mathrm{R}_{k+1}^{\Sigma }\) y por lo tanto \(M(\bar{P})\in \mathrm{R}^{\Sigma }\).

(b) Primero veremos que \(D_{M(\bar{P})}\) es un conjunto \(\Sigma \)-p.r.. Notese que

\(\displaystyle \chi _{D_{M(\bar{P})}}=\lambda \vec{x}\vec{\alpha}\left[ (\exists t\in \omega )_{t\leq f(\vec{x},\vec{\alpha})}\;\bar{P}(t,\vec{x},\vec{\alpha}) \right] \)

lo cual nos dice que
\(\displaystyle \chi _{D_{M(\bar{P})}}=\lambda x\vec{x}\vec{\alpha}\left[ (\exists t\in \omega )_{t\leq x}\;\bar{P}(t,\vec{x},\vec{\alpha})\right] \circ (f,p_{1}^{n,m},...,p_{n+m}^{n,m}) \)

Pero el Lema 39 nos dice que \(\lambda x\vec{x}\vec{\alpha} \left[ (\exists t\in \omega )_{t\leq x}\;\bar{P}(t,\vec{x},\vec{\alpha}) \right] \) es \(\Sigma \)-p.r. por lo cual tenemos que \(\chi _{D_{M(\bar{P})}}\) lo es.
Sea

\(\displaystyle P_{1}=\lambda t\vec{x}\vec{\alpha}\left[ \bar{P}(t,\vec{x},\vec{\alpha} )\wedge (\forall j\in \omega )_{j\leq t}\;j=t\vee \lnot \bar{P}(j,\vec{x}, \vec{\alpha})\right] \)

Note que \(P_{1}\) es \(\Sigma \)-total. Dejamos al lector usando lemas anteriores probar que \(P_{1}\) es \(\Sigma \)-p.r.. Ademas notese que para cada \((\vec{x},\vec{\alpha})\in \omega ^{n}\times \Sigma ^{\ast m}\) tenemos que
\(\displaystyle P_{1}(t,\vec{x},\vec{\alpha})=1\text{ si y solo si }t=M(\bar{P})(\vec{x}, \vec{\alpha}) \)

Esto nos dice que
\(\displaystyle M(\bar{P})=\left( \lambda \vec{x}\vec{\alpha}\left[ \prod_{t=0}^{f(\vec{x}, \vec{\alpha})}t^{P_{1}(t,\vec{x},\vec{\alpha})}\right] \right) \mid _{D_{M( \bar{P})}} \)

por lo cual para probar que \(M(\bar{P})\) es \(\Sigma \)-p.r. solo nos resta probar que
\(\displaystyle F=\lambda \vec{x}\vec{\alpha}\left[ \prod_{t=0}^{f(\vec{x},\vec{\alpha} )}t^{P_{1}(t,\vec{x},\vec{\alpha})}\right] \)

lo es. Pero
\(\displaystyle F=\lambda xy\vec{x}\vec{\alpha}\left[ \prod_{t=x}^{y}t^{P_{1}(t,\vec{x},\vec{ \alpha})}\right] \circ (C_{0}^{n,m},f,p_{1}^{n,m},...,p_{n+m}^{n,m}) \)

y por lo tanto el Lema 38 nos dice que \(F\) es \(\Sigma \)-p.r.. De esta manera hemos probado que \(M(\bar{P})\) es \(\Sigma \)-p.r. y por lo tanto \(M(P)\) lo es. \(\Box\)

\textbf{\underline{Lemma 44:}} Las siguientes funciones son \(\varnothing \)-p.r.:
(a) \( \begin{array}{rll} Q:\omega \times \mathbf{N} & \rightarrow & \omega \\ (x,y) & \rightarrow & \text{cociente de la division de }x\text{ por }y \end{array} \)
(b) \( \begin{array}{rll} R:\omega \times \mathbf{N} & \rightarrow & \omega \\ (x,y) & \rightarrow & \text{resto de la division de }x\text{ por }y \end{array} \)
(c) \( \begin{array}{rll} pr:\mathbf{N} & \rightarrow & \omega \\ n & \rightarrow & n\text{-esimo numero primo} \end{array} \)


\textbf{\underline{Proof:}} (a) Veamos primero veamos que \(Q=M(P)\), donde \(P=\lambda txy\left[ (t+1).y >x \right] \). Notar que

\(\displaystyle \begin{array}{rcl} D_{M(P)} & =& \{(x,y):(\exists t\in \omega )\;P(t,x,y)=1\} \\ & =& \{(x,y):(\exists t\in \omega )\;(t+1).y >x\} \\ & =& \omega \times \mathbf{N} \\ & =& D_{Q} \end{array} \)

Dejamos al lector la facil verificacion de que para cada \((x,y)\in \omega \times \mathbf{N}\), se tiene que
\(\displaystyle Q(x,y)=M(P)(x,y)=\min_{t}(t+1).y >x \)

Esto prueba que \(Q=M(P)\). Ya que \(P\) es \(\varnothing \)-p.r. y
\(\displaystyle Q(x,y)\leq p_{1}^{2,0}(x,y),\text{para cada }(x,y)\in \omega \times \mathbf{N } \)

(b) del Lema 43 implica que \(Q\in \mathrm{PR}^{\varnothing }\).
(b) Notese que

\(\displaystyle R=\lambda xy\left[ x\dot{-}Q(x,y).y\right] \)

y por lo tanto \(R\in \mathrm{PR}^{\varnothing }\).
(c) Para ver que \(pr\) es \(\varnothing \)-p.r., veremos que la extension \( h:\omega \rightarrow \omega \), dada por \(h(0)=0\) y \(h(n)=pr(n)\), \(n\geq 1\), es \(\varnothing \)-p.r.. Primero note que

\(\displaystyle \begin{array}{rcl} h(0) & =& 0 \\ h(x+1) & =& \min\nolimits_{t}\left( t\text{ es primo}\wedge t >h(x)\right) \end{array} \)

O sea que \(h=R\left( C_{0}^{0,0},M(P)\right) \), donde
\(\displaystyle P=\lambda tzx\left[ t\text{ es primo}\wedge t >z\right] \)

Es decir que solo nos resta ver que \(M(P)\) es \(\varnothing \)-p.r.. Claramente \( P\) es \(\varnothing \)-p.r.. Veamos que para cada \((z,x)\in \omega ^{2}\), tenemos que
\(\displaystyle M(P)(z,x)=\min\nolimits_{t}\left( t\text{ es primo}\wedge t >z\right) \leq z!+1 \)

Sea \(p\) primo tal que \(p\) divide a \(z!+1\). Es facil ver que entonces \(p >z\). Pero esto claramente nos dice que
\(\displaystyle \min\nolimits_{t}\left( t\text{ es primo}\wedge t >z\right) \leq p\leq z!+1 \)

O sea que (b) del Lema 43 implica que \(M(P)\) es \(\varnothing \) -p.r. ya que podemos tomar \(f=\lambda zx\left[ z!+1\right] \). \(\Box\)


\textbf{\underline{Lemma 45:}} Las funciones \(\lambda xi\left[ (x)_{i}\right] \) y \(\lambda x\left[ Lt(x) \right] \) son \(\varnothing \)-p.r.

\textbf{\underline{Proof:}} Note que \(D_{\lambda xi\left[ (x)_{i}\right] }=\mathbf{N}\times \mathbf{N}\). Sea

\(\displaystyle P=\lambda txi\left[ \lnot (pr(i)^{t+1}\ \text{divide }x)\right] \)

Note que \(P\) es \(\varnothing \)-p.r. y que \(D_{P}=\omega \times \omega \times \mathbf{N}\). Dejamos al lector la prueba de que \(\lambda xi\left[ (x)_{i} \right] =M(P)\). Ya que \((x)_{i}\leq x\), para todo \(x\in \mathbf{N}\), (b) del Lema 43 implica que \(\lambda xi\left[ (x)_{i}\right] \) es \( \varnothing \)-p.r..
Veamos que \(\lambda x\left[ Lt(x)\right] \) es \(\varnothing \)-p.r.. Sea

\(\displaystyle Q=\lambda tx\left[ (\forall i\in \mathbf{N})_{i\leq x}\;(i\leq t\vee (x)_{i}=0)\right] \)

Notese que \(D_{Q}=\omega \times \mathbf{N}\) y que ademas por el Lema 39 tenemos que \(Q\) es \(\varnothing \)-p.r. (dejamos al lector explicar como se aplica tal lema en este caso). Ademas notese que \(\lambda x \left[ Lt(x)\right] =M(Q)\) y que
\(\displaystyle Lt(x)\leq x,\text{para todo }x\in \mathbf{N} \)

lo cual por (b) del Lema 43 nos dice que \(\lambda x\left[ Lt(x)\right] \) es \(\varnothing \)-p.r.. \(\Box\)
Para \(x_{1},...,x_{n}\in \omega \), escribiremos \(\left\langle x_{1},...,x_{n}\right\rangle \) en lugar de \(\left\langle x_{1},...,x_{n},0,...\right\rangle \).



\textbf{\underline{Lemma 46:}} Sea \(n\geq 1\). La funcion \(\lambda x_{1}...x_{n}\left[ \left\langle x_{1},...,x_{n}\right\rangle \right] \) es \(\varnothing \)-p.r.

\textbf{\underline{Proof:}} Sea \(f_{n}=\lambda x_{1}...x_{n}\left[ \left\langle x_{1},...,x_{n}\right\rangle \right] \). Claramente \(f_{1}\) es \(\varnothing \) -p.r.. Ademas note que para cada \(n\geq 1\), tenemos

\(\displaystyle f_{n+1}=\lambda x_{1}...x_{n+1}\left[ \left( f_{n}(x_{1},...,x_{n})pr(n+1)^{x_{n+1}}\right) \right] \text{.} \)

O sea que podemos aplicar un argumento inductivo. \(\Box\)


\textbf{\underline{Lemma 47:}} Supongamos que \(\Sigma \neq \varnothing \). Sea \(< \) un orden total estricto sobre \(\Sigma \), sean \(n,m\geq 0\) y sea \( P:D_{P}\subseteq \omega ^{n}\times \Sigma ^{\ast m}\times \Sigma ^{\ast }\rightarrow \omega \) un predicado \(\Sigma \)-p.r.. Entonces
(a) \(M^{< }(P)\) es \(\Sigma \)-recursiva.
(b) Si existe una funcion \(\Sigma \)-p.r. \(f:\omega ^{n}\times \Sigma ^{\ast m}\rightarrow \omega \) tal que
\(\displaystyle \left\vert M^{< }(P)(\vec{x},\vec{\alpha})\right\vert =\left\vert \min\nolimits_{\alpha }^{< }P(\vec{x},\vec{\alpha},\alpha )\right\vert \leq f( \vec{x},\vec{\alpha})\text{, para cada }(\vec{x},\vec{\alpha})\in D_{M^{< }(P)}\text{,} \)
entonces \(M^{< }(P)\) es \(\Sigma \)-p.r..

\textbf{\underline{Proof:}} Sea \(Q=P\circ \left( p_{2}^{1+n,m},...,p_{1+n+m}^{1+n,m},\ast ^{< }\circ p_{1}^{1+n,m}\right) \). Note que

\(\displaystyle M^{< }(P)=\ast ^{< }\circ M(Q) \)

lo cual por (a) del Lema 43 implica que \(M^{< }(P)\) es \( \Sigma \)-recursiva.
Sea \(k\) el cardinal de \(\Sigma \). Ya que

\(\displaystyle \left\vert \ast ^{< }(M(Q)(\vec{x},\vec{\alpha}))\right\vert =\left\vert M^{< }(P)(\vec{x},\vec{\alpha})\right\vert \leq f(\vec{x},\vec{\alpha})\text{, } \)

para todo \((\vec{x},\vec{\alpha})\in D_{M^{< }(P)}=D_{M(Q)}\), tenemos que
\(\displaystyle M(Q)(\vec{x},\vec{\alpha}))\leq \sum_{\iota =1}^{i=f(\vec{x},\vec{\alpha} )}k^{i}\text{, para cada }(\vec{x},\vec{\alpha})\in D_{M(Q)}\text{.} \)

O sea que por (a) del Lema 43, \(M(Q)\) es \(\Sigma \)-p.r. y por lo tanto \(M^{< }(P)\) lo es. \(\Box\)
