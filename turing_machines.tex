\section{Maquinas de Turing}


\textbf{\underline{Lemma 79:}} Sea \(L\subseteq \Sigma ^{\ast }.\) entonces \(L=L(M)\) para alguna maquina de Turing \(M\) sii \(L=H(M)\) para alguna maquina de Turing \(M=(Q,\Sigma ,\Gamma ,\delta ,q_{0},B,F)\).

\textbf{\underline{Proof:}} (\(\Rightarrow \)) Dada una maquina \(M=(Q,\Sigma ,\Gamma ,\delta ,q_{0},B,F)\), costruiremos una maquina \(M_{1}=(Q_{1},\Sigma ,\Gamma _{1},\delta _{1}, \tilde{q}_{0},B,\varnothing )\) tal que \(L(M)=H(M_{1}).\) Tomaremos \(\Gamma _{1}=\Gamma \cup \{X\}\), con \(X\) un simbolo nuevo no perteneciente a \(\Gamma \). Para cada \(a\in \Sigma \), sea \(q_{a}\) un estado nuevo, no perteneciente a \(Q.\) Sean \(\tilde{q}_{0},q_{r},q_{d},q_{B}\) estados nuevos no pertenecientes a \(Q.\) Tomemos entonces

\(\displaystyle Q_{1}=Q\cup \{\tilde{q}_{0},q_{r},q_{d},q_{B}\}\cup \{q_{a}:a\in \Sigma \} \)

Finalmente definamos \(\delta _{1}\) de la siguiente manera:
\(\displaystyle \begin{array}{rcl} \delta _{1}(\tilde{q}_{0},B) & =& \{(q_{B},X,R)\} \\ \delta _{1}(q_{B},a) & =& \{(q_{a},B,R)\}\text{, para }a\in \Sigma \\ \delta _{1}(q_{B},B) & =& \{(q_{0},B,K)\} \\ \delta _{1}(q_{a},b) & =& \{(q_{b},a,R)\}\text{, para }a,b\in \Sigma \\ \delta _{1}(q_{a},B) & =& \{(q_{r},a,L)\}\text{, para }a\in \Sigma \\ \delta _{1}(q_{r},a) & =& \{(q_{r},a,L)\}\text{, para }a\in \Sigma \\ \delta _{1}(q_{r},B) & =& \{(q_{0},B,K)\} \\ \delta _{1}(q,X) & =& \{(q,X,K)\}\text{, para }q\in Q \\ \delta _{1}(q,\sigma ) & =& \delta (q,\sigma )\cup \{(q_{d},\sigma ,K)\}\text{ , para }q\in F\text{ y }\sigma \in \Gamma \\ \delta _{1}(q,\sigma ) & =& \delta (q,\sigma )\text{, para }q\in Q-F\text{ y } \sigma \in \Gamma \\ \delta _{1}(q_{d},\sigma ) & =& \varnothing \text{, para }\sigma \in \Gamma \end{array} \)

(\(\delta _{1}\) se define igual a vacio para los casos no contemplados arriba).
(\(\Leftarrow \)) Dada \(M=(Q,\Sigma ,\Gamma ,\delta ,q_{0},B,F)\), dejamos al lector la construccion de una maquina \(M_{1}=(Q_{1},\Sigma ,\Gamma _{1},\delta _{1},\tilde{q}_{0},B,\varnothing )\) tal que \(H(M)=L(M_{1})\). \(\Box\)


\textbf{\underline{Lemma 80:}} El predicado \(\lambda ndd^{\prime }\left[ d\vdash d^{\prime }\right] \) es \( (\Gamma \cup Q)\)-p.r..

\textbf{\underline{Proof:}} Note que \(D_{\lambda dd^{\prime }\left[ d\vdash d^{\prime }\right] }=Des\times Des\). Tambien notese que los predicados

\(\displaystyle \begin{array}{rcl} & & \lambda p\sigma q\gamma \left[ (p,\sigma ,L)\in \delta (q,\gamma )\right] \\ & & \lambda p\sigma q\gamma \left[ (p,\sigma ,R)\in \delta (q,\gamma )\right] \\ & & \lambda p\sigma q\gamma \left[ (p,\sigma ,K)\in \delta (q,\gamma )\right] \end{array} \)

son \((\Gamma \cup Q)\)-p.r. ya que los tres tienen dominio igual a \(Q\times \Gamma \times Q\times \Gamma \) el cual es finito (Corolario 36 ). Sea \(P_{R}:Des\times Des\times \Gamma \times \Gamma ^{\ast }\times \Gamma ^{\ast }\times Q\times Q\rightarrow \omega \) definido por \(P_{R}(d,d^{\prime },\sigma ,\alpha ,\beta ,p,q)=1\) sii
\(\displaystyle d=\alpha p\beta \wedge (q,\sigma ,R)\in \delta \left( p,\left[ \beta B\right] _{1}\right) \wedge d^{\prime }=\alpha \sigma q^{\curvearrowright }\beta \)

Sea \(P_{L}:Des\times Des\times \Gamma \times \Gamma ^{\ast }\times \Gamma ^{\ast }\times Q\times Q\rightarrow \omega \) definido por \(P_{L}(d,d^{\prime },\sigma ,\alpha ,\beta ,p,q)=1\) sii
\(\displaystyle d=\alpha p\beta \wedge (q,\sigma ,L)\in \delta \left( p,\left[ \beta B\right] _{1}\right) \wedge \alpha \neq \varepsilon \wedge d^{\prime }=\left\lfloor \alpha ^{\curvearrowleft }q\left[ \alpha \right] _{\left\vert \alpha \right\vert }\sigma ^{\curvearrowright }\beta \right\rfloor \)

Sea \(P_{K}:Des\times Des\times \Gamma \times \Gamma ^{\ast }\times \Gamma ^{\ast }\times Q\times Q\rightarrow \omega \) definido por \(P_{K}(d,d^{\prime },\sigma ,\alpha ,\beta ,p,q)=1\) sii
\(\displaystyle d=\alpha p\beta \wedge (q,\sigma ,K)\in \delta \left( p,\left[ \beta B\right] _{1}\right) \wedge d^{\prime }=\left\lfloor \alpha q\sigma ^{\curvearrowright }\beta \right\rfloor \)

Se deja al lector la verificacion de que estos predicados son \((\Gamma \cup Q)\)-p.r.. Notese que \(\lambda dd^{\prime }\left[ d\vdash d^{\prime }\right] \) es igual al predicado
\(\displaystyle \lambda dd^{\prime }\left[ (\exists \sigma \in \Gamma )(\exists \alpha ,\beta \in \Gamma ^{\ast })(\exists p,q\in Q)(P_{R}\vee P_{L}\vee P_{K})(d,d^{\prime },\sigma ,\alpha ,\beta ,p,q)\right] \)

lo cual por el Lema 39 nos dice que \(\lambda dd^{\prime } \left[ d\vdash d^{\prime }\right] \) es \((\Gamma \cup Q)\)-p.r. \(\Box\)


\textbf{\underline{Proposition 81:}} \(\lambda ndd^{\prime }\left[ d\overset{n}{\vdash }d^{\prime }\right] \) es \( (\Gamma \cup Q)\)-p.r..

\textbf{\underline{Proof:}} Sea \(Q=\lambda dd^{\prime }\left[ d\vdash d^{\prime }\right] \cup C_{0}^{0,2}\mid _{(\Gamma \cup Q)^{\ast 2}-Des^{2}}\) es decir \(Q\) es el resultado de extender con el valor \(0\) al predicado \(\lambda dd^{\prime } \left[ d\vdash d^{\prime }\right] \) de manera que este definido en todo \( (\Gamma \cup Q)^{\ast 2}\). Sea \(< \) un orden total estricto sobre \(\Gamma \cup Q\) y sea \(Q_{1}:\mathbf{N}\times Des\times Des\rightarrow \omega \) definido por \(Q_{1}(x,d,d^{\prime })=1\) sii

\(\left( (\forall i\in \mathbf{N})_{i\leq Lt(x)}\ast ^{< }((x)_{i})\in Des\right) \wedge \ast ^{< }((x)_{1})=d\wedge \)

\(\ \ \ \ \ \ \ \ \ \ \ \ \ \ \ \ \ \ \ \ \ \ \ast ^{< }((x)_{Lt(x)})=d^{\prime }\wedge \left( (\forall i\in \mathbf{N})_{i\leq Lt(x)\dot{-}1}\;Q(\ast ^{< }((x)_{i}),\ast ^{< }((x)_{i+1}))\right) \)

Notese que dicho rapidamente \(Q_{1}(x,d,d^{\prime })=1\) sii \(x\) codifica una computacion que parte de \(d\) y llega a \(d^{\prime }\). Se deja al lector la verificacion de que este predicado es \((\Gamma \cup Q)\)-p.r.. Notese que

\(\displaystyle \lambda ndd^{\prime }\left[ d\overset{n}{\vdash }d^{\prime }\right] =\lambda ndd^{\prime }\left[ \left( \exists x\in \mathbf{N}\right) \;Lt(x)=n+1\wedge Q_{1}(x,d,d^{\prime })\right] \)

Es decir que solo nos falta acotar el cuantificador existencial, para poder aplicar el lema de cuantificacion acotada. Ya que cuando \( d_{1},...,d_{n+1}\in Des\) son tales que \(d_{1}\vdash d_{2}\vdash ...\vdash d_{n+1}\) tenemos que
\(\displaystyle \left\vert d_{i}\right\vert \leq \left\vert d_{1}\right\vert +n\text{, para } i=1,...,n \)

una posible cota para dicho cuantificador es
\(\displaystyle \prod_{i=1}^{n+1}pr(i)^{\left\vert \Gamma \cup Q\right\vert ^{\left\vert d\right\vert +n}}\text{.} \)

O sea que, por el lema de cuantificacion acotada, tenemos que el predicado \( \lambda ndd^{\prime }\left[ d\overset{n}{\vdash }d^{\prime }\right] \) es \( (\Gamma \cup Q)\)-p.r. \(\Box\)

\textbf{\underline{Theorem 82:}} Sea \(M=\left( Q,\Sigma ,\Gamma ,\delta ,q_{0},B,F\right) \) una maquina de Turing. Entonces \(L(M)\) es \(\Sigma \)-recursivamente enumerable.


\textbf{\underline{Proof:}} Sea \(P\) el siguiente predicado \((\Gamma \cup Q)\)-mixto

\(\displaystyle \lambda n\alpha \left[ (\exists d\in Des)\;\left\lfloor q_{0}B\alpha \right\rfloor \overset{n}{\vdash }d\wedge St(d)\in F\right] \)

Notese que \(D_{P}=\omega \times \Gamma ^{\ast }\). Dejamos al lector probar que \(P\) es \((\Gamma \cup Q)\)-p.r.. Sea \(P^{\prime }=P\mid _{\omega \times \Sigma ^{\ast }}\). Notese que \(P^{\prime }(n,\alpha )=1\) sii \(\alpha \in L(M) \) atestiguado por una computacion de longitud \(n\). Ya que \(P^{\prime }\) es \((\Gamma \cup Q)\)-p.r. (por que?) y ademas es \(\Sigma \)-mixto, el Teorema 51 nos dice que \(P^{\prime }\) es \(\Sigma \)-p.r.. Ya que \( L(M)=D_{M(P^{\prime })}\), el Teorema 71 nos dice que \( L(M)\) es \(\Sigma \)-r.e.. \(\Box\)
