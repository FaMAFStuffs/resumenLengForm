\section{Funciones $\Sigma$-recursivas primitivas}

  \textbf{\underline{Lemma 18:}} Si \(f,f_{1},...,f_{n+m}\) son \(\Sigma \)-efectivamente computables, entonces \( f\circ (f_{1},...,f_{n+m})\) lo es.

  \textbf{\underline{Proof:}} Sean \(\mathbb{P},\mathbb{P}_{1},...,\mathbb{P}_{n+m}\) procedimientos efectivos los cuales computen las funciones \(f,f_{1},...,f_{n+m}\), respectivamente. Usando estos procedimientos es facil definir un procedimiento efectivo el cual compute a \(f\circ (f_{1},...,f_{n+m})\). \(\Box\)

  \textbf{\underline{Lemma 19:}} Si \(f\) y \(g\) son \(\Sigma \)-efectivamente computables, entonces \(R(f,g)\) lo es.

  \textbf{\underline{Proof:}} La Proof es dejada al lector. \(\Box\)

  \textbf{\underline{Lemma 20:}} Si \(f\) y cada \(\mathcal{G}_{a}\) son \(\Sigma \)-efectivamente computables, entonces \(R(f,\mathcal{G})\) lo es.

  \textbf{\underline{Proof:}} Es dejada al lector con la recomendacion de que haga la Proof para el caso \( \Sigma =\{@,\& \}\) \(\Box\)

  \textbf{\underline{Theorem 21:}} Si \(f\in \mathrm{PR}^{\Sigma }\), entonces \(f\) es \(\Sigma \)-efectivamente computable.

  \textbf{\underline{Proof:}} Dejamos al lector la Proof por induccion en \(k\) de que si \(f\in \mathrm{PR} _{k}^{\Sigma }\), entonces \(f\) es \(\Sigma \)-efectivamente computable, la cual sale en forma directa usando los lemas anteriores que garantizan que los constructores de composicion y recursion primitiva preservan la computabilidad efectiva \(\Box\)

  \textbf{\underline{Lemma 22:}}
  (1) \(\varnothing \in \mathrm{PR}^{\varnothing }\).
  (2) \(\lambda xy\left[ x+y\right] \in \mathrm{PR}^{\varnothing }\).
  (3) \(\lambda xy\left[ x.y\right] \in \mathrm{PR}^{\varnothing }\).
  (4) \(\lambda x\left[ x!\right] \in \mathrm{PR}^{\varnothing }\).

  \textbf{\underline{Proof:}} (1) Notese que \(\varnothing =Pred\circ C_{0}^{0,0}\in \mathrm{PR} _{1}^{\varnothing }\)

  (2) Notar que

  \(\displaystyle \begin{array}{rcl} \lambda xy\left[ x+y\right] (0,x_{1}) & =& x_{1}=p_{1}^{1,0}(x_{1}) \\ \lambda xy\left[ x+y\right] (t+1,x_{1}) & =& \lambda xy\left[ x+y\right] (t,x_{1})+1 \\ & =& \left( Suc\circ p_{1}^{3,0}\right) \left( \lambda xy\left[ x+y\right] (t,x_{1}),t,x_{1}\right) \end{array} \)

  lo cual implica que \(\lambda xy\left[ x+y\right] =R\left( p_{1}^{1,0},Suc\circ p_{1}^{3,0}\right) \in \mathrm{PR}_{2}^{\varnothing }.\)
  (3) Primero note que

  \(\displaystyle \begin{array}{rcl} C_{0}^{1,0}(0) & =& C_{0}^{0,0}(\Diamond ) \\ C_{0}^{1,0}(t+1) & =& C_{0}^{1,0}(t) \end{array} \)

  lo cual implica que \(C_{0}^{1,0}=R\left( C_{0}^{0,0},p_{1}^{2,0}\right) \in \mathrm{PR}_{1}^{\varnothing }.\) Tambien note que
  \(\displaystyle \lambda tx\left[ t.x\right] =R\left( C_{0}^{1,0},\lambda xy\left[ x+y\right] \circ \left( p_{1}^{3,0},p_{3}^{3,0}\right) \right) , \)

  lo cual por (1) implica que \(\lambda tx\left[ t.x\right] \in \mathrm{PR} _{3}^{\varnothing }\).
  (4) Note que

  \(\displaystyle \begin{array}{rcl} \lambda x\left[ x!\right] (0) & =& 1=C_{1}^{0,0}(\Diamond ) \\ \lambda x\left[ x!\right] (t+1) & =& \lambda x\left[ x!\right] (t).(t+1), \end{array} \)

  lo cual implica que
  \(\displaystyle \lambda x\left[ x!\right] =R\left( C_{1}^{0,0},\lambda xy\left[ x.y\right] \circ \left( p_{1}^{2,0},Suc\circ p_{2}^{2,0}\right) \right) . \)

  Ya que \(C_{1}^{0,0}=\) \(Suc\circ C_{0}^{0,0}\), tenemos que \(C_{1}^{0,0}\in \mathrm{PR}_{1}^{\varnothing }\). Por (2), tenemos que
  \(\displaystyle \lambda xy\left[ x.y\right] \circ \left( p_{1}^{2,0},Suc\circ p_{2}^{2,0}\right) \in \mathrm{PR}_{4}^{\varnothing }, \)

  obteniendo que \(\lambda x\left[ x!\right] \in \mathrm{PR}_{5}^{\varnothing }\). \(\Box\)


  \textbf{\underline{Lemma 23:}} Supongamos \(\Sigma \) es no vacio.
  (a) \(\lambda \alpha \beta \left[ \alpha \beta \right] \in \mathrm{PR} ^{\Sigma }\)
  (b) \(\lambda \alpha \left[ \left\vert \alpha \right\vert \right] \in \mathrm{PR}^{\Sigma }\)

  \textbf{\underline{Proof:}} (a) Ya que

  \(\displaystyle \begin{array}{rcl} \lambda \alpha \beta \left[ \alpha \beta \right] (\alpha _{1},\varepsilon ) & =& \alpha _{1}=p_{1}^{0,1}(\alpha _{1}) \\ \lambda \alpha \beta \left[ \alpha \beta \right] (\alpha _{1},\alpha a) & =& d_{a}(\lambda \alpha \beta \left[ \alpha \beta \right] (\alpha _{1},\alpha )),a\in \Sigma \end{array} \)

  tenemos que \(\lambda \alpha \beta \left[ \alpha \beta \right] =R\left( p_{1}^{0,1},\mathcal{G}\right) \), donde \(\mathcal{G}_{a}=d_{a}\circ p_{3}^{0,3}\), para cada \(a\in \Sigma \).
  (b) Ya que

  \(\displaystyle \begin{array}{rcl} \lambda \alpha \left[ \left\vert \alpha \right\vert \right] (\varepsilon ) & =& 0=C_{0}^{0,0}(\Diamond ) \\ \lambda \alpha \left[ \left\vert \alpha \right\vert \right] (\alpha a) & =& \lambda \alpha \left[ \left\vert \alpha \right\vert \right] (\alpha )+1 \end{array} \)

  tenemos que \(\lambda \alpha \left[ \left\vert \alpha \right\vert \right] =R\left( C_{0}^{0,0},\mathcal{G}\right) \), donde \(\mathcal{G}_{a}=\) \( Suc\circ p_{1}^{1,1}\), para cada \(a\in \Sigma .\). \(\Box\)


  \textbf{\underline{Lemma 24:}}
  (a) \(C_{k}^{n,m},C_{\alpha }^{n,m}\in \mathrm{PR}^{\Sigma }\), para \( n,m,k\geq 0\), \(\alpha \in \Sigma ^{\ast }\).
  (b) \(C_{k}^{n,0}\in \mathrm{PR}^{\varnothing }\), para \(n,k\geq 0\).


  \textbf{\underline{Proof:}} (a) Note que \(C_{k+1}^{0,0}=\) \(Suc\circ C_{k}^{0,0}\), lo cual implica \( C_{k}^{0,0}\in \mathrm{PR}_{k}^{\Sigma }\), para \(k\geq 0\). Tambien note que \( C_{\alpha a}^{0,0}=d_{a}\circ C_{\alpha }^{0,0}\), lo cual dice que \( C_{\alpha }^{0,0}\in \mathrm{PR}^{\Sigma }\), para \(\alpha \in \Sigma ^{\ast } \). Para ver que \(C_{k}^{0,1}\in \mathrm{PR}^{\Sigma }\) notar que

  \(\displaystyle \begin{array}{rcl} C_{k}^{0,1}(\varepsilon ) & =& k=C_{k}^{0,0}(\Diamond ) \\ C_{k}^{0,1}(\alpha a) & =& C_{k}^{0,1}(\alpha )=p_{1}^{1,1}\left( C_{k}^{0,1}(\alpha ),\alpha \right) \end{array} \)

  lo cual implica que \(C_{k}^{0,1}=R\left( C_{k}^{0,0},\mathcal{G}\right) \), con \(\mathcal{G}_{a}=p_{1}^{1,1}\), \(a\in \Sigma \). En forma similar podemos ver que \(C_{k}^{1,0},C_{\alpha }^{1,0},C_{\alpha }^{0,1}\in \mathrm{PR} ^{\Sigma }\). Supongamos ahora que \(m >0\). Entonces
  \(\displaystyle \begin{array}{rcl} C_{k}^{n,m} & =& C_{k}^{0,1}\circ p_{n+1}^{n,m} \\ C_{\alpha }^{n,m} & =& C_{\alpha }^{0,1}\circ p_{n+1}^{n,m} \end{array} \)

  de lo cual obtenemos que \(C_{k}^{n,m},C_{\alpha }^{n,m}\in \mathrm{PR} ^{\Sigma }\). El caso \(n >0\) es similar
  (b) Use argumentos similares a los usados en la Proof de (a). \(\Box\)


  \textbf{\underline{Lemma 25:}}
  (a) \(\lambda xy\left[ x^{y}\right] \in \mathrm{PR}^{\varnothing }\).
  (b) \(\lambda t\alpha \left[ \alpha ^{t}\right] \in \mathrm{PR} ^{\Sigma }\).


  \textbf{\underline{Proof:}} (a) Note que

  \(\displaystyle \lambda tx\left[ x^{t}\right] =R\left( C_{1}^{1,0},\lambda xy\left[ x.y \right] \circ \left( p_{1}^{3,0},p_{3}^{3,0}\right) \right) \in \mathrm{PR} ^{\varnothing }. \)

  O sea que \(\lambda xy\left[ x^{y}\right] =\lambda tx\left[ x^{t}\right] \circ \left( p_{2}^{2,0},p_{1}^{2,0}\right) \in \mathrm{PR}^{\varnothing }\).
  (b) Note que

  \(\displaystyle \lambda t\alpha \left[ \alpha ^{t}\right] =R\left( C_{\varepsilon }^{0,1},\lambda \alpha \beta \left[ \alpha \beta \right] \circ \left( p_{3}^{1,2},p_{2}^{1,2}\right) \right) \in \mathrm{PR}^{\Sigma }. \)

  \(\Box\)


  \textbf{\underline{Lemma 26:}} Si \(< \) es un orden total estricto sobre un alfabeto no vacio \( \Sigma \), entonces \(s^{< }\), \(\#^{< }\) y \(\ast ^{< }\) pertenecen a \(\mathrm{PR} ^{\Sigma }\)

  \textbf{\underline{Proof:}} Supongamos \(\Sigma =\{a_{1},...,a_{k}\}\) y \(< \) dado por \(a_{1}< ...< a_{k}\). Ya que

  \(\displaystyle \begin{array}{rcl} s^{< }(\varepsilon ) & =& a_{1} \\ s^{< }(\alpha a_{i}) & =& \alpha a_{i+1}\text{, para }i< k \\ s^{< }(\alpha a_{k}) & =& s^{< }(\alpha )a_{1} \end{array} \)

  tenemos que \(s^{< }=R\left( C_{a_{1}}^{0,0},\mathcal{G}\right) \), donde \( \mathcal{G}_{a_{i}}=d_{a_{i+1}}\circ p_{1}^{0,2}\), para \(i=1,...,k-1\) y \( \mathcal{G}_{a_{k}}=d_{a_{1}}\circ p_{2}^{0,2}.\) O sea que \(s^{< }\in \mathrm{ PR}^{\Sigma }.\) Ya que
  \(\displaystyle \begin{array}{rcl} \ast ^{< }(0) & =& \varepsilon \\ \ast ^{< }(t+1) & =& s^{< }(\ast ^{< }(t)) \end{array} \)

  podemos ver que \(\ast ^{< }\in \mathrm{PR}^{\Sigma }.\) Ya que
  \(\displaystyle \begin{array}{rcl} \#^{< }(\varepsilon ) & =& 0 \\ \#^{< }(\alpha a_{i}) & =& \#^{< }(\alpha ).k+i\text{, para }i=1,...,k, \end{array} \)

  tenemos que \(\#^{< }=R\left( C_{0}^{0,0},\mathcal{G}\right) \), donde
  \(\displaystyle \mathcal{G}_{a_{i}}=\lambda xy\left[ x+y\right] \circ \left( \lambda xy\left[ x.y\right] \circ \left( p_{1}^{1,1},C_{k}^{1,1}\right) ,C_{i}^{1,1}\right) \text{, para }i=1,...,k\text{.} \)

  O sea que \(\#^{< }\in \mathrm{PR}^{\Sigma }\). \(\Box\)

  \textbf{\underline{Lemma 27:}}
  (a) \(\lambda xy\left[ x\dot{-}y\right] \in \mathrm{PR}^{\varnothing }.\)
  (b) \(\lambda xy\left[ \max (x,y)\right] \in \mathrm{PR}^{\varnothing }.\)
  (c) \(\lambda xy\left[ x=y\right] \in \mathrm{PR}^{\varnothing }.\)
  (d) \(\lambda xy\left[ x\leq y\right] \in \mathrm{PR}^{\varnothing }.\)
  (e) Si \(\Sigma \) es no vacio, entonces \(\lambda \alpha \beta \left[ \alpha =\beta \right] \in \mathrm{PR}^{\Sigma }\)


  \textbf{\underline{Proof:}} (a) Primero notar que \(\lambda x\left[ x\dot{-}1\right] =R\left( C_{0}^{0,0},p_{2}^{2,0}\right) \in \mathrm{PR}^{\varnothing }.\) Tambien note que

  \(\displaystyle \lambda tx\left[ x\dot{-}t\right] =R\left( p_{1}^{1,0},\lambda x\left[ x\dot{ -}1\right] \circ p_{1}^{3,0}\right) \in \mathrm{PR}^{\varnothing }. \)

  O sea que \(\lambda xy\left[ x\dot{-}y\right] =\lambda tx\left[ x\dot{-}t \right] \circ \left( p_{2}^{2,0},p_{1}^{2,0}\right) \in \mathrm{PR} ^{\varnothing }.\)
  (b) Note que \(\lambda xy\left[ \max (x,y)\right] =\lambda xy\left[ (x+(y\dot{ -}x)\right] .\)

  (c) Note que \(\lambda xy\left[ x=y\right] =\lambda xy\left[ 1\dot{-}((x\dot{- }y)+(y\dot{-}x))\right] .\)

  (d) Note que \(\lambda xy\left[ x\leq y\right] =\lambda xy\left[ 1\dot{-}(x \dot{-}y)\right] .\)

  (e) Sea \(< \) un orden total estricto sobre \(\Sigma .\) Ya que

  \(\displaystyle \alpha =\beta \text{ sii }\#^{< }(\alpha )=\#^{< }(\beta ) \)

  tenemos que
  \(\displaystyle \lambda \alpha \beta \left[ \alpha =\beta \right] =\lambda xy\left[ x=y \right] \circ \left( \#^{< }\circ p_{1}^{0,2},\#^{< }\circ p_{2}^{0,2}\right) . \)

  O sea que podemos aplicar (c) y Lema 28 implica que \(\chi _{S}\) es \( \Sigma \)-p.r.. \(\Box\)


  \textbf{\underline{Lemma 28:}}
    \textbf{Hacer}
  \textbf{\underline{Proof:}}

  \textbf{\underline{Lemma 29:}}
    \textbf{Hacer}

  \textbf{\underline{Proof:}}

  \textbf{\underline{Corollary 30:}}
    \textbf{Hacer}

  \textbf{\underline{Proof:}}

  %% =========== FALTA UNA PARTE ===========%%

  \textbf{\underline{Lemma 31:}} Supongamos \(S_{1},...,S_{n}\subseteq \omega \), \( L_{1},...,L_{m}\subseteq \Sigma ^{\ast }\) son conjuntos no vacios. Entonces \( S_{1}\times ...\times S_{n}\times L_{1}\times ...\times L_{m}\) es \(\Sigma \) -p.r. sii \(S_{1},...,S_{n},L_{1},...,L_{m}\) son \(\Sigma \)-p.r.

  \textbf{\underline{Proof:}} (\(\Rightarrow \)) Veremos por ejemplo que \(L_{1}\) es \(\Sigma \)-p.r.. Sea \( (z_{1},...,z_{n},\zeta _{1},...,\zeta _{m})\) un elemento fijo de \( S_{1}\times ...\times S_{n}\times L_{1}\times ...\times L_{m}.\) Note que

  \(\displaystyle \alpha \in L_{1}\text{ sii }(z_{1},...,z_{n},\alpha ,\zeta _{2},...,\zeta _{m})\in S_{1}\times ...\times S_{n}\times L_{1}\times ...\times L_{m}, \)

  lo cual implica que
  \(\displaystyle \chi _{L_{1}}=\chi _{S_{1}\times ...\times S_{n}\times L_{1}\times ...\times L_{m}}\circ \left( C_{z_{1}}^{0,1},...,C_{z_{n}}^{0,1},p_{1}^{0,1},C_{\zeta _{2}}^{0,1},...,C_{\zeta _{m}}^{0,1}\right) . \)

  (\(\Leftarrow \)) Note que \(\chi _{S_{1}\times ...\times S_{n}\times L_{1}\times ...\times L_{m}}\) es el predicado
  \(\displaystyle \left( \chi _{S_{1}}\circ p_{1}^{n,m}\wedge ...\wedge \chi _{S_{n}}\circ p_{n}^{n,m}\wedge \chi _{L_{1}}\circ p_{n+1}^{n,m}\wedge ...\wedge \chi _{L_{m}}\circ p_{n+m}^{n,m}\right) . \)

  \(\Box\)


  \textbf{\underline{Lemma 32:}} Supongamos \(f:D_{f}\subseteq \omega ^{n}\times \Sigma ^{\ast m}\rightarrow O\) es \(\Sigma \)-p.r., donde \(O\in \{\omega ,\Sigma ^{\ast }\}.\) Si \(S\subseteq D_{f}\) es \(\Sigma \)-p.r., entonces \(f\mid _{S}\) es \(\Sigma \)-p.r..

  \textbf{\underline{Proof:}} Supongamos \(O=\Sigma ^{\ast }\). Entonces

  \(\displaystyle f\mid _{S}=\lambda x\alpha \left[ \alpha ^{x}\right] \circ \left( Suc\circ Pred\circ \chi _{S},f\right) \)

  es \(\Sigma \)-p.r.. El caso \(O=\omega \) es similar usando \(\lambda xy\left[ x^{y}\right] \) en lugar de \(\lambda x\alpha \left[ \alpha ^{x}\right] \). \(\Box\)


  \textbf{\underline{Lemma 33:}} Si \(f:D_{f}\subseteq \omega ^{n}\times \Sigma ^{\ast m}\rightarrow O\) es \(\Sigma \)-p.r., entonces existe una funcion \(\Sigma \) -p.r. \(\bar{f}:\omega ^{n}\times \Sigma ^{\ast m}\rightarrow O\), tal que \(f= \bar{f}\mid _{D_{f}}\).

  \textbf{\underline{Proof:}} Es facil ver por induccion en \(k\) que el enunciado se cumple para cada \(f\in \mathrm{PR}_{k}^{\Sigma }\) \(\Box\)


  \textbf{\underline{Proposition 34:}} Un conjunto \(S\) es \(\Sigma \)-p.r. sii \(S\) es el dominio de una funcion \(\Sigma \)-p.r.\(.\)

  \textbf{\underline{Proof:}} (\(\Rightarrow \)) Note que \(S=D_{Pred\circ \chi _{S}}.\)

  (\(\Leftarrow \)) Probaremos por induccion en \(k\) que \(D_{F}\) es \(\Sigma \) -p.r., para cada \(F\in \mathrm{PR}_{k}^{\Sigma }.\) El caso \(k=0\) es facil\(.\) Supongamos el resultado vale para un \(k\) fijo y supongamos \(F\in \mathrm{PR} _{k+1}^{\Sigma }.\) Veremos entonces que \(D_{F}\) es \(\Sigma \)-p.r.. Hay varios casos. Consideremos primero el caso en que \(F=R(f,g)\), donde

  \(\displaystyle \begin{array}{rcl} f & :& S_{1}\times ...\times S_{n}\times L_{1}\times ...\times L_{m}\rightarrow \Sigma ^{\ast } \\ g & :& \omega \times S_{1}\times ...\times S_{n}\times L_{1}\times ...\times L_{m}\times \Sigma ^{\ast }\rightarrow \Sigma ^{\ast }, \end{array} \)

  con \(S_{1},...,S_{n}\subseteq \omega \) y \(L_{1},...,L_{m}\subseteq \Sigma ^{\ast }\) conjuntos no vacios y \(f,g\in \mathrm{PR}_{k}^{\Sigma }\). Notese que por definicion de \(R(f,g)\), tenemos que
  \(\displaystyle D_{F}=\omega \times S_{1}\times ...\times S_{n}\times L_{1}\times ...\times L_{m}. \)

  Por hipotesis inductiva tenemos que \(D_{f}=S_{1}\times ...\times S_{n}\times L_{1}\times ...\times L_{m}\) es \(\Sigma \)-p.r., lo cual por el Lema 31 nos dice que los conjuntos \(S_{1},...,S_{n}\), \( L_{1},...,L_{m}\) son \(\Sigma \)-p.r.. Ya que \(\omega \) es \(\Sigma \)-p.r., el Lema 31 nos dice que \(D_{F}\) es \(\Sigma \)-p.r..
  Los otros casos de recursion primitiva son dejados al lector.

  Supongamos ahora que \(F=g\circ (g_{1},...,g_{n+m})\), donde

  \(\displaystyle \begin{array}{rcl} g & :& D_{g}\subseteq \omega ^{n}\times \Sigma ^{\ast m}\rightarrow O \\ g_{i} & :& D_{g_{i}}\subseteq \omega ^{k}\times \Sigma ^{\ast l}\rightarrow \omega \text{, }i=1,...,n \\ g_{i} & :& D_{g_{i}}\subseteq \omega ^{k}\times \Sigma ^{\ast l}\rightarrow \Sigma ^{\ast },i=n+1,...,n+m \end{array} \)

  estan en \(\mathrm{PR}_{k}^{\Sigma }.\) Por Lema 33, hay funciones \(\Sigma \)-p.r. \(\bar{g}_{1},...,\bar{g}_{n+m}\) las cuales son \( \Sigma \)-totales y cumplen
  \(\displaystyle g_{i}=\bar{g}_{i}\mid _{D_{g_{i}}}\text{, para }i=1,...,n+m. \)

  Por hipotesis inductiva los conjuntos \(D_{g}\), \(D_{g_{i}}\), \(i=1,...,n+m\), son \(\Sigma \)-p.r. y por lo tanto
  \(\displaystyle S=\bigcap_{i=1}^{n+m}D_{g_{i}} \)

  lo es. Notese que
  \(\displaystyle \chi _{D_{F}}=(\chi _{D_{g}}\circ \left( \bar{g}_{1},...,\bar{g} _{n+m}\right) \wedge \chi _{S}) \)

  lo cual nos dice que \(D_{F}\) es \(\Sigma \)-p.r.. \(\Box\)


  \textbf{\underline{Lemma 35:}} Supongamos \(f_{i}:D_{f_{i}}\subseteq \omega ^{n}\times \Sigma ^{\ast m}\rightarrow O\), \(i=1,...,k\), son funciones \(\Sigma \)-p.r. tales que \(D_{f_{i}}\cap D_{f_{j}}=\varnothing \) para \(i\neq j.\) Entonces \(f_{1}\cup ...\cup f_{k}\) es \(\Sigma \)-p.r..

  \textbf{\underline{Proof:}} Supongamos \(O=\Sigma ^{\ast }\) y \(k=2.\) Sean

  \(\displaystyle \bar{f}_{i}:\omega ^{n}\times \Sigma ^{\ast m}\rightarrow \Sigma ^{\ast },i=1,2, \)

  funciones \(\Sigma \)-p.r. tales que \(\bar{f}_{i}\mid _{D_{f_{i}}}=f_{i}\), \( i=1,2\) (Lema 33)\(.\) Por Lema 34 los conjuntos \(D_{f_{1}}\) y \(D_{f_{2}}\) son \(\Sigma \)-p.r. y por lo tanto lo es \( D_{f_{1}}\cup D_{f_{2}}\). Ya que
  \(\displaystyle f_{1}\cup f_{2}=\left( \lambda \alpha \beta \left[ \alpha \beta \right] \circ (\lambda x\alpha \left[ \alpha ^{x}\right] \circ (\chi _{D_{f_{1}}}, \bar{f}_{1}),\lambda x\alpha \left[ \alpha ^{x}\right] \circ (\chi _{D_{f_{2}}},\bar{f}_{2}))\right) \mid _{D_{f_{1}}\cup D_{f_{2}}} \)

  tenemos que \(f_{1}\cup f_{2}\) es \(\Sigma \)-p.r..
  El caso \(k >2\) puede probarse por induccion ya que

  \(\displaystyle f_{1}\cup ...\cup f_{k}=(f_{1}\cup ...\cup f_{k-1})\cup f_{k}. \)

  \(\Box\)


  \textbf{\underline{Corollary 36:}} Supongamos \(f\) es una funcion \(\Sigma \)-mixta cuyo dominio es finito. Entonces \(f\) es \(\Sigma \)-p.r..

  \textbf{\underline{Proof:}} Supongamos \(f:D_{f}\subseteq \omega ^{n}\times \Sigma ^{\ast m}\rightarrow O\) , con \(D_{f}=\{e_{1},...,e_{k}\}\). Por el Corolario 30, cada \( \{e_{i}\}\) es \(\Sigma \)-p.r. por lo cual el Lema 32 nos dice que \(C_{f(e_{i})}^{n,m}\mid _{\{e_{1}\}}\) es \(\Sigma \)-p.r.. O sea que

  \(\displaystyle f=C_{f(e_{1})}^{n,m}\mid _{\{e_{1}\}}\cup ...\cup C_{f(e_{k})}^{n,m}\mid _{\{e_{k}\}} \)

  es \(\Sigma \)-p.r.. \(\Box\)


  \textbf{\underline{Lemma 37:}} \(\lambda i\alpha \left[ \lbrack \alpha ]_{i}\right] \) es \(\Sigma \)-p.r..

  \textbf{\underline{Proof:}} Note que

  \(\displaystyle \begin{array}{rcl} \lbrack \varepsilon ]_{i} & =& \varepsilon \\ \lbrack \alpha a]_{i} & =& \left\{ \begin{array}{lll} \lbrack \alpha ]_{i} & & \text{si }i\neq \left\vert \alpha \right\vert +1 \\ a & & \text{si }i=\left\vert \alpha \right\vert +1 \end{array} \right. \end{array} \)

  lo cual dice que \(\lambda i\alpha \left[ \lbrack \alpha ]_{i}\right] =R\left( C_{\varepsilon }^{1,0},\mathcal{G}\right) \), donde \(\mathcal{G} _{a}:\omega \times \Sigma ^{\ast }\times \Sigma ^{\ast }\rightarrow \Sigma ^{\ast }\) es dada por
  \(\displaystyle \mathcal{G}_{a}(i,\alpha ,\zeta )=\left\{ \begin{array}{lll} \zeta & & \text{si }i\neq \left\vert \alpha \right\vert +1 \\ a & & \text{si }i=\left\vert \alpha \right\vert +1 \end{array} \right. \)

  O sea que solo resta probar que cada \(\mathcal{G}_{a}\) es \(\Sigma \)-p.r.. Primero note que los conjuntos
  \(\displaystyle \begin{array}{rcl} S_{1} & =& \left\{ (i,\alpha ,\zeta )\in \omega \times \Sigma ^{\ast }\times \Sigma ^{\ast }:i\neq \left\vert \alpha \right\vert +1\right\} \\ S_{2} & =& \left\{ (i,\alpha ,\zeta )\in \omega \times \Sigma ^{\ast }\times \Sigma ^{\ast }:i=\left\vert \alpha \right\vert +1\right\} \end{array} \)

  son \(\Sigma \)-p.r. ya que
  \(\displaystyle \begin{array}{rcl} \chi _{S_{1}} & =& \lambda xy\left[ x\neq y\right] \circ \left( p_{1}^{1,2},Suc\circ \lambda \alpha \left[ \left\vert \alpha \right\vert \right] \circ p_{2}^{1,2}\right) \\ \chi _{S_{2}} & =& \lambda xy\left[ x=y\right] \circ \left( p_{1}^{1,2},Suc\circ \lambda \alpha \left[ \left\vert \alpha \right\vert \right] \circ p_{2}^{1,2}\right) . \end{array} \)

  Ya que
  \(\displaystyle \mathcal{G}_{a}=p_{3}^{1,2}\mid _{S_{1}}\cup C_{a}^{1,2}\mid _{S_{2}}, \)

  el Lema 35 nos dice que \(\mathcal{G}_{a}\) es \(\Sigma \)-p.r., para cada \(a\in \Sigma \). \(\Box\)


  \textbf{\underline{Lemma 38:}} Sean \(n,m\geq 0\).
  (a) Si \(f:\omega \times S_{1}\times ...\times S_{n}\times L_{1}\times ...\times L_{m}\rightarrow \omega \) es \(\Sigma \)-p.r., con \( S_{1},...,S_{n}\subseteq \omega \) y \(L_{1},...,L_{m}\subseteq \Sigma ^{\ast } \) no vacios, entonces lo son las funciones \(\lambda xy\vec{x}\vec{\alpha} \left[ \sum_{t=x}^{t=y}f(t,\vec{x},\vec{\alpha})\right] \) y \(\lambda xy\vec{x }\vec{\alpha}\left[ \prod_{t=x}^{t=y}f(t,\vec{x},\vec{\alpha})\right] \).
  (b) Si \(f:\omega \times S_{1}\times ...\times S_{n}\times L_{1}\times ...\times L_{m}\rightarrow \Sigma ^{\ast }\) es \(\Sigma \)-p.r., con \( S_{1},...,S_{n}\subseteq \omega \) y \(L_{1},...,L_{m}\subseteq \Sigma ^{\ast } \) no vacios, entonces lo es la funcion \(\lambda xy\vec{x}\vec{\alpha}\left[ \subset _{t=x}^{t=y}f(t,\vec{x},\vec{\alpha})\right] \)

  \textbf{\underline{Proof:}} (a) Sea \(G=\lambda tx\vec{x}\vec{\alpha}\left[ \sum_{i=x}^{i=t}f(i,\vec{x}, \vec{\alpha})\right] \). Ya que

  \(\displaystyle \lambda xy\vec{x}\vec{\alpha}\left[ \sum_{i=x}^{i=y}f(i,\vec{x},\vec{\alpha}) \right] =G\circ \left( p_{2}^{n+2,m},p_{1}^{n+2,m},p_{3}^{n+2,m},...,p_{n+m+2}^{n+2,m}\right) \)

  solo tenemos que probar que \(G\) es \(\Sigma \)-p.r.. Primero note que
  \(\displaystyle \begin{array}{rcl} G(0,x,\vec{x},\vec{\alpha}) & =& \left\{ \begin{array}{lll} 0 & & \text{si }x >0 \\ f(0,\vec{x},\vec{\alpha}) & & \text{si }x=0 \end{array} \right. \\ G(t+1,x,\vec{x},\vec{\alpha}) & =& \left\{ \begin{array}{lll} 0 & & \text{si }x >t+1 \\ G(t,x,\vec{x},\vec{\alpha})+f(t+1,\vec{x},\vec{\alpha}) & & \text{si }x\leq t+1 \end{array} \right. \end{array} \)

  Sean
  \(\displaystyle \begin{array}{rcl} D_{1} & =& \left\{ (x,\vec{x},\vec{\alpha})\in \omega \times S_{1}\times ...\times S_{n}\times L_{1}\times ...\times L_{m}:x >0\right\} \\ D_{2} & =& \left\{ (x,\vec{x},\vec{\alpha})\in \omega \times S_{1}\times ...\times S_{n}\times L_{1}\times ...\times L_{m}:x=0\right\} \\ H_{1} & =& \left\{ (z,t,x,\vec{x},\vec{\alpha})\in \omega ^{3}\times S_{1}\times ...\times S_{n}\times L_{1}\times ...\times L_{m}:x >t+1\right\} \\ H_{2} & =& \left\{ (z,t,x,\vec{x},\vec{\alpha})\in \omega ^{3}\times S_{1}\times ...\times S_{n}\times L_{1}\times ...\times L_{m}:x\leq t+1\right\} . \end{array} \)

  Es facil de chequear que estos conjuntos son \(\Sigma \)-p.r.. Veamos que por ejemplo \(H_{1}\) lo es. Es decir debemos ver que \(\chi _{H_{1}}\) es \(\Sigma \) -p.r.. Ya que \(f\) es \(\Sigma \)-p.r. tenemos que \(D_{f}=\omega \times S_{1}\times ...\times S_{n}\times L_{1}\times ...\times L_{m}\) es \(\Sigma \) -p.r., lo cual por el Lema 31 nos dice que los conjuntos \( S_{1},...,S_{n}\), \(L_{1},...,L_{m}\) son \(\Sigma \)-p.r.. Ya que \(\omega \) es \( \Sigma \)-p.r., el Lema 31 nos dice que \(R=\omega ^{3}\times S_{1}\times ...\times S_{n}\times L_{1}\times ...\times L_{m}\) es \(\Sigma \)-p.r.. Notese que \(\chi _{H_{1}}=(\chi _{R}\wedge \lambda ztx\vec{x} \vec{\alpha}\left[ x >t+1\right] )\) por cual \(\chi _{H_{1}}\) es \(\Sigma \) -p.r. ya que es la conjuncion de dos predicados \(\Sigma \)-p.r.
  Ademas note que \(G=R(h,g)\), donde

  \(\displaystyle \begin{array}{rcl} h & =& C_{0}^{n+1,m}\mid _{D_{1}}\cup \lambda x\vec{x}\vec{\alpha}\left[ f(0, \vec{x},\vec{\alpha})\right] \mid _{D_{2}} \\ g & =& C_{0}^{n+3,m}\mid _{H_{1}}\cup \lambda ztx\vec{x}\vec{\alpha}\left[ z+f(t+1,\vec{x},\vec{\alpha})\right] )\mid _{H_{2}} \end{array} \)

  O sea que los Lemas 35 y 32 garantizan que \(G\) es \( \Sigma \)-p.r.. \(\Box\)


  \textbf{\underline{Lemma 39:}} Sean \(n,m\geq 0\).
  (a) Sea \(P:S\times S_{1}\times ...\times S_{n}\times L_{1}\times ...\times L_{m}\rightarrow \omega \) un predicado \(\Sigma \)-p.r. y supongamos \(\bar{S}\subseteq S\) es \(\Sigma \)-p.r.. Entonces \(\lambda x\vec{x}\vec{\alpha }\left[ (\forall t\in \bar{S})_{t\leq x}\;P(t,\vec{x},\vec{\alpha})\right] \) y \(\lambda x\vec{x}\vec{\alpha}\left[ (\exists t\in \bar{S})_{t\leq x}\;P(t, \vec{x},\vec{\alpha})\right] \) son predicados \(\Sigma \)-p.r.. (Note que el dominio de estos predicados es \(\omega \times S_{1}\times ...\times S_{n}\times L_{1}\times ...\times L_{m}\))
  (b) Sea \(P:S_{1}\times ...\times S_{n}\times L_{1}\times ...\times L_{m}\times L\rightarrow \omega \) un predicado \(\Sigma \)-p.r. y supongamos \( \bar{L}\subseteq L\) es \(\Sigma \)-p.r.. Entonces \(\lambda x\vec{x}\vec{\alpha} \left[ (\forall \alpha \in \bar{L})_{\left\vert \alpha \right\vert \leq x}\;P(\vec{x},\vec{\alpha},\alpha )\right] \) y \(\lambda x\vec{x}\vec{\alpha} \left[ (\exists \alpha \in \bar{L})_{\left\vert \alpha \right\vert \leq x}\;P(\vec{x},\vec{\alpha},\alpha )\right] \) son predicados \(\Sigma \)-p.r..

  \textbf{\underline{Proof:}} (a) Sea

  \(\displaystyle \bar{P}=P\mid _{\bar{S}\times S_{1}\times ...\times S_{n}\times L_{1}\times ...\times L_{m}}\cup C_{1}^{1+n,m}\mid _{(\omega -\bar{S})\times S_{1}\times ...\times S_{n}\times L_{1}\times ...\times L_{m}} \)

  Notese que \(\bar{P}\) es \(\Sigma \)-p.r.. Ya que
  \(\displaystyle \begin{array}{rcl} \lambda x\vec{x}\vec{\alpha}\left[ (\forall t\in \bar{S})_{t\leq x}P(t,\vec{x },\vec{\alpha})\right] & =& \lambda x\vec{x}\vec{\alpha}\left[ \prod\limits_{t=0}^{t=x}\bar{P}(t,\vec{x},\vec{\alpha})\right] \\ & =& \lambda xy\vec{x}\vec{\alpha}\left[ \prod\limits_{t=x}^{t=y}\bar{P}(t, \vec{x},\vec{\alpha})\right] \circ \left( C_{0}^{1+n,m},p_{1}^{1+n,m},...,p_{1+n+m}^{1+n,m}\right) \end{array} \)

  el Lema 38 implica que \(\lambda x\vec{x}\vec{\alpha}\left[ (\forall t\in \bar{S})_{t\leq x}\;P(t,\vec{x},\vec{\alpha})\right] \) es \( \Sigma \)-p.r..
  Finalmente note que

  \(\displaystyle \lambda x\vec{x}\vec{\alpha}\left[ (\exists t\in \bar{S})_{t\leq x}\;P(t, \vec{x},\vec{\alpha})\right] =\lnot \lambda x\vec{x}\vec{\alpha}\left[ (\forall t\in \bar{S})_{t\leq x}\;\lnot P(t,\vec{x},\vec{\alpha})\right] \)

  es \(\Sigma \)-p.r..
  (b) Sea \(< \) un orden total estricto sobre \(\Sigma .\) Sea \(k\) el cardinal de \( \Sigma \). Ya que

  \(\displaystyle \left\vert \alpha \right\vert \leq x\text{ sii }\#^{< }(\alpha )\leq \sum_{\iota =1}^{i=x}k^{i}, \)

  (ejercicio) tenemos que
  \(\displaystyle \lambda x\vec{x}\vec{\alpha}\left[ (\forall \alpha \in \bar{L})_{\left\vert \alpha \right\vert \leq x}P(\vec{x},\vec{\alpha},\alpha )\right] =\lambda x \vec{x}\vec{\alpha}\left[ (\forall t\in \#^{< }(\bar{L}))_{t\leq \sum_{\iota =1}^{i=x}k^{i}}P(\vec{x},\vec{\alpha},\ast ^{< }(t))\right] \)

  Sea \(H=\lambda t\vec{x}\vec{\alpha}\left[ P(\vec{x},\vec{\alpha},\ast ^{< }(t))\right] .\) Notese que \(H\) es \(\Sigma \)-p.r. y
  \(\displaystyle D_{H}=\#^{< }(L)\times S_{1}\times ...\times S_{n}\times L_{1}\times ...\times L_{m} \)

  Ademas note que \(\#^{< }(\bar{L})\) es \(\Sigma \)-p.r. (ejercicio), lo cual por (a) implica que
  \(\displaystyle Q=\lambda x\vec{x}\vec{\alpha}\left[ (\forall t\in \#^{< }(\bar{L}))_{t\leq x}H(t,\vec{x},\vec{\alpha})\right] \)

  es \(\Sigma \)-p.r.. O sea que
  \(\displaystyle \lambda x\vec{x}\vec{\alpha}\left[ (\forall \alpha \in \bar{L})_{\left\vert \alpha \right\vert \leq x}\;P(\vec{x},\vec{\alpha},\alpha )\right] =Q\circ \left( \lambda x\vec{x}\vec{\alpha}\left[ \sum\limits_{\iota =1}^{i=x}k^{i} \right] ,p_{1}^{1+n,m},...,p_{1+n+m}^{1+n,m}\right) \)

  es \(\Sigma \)-p.r.. \(\Box\)


  \textbf{\underline{Lemma 40:}}
  (a) El predicado \(\lambda xy\left[ x\text{ divide }y\right] \) es \( \varnothing \)-p.r..
  (b) El predicado \(\lambda x\left[ x\text{ es primo}\right] \) es \( \varnothing \)-p.r..
  (c) El predicado \(\lambda \alpha \beta \left[ \alpha \text{\ }\mathrm{ inicial}\ \beta \right] \) es \(\Sigma \)-p.r..

  \textbf{\underline{Proof:}} (a) Si tomamos \(P=\lambda tx_{1}x_{2}\left[ x_{2}=t.x_{1}\right] \in \mathrm{PR}^{\varnothing }\), tenemos que

  \(\displaystyle \begin{array}{rcl} \lambda x_{1}x_{2}\left[ x_{1}\text{ divide }x_{2}\right] & =& \lambda x_{1}x_{2}\left[ (\exists t\in \omega )_{t\leq x_{2}}\;P(t,x_{1},x_{2}) \right] \\ & =& \lambda xx_{1}x_{2}\left[ (\exists t\in \omega )_{t\leq x}\;P(t,x_{1},x_{2})\right] \circ \left( p_{2}^{2,0},p_{1}^{2,0},p_{2}^{2,0}\right) \end{array} \)

  por lo que podemos aplicar el lema anterior.
  (b) Ya que

  \(\displaystyle x\text{ es primo sii }x >1\wedge \left( (\forall t\in \omega )_{t\leq x}\;t=1\vee t=x\vee \lnot (t\text{ divide }x)\right) \)

  podemos usar un argumento similar al de la prueba de (a).
  (c) es dejado al lector. \(\Box\)
