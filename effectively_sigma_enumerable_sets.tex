\section{Conjuntos $\Sigma$-efectivamente enumerables}

  \textbf{\underline{Lemma 14:}} Sean \(S_{1},S_{2}\subseteq \omega ^{n}\times \Sigma ^{\ast m}\) conjuntos \( \Sigma \)-efectivamente enumerables. Entonces \(S_{1}\cup S_{2}\) y \(S_{1}\cap S_{2}\) son \(\Sigma \)-efectivamente enumerables.

  \textbf{\underline{Proof:}} El caso en el que alguno de los conjuntos es vacio es trivial. Supongamos que ambos conjuntos son no vacios y sean \(\mathbb{P}_{1}\) y \(\mathbb{P}_{2}\) procedimientos que enumeran a \(S_{1}\) y \(S_{2}\). El siguiente procedimiento enumera al conjunto \(S_{1}\cup S_{2}\):

- Si \(x\) es par realizar \(\mathbb{P}_{1}\) partiendo de \(x/2\) y dar el elemento de \(S_{1}\) obtenido como salida. Si \(x\) es impar realizar \(\mathbb{P }_{2}\) partiendo de \((x-1)/2\) y dar el elemento de \(S_{2}\) obtenido como salida.
Veamos ahora que \(S_{1}\cap S_{2}\) es \(\Sigma \)-efectivamente enumerable. Si \(S_{1}\cap S_{2}=\varnothing \) entonces no hay nada que probar. Supongamos entonces que \(S_{1}\cap S_{2}\) en no vacio. Sea \(z_{0}\) un elemento fijo de \( S_{1}\cap S_{2}.\) Sea \(\mathbb{P}\) un procedimiento efectivo el cual enumere a \(\omega \times \omega \) (ver el ejemplo de mas arriba). Un procedimiento que enumera a \(S_{1}\cap S_{2}\) es el siguiente

Etapa 1

Realizar \(\mathbb{P}\) con dato de entrada \(x\), para obtener un par \((x_{1},x_{2})\in \omega \times \omega \).

Etapa 2

Realizar \(\mathbb{P}_{1}\) con dato de entrada \(x_{1}\) para obtener un elemento \(z_{1}\in S_{1}\)

Etapa 3

Realizar \(\mathbb{P}_{2}\) con dato de entrada \(x_{2}\) para obtener un elemento \(z_{2}\in S_{2}\)

Etapa 4

Si \(z_{1}=z_{2}\), entonces dar como dato de salida \(z_{1}.\) En caso contrario dar como dato de salida \(z_{0}\). \(\Box\)
