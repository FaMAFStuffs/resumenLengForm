
  % Lemma 35: Con prueba. Solo el caso k = 2 y O = w.
  \begin{lemma}
    \PN Supongamos $f_{i}: D_{f_{i}} \subseteq \omega^{n} \times \Sigma^{\ast m} \rightarrow O, i = 1, \dotsc, k$, son
    funciones $\Sigma$-PR tales que $D_{f_{i}} \cap D_{f_{j}} = \emptyset$ para $i \neq j$, entonces $f_{1} \cup \dotsc
    \cup f_{k}$ es $\Sigma$-PR.
  \end{lemma}
  \begin{proof}
    \PN Vamos a probar solo el caso en que $k = 2$ y $O = \omega$. Sean:
    \begin{eqnarray*}
      \bar{f}_{1}: \omega^{n} \times \Sigma^{\ast m} \rightarrow \SIGMA \qquad \text{ tal que } \bar{f}_{1}
        \mid_{D_{f_{1}}} = f_{1} \\
      \bar{f}_{2}: \omega^{n} \times \Sigma^{\ast m} \rightarrow \SIGMA \qquad \text{ tal que } \bar{f}_{2}
        \mid_{D_{f_{2}}} = f_{2}
    \end{eqnarray*}

    \PN funciones $\Sigma$-PR por \textbf{Lemma 33}. Luego, por el \textbf{Lemma 34} los conjuntos $D_{f_{1}}$ y
    $D_{f_{2}}$ son $\Sigma$-PR y por lo tanto lo es $ D_{f_{1}} \cup D_{f_{2}}$. Ya que:
    \begin{eqnarray*}
      f_{1} \cup f_{2} &=& \left(\bar{f_{1}}^{\chi_{D_{f_{1}}}} . \bar{f_{2}}^{\chi_{D_{f_{2}}}}\right)
        \mid_{D_{f_{1}} \cup D_{f_{2}}} \\
      &=& \lambda xy \left[x.y\right] \circ \left(\lambda xy \left[x^{y}\right] \circ (\bar{f_{1}},
        \chi_{D_{f_{1}}}), \lambda xy \left[x^{y}\right] \circ (\bar{f_{2}}, \chi_{D_{f_{2}}}) \right) \mid_{D_{f_{1}}
        \cup D_{f_{2}}}
    \end{eqnarray*}

    \PN Por lo tanto, $f_{1} \cup f_{2}$ es $\Sigma$-PR.
  \end{proof}

  % Corollary 36: Sin prueba.
  \begin{corollary}
    \PN Supongamos $f$ es una función $\Sigma$-mixta cuyo dominio es finito, entonces $f$ es $\Sigma$-PR.
  \end{corollary}

  % Lemma 37: Sin prueba
  \begin{lemma}
    \PN $\lambda i\alpha \left[\lbrack \alpha]_{i}\right]$ es $\Sigma$-PR.
  \end{lemma}

  % Lemma 38: Con prueba. Solo el inciso (a).
  \begin{lemma}
    \PN Sean $n, m \geq 0$.

    \begin{enumerate}[a)]
      \item Si $f: \omega \times S_{1} \times \dotsc \times S_{n} \times L_{1} \times \dotsc \times L_{m} \rightarrow
        \omega$ es $\Sigma$-PR, con $ S_{1}, \dotsc, S_{n} \subseteq \omega$ y $L_{1}, \dotsc, L_{m} \subseteq \SIGMA$
        no vacíos, entonces lo son las funciones:
        \begin{eqnarray*}
          \text{\textbf{Sumatoria:} } \lambda xy\vec{x}\vec{\alpha} \left[\sum_{t=x}^{t=y} f(t,\vec{x},\vec{\alpha})
            \right] \\
          \text{\textbf{Productoria:} } \lambda xy\vec{x}\vec{\alpha} \left[\prod_{t=x}^{t=y}f(t,\vec{x},\vec{\alpha})
            \right]
        \end{eqnarray*}

      \item Si $f: \omega \times S_{1} \times \dotsc \times S_{n} \times L_{1} \times \dotsc \times L_{m} \rightarrow
        \SIGMA$ es $\Sigma$-PR, con $ S_{1}, \dotsc, S_{n} \subseteq \omega$ y $L_{1}, \dotsc, L_{m} \subseteq \SIGMA$
        no vacíos, entonces lo es la función:
        \[
          \text{\textbf{Concatenatoria:} } \lambda xy\vec{x}\vec{\alpha}\left[C_{t=x}^{t=y} f(t,\vec{x},\vec{\alpha})
            \right]
        \]
    \end{enumerate}
  \end{lemma}
  \begin{proof}
    \PN Se probará solamente el inciso (a).
    \PN Sea $G = \lambda tx\vec{x}\vec{\alpha} \left[\sum_{i=x}^{i=t} f(i,\vec{x},\vec{\alpha})\right]$. Ya que
    \[
      \lambda xy\vec{x}\vec{\alpha}\left[\sum_{i=x}^{i=y} f(i, \vec{x}, \vec{\alpha})\right] = G \circ \left(
      p_{2}^{n+2,m}, p_{1}^{n+2,m}, p_{3}^{n+2,m}, \dotsc, p_{n+m+2}^{n+2,m}\right)
    \]

    \PN solo tenemos que probar que $G$ es $\Sigma$-PR. Primero note que:
    \begin{eqnarray*}
      G(0,x,\vec{x},\vec{\alpha}) &=& \left\{
        \begin{array}{lll}
          0 && \text{si } x > 0 \\
          f(0,\vec{x},\vec{\alpha}) && \text{si } x = 0
        \end{array}\right. \\
      G(t+1,x,\vec{x},\vec{\alpha}) &=& \left\{
        \begin{array}{lll}
          0 && \text{si } x > t+1 \\
          G(t,x,\vec{x},\vec{\alpha}) + f(t+1,\vec{x},\vec{\alpha}) && \text{si } x \leq t+1
        \end{array} \right.
    \end{eqnarray*}

    \PN Sean:
    \begin{eqnarray*}
      P_{1} &=& \left\{(x,\vec{x},\vec{\alpha}) \in \omega \times S_{1} \times \dotsc \times S_{n} \times L_{1} \times
        \dotsc \times L_{m}: x > 0 \right\} \\
      P_{2} &=& \left\{(x,\vec{x},\vec{\alpha}) \in \omega \times S_{1} \times \dotsc \times S_{n} \times L_{1} \times
        \dotsc \times L_{m}: x = 0 \right\} \\
      Q_{1} &=& \left\{(z,t,x,\vec{x},\vec{\alpha}) \in \omega^{3} \times S_{1} \times \dotsc \times S_{n} \times L_{1}
        \times \dotsc \times L_{m}: x > t+1\right\} \\
      Q_{2} &=& \left\{(z,t,x,\vec{x},\vec{\alpha}) \in \omega^{3} \times S_{1} \times \dotsc \times S_{n} \times L_{1}
        \times \dotsc \times L_{m}: x \leq t+1\right\}
    \end{eqnarray*}

    \PN Notar que $P_{1}, P_{2}, Q_{1}, Q_{2}$ son conjuntos $\Sigma$-PR, probaremos solo $Q_{1}$. Debemos ver que
    $\chi_{Q_{1}}$ es $\Sigma$-PR.
    \begin{eqnarray*}
      f \text{ es } \Sigma-PR &\Rightarrow& D_{f} = \omega \times S_{1} \times \dotsc \times S_{n} \times L_{1} \times
        \dotsc \times L_{m} \text{ es } \Sigma-PR \qquad\; \text{\textbf{Proposition 34}} \\
      &\Rightarrow& S_{1}, \dotsc, S_{n}, L_{1}, \dotsc, L_{m} \text{ son  } \Sigma-PR \qquad\qquad\qquad\qquad\qquad
        \text{\textbf{Lemma 31}} \\
      \\
      \omega \text{ es } \Sigma-PR &\Rightarrow& R = \omega^{3} \times S_{1} \times \dotsc \times S_{n} \times L_{1}
        \times \dotsc \times L_{m} \text{ es } \Sigma-PR
    \end{eqnarray*}

    \PN Notar que:
    \[
      \chi_{Q_{1}} = (\chi_{R} \wedge \lambda ztx\vec{x} \vec{\alpha}\left[x > t+1\right])
    \]

    \PN por cual $\chi_{Q_{1}}$ es $\Sigma$-PR ya que es la conjunción de dos predicados $\Sigma$-PR.

    \PN Además notar que $G = R(g, h)$, donde:
    \begin{eqnarray*}
      g &=& C_{0}^{n+1,m} \mid_{P_{1}} \cup \; \lambda x\vec{x}\vec{\alpha}\left[f(0,\vec{x},\vec{\alpha})\right]
        \mid_{P_{2}} \\
      h &=& C_{0}^{n+3,m} \mid_{Q_{1}} \cup \; \lambda ztx\vec{x}\vec{\alpha}\left[z+f(t+1,\vec{x},\vec{\alpha})\right]
        \mid_{Q_{2}}
    \end{eqnarray*}

    \PN Por lo tanto, el \textbf{Lemma 35} y el \textbf{Lemma 32} garantizan que $G$ es $\Sigma$-PR.
  \end{proof}

  % Lemma 39: Con prueba. Solo el inciso (a).
  \begin{lemma}
    \PN Sean $n, m \geq 0$.

    \begin{enumerate}[a)]
      \item Sea $P: S \times S_{1} \times \dotsc \times S_{n} \times L_{1} \times \dotsc \times L_{m} \rightarrow
        \omega$ un predicado $\Sigma$-PR y supongamos $\bar{S} \subseteq S$ es $\Sigma$-PR, entonces:
        \begin{eqnarray*}
          \lambda x\vec{x}\vec{\alpha} \left[(\forall t \in \bar{S})_{t\leq x} \; P(t,\vec{x},\vec{\alpha})
            \right] \\
          \lambda x\vec{x}\vec{\alpha} \left[(\exists t \in \bar{S})_{t\leq x} \; P(t,\vec{x},\vec{\alpha})
            \right]
        \end{eqnarray*}

        \PN son predicados $\Sigma$-PR.

      \item Sea $P: S_{1} \times \dotsc \times S_{n} \times L_{1} \times \dotsc \times L_{m} \times L \rightarrow
        \omega$ un predicado $\Sigma$-PR y supongamos $\bar{L} \subseteq L$ es $\Sigma$-PR, entonces:
        \begin{eqnarray*}
          \lambda x\vec{x}\vec{\alpha} \left[(\forall \alpha \in \bar{L})_{\lvert \alpha \rvert
            \leq x} \; P(\vec{x},\vec{\alpha},\alpha)\right] \\
          \lambda x\vec{x}\vec{\alpha} \left[(\exists \alpha \in \bar{L})_{\lvert \alpha \rvert
            \leq x} \; P(\vec{x},\vec{\alpha},\alpha)\right]
        \end{eqnarray*}

        \PN son predicados $\Sigma$-PR.
    \end{enumerate}
  \end{lemma}
  \begin{proof}
    \PN Se probará solamente el inciso (a). Sea:
    \[
      \bar{P} = P \mid_{\bar{S} \times S_{1} \times \dotsc \times S_{n} \times L_{1} \times \dotsc \times L_{m}} \cup \;
      C_{1}^{1+n,m} \mid_{(\omega -\bar{S}) \times S_{1} \times \dotsc \times S_{n} \times L_{1} \times \dotsc \times
      L_{m}}
    \]

    \PN Notese que $\bar{P}$ es $\Sigma$-PR. Ya que:
    \begin{eqnarray*}
      \lambda x\vec{x}\vec{\alpha}\left[(\forall t \in \bar{S})_{t\leq x} \; P(t,\vec{x},\vec{\alpha})\right]
        &=& \lambda x\vec{x}\vec{\alpha}\left[\prod\limits_{t=0}^{t=x}\bar{P}(t,\vec{x},\vec{\alpha})\right] \\
      &=& \lambda xy\vec{x}\vec{\alpha}\left[\prod\limits_{t=x}^{t=y} \bar{P}(t,\vec{x},\vec{\alpha})\right]
        \circ \left(C_{0}^{1+n,m}, p_{1}^{1+n,m}, \dotsc, p_{1+n+m}^{1+n,m}\right)
    \end{eqnarray*}

    \PN el \textbf{Lemma 38} implica que $\lambda x\vec{x}\vec{\alpha}\left[(\forall t\in \bar{S})_{t\leq x} \;
    P(t,\vec{x},\vec{\alpha})\right]$ es $\Sigma$-PR.

    \PN Finalmente note que:
    \[
      \lambda x\vec{x}\vec{\alpha}\left[(\exists t \in \bar{S})_{t\leq x} \; P(t,\vec{x},\vec{\alpha})\right] = \lnot
      \lambda x\vec{x}\vec{\alpha}\left[(\forall t \in \bar{S})_{t\leq x} \; \lnot P(t,\vec{x},\vec{\alpha})\right]
    \]

    \PN es $\Sigma$-PR.
  \end{proof}

  % Lemma 40: Con prueba.
  \begin{lemma}
    \begin{enumerate}[a)]
      \item El predicado $\lambda xy\left[x \text{ divide } y\right]$ es $\emptyset$-PR.
      \item El predicado $\lambda x\left[x \text{ es primo}\right]$ es $\emptyset$-PR.
      \item El predicado $\lambda \alpha\beta \left[\alpha \text{\ }\mathrm{ inicial}\ \beta \right]$ es $\Sigma$-PR.
    \end{enumerate}
  \end{lemma}
  \begin{proof}
    \begin{enumerate}[a)]
      \item Si tomamos $P = \lambda tx_{1}x_{2}\left[x_{2}=t.x_{1}\right] \in \mathrm{PR}^{\emptyset}$, tenemos que:
        \begin{eqnarray*}
          \lambda x_{1}x_{2} \left[x_{1}\text{ divide } x_{2}\right] &=& \lambda x_{1}x_{2}\left[(\exists t
            \in \omega)_{t\leq x_{2}} \; P(t,x_{1},x_{2}) \right] \\
          &=& \lambda xx_{1}x_{2}\left[(\exists t \in \omega)_{t\leq x} \; P(t,x_{1},x_{2})\right] \circ
            \left(p_{2}^{2,0}, p_{1}^{2,0}, p_{2}^{2,0}\right)
        \end{eqnarray*}

        \PN por el \textbf{Lemma 39}, $\lambda xy\left[x \text{ divide } y\right]$ es $\emptyset$-PR.

      \item Ya que:
        \[
          x \text{ es primo } \Leftrightarrow x > 1 \wedge \left((\forall t \in \omega)_{t\leq x} \; t=1 \vee t=x \vee
          \lnot (t\text{ divide } x)\right)
        \]

        \PN tomamos $P = \lambda tx \left[ t=1 \vee t=x \vee (t \text{ divide } x)\right]$. Luego, tenemos que:
        \begin{eqnarray*}
          \lambda x \left[x \text{ es primo}\right] &=& \lambda x \left[x > 1\right] \wedge \lambda x
            \left[(\forall t \in \omega)_{t \leq x} \; P(t,x) \right] \\
          &=& \lambda x \left[x > 1\right] \wedge \lambda x_{1} x_{2} \left[(\forall t \in \omega)_{t \leq
            x_{1}} \; P(t,x_{2}) \right] \circ (p_{1}^{1,0}, p_{1}^{1,0})
        \end{eqnarray*}

        \PN por lo tanto, $\lambda x\left[x \text{ es primo}\right]$ es $\emptyset$-PR.

      \item Sea $P = \lambda \alpha\beta\gamma \left[\beta = \alpha\gamma\right]$, entonces:
        \begin{eqnarray*}
          \lambda \alpha\beta \left[\alpha \text{\ }\mathrm{ inicial}\ \beta \right] &=& \lambda \alpha\beta
            \left[(\exists \gamma \in \SIGMA)_{\lvert \gamma \rvert \leq \lvert \beta \rvert} \; P(\alpha, \beta,
            \gamma)\right] \\
          &=& \lambda x\alpha\beta \left[(\exists \gamma \in \SIGMA)_{\lvert \gamma \rvert \leq x} \; P(\alpha,\beta,
            \gamma)\right] \circ (\lambda \alpha \left[\lvert \alpha\rvert\right] \circ p_{2}^{0,2},p_{1}^{0,2},
            p_{2}^{0,2})
        \end{eqnarray*}

        \PN luego, $\lambda \alpha\beta \left[\alpha \text{\ }\mathrm{ inicial}\ \beta \right]$ es $\Sigma$-PR.
    \end{enumerate}
  \end{proof}

  % Lemma 41: Con prueba.
  \begin{lemma}
    \PN Si $P: D_{P} \subseteq \omega \times \omega^{n} \times \Sigma^{\ast m} \rightarrow \omega$ es un predicado
    $\Sigma$-efectivamente computable y $ D_{P}$ es $\Sigma$-efectivamente computable, entonces la función $M(P)$ es
    $\Sigma$-efectivamente computable.
  \end{lemma}
  \begin{proof}
    \PN Sean:

    \begin{itemize}
      \item $\mathbb{P}_{1}$ un procedimiento efectivo que compute $\chi_{D_{P}}$.
      \item $\mathbb{P}_{2}$ un procedimiento efectivo que compute el predicado $P$.
    \end{itemize}

    \PN El siguiente procedimiento computa $M(P)$:

    \vspace{3mm}
    \PN \textbf{Etapa 1:}
    Darle a la variable $T$ el valor $0$.

    \PN \textbf{Etapa 2:}
    Realizar el procedimiento $\mathbb{P}_{1}$ con entrada $(T,\vec{x},\vec{\alpha})$ para obtener el valor
    \textit{Booleano} $p$

    $\qquad\;\;\;\;$de salida.

    \PN \textbf{Etapa 3:}
    \textbf{Si $p=1$:} realizar $\mathbb{P}_{2}$ con entrada $(T,\vec{x},\vec{\alpha})$ para obtener el valor
    \textit{Booleano} $e$ de salida.

    $\qquad\;\;\;\;$\textbf{Si $p=0$:} aumentar en $1$ el valor de $T$, y dirigirse a la Etapa 2.

    \PN \textbf{Etapa 4:}
    \textbf{Si $e=1$:} dar como dato de salida $T$.

    $\qquad\;\;\;\;$\textbf{Si $e=0$:} aumentar en $1$ el valor de $T$, y dirigirse a la Etapa 2.
  \end{proof}

  % Theorem 42: Con prueba.
  \begin{theorem}
    \PN Si $f \in \mathrm{R}^{\Sigma}$, entonces $f$ es $\Sigma$-efectivamente computable.
  \end{theorem}
  \begin{proof}
    \PN Recordemos que $R^{\Sigma} = \bigcup\limits_{k \geq 0} R_{k}^{\Sigma}$. Supongamos que $f \in
    R_{k}^{\Sigma}$, probaremos este teorema por inducción en $k$.

    \vspace{3mm}
    \PN \underline{Caso Base:} \begin{tabular}{|c|} \hline $k = 0$ \\\hline \end{tabular}

    \vspace{1mm}
    \PN Luego $f \in R_{0}^{\Sigma} = PR_{0}^{\Sigma}$, es decir:
    \[
      f \in \{Suc, Pred, C_{0}^{0,0}, C_{\varepsilon}^{0,0}\} \cup \{d_{a}: a \in \Sigma\} \cup \{p_{j}^{n,m} : 1 \leq j
      \leq n+m\}
    \]

    \PN Por lo tanto, $f$ es $\Sigma$-efectivamente computable.

    \vspace{3mm}
	  \PN \underline{Caso Inductivo:} \begin{tabular}{|c|} \hline $k > 0$ \\\hline \end{tabular}

    \PN Supongamos ahora que si $f \in \mathrm{R}_{k}^{\Sigma} \Rightarrow f$ es $\Sigma$-efectivamente computable,
    veamos que $f \in \mathrm{R}_{k+1}^{\Sigma} \Rightarrow f$ es $\Sigma$-efectivamente computable.

    \PN Dado que las funciones de $R_{k}^{\Sigma}$ son $\Sigma$-efectivamente computable por hipótesis inductiva, y que
    $R_{k+1}^{\Sigma}$ se construye a partir de las mismas, a través de recursiones, composiciones o minimizaciones,
    las cuales probamos son $\Sigma$-efectivamente computables en el \textbf{Lemma 18}, \textbf{Lemma 19}, y
    \textbf{Lemma 41} respectivamente, entonces concluimos que $f$ es $\Sigma$-efectivamente computable.
  \end{proof}

  % Lemma 43: Con prueba.
  \begin{lemma}
    \PN Sean $n, m \geq 0$. Sea $P: D_{P} \subseteq \omega \times \omega^{n} \times \Sigma^{\ast m} \rightarrow \omega$
    un predicado $\Sigma$-PR, entonces:

    \begin{enumerate}[a)]
      \item $M(P)$ es $\Sigma$-R.
      \item Si existe una función $\Sigma$-PR $f: \omega^{n} \times \Sigma^{\ast m} \rightarrow \omega$ tal que:
        \[
          M(P)(\vec{x},\vec{\alpha}) = \min_{t}P(t,\vec{x},\vec{\alpha}) \leq f(\vec{x},\vec{\alpha}),
          \text{ para cada }(\vec{x},\vec{\alpha}) \in D_{M(P)}
        \]

        \PN entonces $M(P)$ es $\Sigma$-PR.
    \end{enumerate}
  \end{lemma}
  \begin{proof}
    \PN Sea $\bar{P} = P \mid_{D_{P}} \cup \; C_{0}^{n+1,m} \mid_{(\omega^{n+1} \times \Sigma^{\ast m})-D_{P}}$.
    \begin{enumerate}[a)]
      \item Veamos primero que $M(P) = M(\bar{P})$, es decir, que los dominios y las reglas de asignación son las
        mismas.
        \begin{eqnarray*}
          D_{M_{P}} &=& \{(\vec{x},\vec{\alpha}) \in \omega^{n} \times \Sigma^{\ast m}: (\exists t \in \omega)
            \; P(t,\vec{x},\vec{\alpha})\} \\
          D_{M_{\bar{P}}} &=& \{(\vec{x},\vec{\alpha}) \in \omega^{n} \times \Sigma^{\ast m}: (\exists t \in
            \omega) \; \bar{P}(t,\vec{x},\vec{\alpha})\}
        \end{eqnarray*}

        \PN notar que:

        \begin{center} \begin{tabular}{|c|} \hline $\bar{P}(t,\vec{x},\vec{\alpha}) = 1 \Leftrightarrow P \mid_{D_{P}} =
        1$ \\\hline \end{tabular} $(\star)$ \end{center}

        \PN Luego, $D_{M(P)} = D_{M(\bar{P})}$ y $M(P) = M(\bar{P})$. Veamos ahora que $M(\bar{P})$ es $\Sigma$-R.

        \PN Sea $k$ tal que $\bar{P} \in \mathrm{PR}_{k}^{\Sigma}$, ya que $\bar{P}$ es $\Sigma$-total y $\bar{P} \in
        \mathrm{PR}_{k}^{\Sigma} \subseteq \mathrm{R}_{k}^{\Sigma}$, tenemos que $M(\bar{P}) \in
        \mathrm{R}_{k+1}^{\Sigma}$ y por lo tanto $M(\bar{P}) \in \mathrm{R}^{\Sigma}$, es decir, $M(P)$ es $\Sigma$-R.

      \item Primero veremos que $D_{M(\bar{P})}$ es un conjunto $\Sigma$-PR. Notese que:
        \[
          \chi_{D_{M(\bar{P})}} = \lambda \vec{x}\vec{\alpha} \left[(\exists t \in \omega)_{t \leq
          f(\vec{x},\vec{\alpha})} \; \bar{P}(t,\vec{x},\vec{\alpha})\right]
        \]

        \PN lo cual nos dice que:
        \[
          \chi_{D_{M(\bar{P})}} = \lambda x\vec{x}\vec{\alpha} \left[(\exists t \in \omega)_{t\leq x} \;
          \bar{P}(t,\vec{x},\vec{\alpha})\right] \circ (f,p_{1}^{n,m},\dotsc,p_{n+m}^{n,m})
        \]

        \PN pero el \textbf{Lemma 39} nos dice que $\lambda x\vec{x}\vec{\alpha} \left[(\exists t \in \omega)_{t\leq x}
        \; \bar{P}(t,\vec{x},\vec{\alpha})\right]$ es $\Sigma$-PR por lo cual tenemos que $\chi_{D_{M(\bar{P})}}$ lo es.

        \PN Sea:
        \[
          Q = \lambda t\vec{x}\vec{\alpha}\left[\bar{P}(t,\vec{x},\vec{\alpha}) \wedge (\forall j \in \omega)_{j \leq t}
          \; j=t \vee \lnot \bar{P}(j,\vec{x},\vec{\alpha})\right]
        \]

        \PN notar que $Q$ es $\Sigma$-total, veamos que es $\Sigma$-PR. Sea:
        \[
          R = \lambda jt\vec{x}\vec{\alpha} \left[j=t \; \vee \neg \bar{P}(j,\vec{x},\vec{\alpha})\right]
        \]

        \PN luego, por el \textbf{Lemma 39}:
        \[
          \lambda t\vec{x}\vec{\alpha} \left[(\forall j \in \omega)_{j \leq t} \; R(j,t,\vec{x},\vec{\alpha})\right]
        \]

        \PN es $\Sigma$-PR y por lo tanto $Q$ es $\Sigma$-PR. Además notese que para cada $(\vec{x},\vec{\alpha}) \in
        \omega^{n} \times \Sigma^{\ast m}$ tenemos:
        \[
          Q(t,\vec{x},\vec{\alpha}) = 1 \Leftrightarrow t = M(\bar{P})(\vec{x},\vec{\alpha})
        \]

        \PN Esto nos dice que
        \[
          M(\bar{P}) = \left(\lambda \vec{x}\vec{\alpha}\left[\prod_{t=0}^{f(\vec{x},\vec{\alpha})}t^{Q(t,\vec{x},
          \vec{\alpha})}\right]\right) \mid_{D_{M(\bar{P})}}
        \]

        \PN por lo cual para probar que $M(\bar{P})$ es $\Sigma$-PR solo nos resta probar que
        \[
          F = \lambda \vec{x}\vec{\alpha}\left[\prod_{t=0}^{f(\vec{x},\vec{\alpha})}t^{Q(t,\vec{x},\vec{\alpha})}
          \right]
        \]

        \PN es $\Sigma$-PR. Pero
        \[
          F = \lambda xy\vec{x}\vec{\alpha}\left[\prod_{t=x}^{y}t^{Q(t,\vec{x},\vec{\alpha})}\right] \circ
          (C_{0}^{n,m},f,p_{1}^{n,m},\dotsc,p_{n+m}^{n,m})
        \]

        \PN y por lo tanto el \textbf{Lemma 38} nos dice que $F$ es $\Sigma$-PR. De esta manera hemos probado que
        $M(\bar{P})$ es $\Sigma$-PR y por lo tanto $M(P)$ lo es.
    \end{enumerate}
  \end{proof}

  % Lemma 44: Con prueba.
  \begin{lemma}
    \PN Las siguientes funciones son $\emptyset$-PR:

    \begin{enumerate}[a)]
      \item
        $\begin{array}{rll}
          Q: \omega \times \mathbb{N} &\rightarrow& \omega \\
          (x,y) & \rightarrow & \text{cociente de la division de } x \text{ por } y
        \end{array}$
      \item
        $\begin{array}{rll}
          R: \omega \times \mathbb{N} &\rightarrow& \omega \\
          (x,y) &\rightarrow& \text{resto de la division de } x \text{ por } y
        \end{array}$
      \item
        $\begin{array}{rll}
          pr: \mathbb{N} &\rightarrow& \omega \\
          n & \rightarrow & n\text{-esimo numero primo}
        \end{array}$
    \end{enumerate}
  \end{lemma}
  \begin{proof}
    \begin{enumerate}[a)]
      \item Veamos primero veamos que $Q=M(P)$, donde $P=\lambda txy\left[(t+1).y > x\right]$. Notar que:
        \begin{eqnarray}
          \nonumber D_{M(P)} &=& \{(x,y) \in \omega \times \mathbb{N}: (\exists t \in \omega) \; P(t,x,y) = 1\} \\
          \nonumber &=& \{(x,y) \in \omega \times \mathbb{N}: (\exists t \in \omega) \; (t+1).y > x \} \\
          \nonumber &=& \omega \times \mathbb{N} \\
          \nonumber &=& D_{Q}
        \end{eqnarray}

        \PN Luego, para cada $(x,y) \in \omega \times \mathbb{N}$, se tiene que:
        \[
          Q(x,y) = M(P)(x,y) = \min_{t} \; (t+1).y > x
        \]

        \PN es decir, $Q = M(P)$. Ya que $P$ es $\emptyset$-PR y además:
        \[
          Q(x,y) \leq p_{1}^{2,0}(x,y), \text{para cada }(x,y) \in \omega \times \mathbb{N}
        \]

        \PN el inciso (b) del \textbf{Lemma 43} implica que $Q \in \mathrm{PR}^{\emptyset}$.
      \item Notese que:
        \begin{eqnarray*}
          R &=& \lambda xy\left[x \dot{-}Q(x,y).y\right] \\
          &=& \lambda xy \left[x \dot{-}y\right] \circ (p_{1}^{2,0}, \lambda xy \left[x.y\right] \circ (Q
            \circ (p_{1}^{2,0}, p_{2}^{2,0}), p_{2}^{2,0}))
        \end{eqnarray*}

        \PN y por lo tanto $R \in \mathrm{PR}^{\emptyset}$.
      \item Para ver que $pr$ es $\emptyset$-PR, veremos que la extensión $h: \omega \rightarrow \omega$, dada por:
        \begin{eqnarray*}
          h(0) &=& 0 \\
          h(n) &=& pr(n), \; n \geq 1
        \end{eqnarray*}

        \PN es $\emptyset$-PR. Primero notar que:
        \begin{eqnarray*}
          h(0) &=& 0 \\
          h(x+1) &=& \min_{t}\left(t \text{ es primo} \wedge t > h(x)\right)
        \end{eqnarray*}

        \PN Osea que $h = R \left(C_{0}^{0,0},M(P)\right)$, donde:
        \[
          P = \lambda tzx\left[t \text{ es primo} \wedge t > z\right]
        \]

        \PN Solo resta ver que $M(P)$ es $\emptyset$-PR. Claramente, $P$ es $\emptyset$-PR.
        \PN Veamos que para cada $(z,x) \in \omega^{2}$, tenemos que:
        \[
          M(P)(z,x) = \min_{t}\left(t \text{ es primo} \wedge t > z\right) \leq z! + 1
        \]

        \PN Sea $p$ primo tal que $p$ divide a $z!+1$, luego $p > z$. Esto nos dice que:
        \[
          \min_{t}\left(t \text{ es primo} \wedge t > z\right) \leq p \leq z! + 1
        \]

        \PN Luego, $f = \lambda zx\left[z! + 1\right]$ y utilizando el \textbf{Lemma 43} tenemos que $M(P)$ es
        $\emptyset$-PR.
    \end{enumerate}
  \end{proof}

  % Lemma 45: Sin prueba.
  \begin{lemma}
    \PN Las funciones $\lambda xi\left[(x)_{i}\right]$ y $\lambda x\left[Lt(x)\right]$ son $\emptyset$-PR.
  \end{lemma}

  % Lemma 46: Nada.
  \begin{lemma}
    \PN Este lema no se evalua.
  \end{lemma}

  % Lemma 47: Sin prueba.
  \begin{lemma}
    \PN Supongamos que $\Sigma \neq \emptyset$. Sea $<$ un orden total estricto sobre $\Sigma$. Sean $n, m \geq 0$ y
    sea $P: D_{P} \subseteq \omega^{n} \times \Sigma^{\ast m} \times \SIGMA \rightarrow \omega$ un predicado
    $\Sigma$-PR, entonces:

    \begin{enumerate}[a)]
      \item $M^{<}(P)$ es $\Sigma$-R.
      \item Si existe una función $\Sigma$-PR $f: \omega^{n} \times \Sigma^{\ast m} \rightarrow \omega$ tal que:

        \[
          \lvert M^{<}(P)(\vec{x},\vec{\alpha})\rvert = \lvert \min_{\alpha}^{<} P(\vec{x},
          \vec{\alpha},\alpha)\rvert \leq f(\vec{x},\vec{\alpha})
        \]

        \PN para cada $(\vec{x},\vec{\alpha}) \in D_{M^{< }(P)}$ entonces $M^{<}(P)$ es $\Sigma$-PR.
    \end{enumerate}
  \end{lemma}

  % Lemma 48: Nada.
  \begin{lemma}
    \PN Este lema no se evalua.
  \end{lemma}

  % Lemma 49: Nada.
  \begin{lemma}
    \PN Este lema no se evalua.
  \end{lemma}

  % Lemma 50: Nada.
  \begin{lemma}
    \PN Este lema no se evalua.
  \end{lemma}

  % Theorem 51: Sin prueba.
  \begin{theorem}
    \PN Sean $\Sigma$ y $\Gamma$ alfabetos cualesquiera.

    \begin{enumerate}[a)]
      \item Supongamos una función $f$ es $\Sigma$-mixta y $\Gamma$-mixta, entonces $f$ es $\Sigma$-R (respectivamente
        $\Sigma$-PR) $\Leftrightarrow f$ es $\Gamma$-R (respectivamente $\Gamma$-PR).
      \item Supongamos un conjunto $S$ es $\Sigma$-mixto y $\Gamma$-mixto, entonces $S$ es $\Sigma$-PR $\Leftrightarrow
        S$ es $\Gamma$-PR.
    \end{enumerate}
  \end{theorem}
