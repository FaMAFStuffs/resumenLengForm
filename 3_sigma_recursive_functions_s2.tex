
  % Lemma 35
  \begin{lemma}
    \par Supongamos $f_{i}: D_{f_{i}} \subseteq \omega^{n} \times \Sigma^{\ast m} \rightarrow O, i = 1, \dotsc, k$, son
    funciones $\Sigma$-PR tales que $D_{f_{i}} \cap D_{f_{j}} = \emptyset$ para $i \neq j$, entonces $f_{1} \cup \dotsc
    \cup f_{k}$ es $\Sigma$-PR.
  \end{lemma}
  \begin{proof}
    \par Supongamos $O = \SIGMA$ y $k = 2$. Sean:

    \[
      \bar{f}_{i}: \omega^{n} \times \Sigma^{\ast m} \rightarrow \SIGMA \; i = 1, 2
    \]

    \par funciones $\Sigma$-PR tales que $\bar{f}_{i} \mid_{D_{f_{i}}} = f_{i}$, $i = 1, 2$ (por \textbf{Lemma 33}).
    \par Por \textbf{Lemma 34} los conjuntos $D_{f_{1}}$ y $D_{f_{2}}$ son $\Sigma$-PR y por lo tanto lo es $ D_{f_{1}}
    \cup D_{f_{2}}$. Ya que:

    \[
      f_{1} \cup f_{2} = \left(\lambda \alpha \beta \left[\alpha \beta\right] \circ (\lambda x\alpha
      \left[\alpha^{x}\right] \circ (\chi_{D_{f_{1}}}, \bar{f}_{1}), \lambda x\alpha \left[\alpha^{x}\right] \circ
      (\chi_{D_{f_{2}}}, \bar{f}_{2}))\right) \mid_{D_{f_{1}} \cup D_{f_{2}}}
    \]

    \par tenemos que $f_{1} \cup f_{2}$ es $\Sigma$-PR.
    \par El caso $k > 2$ puede probarse por inducción ya que:

    \[
      f_{1} \cup \dotsc \cup f_{k} = (f_{1} \cup \dotsc \cup f_{k-1}) \cup f_{k}
    \]
  \end{proof}

  % Corollary 36
  \begin{corollary}
    \par Supongamos $f$ es una función $\Sigma$-mixta cuyo dominio es finito, entonces $f$ es $\Sigma$-PR.
  \end{corollary}

  % Lemma 37
  \begin{lemma}
    \par $\lambda i\alpha \left[\lbrack \alpha]_{i}\right]$ es $\Sigma$-PR.
  \end{lemma}

  % Lemma 38
  \begin{lemma}
    \par Sean $n, m \geq 0$.

    \begin{enumerate}[a)]
      \item Si $f: \omega \times S_{1} \times \dotsc \times S_{n} \times L_{1} \times \dotsc \times L_{m} \rightarrow
        \omega$ es $\Sigma$-PR, con $ S_{1}, \dotsc, S_{n} \subseteq \omega$ y $L_{1}, \dotsc, L_{m} \subseteq \SIGMA$
        no vacíos, entonces lo son las funciones $\lambda xy\vec{x}\vec{\alpha} \left[\sum_{t=x}^{t=y} f(t, \vec{x},
        \vec{\alpha})\right]$ y $\lambda xy\vec{x}\vec{\alpha} \left[\prod_{t=x}^{t=y}f(t, \vec{x}, \vec{\alpha})
        \right]$.
      \item Si $f: \omega \times S_{1} \times \dotsc \times S_{n} \times L_{1} \times \dotsc \times L_{m} \rightarrow
        \SIGMA$ es $\Sigma$-PR, con $ S_{1}, \dotsc, S_{n} \subseteq \omega$ y $L_{1}, \dotsc, L_{m} \subseteq \SIGMA$
        no vacíos, entonces lo es la función $\lambda xy\vec{x}\vec{\alpha}\left[\subset_{t=x}^{t=y} f(t, \vec{x},
        \vec{\alpha})\right]$.
    \end{enumerate}
  \end{lemma}
  \begin{proof}
    \par (a): Sea $G = \lambda tx\vec{x}\vec{\alpha}\left[\sum_{i=x}^{i=t} f(i, \vec{x}, \vec{\alpha})\right]$. Ya que:

    \[
      \lambda xy\vec{x}\vec{\alpha}\left[\sum_{i=x}^{i=y} f(i, \vec{x}, \vec{\alpha})\right] = G \circ \left(
      p_{2}^{n+2,m}, p_{1}^{n+2,m}, p_{3}^{n+2,m}, \dotsc, p_{n+m+2}^{n+2,m}\right)
    \]

    \par solo tenemos que probar que $G$ es $\Sigma$-PR. Primero note que:

    \begin{eqnarray}
      \nonumber G(0,x,\vec{x},\vec{\alpha}) &=& \left\{
        \begin{array}{lll}
          0 && \text{si } x > 0 \\
          f(0,\vec{x},\vec{\alpha}) && \text{si } x = 0
        \end{array}\right. \\
      \nonumber G(t+1,x,\vec{x},\vec{\alpha}) &=& \left\{
        \begin{array}{lll}
          0 && \text{si } x > t+1 \\
          G(t,x,\vec{x},\vec{\alpha}) + f(t+1,\vec{x},\vec{\alpha}) && \text{si } x \leq t+1
        \end{array} \right.
    \end{eqnarray}

    \par Sean:

    \begin{eqnarray}
      \nonumber D_{1} &=& \left\{(x,\vec{x},\vec{\alpha}) \in \omega \times S_{1} \times \dotsc \times S_{n} \times
        L_{1} \times \dotsc \times L_{m}: x > 0 \right\} \\
      \nonumber D_{2} &=& \left\{(x,\vec{x},\vec{\alpha}) \in \omega \times S_{1} \times \dotsc \times S_{n} \times
        L_{1} \times \dotsc \times L_{m}: x = 0 \right\} \\
      \nonumber H_{1} &=& \left\{(z,t,x,\vec{x},\vec{\alpha}) \in \omega^{3} \times S_{1} \times \dotsc \times S_{n}
        \times L_{1} \times \dotsc \times L_{m}: x > t+1\right\} \\
      \nonumber H_{2} &=& \left\{(z,t,x,\vec{x},\vec{\alpha}) \in \omega^{3} \times S_{1} \times \dotsc \times S_{n}
        \times L_{1} \times \dotsc \times L_{m}: x \leq t+1\right\}
    \end{eqnarray}

    \par Es fácil de chequear que estos conjuntos son $\Sigma$-PR. Veamos que por ejemplo $H_{1}$ lo es. Es decir
    debemos ver que $\chi_{H_{1}}$ es $\Sigma$-PR. Ya que $f$ es $\Sigma$-PR tenemos que $D_{f} = \omega \times S_{1}
    \times \dotsc \times S_{n} \times L_{1} \times \dotsc \times L_{m}$ es $\Sigma$-PR, lo cual por el \textbf{Lemma 31}
    nos dice que los conjuntos $S_{1}, \dotsc, S_{n}, L_{1}, \dotsc, L_{m}$ son $\Sigma$-PR. Ya que $\omega$ es
    $\Sigma$-PR, el \textbf{Lemma 31} nos dice que $R = \omega^{3} \times S_{1} \times \dotsc \times S_{n} \times L_{1}
    \times \dotsc \times L_{m}$ es $\Sigma$-PR. Notese que $\chi_{H_{1}} = (\chi_{R} \wedge \lambda ztx\vec{x}
    \vec{\alpha}\left[x > t+1\right])$ por cual $\chi_{H_{1}}$ es $\Sigma$-PR ya que es la conjunción de dos predicados
    $\Sigma$-PR.

    \par Además note que $G = R(h,g)$, donde:

    \begin{eqnarray}
      \nonumber h &=& C_{0}^{n+1,m} \mid_{D_{1}} \cup \lambda x\vec{x}\vec{\alpha}\left[f(0,\vec{x},\vec{\alpha})\right]
      \mid_{D_{2}} \\
      \nonumber g &=& C_{0}^{n+3,m} \mid_{H_{1}} \cup \lambda ztx\vec{x}\vec{\alpha}\left[z+f(t+1,\vec{x},\vec{\alpha})
      \right]) \mid_{H_{2}}
    \end{eqnarray}

    \par O sea que el \textbf{Lemma 35} y el \textbf{Lemma 32} garantizan que $G$ es $\Sigma$-PR.
  \end{proof}

  % Lemma 39
  \begin{lemma}
    \par Sean $n, m \geq 0$.

    \begin{enumerate}[a)]
      \item Sea $P: S \times S_{1} \times \dotsc \times S_{n} \times L_{1} \times \dotsc \times L_{m} \rightarrow
        \omega$ un predicado $\Sigma$-PR y supongamos $\bar{S} \subseteq S$ es $\Sigma$-PR, entonces $\lambda
        x\vec{x}\vec{\alpha} \left[(\forall t\in \bar{S})_{t\leq x} \; P(t,\vec{x},\vec{\alpha})\right]$ y $\lambda
        x\vec{x}\vec{\alpha} \left[(\exists t\in \bar{S})_{t\leq x} \; P(t,\vec{x},\vec{\alpha})\right]$ son predicados
        $\Sigma$-PR. (Notar que el dominio de estos predicados es $\omega \times S_{1} \times \dotsc \times S_{n} \times
        L_{1} \times \dotsc \times L_{m}$).
      \item Sea $P: S_{1} \times \dotsc \times S_{n} \times L_{1} \times \dotsc \times L_{m} \times L \rightarrow
        \omega$ un predicado $\Sigma$-PR y supongamos $\bar{L} \subseteq L$ es $\Sigma$-PR, entonces $\lambda
        x\vec{x}\vec{\alpha} \left[(\forall \alpha \in \bar{L})_{\left\vert \alpha \right\vert \leq x} \;
        P(\vec{x},\vec{\alpha},\alpha)\right]$ y $\lambda x\vec{x}\vec{\alpha} \left[(\exists \alpha \in
        \bar{L})_{\left\vert \alpha \right\vert \leq x} \; P(\vec{x},\vec{\alpha},\alpha )\right]$ son predicados
        $\Sigma$-PR.
    \end{enumerate}
  \end{lemma}
  \begin{proof}
    \par Se probará solamente el inciso (a). Sea:

    \[
      \bar{P} = P \mid_{\bar{S} \times S_{1} \times \dotsc \times S_{n} \times L_{1} \times \dotsc \times L_{m}} \cup
      C_{1}^{1+n,m} \mid_{(\omega -\bar{S}) \times S_{1} \times \dotsc \times S_{n} \times L_{1} \times \dotsc \times
      L_{m}}
    \]

    \par Notese que $\bar{P}$ es $\Sigma$-PR. Ya que:

    \begin{eqnarray}
      \nonumber \lambda x\vec{x}\vec{\alpha}\left[(\forall t\in \bar{S})_{t\leq x} P(t,\vec{x},\vec{\alpha})\right] &=&
        \lambda x\vec{x}\vec{\alpha}\left[\prod\limits_{t=0}^{t=x}\bar{P}(t,\vec{x},\vec{\alpha})\right] \\
      \nonumber &=& \lambda xy\vec{x}\vec{\alpha}\left[\prod\limits_{t=x}^{t=y}\bar{P}(t,\vec{x},\vec{\alpha})\right]
        \circ \left(C_{0}^{1+n,m}, p_{1}^{1+n,m}, \dotsc, p_{1+n+m}^{1+n,m}\right)
    \end{eqnarray}

    \par el \textbf{Lemma 38} implica que $\lambda x\vec{x}\vec{\alpha}\left[(\forall t\in \bar{S})_{t\leq x} \;
    P(t,\vec{x},\vec{\alpha})\right]$ es $\Sigma$-PR.

    \par Finalmente note que:

    \[
      \lambda x\vec{x}\vec{\alpha}\left[(\exists t\in \bar{S})_{t\leq x} \; P(t,\vec{x},\vec{\alpha})\right] = \lnot
      \lambda x\vec{x}\vec{\alpha}\left[(\forall t\in \bar{S})_{t\leq x} \; \lnot P(t,\vec{x},\vec{\alpha})\right]
    \]

    \par es $\Sigma$-PR.
  \end{proof}

  % Lemma 40
  \begin{lemma}
    \begin{enumerate}[a)]
      \item El predicado $\lambda xy\left[x \text{ divide } y\right]$ es $\emptyset$-PR.
      \item El predicado $\lambda x\left[x \text{ es primo}\right]$ es $\emptyset$-PR.
      \item El predicado $\lambda \alpha\beta \left[\alpha \text{\ }\mathrm{ inicial}\ \beta \right]$ es $\Sigma$-PR.
    \end{enumerate}
  \end{lemma}
  \begin{proof}
    \begin{enumerate}[a)]
      \item Si tomamos $P = \lambda tx_{1}x_{2}\left[x_{2}=t.x_{1}\right] \in \mathrm{PR}^{\emptyset}$, tenemos que:

        \begin{eqnarray}
          \nonumber \lambda x_{1}x_{2} \left[x_{1}\text{ divide } x_{2}\right] &=& \lambda x_{1}x_{2}\left[(\exists t
            \in \omega)_{t\leq x_{2}} \; P(t,x_{1},x_{2}) \right] \\
          \nonumber &=& \lambda xx_{1}x_{2}\left[(\exists t \in \omega)_{t\leq x} \; P(t,x_{1},x_{2})\right] \circ
            \left(p_{2}^{2,0}, p_{1}^{2,0}, p_{2}^{2,0}\right)
        \end{eqnarray}

        \par por lo que podemos aplicar el \textbf{Lemma 39} anterior.

      \item Ya que:

        \[
          x \text{ es primo sii } x > 1 \wedge \left((\forall t \in \omega)_{t\leq x} \; t=1 \vee t=x \vee \lnot
          (t\text{ divide } x)\right)
        \]

        \par podemos usar un argumento similar al de la prueba de (a).
      \item es dejado al lector.
    \end{enumerate}
  \end{proof}

  % Lemma 41
  \begin{lemma}
    \par Si $P: D_{P} \subseteq \omega \times \omega^{n} \times \Sigma^{\ast m} \rightarrow \omega$ es un predicado
    $\Sigma$-EC y $ D_{P}$ es $\Sigma$-EC, entonces la función $M(P)$ es $\Sigma$-EC.
  \end{lemma}
  \begin{proof}
    \par Ejercicio.
  \end{proof}

  % Theorem 42:
  \begin{theorem}
    \par Si $f \in \mathrm{R}^{\Sigma}$, entonces $f$ es $\Sigma$-EC.
  \end{theorem}
  \begin{proof}
    \par Dejamos al lector la prueba por inducción en $k$ de que si $f\in \mathrm{R}_{k}^{\Sigma}$, entonces $f$ es
    $\Sigma$-EC.
  \end{proof}

  % Lemma 43
  \begin{lemma}
    \par Sean $n, m \geq 0$. Sea $P: D_{P} \subseteq \omega \times \omega^{n} \times \Sigma^{\ast m} \rightarrow \omega$
    un predicado $\Sigma$-PR, ntonces:

    \begin{enumerate}[a)]
      \item $M(P)$ es $\Sigma$-R.
      \item Si hay una función $\Sigma$-PR $f: \omega^{n} \times \Sigma^{\ast m} \rightarrow \omega$ tal que:

        \[
          M(P)(\vec{x},\vec{\alpha}) = \min_{t}P(t,\vec{x},\vec{\alpha}) \leq f(\vec{x},\vec{\alpha}),
          \text{ para cada }(\vec{x},\vec{\alpha}) \in D_{M(P)}
        \]

        \par entonces $M(P)$ es $\Sigma$-PR.
    \end{enumerate}
  \end{lemma}
  \begin{proof}
    \begin{enumerate}[a)]
      \item Sea $\bar{P} = P \mid_{D_{P}} \cup C_{0}^{n+1,m} \mid_{(\omega^{n+1} \times \Sigma^{\ast m})-D_{P}}$.
        Dejamos al lector verificar cuidadosamente que $M(P) = M(\bar{P})$. Veremos entonces que $M(\bar{P})$ es
        $\Sigma$-R. Note que $\bar{P}$ es $\Sigma$-PR. Sea $k$ tal que $\bar{P} \in \mathrm{PR}_{k}^{\Sigma}$. Ya que
        $\bar{P}$ es $\Sigma$-total y $\bar{P} \in \mathrm{PR}_{k}^{\Sigma} \subseteq \mathrm{R}_{k}^{\Sigma}$, tenemos
        que $M(\bar{P}) \in \mathrm{R}_{k+1}^{\Sigma}$ y por lo tanto $M(\bar{P}) \in \mathrm{R}^{\Sigma}$.

      \item Primero veremos que $D_{M(\bar{P})}$ es un conjunto $\Sigma$-PR. Notese que:

        \[
          \chi_{D_{M(\bar{P})}} = \lambda \vec{x}\vec{\alpha}\left[(\exists t\in \omega)_{t\leq f(\vec{x},\vec{\alpha})}
          \; \bar{P}(t,\vec{x},\vec{\alpha})\right]
        \]

        \par lo cual nos dice que:

        \[
          \chi_{D_{M(\bar{P})}} = \lambda x\vec{x}\vec{\alpha}\left[(\exists t\in \omega)_{t\leq x} \;
          \bar{P}(t,\vec{x},\vec{\alpha})\right] \circ (f,p_{1}^{n,m},\dotsc,p_{n+m}^{n,m})
        \]

        \par Pero el \textbf{Lemma 39} nos dice que $\lambda x\vec{x}\vec{\alpha} \left[(\exists t\in \omega)_{t\leq x}
        \; \bar{P}(t,\vec{x},\vec{\alpha})\right]$ es $\Sigma$-PR por lo cual tenemos que $\chi _{D_{M(\bar{P})}}$ lo es.

        \par Sea:

        \[
          P_{1} = \lambda t\vec{x}\vec{\alpha}\left[\bar{P}(t,\vec{x},\vec{\alpha}) \wedge (\forall j\in \omega)_{j\leq
          t} \; j=t \vee \lnot \bar{P}(j,\vec{x},\vec{\alpha})\right]
        \]

        \par Note que $P_{1}$ es $\Sigma$-total. Dejamos al lector usando Lemmas anteriores probar que $P_{1}$ es
        $\Sigma$-PR. Además notese que para cada $(\vec{x},\vec{\alpha}) \in \omega^{n} \times \Sigma^{\ast m}$ tenemos
        que:

        \[
          P_{1}(t,\vec{x},\vec{\alpha}) = 1 \text{ si y solo si } t = M(\bar{P})(\vec{x},\vec{\alpha})
        \]

        \par Esto nos dice que:

        \[
          M(\bar{P}) = \left(\lambda \vec{x}\vec{\alpha}\left[\prod_{t=0}^{f(\vec{x},\vec{\alpha})}t^{P_{1}(t,\vec{x},
          \vec{\alpha})}\right]\right) \mid_{D_{M(\bar{P})}}
        \]

        \par por lo cual para probar que $M(\bar{P})$ es $\Sigma$-PR solo nos resta probar que:

        \[
          F = \lambda \vec{x}\vec{\alpha}\left[\prod_{t=0}^{f(\vec{x},\vec{\alpha})}t^{P_{1}(t,\vec{x},\vec{\alpha})}
          \right]
        \]

        \par lo es. Pero:

        \[
          F = \lambda xy\vec{x}\vec{\alpha}\left[\prod_{t=x}^{y}t^{P_{1}(t,\vec{x},\vec{\alpha})}\right] \circ
          (C_{0}^{n,m}, f, p_{1}^{n,m}, \dotsc, p_{n+m}^{n,m})
        \]

        \par y por lo tanto el \textbf{Lemma 38} nos dice que $F$ es $\Sigma$-PR. De esta manera hemos probado que
        $M(\bar{P})$ es $\Sigma$-PR y por lo tanto $M(P)$ lo es.
    \end{enumerate}
  \end{proof}

  % Lemma 44
  \begin{lemma}
    \par Las siguientes funciones son $\emptyset$-PR:

    \begin{enumerate}[a)]
      \item
        $\begin{array}{rll}
          Q: \omega \times \mathbf{N} &\rightarrow& \omega \\
          (x,y) & \rightarrow & \text{cociente de la division de } x \text{ por } y
        \end{array}$
      \item
        $\begin{array}{rll}
          R: \omega \times \mathbf{N} &\rightarrow& \omega \\
          (x,y) &\rightarrow& \text{resto de la division de } x \text{ por } y
        \end{array}$
      \item
        $\begin{array}{rll}
          pr: \mathbf{N} &\rightarrow& \omega \\
          n & \rightarrow & n\text{-esimo numero primo}
        \end{array}$
    \end{enumerate}
  \end{lemma}
  \begin{proof}
    \begin{enumerate}[a)]
      \item Veamos primero veamos que $Q=M(P)$, donde $P=\lambda txy\left[(t+1).y > x\right]$. Notar que:

        \begin{eqnarray}
          \nonumber D_{M(P)} &=& \{(x,y): (\exists t \in \omega) \; P(t,x,y) = 1\} \\
          \nonumber &=& \{(x,y): (\exists t \in \omega) \; (t+1).y > x \} \\
          \nonumber &=& \omega \times \mathbf{N} \\
          \nonumber &=& D_{Q}
        \end{eqnarray}

        \par Dejamos al lector la fácil verificación de que para cada $(x,y) \in \omega \times \mathbf{N}$, se tiene que:

        \[
          Q(x,y) = M(P)(x,y) = \min_{t}(t+1).y > x
        \]

        \par Esto prueba que $Q=M(P)$. Ya que $P$ es $\emptyset$-PR y además:

        \[
          Q(x,y) \leq p_{1}^{2,0}(x,y), \text{para cada }(x,y) \in \omega \times \mathbf{N}
        \]

        \par el inciso (b) del \textbf{Lemma 43} implica que $Q \in \mathrm{PR}^{\emptyset}$.
      \item Notese que:

        \[
          R = \lambda xy\left[x \dot{-} Q(x,y).y\right]
        \]

        \par y por lo tanto $R \in \mathrm{PR}^{\emptyset}$.
      \item Para ver que $pr$ es $\emptyset$-PR, veremos que la extensión $ h: \omega \rightarrow \omega$, dada por
        $h(0)=0$ y $h(n)=pr(n)$, $n \geq 1$, es $\emptyset$-PR. Primero note que:

        \begin{eqnarray}
          \nonumber h(0) &=& 0 \\
          \nonumber h(x+1) &=& \min\nolimits_{t}\left(t \text{ es primo} \wedge t > h(x)\right)
        \end{eqnarray}

        \par Osea que $h = R \left(C_{0}^{0,0},M(P)\right)$, donde:

        \[
          P = \lambda tzx\left[t \text{ es primo} \wedge t > z\right]
        \]

        \par Es decir que solo nos resta ver que $M(P)$ es $\emptyset$-PR.
        \par Claramente $P$ es $\emptyset$-PR. Veamos que para cada $(z,x) \in \omega^{2}$, tenemos que:

        \[
          M(P)(z,x) = \min\nolimits_{t}\left(t \text{ es primo} \wedge t > z\right) \leq z! + 1
        \]

        \par Sea $p$ primo tal que $p$ divide a $z!+1$. Es fácil ver que entonces $p > z$. Pero esto claramente nos dice
        que:

        \[
          \min\nolimits_{t}\left(t \text{ es primo} \wedge t > z\right) \leq p \leq z! + 1
        \]

        \par Osea que el inciso (b) del \textbf{Lemma 43} implica que $M(P)$ es $\emptyset$-PR ya que podemos tomar
        $f = \lambda zx\left[z! + 1\right]$.
    \end{enumerate}
  \end{proof}

  % Lemma 45
  \begin{lemma}
    \par Las funciones $\lambda xi\left[(x)_{i}\right]$ y $\lambda x\left[Lt(x)\right]$ son $\emptyset$-PR.
  \end{lemma}

  % Lemma 46
  \begin{lemma}
    \par Este lemma no se evalua.
  \end{lemma}

  % Lemma 47
  \begin{lemma}
    \par Supongamos que $\Sigma \neq \emptyset$. Sea $<$ un orden total estricto sobre $\Sigma$, sean $n, m \geq 0$ y
    sea $P: D_{P} \subseteq \omega^{n} \times \Sigma^{\ast m} \times \SIGMA \rightarrow \omega$ un predicado
    $\Sigma$-PR, entonces:

    \begin{enumerate}[a)]
      \item $M^{<}(P)$ es $\Sigma$-R.
      \item Si existe una función $\Sigma$-PR $f: \omega^{n} \times \Sigma^{\ast m} \rightarrow \omega$ tal que:

        \[
          \left\vert M^{<}(P)(\vec{x},\vec{\alpha})\right\vert = \left\vert \min\nolimits_{\alpha}^{<} P(\vec{x},
          \vec{\alpha},\alpha)\right\vert \leq f(\vec{x},\vec{\alpha}) \text{, para cada } (\vec{x},\vec{\alpha}) \in
          D_{M^{< }(P)}
        \]

        \par entonces $M^{<}(P)$ es $\Sigma$-PR.
    \end{enumerate}
  \end{lemma}

  % Lemma 48
  \begin{lemma}
    \par Este lemma no se evalua.
  \end{lemma}

  % Lemma 49
  \begin{lemma}
    \par Este lemma no se evalua.
  \end{lemma}

  % Lemma 50
  \begin{lemma}
    \par Este lemma no se evalua.
  \end{lemma}

  % Theorem 51
  \begin{theorem}
    \par Sean $\Sigma$ y $\Gamma$ alfabetos cualesquiera.

    \begin{enumerate}[a)]
      \item Supongamos una función $f$ es $\Sigma$-mixta y $\Gamma$-mixta, entonces $f$ es $\Sigma$-R (respectivamente
        $\Sigma$-PR) sii $f$ es $\Gamma$-R (respectivamente $\Gamma$-PR).
      \item Supongamos un conjunto $S$ es $\Sigma$-mixto y $\Gamma$-mixto, entonces $S$ es $\Sigma$-PR sii $S$ es
        $\Gamma$-PR.
    \end{enumerate}
  \end{theorem}
