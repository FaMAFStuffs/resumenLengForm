\section{Notación y conceptos básicos}

  % Lemma 1: Sin prueba.
  \begin{lemma}
    \PN Sea $S \subseteq \omega \times \SIGMA$, entonces $S$ es rectangular si y solo si se cumple la siguiente
    propiedad:
    \[
      \text{Si } (x, \alpha), \; (y, \beta) \in S \Rightarrow (x, \beta) \in S
    \]
  \end{lemma}

  % Lemma 2: Sin prueba.
  \begin{lemma}
    \PN La relación $<$ es un orden total estricto sobre $\SIGMA$.
  \end{lemma}

  % Lemma 3: Sin prueba.
  \begin{lemma}
    \PN La función $s^{<}: \SIGMA \rightarrow \SIGMA$, definida recursivamente de la siguiente manera:
    \begin{eqnarray*}
  		s^{<}(\varepsilon) &=& a_{1} \\
  		s^{<}(\alpha a_{i}) &=& \alpha a_{i+1} \qquad \text{para } i < n \\
  		s^{<}(\alpha a_{n}) &=& s^{<}(\alpha) a_{1}
    \end{eqnarray*}

    \PN tiene la siguiente propiedad:

    \[
      s^{<}(\alpha) = \min \{\beta \in \SIGMA: \alpha < \beta\}
    \]
  \end{lemma}

  % Corollary 4: Sin prueba.
  \begin{corollary}
    \PN $s^{<}$ es inyectiva.
  \end{corollary}

  % Lemma 5: Sin prueba.
  \begin{lemma}
    \PN Se tiene que:

    \begin{enumerate}
      \item $s^{<}(\alpha) \neq \varepsilon$, para cada $\alpha \in \SIGMA$.
      \item Si $\alpha \neq \varepsilon$, entonces $\alpha = s^{<}(\beta)$ para algún $\beta$.
      \item Si $S\subseteq \SIGMA \neq \emptyset$, entonces $\exists \alpha \in S$ tal que $\alpha < \beta$, para cada
      $\beta \in S - \{\alpha\}$.
    \end{enumerate}
  \end{lemma}

  % Lemma 6: Sin prueba.
  \begin{lemma}
    \PN Tenemos que:
    \[
      \SIGMA = \{\ast^{<}(0), \ast^{<}(1), \dotsc\}
    \]

    \PN Mas aún la función $\ast^{<}$ es biyectiva.
  \end{lemma}

  % Lemma 7: Sin prueba.
  \begin{lemma}
    \PN Sea $n \geq 1$ fijo, entonces cada $x \geq 1$ se escribe en forma única de la siguiente manera:
    \[
      x = i_{0} n^{0} + \dotsc + i_{k-1} n^{k-1} + i_{k} n^{k}
    \]

    \PN con $k\geq 0$ y $1 \leq i_{0}, \dotsc, i_{k-1}, i_{k} \leq n$.
  \end{lemma}

  % Lemma 8: Sin prueba.
  \begin{lemma}
    \PN La función $\#^{<}$ es biyectiva.
  \end{lemma}

  % Lemma 9: Sin prueba.
  \begin{lemma}
    \PN Las funciones $\#^{<}$ y $\ast^{<}$ son una inversa de la otra.
  \end{lemma}

  % Lemma 10: Sin prueba.
  \begin{lemma}
    \PN Si $p, p_{1}, \dotsc, p_{n}$ son números primos y $p$ divide a $\underset{i=1}{\overset{n}{\Pi}} \; p_{i}$,
    entonces $p = p_{i}$, para algún $i$.
  \end{lemma}

  % Theorem 11: Sin prueba.
  \begin{theorem}
    \PN Para cada $x \in \mathbb{N}$, hay una única sucesión $(s_{1}, s_{2}, \dotsc) \in
    \omega^{\left[\mathbb{N}\right]}$ tal que:
    \[
      x = \underset{i=1}{\overset{\infty}{\Pi}} \; pr(i)^{s_{i}}
    \]

    \PN Notar que $\underset{i=1}{\overset{\infty}{\Pi}} \; pr(i)^{s_{i}}$ tiene sentido ya que es un producto que solo
    tiene una cantidad finita de factores no iguales a 1.
  \end{theorem}

  % Lemma 12: Sin prueba.
  \begin{lemma}
    \PN Las funciones:
    \begin{equation*}
      \begin{aligned}
        \mathbb{N} &\rightarrow \omega^{\left[\mathbb{N}\right]} \\
      	x &\rightarrow ((x)_{1}, (x)_{2}, \dotsc) \\
      \end{aligned}
      \qquad\qquad\qquad
      \begin{aligned}
        \omega^{\left[\mathbb{N}\right]} &\rightarrow \mathbb{N} \\
        (s_{1}, s_{2}, \dotsc) &\rightarrow \left\langle s_{1}, s_{2}, \dotsc \right\rangle \\
      \end{aligned}
    \end{equation*}

    \PN son biyecciones una inversa de la otra.
  \end{lemma}

  % Lemma 13: Sin prueba.
  \begin{lemma}
    \PN Recordemos, para cada $x \in \mathbb{N}$ se define:
    \[
      Lt(x) = \left\{\begin{array}{lll}
                      \max_{i} \; (x)_{i} \neq 0 && \text{si } x \neq 1 \\
                      0 & & \text{si } x = 1
                      \end{array} \right.
    \]

    \PN Luego, para cada $x \in \mathbb{N}$:

    \begin{enumerate}
      \item $Lt(x) = 0 \Leftrightarrow x = 1$
      \item $x = \underset{i=1}{\overset{Lt(x)}{\Pi}} \; pr(i)^{(x)_{i}}$
    \end{enumerate}

    \PN Cabe destacar entonces que la función $\lambda ix[(x)_{i}]$ tiene dominio igual a $\mathbb{N}^{2}$ y la
    función $\lambda x[Lt(x)]$ tiene dominio igual a $\mathbb{N}$.
  \end{lemma}
