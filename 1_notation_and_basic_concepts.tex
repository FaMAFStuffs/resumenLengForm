\section{Notación y conceptos básicos}

  % Lemma 1
  \begin{lemma}
    \par Sea $S \subseteq \omega \times \SIGMA$, entonces $S$ es rectangular si y solo si se cumple la siguiente
    propiedad:

    \[
      \text{Si } (x, \alpha), \; (y, \beta) \in S \Rightarrow (x, \beta) \in S
    \]
  \end{lemma}

  % Lemma 2
  \begin{lemma}
    \par La relación $<$ es un orden total estricto sobre $\SIGMA$.
  \end{lemma}

  % Lemma 3
  \begin{lemma}
    \par La función $s^{<}: \SIGMA \rightarrow \SIGMA$, definida recursivamente de la siguiente manera:

    \begin{eqnarray}
  		\nonumber s^{<}(\varepsilon) &=& a_{1} \\
  		\nonumber s^{<}(\alpha a_{i}) &=& \alpha a_{i+1} \qquad \text{para } i < n \\
  		\nonumber s^{<}(\alpha a_{n}) &=& s^{<}(\alpha) a_{1}
    \end{eqnarray}

    \par tiene la siguiente propiedad:

    \[
      s^{<}(\alpha) = \min \{\beta \in \SIGMA: \alpha < \beta\}
    \]
  \end{lemma}

  % Corollary 4
  \begin{corollary}
    \par $s^{<}$ es inyectiva.
  \end{corollary}

  % Lemma 5
  \begin{lemma}
    \par Se tiene que:

    \begin{enumerate}
      \item $s^{<}(\alpha) \neq \varepsilon$, para cada $\alpha \in \SIGMA$.
      \item Si $\alpha \neq \varepsilon$, entonces $\alpha = s^{<}(\beta)$ para algún $\beta$.
      \item Si $S\subseteq \SIGMA \neq \emptyset$, entonces $\exists \alpha \in S$ tal que $\alpha < \beta$, para cada
      $\beta \in S - \{\alpha\}$.
    \end{enumerate}
  \end{lemma}

  % Lemma 6
  \begin{lemma}
    \par Tenemos que:

    \[
      \SIGMA = \{\ast^{<}(0), \ast^{<}(1), \dotsc\}
    \]

    \par Mas aún la función $\ast^{<}$ es biyectiva.
  \end{lemma}

  % Lemma 7
  \begin{lemma}
    \par Sea $n \geq 1$ fijo, entonces cada $x \geq 1$ se escribe en forma única de la siguiente manera:

    \[
      x = i_{0} n^{0} + \dotsc + i_{k-1} n^{k-1} + i_{k} n^{k}
    \]

    \par con $k\geq 0$ y $1 \leq i_{0}, \dotsc, i_{k-1}, i_{k} \leq n$.
  \end{lemma}

  % Lemma 8
  \begin{lemma}
    \par La función $\#^{<}$ es biyectiva.
  \end{lemma}

  % Lemma 9
  \begin{lemma}
    \par Las funciones $\#^{<}$ y $\ast^{<}$ son una inversa de la otra.
  \end{lemma}

  % Lemma 10
  \begin{lemma}
    \par Si $p, p_{1}, \dotsc, p_{n}$ son números primos y $p$ divide a $p_{1} \dotsc p_{n}$, entonces $p = p_{i}$,
    para algún $i$.
  \end{lemma}

  % Theorem 11
  \begin{theorem}
    \par Para cada $x \in \mathbb{N}$, hay una única sucesión $(s_{1}, s_{2}, \dotsc) \in
    \omega^{\left[\mathbb{N}\right]}$ tal que:

    \[
      x = \underset{i=1}{\overset{\infty}{\Pi}} \; pr(i)^{s_{i}}
    \]

    \par Notar que $\underset{i=1}{\overset{\infty}{\Pi}} \; pr(i)^{s_{i}}$ tiene sentido ya que es un producto que solo
    tiene una cantidad finita de factores no iguales a 1.
  \end{theorem}

  % Lemma 12
  \begin{lemma}
    \par Las funciones:

    \begin{eqnarray}
    	\nonumber \mathbb{N} &\rightarrow& \omega^{\left[\mathbb{N}\right]} \\
    	\nonumber x &\rightarrow& ((x)_{1}, (x)_{2}, \dotsc) \\
      \nonumber \\
      \nonumber \omega^{\left[\mathbb{N}\right]} &\rightarrow& \mathbb{N} \\
      \nonumber (s_{1}, s_{2}, \dotsc) &\rightarrow& \left\langle s_{1}, s_{2}, \dotsc \right\rangle
  	\end{eqnarray}

    \par son biyecciones una inversa de la otra.
  \end{lemma}

  % Lemma 13
  \begin{lemma}
    \par Para cada $x \in \mathbb{N}$:

    \begin{enumerate}
      \item $Lt(x) = 0 \Leftrightarrow x = 1$
      \item $x = \underset{i=1}{\overset{Lt(x)}{\Pi}} \; pr(i)^{(x)_{i}}$
    \end{enumerate}

    \par Cabe destacar entonces que la función $\lambda ix[(x)_{i}]$ tiene dominio igual a $\mathbb{N}^{2}$ y la
    función $\lambda ix[Lt(x)]$ tiene dominio igual a $\mathbb{N}$.
  \end{lemma}
