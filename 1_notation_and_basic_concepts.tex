\section{Notación y conceptos básicos}

  \textbf{\underline{Lemma 1:}} Sea $S\subseteq \omega \times \Sigma^{\ast}$, entonces $S$ es rectangular si y
    solo si se cumple la siguiente propiedad:

    \[
      \textup{Si } (x, \alpha), (y, \beta) \in S \Rightarrow (x, \beta) \in S
    \]

  \PROOF Ejercicio.

  \QED


  \textbf{\underline{Lemma 2:}} La relación $<$ es un orden total estricto sobre $\Sigma^{\ast}$.

  \PROOF Ejercicio.

  \QED


  \textbf{\underline{Lemma 3:}} La función $s^{<}: \Sigma^{\ast} \rightarrow \Sigma^{\ast}$, definida recursivamente
    de la siguiente manera:

    \begin{eqnarray}
  		\nonumber s^{<}(\varepsilon) &=& a_{1} \\
  		\nonumber s^{<}(\alpha a_{i}) &=& \alpha a_{i+1} \text{, } i < n \\
  		\nonumber s^{<}(\alpha a_{n}) &=& s^{<}(\alpha) a_{1}
    \end{eqnarray}

    \par tiene la siguiente propiedad:

    \[
      s^{<}(\alpha) = \min \{\beta \in \Sigma^{\ast}: \alpha < \beta\}
    \]

  \PROOF Recordemos primero la definición de un \textit{orden total estricto sobre} un conjunto A.

    \vspace{5mm}
    \par \underline{Definición:} Sea $A$ un conjunto no vacío cualquiera, una relación binaria $<$ sobre $A$ será
    llamada un orden total estricto sobre $A$ si se cumplen las siguientes condiciones:

    \begin{itemize}
      \item $\forall a \in A$, no se da que $a < a$
      \item $\forall a, b \in A$, si $a \neq b \Rightarrow a < b$ ó $b < a$
      \item $\forall a, b, c \in A$, si $a < b$ y $b < c \Rightarrow a < c$
    \end{itemize}

    \par Supongamos que $\alpha < \beta$. Probaremos entonces que $s^{<}(\alpha) \leq \beta$.
    Consideraremos los dos posibles casos, i.e, $\left\vert \alpha \right\vert < \left\vert \beta \right\vert$ y
    $\left\vert \alpha \right\vert = \left\vert \beta \right\vert$. Veamos esto:

    \vspace{3mm}
    \begin{tabular}{|c|}
      \hline Caso $\left\vert \alpha \right\vert < \left\vert \beta \right\vert$\\\hline
    \end{tabular}

    \vspace{3mm}
    \par Se puede ver fácilmente que $\left\vert \alpha \right\vert = \left\vert s^{<}(\alpha) \right\vert$ salvo en el
    caso en que $\alpha \in \{a_{n}\}^{\ast}$, por lo cual solo resta ver el caso $\alpha \in \{a_{n}\}^{\ast}$.
    Supongamos $\alpha = a_{n}^{\left\vert \alpha \right\vert}$, entonces $s^{<}(\alpha) = a_{1}^{\left\vert \alpha
    \right\vert + 1}$.

    \begin{itemize}
      \item Si $\left\vert \beta \right\vert = \left\vert \alpha \right\vert + 1$ entonces es fácil ver
    usando el ítem 2 de la definición del orden de $\Sigma^{\ast}$ que $s^{<}(\alpha) = a_{1}^{\left\vert \alpha
    \right\vert + 1} \leq \beta$.
      \item Si $\left\vert \beta \right\vert > \left\vert \alpha \right\vert + 1$, entonces por
    el ítem 1, de tal definición tenemos que $s^{<}(\alpha) = a_{1}^{\left\vert \alpha \right\vert +1}< \beta$.
    \end{itemize}

    \vspace{5mm}
    \begin{tabular}{|c|}
      \hline Caso $\left\vert \alpha \right\vert =\left\vert \beta \right\vert$\\\hline
    \end{tabular}

    \vspace{3mm}
    \par Tenemos entonces que:

    \begin{eqnarray}
      \nonumber \alpha &=& \alpha_{1} a_{i} \gamma_{1} \\
      \nonumber \beta &=& \alpha_{1} a_{j} \gamma_{2}
    \end{eqnarray}

    \par con $i < j$ y $\left\vert \gamma_{1} \right\vert = \left\vert \gamma_{2}\right\vert$.

    \begin{itemize}
      \item Si $\gamma_{1} = \gamma_{2} = \varepsilon$ entonces es claro que $s^{<}(\alpha) \leq \beta$.
      \item El caso en el que $\gamma_{1}$ termina con $a_{l}$ para algún $l < n$ es fácil.
      \item Veamos el caso en que $\gamma_{1} = a_{n}^{k}$ con $k \geq 1.$ Tenemos que:

        \begin{eqnarray}
          \nonumber s^{<}(\alpha ) & = & s^{<}(\alpha _{1}a_{i}a_{n}^{k}) \\
          \nonumber &=& s^{<}(\alpha_{1} a_{i} a_{n}^{k - 1}) a_{1} \\
          \nonumber &\vdots& \;\;\;\;\;\;\vdots \\
          \nonumber &=& s^{<}(\alpha_{1} a_{i}) a_{1}^{k} \\
          \nonumber &=& \alpha_{1} a_{i + 1} a_{1}^{k} \\
          \nonumber &\leq& \alpha_{1} a_{j} \gamma_{2} \\
          \nonumber &=& \beta
        \end{eqnarray}

      \item Supongamos finalmente que $\gamma_{1} = \rho_{1} a_{l} a_{n}^{k}$ con $k \geq 1$ y $l < n$. Tenemos que:

      \begin{eqnarray}
        \nonumber s^{<}(\alpha) &=& s^{<}(\alpha_{1} a_{i} \rho_{1} a_{l} a_{n}^{k}) \\
        \nonumber &=& s^{<}(\alpha_{1} a_{i} \rho_{1} a_{l} a_{n}^{k - 1}) a_{1} \\
        \nonumber &\vdots& \;\;\;\;\;\;\vdots \\
        \nonumber &=& s^{<}(\alpha_{1} a_{i} \rho_{1} a_{l}) a_{1}^{k} \\
        \nonumber &=& \alpha_{1} a_{i} \rho_{1} a_{l + 1} a_{1}^{k} \\
        \nonumber &\leq& \beta
      \end{eqnarray}
    \end{itemize}

    \par Para completar nuestra demostración debemos probar que $\alpha < s^{<}(\alpha)$, para cada $\alpha \in
    \Sigma ^{\ast}$. Dejamos al lector como ejercicio esta prueba la cual puede ser hecha por inducción en $\left\vert
    \alpha \right\vert$ usando argumentos parecidos a los usados anteriormente.

  \QED


  \textbf{\underline{Corollary 4:}} $s^{<}$ es inyectiva.

  \vspace{3mm}
  \PROOF Supongamos $\alpha \neq \beta$. Ya que el orden de $\Sigma^{\ast}$ es total podemos
    suponer sin pérdida de generalidad que $\alpha < \beta$. Por el lema anterior tenemos que $s^{<}(\alpha) \leq \beta
    < s^{<}(\beta)$ y ya que $<$ es transitiva obtenemos que $s^{<}(\alpha)< s^{<}(\beta)$, lo cual nos dice $s^{<}
    (\alpha) \neq s^{<}(\beta)$.

  \QED


  \textbf{\underline{Lemma 5:}} Se tiene que:
    \begin{enumerate}
      \item $\varepsilon \neq s^{<}(\alpha)$, para cada $\alpha \in \Sigma^{\ast}$.
      \item Si $\alpha \neq \varepsilon$, entonces $\alpha = s^{<}(\beta)$ para algún $\beta$.
      \item Si $S\subseteq \Sigma^{\ast}$ es no vacío, entonces $\exists \alpha \in S$ tal que $\alpha < \beta$, para
      cada $\beta \in S-\{\alpha\}$.
    \end{enumerate}

  \PROOF

    \begin{enumerate}
      \item Ejercicio
      \item Ejercicio
      \item Sea $k = \min \{\left\vert \alpha \right\vert: \alpha \in S\}$. Notese que hay una cantidad finita de
      palabras de $S$ con longitud igual a $k$ y que la menor de ellas es justamente la menor palabra de $S$.
    \end{enumerate}

  \QED


  \textbf{\underline{Lemma 6:}} Tenemos que:

    \[
      \Sigma^{\ast} = \{\ast^{<}(0), \ast^{<}(1), ...\}
    \]

    \par Mas aún la función $\ast^{<}$ es biyectiva.

  \vspace{3mm}
  \PROOF

    \begin{itemize}
      \item \underline{Inyectiva:} Supongamos $\ast^{<}(x) = \ast^{<}(y)$ con $x > y$. Note que $y \neq 0$ ya que
      $\varepsilon$ no es el sucesor de ninguna palabra. Osea que $s^{<}(\ast^{<}(x-1)) = s^{<}(\ast^{<}(y-1))$ lo
      cual, ya que $\ast^{<}$ es inyectiva, nos dice que $\ast^{<}(x-1) = \ast^{<}(y-1)$. Iterando este razonamiento
      llegamos a que $\ast^{<}(z) = \ast^{<}(0) = \varepsilon$ para algún $z > 0$, lo cual es absurdo. Por lo tanto,
      $\ast^{<}$ es inyectiva.

      \item \underline{Sobreyectiva:} Supongamos no lo es, es decir supongamos que $\Sigma^{\ast}-I_{\ast^{<}} \neq
      \varnothing$. Por (3) del lema anterior $\Sigma^{\ast}-I_{\ast^{<}}$ tiene un menor elemento $\alpha$. Ya que
      $\alpha \neq \varepsilon$, tenemos que $\alpha = s^{<}(\beta)$, para algún $\beta$. Ya que $\beta < \alpha$
      tenemos que $\beta \notin \Sigma^{\ast}-I_{\ast^{<}}$, es decir que $\beta = \ast^{<}(x)$, para algún $x \in
      \omega$. Esto nos dice que $\alpha = s^{<}(\ast^{<}(x))$, lo cual por la definición de $\ast^{<}$ nos dice que
      $\alpha = \ast^{<}(x+1)$. Pero esto es absurdo ya que $\alpha \notin I_{\ast^{<}}$.

    \end{itemize}

  \QED


  \textbf{\underline{Lemma 7:}} Sea $n \geq 1$ fijo, entonces cada $x \geq 1$ se escribe en forma única de la siguiente
    manera:

    \[
      x = i_{k} n^{k} + i_{k-1} n^{k-1} + ... + i_{0} n^{0}
    \]

    \par con $k\geq 0$ y $1\leq i_{k}, i_{k-1},...,i_{0}\leq n$.

  \PROOF Veamos primero la unicidad. Supongamos que:

    \[
      i_{k} n^{k} + i_{k-1} n^{k-1} + ... + i_{0} n^{0} = j_{m} n^{m} + j_{m-1} n^{m-1} + ... + j_{0} n^{0}
    \]

    \par con $k, m \geq 0$ y $1 \leq i_{k}, i_{k-1}, ..., i_{0}, j_{m}, ..., j_{0} \leq n$.

    \vspace{3mm}
    \par Supongamos $k < m$. Llegaremos a un absurdo. Notese que:

    \begin{eqnarray}
    	\nonumber i_{k} n^{k} + i_{k-1} n^{k-1} + ... + i_{0} n^{0} &\leq& n n^{k} + n n^{k-1} + ... + n n^{0} \\
    	\nonumber &\leq& n^{k+1} + n^{k} + ... + n^{1} \\
      \nonumber &<& n^{k+1} + n^{k}+ ... + n^{1} + n^{0} \\
      \nonumber &\leq& n^{m} + n^{m-1} + ... + n^{0} \\
      \nonumber &\leq& j_{m}n ^{m} + j_{m-1} n^{m-1} + ... + j_{0} n^{0}
		\end{eqnarray}

    \par lo cual contradice la primera igualdad.

    \par Probaremos por inducción en $x$ que: \begin{tabular}{|c|} \hline $ (\dag) \; \exists k \geq 0$ y $i_{k},
    i_{k-1}, ..., i_{0} \in \{1, ..., n\}$\\\hline\end{tabular} tales que:

    \[
      x = i_{k} n^{k} + i_{k-1} n^{k-1} + ... + i_{0} n^{0}
    \]

    El caso $x = 1$ es trivial. Supongamos que ($\dag$) vale para $x$, probaremos que vale para $x + 1$. Existen varios
    casos:

    \vspace{3mm}
    \begin{tabular}{|c|}
      \hline Caso $i_{0} < n$\\\hline
    \end{tabular}

    \begin{eqnarray}
      \nonumber x + 1 &=& \left(i_{k} n^{k} + i_{k-1} n^{k-1} + ... + i_{0} n^{0}\right) + 1 \\
      \nonumber &=& i_{k} n^{k} + i_{k-1} n^{k-1} + ... + (i_{0} + 1) n^{0}
    \end{eqnarray}

    \begin{tabular}{|c|}
      \hline Caso $i_{k}=i_{k-1}=...=i_{0}=n$\\\hline
    \end{tabular}

    \begin{eqnarray}
      \nonumber x + 1 &=& \left(i_{k} n^{k} + i_{k-1} n^{k-1} + ... + i_{0}n^{0}\right) + 1 \\
      \nonumber &=& \left(n n^{k} + n n^{k-1} + ... + n n^{0}\right) + 1 \\
      \nonumber &=& 1 n^{k+1} + 1 n^{k} + ... + 1 n^{1} + 1 n^{0}
    \end{eqnarray}

    \begin{tabular}{|c|}
      \hline Caso $i_{0} = i_{1} = ... = i_{h} = n$, $i_{h+1} \not= n$ \text{ para algún } $0 \leq h < k$\\\hline
    \end{tabular}

    \begin{eqnarray}
      \nonumber x+1 &=& \left(i_{k} n^{k} + ... + i_{h+2}n^{h+2} + i_{h+1} n^{h+1} + n n^{h} + ... + n n^{0}\right) + 1 \\
      \nonumber &=& \left(i_{k} n^{k} + ... + i_{h+2} n^{h+2} + i_{h+1} n^{h+1} + n^{h+1} + n^{h} + ... + n^{1}\right) + 1 \\
      \nonumber &=& i_{k} n^{k} + ... + i_{h+2} n^{h+2} + (i_{h+1} + 1 ) n^{h+1} + 1 n^{h} + ... + 1 n^{1} + 1 n^{0}
    \end{eqnarray}

  \QED


  \textbf{\underline{Lemma 8:}} La función $\#^{<}$ es biyectiva.

  \PROOF Ejercicio.

  \QED


  \textbf{\underline{Lemma 9:}} Las funciones $\#^{<}$ y $\ast^{<}$ son una inversa de la otra.

  \PROOF Probaremos por inducción en $x$ que para cada $x \in \omega$, se tiene que
    $\#^{<}(\ast^{<}(x)) = x$. El caso $x = 0$ es trivial. Supongamos que $\#^{<}(\ast^{<}(x)) = x$, veremos entonces
    que $\#^{<}(\ast^{<}(x + 1)) = x + 1$.

    \par Sean $k \geq 0$ y $i_{k}, ..., i_{0}$ tales que $\ast^{<}(x) = a_{i_{0}} ... a_{i_{0}}$. Ya que $\#^{<}
    (\ast^{<}(x)) = x$ tenemos que $x = i_{k} n^{k} + ... + i_{0} n^{0}$. Existen varios casos:

    \vspace{3mm}
    \begin{tabular}{|c|}
      \hline Caso $i_{0}< n$\\\hline
    \end{tabular}
    entonces $\ast^{<}(x+1) = s^{<}(\ast^{<}(x)) = a_{i_{k}} ... a_{i_{0} + 1}$ por lo cual:

    \begin{eqnarray}
      \nonumber \#^{<}(\ast^{<}(x+1)) &=& i_{k} n^{k} + i_{k-1} n^{k-1} + ... + (i_{0} + 1) n^{0} \\
      \nonumber &=& (i_{k} n^{k} + i_{k-1} n^{k-1} + ... + i_{0} n^{0}) + 1 \\
      \nonumber &=& x + 1
    \end{eqnarray}

    \begin{tabular}{|c|}
      \hline Caso $i_{k}=i_{k-1}=...=i_{0}=n$.\\\hline
    \end{tabular}
    entonces $\ast^{<}(x+1) = s^{<}(\ast^{<}(x)) = a_{1}^{k+2}$ por lo cual:

    \begin{eqnarray}
      \nonumber \#^{<}(\ast^{<}(x+1)) &=& 1 n^{k+1} + 1 n^{k} + ... + 1 n^{1} + 1 n^{0} \\
      \nonumber &=& (n n^{k} + n n^{k-1} + ... + n n^{0}) + 1 \\
      \nonumber &=& x + 1
    \end{eqnarray}

    \begin{tabular}{|c|}
      \hline Caso $i_{0}=i_{1}=...=i_{h}=n$, $\;i_{h+1}\not=n$, para algun $ 0\leq h< k$.\\\hline
    \end{tabular}
    entonces $\ast^{<}(x+1) = s^{<}(\ast^{<}(x)) = a_{i_{k}} ... a_{i_{h+2}} a_{i_{h+1}+1} a_{1} ... a_{1}$ por lo cual

    \begin{eqnarray}
      \nonumber \#^{<}(\ast^{<}(x+1)) &=& i_{k} n^{k} + ... + i_{h+2} n^{h+2} + (i_{h+1} + 1) n^{h+1} + 1 n^{h} + ... +
      1 n^{1} + 1 n^{0} \\
      \nonumber &=& (i_{k} n^{k} + ... + i_{h+2} n^{h+2} + i_{h+1} n^{h+1} + n^{h+1} + n^{h} + ... + n^{1}) +1 \\
      \nonumber &=& (i_{k} n^{k} + ... + i_{h+2} n^{h+2} + i_{h+1} n^{h+1} + n n^{h} + ... + n n^{0}) +1 \\
      \nonumber &=& x + 1
    \end{eqnarray}

  \QED


  \textbf{\underline{Lemma 10:}} Si $p, p_{1}, ..., p_{n}$ son numeros primos y $p$ divide a $p_{1}...p_{n}$, entonces
    $p=p_{i}$, para algún $i$.

  \PROOF Ejercicio.

  \QED


  \textbf{\underline{Theorem 11:}} Para cada $x \in \mathbf{N}$, hay una única sucesión $(s_{1}, s_{2}, ...) \in
    \omega^{\left[\mathbf{N}\right]}$ tal que:

    \[
      x = \underset{i=1}{\overset{\infty}{\Pi}} pr(i)^{s_{i}}
    \]

    \par Notese que $\underset{i=1}{\overset{\infty}{\Pi}} pr(i)^{s_{i}}$ tiene sentido ya que es un producto que solo
    tiene una cantidad finita de factores no iguales a $1$.

  \PROOF

    \begin{itemize}
      \item \underline{Existencia:} por inducción en $x$. Claramente
        $1 =\underset{i=1}{\overset{\infty}{\Pi}} pr(i)^{0}$, con lo cual el caso $x=1$ esta probado. Supongamos que la
        existencia vale para cada $y < x$, veremos que entonces vale para $x$. Si $x$ es primo, entonces $x=pr(i_{0})$
        para algún $i_{0}$ por lo cual tenemos que $x = \underset{i=1}{\overset{\infty}{\Pi}} pr(i)^{s_{i}}$, tomando
        $s_{i} = 0$ si $i \neq i_{0}$ y $s_{i_{0}} = 1$. Si $x$ no es primo, entonces $x = y_{1} . y_{2}$ con
        $y_{1}, y_{2} < x$. Por HI tenemos que hay
        $(s_{1}, s_{2}, ...), (t_{1}, t_{2}, ...) \in \omega^{\left[\mathbf{N}\right]}$ tales que
        $y_{1} = \underset{i=1}{\overset{\infty}{\Pi}} pr(i)^{s_{i}}$ y
        $y_{2} = \underset{i=1}{\overset{\infty}{\Pi}} pr(i)^{t_{i}}$. Tenemos entonces que
        $x=\underset{i=1}{\overset{\infty}{\Pi}} pr(i)^{s_{i} + t_{i}}$ lo cual concluye la prueba de la existencia.

      \item \underline{Unicidad:} Supongamos que
        $\underset{i=1}{\overset{\infty}{\Pi}} pr(i)^{s_{i}} = \underset{i=1}{\overset{\infty}{\Pi}} pr(i)^{t_{i}}$

        \par Si $s_{i} > t_{i}$ entonces dividiendo ambos miembros por $pr(i)^{t_{i}}$ obtenemos que $pr(i)$ divide a
        un producto de primos todos distintos de él, lo cual es absurdo por el lema anterior. Análogamente llegamos a
        un absurdo si suponemos que $t_{i} > s_{i}$, lo cual nos dice que $s_{i} = t_{i}$, para cada
        $i \in \mathbf{N}$.
    \end{itemize}

  \QED


  \textbf{\underline{Lemma 12:}} Las funciones

    \begin{eqnarray}
    	\nonumber \mathbf{N} &\rightarrow& \omega^{\left[\mathbf{N}\right]} \\
    	\nonumber x &\rightarrow& ((x)_{1}, (x)_{2}, ...) \\
      \nonumber \\
      \nonumber \omega^{\left[\mathbf{N}\right]} &\rightarrow& \mathbf{N} \\
      \nonumber (s_{1}, s_{2}, ...) &\rightarrow& \left\langle s_{1}, s_{2}, ... \right\rangle
  	\end{eqnarray}

    \par son biyecciones una inversa de la otra.

  \PROOF Notese que para cada $x \in \mathbf{N}$, tenemos que $\left\langle (x)_{1}, (x)_{2}, ...\right\rangle = x$.
    Además para cada $(s_{1}, s_{2}, ...) \in \omega^{\left[\mathbf{N}\right]}$, tenemos que
    $((\left\langle s_{1}, s_{2}, ...\right\rangle)_{1}, (\left\langle s_{1}, s_{2}, ...\right\rangle)_{2}, ...) =
    (s_{1}, s_{2}, ...)$.
    Es claro que lo anterior garantiza que los mapeos en cuestión son uno inversa del otro.

  \QED


  \textbf{\underline{Lemma 13}} Para cada $x \in \mathbf{N}$:

    \begin{enumerate}
      \item $Lt(x) = 0 \Leftrightarrow x = 1$
      \item $x = \prod\nolimits_{i=1}^{Lt(x)} pr(i)^{(x)_{i}}$
    \end{enumerate}

    \par Cabe destacar entonces que la función $\lambda ix[(x)_{i}]$ tiene dominio igual a $\mathbf{N}^{2}$ y la
    función $\lambda ix[Lt(x)]$ tiene dominio igual a $\mathbf{N}$.

  \PROOF Ejercicio.

  \QED
