\section{Procedimientos efectivos}

  % Lemma 14
  \begin{lemma}
    \par Sean $S_{1}, S_{2} \subseteq \omega^{n} \times \SIGMA$ conjuntos $\Sigma$-efectivamente enumerables, entonces
    $S_{1} \cup S_{2}$ y $S_{1} \cap S_{2}$ son $\Sigma$-efectivamente enumerables.
  \end{lemma}

  % Lemma 15
  \begin{lemma}
    \par Si $S \subseteq \omega^{n} \times \SIGMA$ es $\Sigma$-efectivamente computable entonces $S$ es
    $\Sigma$-efectivamente enumerable.
  \end{lemma}

  % Theorem 16
  \begin{theorem}
    \par Sea $S \subseteq \omega^{n}\times \SIGMA$. Son equivalentes:

    \begin{enumerate}[a)]
      \item $S$ es $\Sigma$-efectivamente computable.
      \item $S$ y $(\omega^{n}\times \SIGMA)-S$ son $\Sigma$-efectivamente enumerables.
    \end{enumerate}
  \end{theorem}

  % Theorem 17
  \begin{theorem}
    \par Dado $S \subseteq \omega^{n} \times \SIGMA$, son equivalentes:

    \begin{enumerate}
      \item $S$ es $\Sigma$-efectivamente enumerable.
      \item $S = \emptyset$ ó $S = I_{F}$, para alguna $F: \omega \rightarrow \omega^{n} \times \SIGMA$ tal que cada
        $F_{i}$ es $\Sigma$-efectivamente computable.
      \item $S = I_{F}$, para alguna $F:D_{F} \subseteq \omega^{k} \times \Sigma^{\ast l} \rightarrow \omega^{n} \times
        \SIGMA$ tal que cada $F_{i}$ es $\Sigma$-efectivamente computable.
      \item $S = D_{f}$, para alguna función $f$ la cual es $\Sigma$-efectivamente computable.
    \end{enumerate}
  \end{theorem}
