\section{Procedimientos efectivos}

  % Lemma 14: Con prueba.
  \begin{lemma}
    \PN Sean $S_{1}, S_{2} \subseteq \omega^{n} \times \Sigma^{\ast m}$ conjuntos $\Sigma$-efectivamente enumerables,
    entonces:

    \begin{enumerate}[a)]
      \item $S_{1} \cup S_{2}$ es $\Sigma$-efectivamente enumerable.
      \item $S_{1} \cap S_{2}$ es $\Sigma$-efectivamente enumerable.
    \end{enumerate}
  \end{lemma}
  \begin{proof}
    \PN El caso en el que alguno de los conjuntos es vacío es trivial. Supongamos que $S_{1}, S_{2} \neq \emptyset$ y
    sean $\mathbb{P}_{1}$ y $\mathbb{P}_{2}$ procedimientos que enumeran a $S_{1}$ y $S_{2}$.

    \begin{enumerate}[a)]
      \item El siguiente procedimiento enumera al conjunto $S_{1} \cup S_{2}$:

        \textbf{Si $x$ es par:} realizar $\mathbb{P}_{1}$ partiendo de $x/2$ y dar el elemento de $S_{1}$ obtenido como
        salida.
        \textbf{Si $x$ es impar:} realizar $\mathbb{P}_{2}$ partiendo de $(x-1)/2$ y dar el elemento de $S_{2}$ obtenido
        como salida.

      \item Veamos ahora que $S_{1} \cap S_{2}$ es $\Sigma$-efectivamente enumerable:

        \textbf{Si $S_{1} \cap S_{2} = \emptyset$:} entonces no hay nada que probar.

        \textbf{Si $S_{1} \cap S_{2} \neq \emptyset$:} sea $z_{0}$ un elemento fijo de $S_{1} \cap S_{2}$. Sea
        $\mathbb{P}$ un procedimiento efectivo el cual enumere a $\omega \times \omega$.

        \vspace{3mm}
        \PN El siguiente procedimiento enumera al conjunto $S_{1} \cap S_{2}$:

        \textbf{Etapa 1:}
        Realizar $\mathbb{P}$ con dato de entrada $x$, para obtener un par $(x_{1}, x_{2}) \in \omega \times \omega $.

        \textbf{Etapa 2:}
        Realizar $\mathbb{P}_{1}$ con dato de entrada $x_{1}$ para obtener un elemento $z_{1} \in S_{1}$.

        \textbf{Etapa 3:}
        Realizar $\mathbb{P}_{2}$ con dato de entrada $x_{2}$ para obtener un elemento $z_{2} \in S_{2}$.

        \textbf{Etapa 4:}
        Si $z_{1} = z_{2}$, entonces dar como dato de salida $z_{1}$. En caso contrario dar como dato de salida $z_{0}$.
    \end{enumerate}
  \end{proof}

  % Lemma 15: Con prueba.
  \begin{lemma}
    \PN Si $S \subseteq \omega^{n} \times \Sigma^{\ast m}$ es $\Sigma$-efectivamente computable entonces $S$ es
    $\Sigma$-efectivamente enumerable.
  \end{lemma}
  \begin{proof}
    \PN El caso en el que S es vacío es trivial. Supongamos $S \neq \emptyset$. Sea $(\vec{z}, \vec{\gamma}) \in S$,
    fijo. Recordemos que $\omega^{n} \times \Sigma^{\ast m}$ es $\Sigma$-efectivamente enumerable. Sean:

    \begin{itemize}
      \item $\mathbb{P}_{1}$ un procedimiento efectivo que enumere a $\omega^{n} \times \Sigma^{\ast m}$
      \item $\mathbb{P}_{2}$ un procedimiento efectivo que compute a $\chi_{S}$.
    \end{itemize}

    \PN El siguiente procedimiento enumera a $S$:

    \vspace{3mm}
    \textbf{Etapa 1:}
    Realizar $\mathbb{P}_{1}$ con $x$ de entrada para obtener $(\vec{x}, \vec{\alpha}) \in \omega^{n}\times
    \Sigma^{\ast m}$.

    \textbf{Etapa 2:}
    Realizar $\mathbb{P}_{2}$ con $(\vec{x}, \vec{\alpha})$ de entrada para obtener el valor \textit{Booleano} $e$ de
    salida.

    \textbf{Etapa 3:}
    \textbf{Si $e=1$:} dar como dato de salida $(\vec{x}, \vec{\alpha})$.

    $\qquad\qquad\;\;$\textbf{Si $e=0$:} dar como dato de salida $(\vec{z}, \vec{\gamma})$.
  \end{proof}

  % Theorem 16: Con prueba.
  \begin{theorem}
    \PN Sea $S \subseteq \omega^{n}\times \Sigma^{\ast m}$. Son equivalentes:

    \begin{enumerate}[a)]
      \item $S$ es $\Sigma$-efectivamente computable.
      \item $S$ y $(\omega^{n}\times \Sigma^{\ast m})-S$ son $\Sigma$-efectivamente enumerables.
    \end{enumerate}
  \end{theorem}
  \begin{proof}
    \begin{tabular}{|c|}\hline $(a) \Rightarrow (b)$\\\hline\end{tabular} Si $S$ es $\Sigma$-efectivamente computable,
    por el \textbf{Lemma 15} tenemos que $S$ es $\Sigma$-efectivamente enumerable. Notese además que, dado que $S$ es
    $\Sigma$-efectivamente computable, $(\omega^{n} \times \Sigma^{\ast m}) - S$ también lo es, es decir, que aplicando
    nuevamente el \textbf{Lemma 15} tenemos que $(\omega^{n} \times \Sigma^{\ast m})-S$ es
    $\Sigma$-efectivamente enumerable.

    \vspace{3mm}
    \begin{tabular}{|c|}\hline $(b) \Rightarrow (a)$\\\hline\end{tabular} Sean:

    \begin{itemize}
      \item $\mathbb{P}_{1}$ un procedimiento efectivo que enumere a $S$.
      \item $\mathbb{P}_{2}$ un procedimiento efectivo que enumere a $(\omega^{n}\times \Sigma^{\ast m}) - S$.
    \end{itemize}

    \PN El siguiente procedimiento computa el predicado $\chi_{S}$:

    \vspace{3mm}
    \textbf{Etapa 1:}
    Darle a la variable $T$ el valor $0$.

    \textbf{Etapa 2:}
    Realizar $\mathbb{P}_{1}$ con el valor de $T$ como entrada para obtener de salida la upla $(\vec{y}, \vec{\beta})$.

    \textbf{Etapa 3:}
    Realizar $\mathbb{P}_{2}$ con el valor de $T$ como entrada para obtener de salida la upla $(\vec{z}, \vec{\gamma})$.

    \textbf{Etapa 4:}
    \textbf{Si $(\vec{y}, \vec{\beta}) = (\vec{x}, \vec{\alpha})$:} entonces detenerse y dar como dato de salida el
    valor $1$.

    $\qquad\qquad\;\;$\textbf{Si $(\vec{z}, \vec{\gamma}) = (\vec{x}, \vec{\alpha})$:} entonces detenerse y dar como
    dato de salida el valor $0$.

    $\qquad\qquad\;\;$\textbf{Si no sucede ninguna de las dos posibilidades:} aumentar en $1$ el valor de

    $\qquad\qquad\;\;$la variable $T$ y dirijirse a la Etapa 2.
  \end{proof}

  % Theorem 17: Sin prueba.
  \begin{theorem}
    \PN Dado $S \subseteq \omega^{n} \times \Sigma^{\ast m}$, son equivalentes:

    \begin{enumerate}
      \item $S$ es $\Sigma$-efectivamente enumerable.
      \item $S = \emptyset$ ó $S = I_{F}$, para alguna $F: \omega \rightarrow \omega^{n} \times \Sigma^{\ast m}$ tal que
        cada $F_{i}$ es $\Sigma$-efectivamente computable.
      \item $S = I_{F}$, para alguna $F:D_{F} \subseteq \omega^{k} \times \Sigma^{\ast l} \rightarrow \omega^{n} \times
        \Sigma^{\ast m} $ tal que cada $F_{i}$ es $\Sigma$-efectivamente computable.
      \item $S = D_{f}$, para alguna función $f$ la cual es $\Sigma$-efectivamente computable.
    \end{enumerate}
  \end{theorem}
