\section{Procedimientos efectivos}

  \textbf{\underline{Lemma 14:}} Sean $S_{1}, S_{2}\subseteq \omega^{n}\times \Sigma^{\ast m}$ conjuntos
    $\Sigma$-efectivamente enumerables, entonces $S_{1} \cup S_{2}$ y $S_{1} \cap S_{2}$ son $\Sigma$-efectivamente
    enumerables.

  \PROOF El caso en el que alguno de los conjuntos es vacío es trivial. Supongamos que ambos conjuntos son no vacíos y
    sean $\mathbb{P}_{1}$ y $\mathbb{P}_{2}$ procedimientos que enumeran a $S_{1}$ y $S_{2}$.

    \vspace{3mm}
    \par El siguiente procedimiento enumera al conjunto $S_{1} \cup S_{2}$:

    \textbf{Si $x$ es par:} realizar $\mathbb{P}_{1}$ partiendo de $x/2$ y dar el elemento de $S_{1}$ obtenido como
    salida.

    \textbf{Si $x$ es impar:} realizar $\mathbb{P}_{2}$ partiendo de $(x-1)/2$ y dar el elemento de $S_{2}$ obtenido
    como salida.

    \vspace{3mm}
    \par Veamos ahora que $S_{1} \cap S_{2}$ es $\Sigma$-efectivamente enumerable:

    \textbf{Si $S_{1} \cap S_{2} = \emptyset$:} entonces no hay nada que probar.

    \textbf{Si $S_{1} \cap S_{2} \neq \emptyset$:} sea $z_{0}$ un elemento fijo de $S_{1} \cap S_{2}$. Sea
    $\mathbb{P}$ un procedimiento efectivo el cual enumere a $\omega \times \omega$.

    \vspace{3mm}
    \par Un procedimiento que enumera a $S_{1} \cap S_{2}$ es el siguiente:

    \textbf{Etapa 1:}
    Realizar $\mathbb{P}$ con dato de entrada $x$, para obtener un par $(x_{1}, x_{2}) \in \omega \times \omega $.

    \textbf{Etapa 2:}
    Realizar $\mathbb{P}_{1}$ con dato de entrada $x_{1}$ para obtener un elemento $z_{1} \in S_{1}$.

    \textbf{Etapa 3:}
    Realizar $\mathbb{P}_{2}$ con dato de entrada $x_{2}$ para obtener un elemento $z_{2} \in S_{2}$.

    \textbf{Etapa 4:}
    Si $z_{1} = z_{2}$, entonces dar como dato de salida $z_{1}$. En caso contrario dar como dato de salida $z_{0}$.


  \QED


  \textbf{\underline{Lemma 15:}} Si $S \subseteq \omega^{n} \times \Sigma^{\ast m}$ es $\Sigma$-efectivamente
    computable entonces $S$ es $\Sigma$-efectivamente enumerable.

  \PROOF Supongamos $S \neq \emptyset$. Sea $(\vec{z}, \gamma) \in S$, fijo. Sea $\mathbb{P}$ un procedimiento efectivo
    que compute a $\chi_{S}$. Ya vimos que $\omega^{2} \times \Sigma^{\ast 3}$ es $\Sigma$-efectivamente enumerable. En
    forma similar se puede ver que $\omega^{n} \times \Sigma^{\ast m}$ lo es. Sea $\mathbb{P}_{1}$ un procedimiento
    efectivo que enumere a $\omega^{n} \times \Sigma^{\ast m}$, entonces el siguiente procedimiento enumera a $S$:

    \vspace{3mm}
    \textbf{Etapa 1:}
    Realizar $\mathbb{P}_{1}$ con $x$ de entrada para obtener $(\vec{x}, \vec{\alpha}) \in \omega^{n}\times
    \Sigma^{\ast m}$.

    \textbf{Etapa 2:}
    Realizar $\mathbb{P}$ con $(\vec{x}, \vec{\alpha})$ de entrada para obtener el valor Booleano $e$ de salida.

    \textbf{Etapa 3:}
    \textbf{Si $e=1$:} dar como dato de salida $(\vec{x}, \vec{\alpha})$.

    $\qquad\qquad\;\;$\textbf{Si $e=0$:} dar como dato de salida $(\vec{z}, \gamma)$.

  \QED


  \textbf{\underline{Theorem 16:}} Sea $S \subseteq \omega^{n}\times \Sigma^{\ast m}$. Son equivalentes:

    \begin{enumerate}[a)]
      \item $S$ es $\Sigma$-efectivamente computable
      \item $S$ y $(\omega^{n}\times \Sigma^{\ast m})-S$ son $\Sigma$-efectivamente enumerables
    \end{enumerate}

  \PROOF \begin{tabular}{|c|}\hline $(a) \Rightarrow (b)$\\\hline\end{tabular} Por el lema anterior tenemos que $S$ es
    $\Sigma$-efectivamente enumerable. Notese además que, dado que $S$ es $\Sigma$-efectivamente computable,
    $(\omega^{n} \times \Sigma^{\ast m})-S$ también lo es, es decir, que aplicando nuevamente el lema anterior tenemos
    que $(\omega^{n} \times \Sigma^{\ast m})-S$ es $\Sigma$-efectivamente enumerable.

    \vspace{3mm}
    \begin{tabular}{|c|}\hline $(b) \Rightarrow (a)$\\\hline\end{tabular} Sea $\mathbb{P}_{1}$ un procedimiento
      efectivo que enumere a $S$ y sea $\mathbb{P}_{2}$ un procedimiento efectivo que enumere a
      $(\omega^{n}\times \Sigma^{\ast m})-S$. Es fácil ver que el siguiente procedimiento computa el predicado
      $\chi _{S}$:

    \vspace{3mm}
    \textbf{Etapa 1:}
    Darle a la variable $T$ el valor $0$.

    \textbf{Etapa 2:}
    Realizar $\mathbb{P}_{1}$ con el valor de $T$ como entrada para obtener de salida la upla $(\vec{y}, \vec{\beta})$.

    \textbf{Etapa 3:}
    Realizar $\mathbb{P}_{2}$ con el valor de $T$ como entrada para obtener de salida la upla $(\vec{z}, \vec{\gamma})$.

    \textbf{Etapa 4:}
    \textbf{Si $(\vec{y}, \vec{\beta}) = (\vec{x}, \vec{\alpha})$:} entonces detenerse y dar como dato de salida el
    valor $1$.

    $\qquad\qquad\;\;$\textbf{Si $(\vec{z}, \vec{\gamma}) = (\vec{x}, \vec{\alpha})$:} entonces detenerse y dar como
    dato de salida el valor $0$.


    $\qquad\qquad\;\;$\textbf{Si no suceden ninguna de las dos posibilidades:} aumentar en $1$ el valor de

    $\qquad\qquad\;\;$la variable $T$ y dirijirse a la Etapa 2.

  \QED


  \textbf{\underline{Theorem 17:}} Dado $S \subseteq \omega^{n}\times \Sigma^{\ast m}$, son equivalentes:

    \begin{enumerate}
      \item $S$ es $\Sigma$-efectivamente enumerable
      \item $S = \emptyset$ ó $S = I_{F}$, para alguna $F:\omega \rightarrow \omega^{n} \times \Sigma^{\ast m}$ tal que
      cada $F_{i}$ es $\Sigma$-efectivamente computable.
      \item $S = I_{F}$, para alguna $F:D_{F} \subseteq \omega^{k} \times \Sigma^{\ast l} \rightarrow \omega^{n} \times
      \Sigma^{\ast m}$ tal que cada $F_{i}$ es $\Sigma$-efectivamente computable.
      \item $S = D_{f}$, para alguna función $f$ la cual es $\Sigma$-efectivamente computable.
    \end{enumerate}

  \PROOF \begin{tabular}{|c|}\hline $(1) \Rightarrow (2)$\\\hline\end{tabular} y
    \begin{tabular}{|c|}\hline $(2) \Rightarrow (1)$\\\hline\end{tabular} son muy naturales y son dejadas al lector.

    \vspace{3mm}
    \begin{tabular}{|c|}\hline $(2) \Rightarrow (3)$\\\hline\end{tabular} es trivial.

    \vspace{3mm}
    \begin{tabular}{|c|}\hline $(3) \Rightarrow (4)$\\\hline\end{tabular} Para $i = 1, ..., n + m$, sea
    $\mathbb{P}_{i}$ un procedimiento el cual computa a $F_{i}$ y sea $\mathbb{P}$ un procedimiento el cual enumere a
    $\omega \times \omega^{k} \times \Sigma^{\ast l}$. El siguiente procedimiento computa la función
    $f:I_{F} \rightarrow \{1\}$:

    \vspace{3mm}
    \textbf{Etapa 1:}
    Darle a la variable $T$ el valor $0$.

    \textbf{Etapa 2:}
    Hacer correr $\mathbb{P}$ con dato de entrada $T$ y obtener $ (t, z_{1}, ..., z_{k}, \gamma_{1}, ..., \gamma_{l})$
    como dato de salida.

    \textbf{Etapa 3:}
    Para cada $i = 1, ..., n + m$, hacer correr $\mathbb{P}_{i}$ durante $t$ pasos, con dato de entrada $(z_{1}, ...,
    z_{k}, \gamma_{1}, ..., \gamma_{l})$. Si cada procedimiento $\mathbb{P}_{i}$ al cabo de los $t$ pasos termino y dió
    como resultado el valor $o_{i}$, entonces comparar $(\vec{x}, \vec{\alpha})$ con $(o_{1}, ..., o_{n+m})$ y en caso
    de que sean iguales detenerse y dar como dato de salida el valor $1$. En el caso en que no son iguales, aumentar en
    $1$ el valor de la variable $T$ y dirijirse a la Etapa 2. Si algún procedimiento $\mathbb{P}_{i}$ al cabo de los
    $t$ pasos no terminó, entonces aumentar en $1$ el valor de la variable $T$ y dirijirse a la Etapa 2.

    \vspace{3mm}
    \begin{tabular}{|c|}\hline $(4) \Rightarrow (1)$\\\hline\end{tabular} Supongamos $S \neq \emptyset$. Sea
      $(\vec{z}, \vec{\gamma})$ un elemento fijo de $S$. Sea $\mathbb{P}$ un procedimiento el cual compute a $f$. Sea
      $\mathbb{P}_{1}$ un procedimiento el cual enumere a $\omega \times \omega^{n}\times \Sigma^{\ast m}.$ Dejamos al
      lector el diseño de un procedimiento efectivo el cual enumere $D_{f}$.

  \QED
