Lema 33 Si \(f:D_{f}\subseteq \omega ^{n}\times \Sigma ^{\ast m}\rightarrow O\) es \(\Sigma \)-p.r., entonces existe una funcion \(\Sigma \) -p.r. \(\bar{f}:\omega ^{n}\times \Sigma ^{\ast m}\rightarrow O\), tal que \(f= \bar{f}\mid _{D_{f}}\).
Prueba: Es facil ver por induccion en \(k\) que el enunciado se cumple para cada \(f\in \mathrm{PR}_{k}^{\Sigma }\) \(\Box\)

Proposición 34 Un conjunto \(S\) es \(\Sigma \)-p.r. sii \(S\) es el dominio de una funcion \(\Sigma \)-p.r.\(.\)
Prueba: (\(\Rightarrow \)) Note que \(S=D_{Pred\circ \chi _{S}}.\)

(\(\Leftarrow \)) Probaremos por induccion en \(k\) que \(D_{F}\) es \(\Sigma \) -p.r., para cada \(F\in \mathrm{PR}_{k}^{\Sigma }.\) El caso \(k=0\) es facil\(.\) Supongamos el resultado vale para un \(k\) fijo y supongamos \(F\in \mathrm{PR} _{k+1}^{\Sigma }.\) Veremos entonces que \(D_{F}\) es \(\Sigma \)-p.r.. Hay varios casos. Consideremos primero el caso en que \(F=R(f,g)\), donde

\(\displaystyle \begin{array}{rcl} f & :& S_{1}\times ...\times S_{n}\times L_{1}\times ...\times L_{m}\rightarrow \Sigma ^{\ast } \\ g & :& \omega \times S_{1}\times ...\times S_{n}\times L_{1}\times ...\times L_{m}\times \Sigma ^{\ast }\rightarrow \Sigma ^{\ast }, \end{array} \)

con \(S_{1},...,S_{n}\subseteq \omega \) y \(L_{1},...,L_{m}\subseteq \Sigma ^{\ast }\) conjuntos no vacios y \(f,g\in \mathrm{PR}_{k}^{\Sigma }\). Notese que por definicion de \(R(f,g)\), tenemos que
\(\displaystyle D_{F}=\omega \times S_{1}\times ...\times S_{n}\times L_{1}\times ...\times L_{m}. \)

Por hipotesis inductiva tenemos que \(D_{f}=S_{1}\times ...\times S_{n}\times L_{1}\times ...\times L_{m}\) es \(\Sigma \)-p.r., lo cual por el Lema 31 nos dice que los conjuntos \(S_{1},...,S_{n}\), \( L_{1},...,L_{m}\) son \(\Sigma \)-p.r.. Ya que \(\omega \) es \(\Sigma \)-p.r., el Lema 31 nos dice que \(D_{F}\) es \(\Sigma \)-p.r..
Los otros casos de recursion primitiva son dejados al lector.

Supongamos ahora que \(F=g\circ (g_{1},...,g_{n+m})\), donde

\(\displaystyle \begin{array}{rcl} g & :& D_{g}\subseteq \omega ^{n}\times \Sigma ^{\ast m}\rightarrow O \\ g_{i} & :& D_{g_{i}}\subseteq \omega ^{k}\times \Sigma ^{\ast l}\rightarrow \omega \text{, }i=1,...,n \\ g_{i} & :& D_{g_{i}}\subseteq \omega ^{k}\times \Sigma ^{\ast l}\rightarrow \Sigma ^{\ast },i=n+1,...,n+m \end{array} \)

estan en \(\mathrm{PR}_{k}^{\Sigma }.\) Por Lema 33, hay funciones \(\Sigma \)-p.r. \(\bar{g}_{1},...,\bar{g}_{n+m}\) las cuales son \( \Sigma \)-totales y cumplen
\(\displaystyle g_{i}=\bar{g}_{i}\mid _{D_{g_{i}}}\text{, para }i=1,...,n+m. \)

Por hipotesis inductiva los conjuntos \(D_{g}\), \(D_{g_{i}}\), \(i=1,...,n+m\), son \(\Sigma \)-p.r. y por lo tanto
\(\displaystyle S=\bigcap_{i=1}^{n+m}D_{g_{i}} \)

lo es. Notese que
\(\displaystyle \chi _{D_{F}}=(\chi _{D_{g}}\circ \left( \bar{g}_{1},...,\bar{g} _{n+m}\right) \wedge \chi _{S}) \)

lo cual nos dice que \(D_{F}\) es \(\Sigma \)-p.r.. \(\Box\)

\subsubsection{Lema de division por casos}

Una observacion interesante es que si \(f_{i}:D_{f_{i}}\rightarrow O\), \( i=1,...,k\), son funciones tales que \(D_{f_{i}}\cap D_{f_{j}}=\varnothing \) para \(i\neq j\), entonces \(f_{1}\cup ...\cup f_{k}\) es la funcion

\(\displaystyle \begin{array}{rll} D_{f_{1}}\cup ...\cup D_{f_{k}} & \rightarrow & O \\ e & \rightarrow & \left\{ \begin{array}{clc} f_{1}(e) & & \text{si }e\in D_{f_{1}} \\ \vdots & & \vdots \\ f_{k}(e) & & \text{si }e\in D_{f_{k}} \end{array} \right. \end{array} \)


Lema 35 Supongamos \(f_{i}:D_{f_{i}}\subseteq \omega ^{n}\times \Sigma ^{\ast m}\rightarrow O\), \(i=1,...,k\), son funciones \(\Sigma \)-p.r. tales que \(D_{f_{i}}\cap D_{f_{j}}=\varnothing \) para \(i\neq j.\) Entonces \(f_{1}\cup ...\cup f_{k}\) es \(\Sigma \)-p.r..
Prueba: Supongamos \(O=\Sigma ^{\ast }\) y \(k=2.\) Sean

\(\displaystyle \bar{f}_{i}:\omega ^{n}\times \Sigma ^{\ast m}\rightarrow \Sigma ^{\ast },i=1,2, \)

funciones \(\Sigma \)-p.r. tales que \(\bar{f}_{i}\mid _{D_{f_{i}}}=f_{i}\), \( i=1,2\) (Lema 33)\(.\) Por Lema 34 los conjuntos \(D_{f_{1}}\) y \(D_{f_{2}}\) son \(\Sigma \)-p.r. y por lo tanto lo es \( D_{f_{1}}\cup D_{f_{2}}\). Ya que
\(\displaystyle f_{1}\cup f_{2}=\left( \lambda \alpha \beta \left[ \alpha \beta \right] \circ (\lambda x\alpha \left[ \alpha ^{x}\right] \circ (\chi _{D_{f_{1}}}, \bar{f}_{1}),\lambda x\alpha \left[ \alpha ^{x}\right] \circ (\chi _{D_{f_{2}}},\bar{f}_{2}))\right) \mid _{D_{f_{1}}\cup D_{f_{2}}} \)

tenemos que \(f_{1}\cup f_{2}\) es \(\Sigma \)-p.r..
El caso \(k >2\) puede probarse por induccion ya que

\(\displaystyle f_{1}\cup ...\cup f_{k}=(f_{1}\cup ...\cup f_{k-1})\cup f_{k}. \)

\(\Box\)


Corolario 36 Supongamos \(f\) es una funcion \(\Sigma \)-mixta cuyo dominio es finito. Entonces \(f\) es \(\Sigma \)-p.r..
Prueba: Supongamos \(f:D_{f}\subseteq \omega ^{n}\times \Sigma ^{\ast m}\rightarrow O\) , con \(D_{f}=\{e_{1},...,e_{k}\}\). Por el Corolario 30, cada \( \{e_{i}\}\) es \(\Sigma \)-p.r. por lo cual el Lema 32 nos dice que \(C_{f(e_{i})}^{n,m}\mid _{\{e_{1}\}}\) es \(\Sigma \)-p.r.. O sea que

\(\displaystyle f=C_{f(e_{1})}^{n,m}\mid _{\{e_{1}\}}\cup ...\cup C_{f(e_{k})}^{n,m}\mid _{\{e_{k}\}} \)

es \(\Sigma \)-p.r.. \(\Box\)
Recordemos que dados \(i\in \omega \) y \(\alpha \in \Sigma ^{\ast }\), definimos

\(\displaystyle \left[ \alpha \right] _{i}=\left\{ \begin{array}{lll} i\text{-esimo elemento de }\alpha & & \text{si }1\leq i\leq \left\vert \alpha \right\vert \\ \varepsilon & & \text{caso contrario} \end{array} \right. \)



Lema 37 \(\lambda i\alpha \left[ \lbrack \alpha ]_{i}\right] \) es \(\Sigma \)-p.r..
Prueba: Note que

\(\displaystyle \begin{array}{rcl} \lbrack \varepsilon ]_{i} & =& \varepsilon \\ \lbrack \alpha a]_{i} & =& \left\{ \begin{array}{lll} \lbrack \alpha ]_{i} & & \text{si }i\neq \left\vert \alpha \right\vert +1 \\ a & & \text{si }i=\left\vert \alpha \right\vert +1 \end{array} \right. \end{array} \)

lo cual dice que \(\lambda i\alpha \left[ \lbrack \alpha ]_{i}\right] =R\left( C_{\varepsilon }^{1,0},\mathcal{G}\right) \), donde \(\mathcal{G} _{a}:\omega \times \Sigma ^{\ast }\times \Sigma ^{\ast }\rightarrow \Sigma ^{\ast }\) es dada por
\(\displaystyle \mathcal{G}_{a}(i,\alpha ,\zeta )=\left\{ \begin{array}{lll} \zeta & & \text{si }i\neq \left\vert \alpha \right\vert +1 \\ a & & \text{si }i=\left\vert \alpha \right\vert +1 \end{array} \right. \)

O sea que solo resta probar que cada \(\mathcal{G}_{a}\) es \(\Sigma \)-p.r.. Primero note que los conjuntos
\(\displaystyle \begin{array}{rcl} S_{1} & =& \left\{ (i,\alpha ,\zeta )\in \omega \times \Sigma ^{\ast }\times \Sigma ^{\ast }:i\neq \left\vert \alpha \right\vert +1\right\} \\ S_{2} & =& \left\{ (i,\alpha ,\zeta )\in \omega \times \Sigma ^{\ast }\times \Sigma ^{\ast }:i=\left\vert \alpha \right\vert +1\right\} \end{array} \)

son \(\Sigma \)-p.r. ya que
\(\displaystyle \begin{array}{rcl} \chi _{S_{1}} & =& \lambda xy\left[ x\neq y\right] \circ \left( p_{1}^{1,2},Suc\circ \lambda \alpha \left[ \left\vert \alpha \right\vert \right] \circ p_{2}^{1,2}\right) \\ \chi _{S_{2}} & =& \lambda xy\left[ x=y\right] \circ \left( p_{1}^{1,2},Suc\circ \lambda \alpha \left[ \left\vert \alpha \right\vert \right] \circ p_{2}^{1,2}\right) . \end{array} \)

Ya que
\(\displaystyle \mathcal{G}_{a}=p_{3}^{1,2}\mid _{S_{1}}\cup C_{a}^{1,2}\mid _{S_{2}}, \)

el Lema 35 nos dice que \(\mathcal{G}_{a}\) es \(\Sigma \)-p.r., para cada \(a\in \Sigma \). \(\Box\)


\subsubsection{Sumatoria, productoria y concatenatoria}

Sea \(f:\omega \times S_{1}\times ...\times S_{n}\times L_{1}\times ...\times L_{m}\rightarrow \omega \), donde \(S_{1},...,S_{n}\subseteq \omega \) y \( L_{1},...,L_{m}\subseteq \Sigma ^{\ast }\) son no vacios. Para \(x,y\in \omega \) y \((\vec{x},\vec{\alpha})\in S_{1}\times ...\times S_{n}\times L_{1}\times ...\times L_{m}\), definamos

\(\displaystyle \begin{array}{rcl} \sum\limits_{t=x}^{t=y}f(t,\vec{x},\vec{\alpha}) & =& \left\{ \begin{array}{lll} 0 & & \text{si }x >y \\ f(x,\vec{x},\vec{\alpha})+f(x+1,\vec{x},\vec{\alpha})+...+f(y,\vec{x},\vec{ \alpha}) & & \text{si }x\leq y \end{array} \right. \\ \prod\limits_{t=x}^{t=y}f(t,\vec{x},\vec{\alpha}) & =& \left\{ \begin{array}{lll} 1 & & \text{si }x >y \\ f(x,\vec{x},\vec{\alpha}).f(x+1,\vec{x},\vec{\alpha})....f(y,\vec{x},\vec{ \alpha}) & & \text{si }x\leq y \end{array} \right. \end{array} \)

En forma similar, cuando \(I_{f}\subseteq \Sigma ^{\ast }\), definamos
\(\displaystyle \overset{t=y}{\underset{t=x}{\subset }}f(t,\vec{x},\vec{\alpha})=\left\{ \begin{array}{lll} \varepsilon & & \text{si }x >y \\ f(x,\vec{x},\vec{\alpha})f(x+1,\vec{x},\vec{\alpha})....f(y,\vec{x},\vec{ \alpha}) & & \text{si }x\leq y \end{array} \right. \)

Note que, en virtud de la definicion anterior, el dominio de las funciones
\(\displaystyle \lambda xy\vec{x}\vec{\alpha}\left[ \sum_{t=x}^{t=y}f(t,\vec{x},\vec{\alpha}) \right] \ \ \ \ \ \ \ \ \ \ \ \ \lambda xy\vec{x}\vec{\alpha}\left[ \prod_{t=x}^{t=y}f(t,\vec{x},\vec{\alpha})\right] \ \ \ \ \ \ \ \ \ \ \ \ \lambda xy\vec{x}\vec{\alpha}\left[ \subset _{t=x}^{t=y}f(t,\vec{x},\vec{ \alpha})\right] \)

es \(\omega \times \omega \times S_{1}\times ...\times S_{n}\times L_{1}\times ...\times L_{m}\).


Lema 38 Sean \(n,m\geq 0\).
(a) Si \(f:\omega \times S_{1}\times ...\times S_{n}\times L_{1}\times ...\times L_{m}\rightarrow \omega \) es \(\Sigma \)-p.r., con \( S_{1},...,S_{n}\subseteq \omega \) y \(L_{1},...,L_{m}\subseteq \Sigma ^{\ast } \) no vacios, entonces lo son las funciones \(\lambda xy\vec{x}\vec{\alpha} \left[ \sum_{t=x}^{t=y}f(t,\vec{x},\vec{\alpha})\right] \) y \(\lambda xy\vec{x }\vec{\alpha}\left[ \prod_{t=x}^{t=y}f(t,\vec{x},\vec{\alpha})\right] \).
(b) Si \(f:\omega \times S_{1}\times ...\times S_{n}\times L_{1}\times ...\times L_{m}\rightarrow \Sigma ^{\ast }\) es \(\Sigma \)-p.r., con \( S_{1},...,S_{n}\subseteq \omega \) y \(L_{1},...,L_{m}\subseteq \Sigma ^{\ast } \) no vacios, entonces lo es la funcion \(\lambda xy\vec{x}\vec{\alpha}\left[ \subset _{t=x}^{t=y}f(t,\vec{x},\vec{\alpha})\right] \)
Prueba: (a) Sea \(G=\lambda tx\vec{x}\vec{\alpha}\left[ \sum_{i=x}^{i=t}f(i,\vec{x}, \vec{\alpha})\right] \). Ya que

\(\displaystyle \lambda xy\vec{x}\vec{\alpha}\left[ \sum_{i=x}^{i=y}f(i,\vec{x},\vec{\alpha}) \right] =G\circ \left( p_{2}^{n+2,m},p_{1}^{n+2,m},p_{3}^{n+2,m},...,p_{n+m+2}^{n+2,m}\right) \)

solo tenemos que probar que \(G\) es \(\Sigma \)-p.r.. Primero note que
\(\displaystyle \begin{array}{rcl} G(0,x,\vec{x},\vec{\alpha}) & =& \left\{ \begin{array}{lll} 0 & & \text{si }x >0 \\ f(0,\vec{x},\vec{\alpha}) & & \text{si }x=0 \end{array} \right. \\ G(t+1,x,\vec{x},\vec{\alpha}) & =& \left\{ \begin{array}{lll} 0 & & \text{si }x >t+1 \\ G(t,x,\vec{x},\vec{\alpha})+f(t+1,\vec{x},\vec{\alpha}) & & \text{si }x\leq t+1 \end{array} \right. \end{array} \)

Sean
\(\displaystyle \begin{array}{rcl} D_{1} & =& \left\{ (x,\vec{x},\vec{\alpha})\in \omega \times S_{1}\times ...\times S_{n}\times L_{1}\times ...\times L_{m}:x >0\right\} \\ D_{2} & =& \left\{ (x,\vec{x},\vec{\alpha})\in \omega \times S_{1}\times ...\times S_{n}\times L_{1}\times ...\times L_{m}:x=0\right\} \\ H_{1} & =& \left\{ (z,t,x,\vec{x},\vec{\alpha})\in \omega ^{3}\times S_{1}\times ...\times S_{n}\times L_{1}\times ...\times L_{m}:x >t+1\right\} \\ H_{2} & =& \left\{ (z,t,x,\vec{x},\vec{\alpha})\in \omega ^{3}\times S_{1}\times ...\times S_{n}\times L_{1}\times ...\times L_{m}:x\leq t+1\right\} . \end{array} \)

Es facil de chequear que estos conjuntos son \(\Sigma \)-p.r.. Veamos que por ejemplo \(H_{1}\) lo es. Es decir debemos ver que \(\chi _{H_{1}}\) es \(\Sigma \) -p.r.. Ya que \(f\) es \(\Sigma \)-p.r. tenemos que \(D_{f}=\omega \times S_{1}\times ...\times S_{n}\times L_{1}\times ...\times L_{m}\) es \(\Sigma \) -p.r., lo cual por el Lema 31 nos dice que los conjuntos \( S_{1},...,S_{n}\), \(L_{1},...,L_{m}\) son \(\Sigma \)-p.r.. Ya que \(\omega \) es \( \Sigma \)-p.r., el Lema 31 nos dice que \(R=\omega ^{3}\times S_{1}\times ...\times S_{n}\times L_{1}\times ...\times L_{m}\) es \(\Sigma \)-p.r.. Notese que \(\chi _{H_{1}}=(\chi _{R}\wedge \lambda ztx\vec{x} \vec{\alpha}\left[ x >t+1\right] )\) por cual \(\chi _{H_{1}}\) es \(\Sigma \) -p.r. ya que es la conjuncion de dos predicados \(\Sigma \)-p.r.
Ademas note que \(G=R(h,g)\), donde

\(\displaystyle \begin{array}{rcl} h & =& C_{0}^{n+1,m}\mid _{D_{1}}\cup \lambda x\vec{x}\vec{\alpha}\left[ f(0, \vec{x},\vec{\alpha})\right] \mid _{D_{2}} \\ g & =& C_{0}^{n+3,m}\mid _{H_{1}}\cup \lambda ztx\vec{x}\vec{\alpha}\left[ z+f(t+1,\vec{x},\vec{\alpha})\right] )\mid _{H_{2}} \end{array} \)

O sea que los Lemas 35 y 32 garantizan que \(G\) es \( \Sigma \)-p.r.. \(\Box\)
Cuantificacion acotada de predicados con dominio rectangular.


Lema 39 Sean \(n,m\geq 0\).
(a) Sea \(P:S\times S_{1}\times ...\times S_{n}\times L_{1}\times ...\times L_{m}\rightarrow \omega \) un predicado \(\Sigma \)-p.r. y supongamos \(\bar{S}\subseteq S\) es \(\Sigma \)-p.r.. Entonces \(\lambda x\vec{x}\vec{\alpha }\left[ (\forall t\in \bar{S})_{t\leq x}\;P(t,\vec{x},\vec{\alpha})\right] \) y \(\lambda x\vec{x}\vec{\alpha}\left[ (\exists t\in \bar{S})_{t\leq x}\;P(t, \vec{x},\vec{\alpha})\right] \) son predicados \(\Sigma \)-p.r.. (Note que el dominio de estos predicados es \(\omega \times S_{1}\times ...\times S_{n}\times L_{1}\times ...\times L_{m}\))
(b) Sea \(P:S_{1}\times ...\times S_{n}\times L_{1}\times ...\times L_{m}\times L\rightarrow \omega \) un predicado \(\Sigma \)-p.r. y supongamos \( \bar{L}\subseteq L\) es \(\Sigma \)-p.r.. Entonces \(\lambda x\vec{x}\vec{\alpha} \left[ (\forall \alpha \in \bar{L})_{\left\vert \alpha \right\vert \leq x}\;P(\vec{x},\vec{\alpha},\alpha )\right] \) y \(\lambda x\vec{x}\vec{\alpha} \left[ (\exists \alpha \in \bar{L})_{\left\vert \alpha \right\vert \leq x}\;P(\vec{x},\vec{\alpha},\alpha )\right] \) son predicados \(\Sigma \)-p.r..
Prueba: (a) Sea

\(\displaystyle \bar{P}=P\mid _{\bar{S}\times S_{1}\times ...\times S_{n}\times L_{1}\times ...\times L_{m}}\cup C_{1}^{1+n,m}\mid _{(\omega -\bar{S})\times S_{1}\times ...\times S_{n}\times L_{1}\times ...\times L_{m}} \)

Notese que \(\bar{P}\) es \(\Sigma \)-p.r.. Ya que
\(\displaystyle \begin{array}{rcl} \lambda x\vec{x}\vec{\alpha}\left[ (\forall t\in \bar{S})_{t\leq x}P(t,\vec{x },\vec{\alpha})\right] & =& \lambda x\vec{x}\vec{\alpha}\left[ \prod\limits_{t=0}^{t=x}\bar{P}(t,\vec{x},\vec{\alpha})\right] \\ & =& \lambda xy\vec{x}\vec{\alpha}\left[ \prod\limits_{t=x}^{t=y}\bar{P}(t, \vec{x},\vec{\alpha})\right] \circ \left( C_{0}^{1+n,m},p_{1}^{1+n,m},...,p_{1+n+m}^{1+n,m}\right) \end{array} \)

el Lema 38 implica que \(\lambda x\vec{x}\vec{\alpha}\left[ (\forall t\in \bar{S})_{t\leq x}\;P(t,\vec{x},\vec{\alpha})\right] \) es \( \Sigma \)-p.r..
Finalmente note que

\(\displaystyle \lambda x\vec{x}\vec{\alpha}\left[ (\exists t\in \bar{S})_{t\leq x}\;P(t, \vec{x},\vec{\alpha})\right] =\lnot \lambda x\vec{x}\vec{\alpha}\left[ (\forall t\in \bar{S})_{t\leq x}\;\lnot P(t,\vec{x},\vec{\alpha})\right] \)

es \(\Sigma \)-p.r..
(b) Sea \(< \) un orden total estricto sobre \(\Sigma .\) Sea \(k\) el cardinal de \( \Sigma \). Ya que

\(\displaystyle \left\vert \alpha \right\vert \leq x\text{ sii }\#^{< }(\alpha )\leq \sum_{\iota =1}^{i=x}k^{i}, \)

(ejercicio) tenemos que
\(\displaystyle \lambda x\vec{x}\vec{\alpha}\left[ (\forall \alpha \in \bar{L})_{\left\vert \alpha \right\vert \leq x}P(\vec{x},\vec{\alpha},\alpha )\right] =\lambda x \vec{x}\vec{\alpha}\left[ (\forall t\in \#^{< }(\bar{L}))_{t\leq \sum_{\iota =1}^{i=x}k^{i}}P(\vec{x},\vec{\alpha},\ast ^{< }(t))\right] \)

Sea \(H=\lambda t\vec{x}\vec{\alpha}\left[ P(\vec{x},\vec{\alpha},\ast ^{< }(t))\right] .\) Notese que \(H\) es \(\Sigma \)-p.r. y
\(\displaystyle D_{H}=\#^{< }(L)\times S_{1}\times ...\times S_{n}\times L_{1}\times ...\times L_{m} \)

Ademas note que \(\#^{< }(\bar{L})\) es \(\Sigma \)-p.r. (ejercicio), lo cual por (a) implica que
\(\displaystyle Q=\lambda x\vec{x}\vec{\alpha}\left[ (\forall t\in \#^{< }(\bar{L}))_{t\leq x}H(t,\vec{x},\vec{\alpha})\right] \)

es \(\Sigma \)-p.r.. O sea que
\(\displaystyle \lambda x\vec{x}\vec{\alpha}\left[ (\forall \alpha \in \bar{L})_{\left\vert \alpha \right\vert \leq x}\;P(\vec{x},\vec{\alpha},\alpha )\right] =Q\circ \left( \lambda x\vec{x}\vec{\alpha}\left[ \sum\limits_{\iota =1}^{i=x}k^{i} \right] ,p_{1}^{1+n,m},...,p_{1+n+m}^{1+n,m}\right) \)

es \(\Sigma \)-p.r.. \(\Box\)
Algunos ejemplos en los cuales cuantificacion acotada se aplica naturalmente son


Lema 40
(a) El predicado \(\lambda xy\left[ x\text{ divide }y\right] \) es \( \varnothing \)-p.r..
(b) El predicado \(\lambda x\left[ x\text{ es primo}\right] \) es \( \varnothing \)-p.r..
(c) El predicado \(\lambda \alpha \beta \left[ \alpha \text{\ }\mathrm{ inicial}\ \beta \right] \) es \(\Sigma \)-p.r..
Prueba: (a) Si tomamos \(P=\lambda tx_{1}x_{2}\left[ x_{2}=t.x_{1}\right] \in \mathrm{PR}^{\varnothing }\), tenemos que

\(\displaystyle \begin{array}{rcl} \lambda x_{1}x_{2}\left[ x_{1}\text{ divide }x_{2}\right] & =& \lambda x_{1}x_{2}\left[ (\exists t\in \omega )_{t\leq x_{2}}\;P(t,x_{1},x_{2}) \right] \\ & =& \lambda xx_{1}x_{2}\left[ (\exists t\in \omega )_{t\leq x}\;P(t,x_{1},x_{2})\right] \circ \left( p_{2}^{2,0},p_{1}^{2,0},p_{2}^{2,0}\right) \end{array} \)

por lo que podemos aplicar el lema anterior.
(b) Ya que

\(\displaystyle x\text{ es primo sii }x >1\wedge \left( (\forall t\in \omega )_{t\leq x}\;t=1\vee t=x\vee \lnot (t\text{ divide }x)\right) \)

podemos usar un argumento similar al de la prueba de (a).
(c) es dejado al lector. \(\Box\)
