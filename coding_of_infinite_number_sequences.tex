\section{Codificacion de sucesiones infinitas de numeros}

  \textbf{\underline{Lemma 10:}} Si \(p,p_{1},...,p_{n}\) son numeros primos y \(p\) divide a \(p_{1}...p_{n}\), entonces \(p=p_{i}\), para algun \(i\).
Ahora la version del Teorema Fundamental de la Aritmetica.

  \textbf{\underline{Theorem 11:}} Para cada \(x\in \mathbf{N}\), hay una unica sucesion \((s_{1},s_{2},...)\in \omega ^{\left[ \mathbf{N}\right] }\) tal que
\(\displaystyle x=\underset{i=1}{\overset{\infty }{\Pi }}pr(i)^{s_{i}} \)
(Notese que \(\underset{i=1}{\overset{\infty }{\Pi }}pr(i)^{s_{i}}\) tiene sentido ya que es un producto que solo tiene una cantidad finita de factores no iguales a \(1\). )

\textbf{\underline{Proof:}} Primero probaremos la existencia por induccion en \(x\). Claramente \(1= \underset{i=1}{\overset{\infty }{\Pi }}pr(i)^{0}\), con lo cual el caso \(x=1\) esta probado. Supongamos la existencia vale para cada \(y\) menor que \(x\), veremos que entonces vale para \(x\). Si \(x\) es primo, entonces \(x=pr(i_{0})\) para algun \(i_{0}\) por lo cual tenemos que \(x=\underset{i=1}{\overset{\infty }{\Pi }}pr(i)^{s_{i}}\), tomando \(s_{i}=0\) si \(i\neq i_{0}\) y \(s_{i_{0}}=1\). Si \(x\) no es primo, entonces \(x=y_{1}.y_{2}\) con \(y_{1},y_{2}< x\). Por hipotesis inductiva tenemos que hay \((s_{1},s_{2},...),(t_{1},t_{2},...)\in \omega ^{\left[ \mathbf{N}\right] }\) tales que \(y_{1}=\underset{i=1}{\overset {\infty }{\Pi }}pr(i)^{s_{i}}\) y \(y_{2}=\underset{i=1}{\overset{\infty }{\Pi }}pr(i)^{t_{i}}\). Tenemos entonces que \(x=\underset{i=1}{\overset{\infty }{ \Pi }}pr(i)^{s_{i}+t_{i}}\) lo cual concluye la prueba de la existencia.

Veamos ahora la unicidad. Suponganos que

\(\displaystyle \underset{i=1}{\overset{\infty }{\Pi }}pr(i)^{s_{i}}=\underset{i=1}{\overset{ \infty }{\Pi }}pr(i)^{t_{i}} \)

Si \(s_{i} >t_{i}\) entonces dividiendo ambos miembros por \(pr(i)^{t_{i}}\) obtenemos que \(pr(i)\) divide a un producto de primos todos distintos de el, lo cual es absurdo por el lema anterior. Analogamente llegamos a un absurdo si suponemos que \(t_{i} >s_{i}\), lo cual nos dice que \(s_{i}=t_{i}\), para cada \(i\in \mathbf{N}\) \(\Box\)

  \textbf{\underline{Lemma 12:}} Las funciones
\(\displaystyle \begin{array}{lll} \mathbf{N} & \rightarrow & \omega ^{\left[ \mathbf{N}\right] } \\ x & \rightarrow & ((x)_{1},(x)_{2},...) \end{array} \ \ \ \ \ \ \ \ \ \ \ \ \ \ \ \ \ \ \begin{array}{rll} \omega ^{\left[ \mathbf{N}\right] } & \rightarrow & \mathbf{N} \\ (s_{1},s_{2},...) & \rightarrow & \left\langle s_{1},s_{2},...\right\rangle \end{array} \)
son biyecciones una inversa de la otra.
\textbf{\underline{Proof:}} Notese que para cada \(x\in \mathbf{N}\), tenemos que \(\left\langle (x)_{1},(x)_{2},...\right\rangle =x\). Ademas para cada \((s_{1},s_{2},...)\in \omega ^{\left[ \mathbf{N}\right] }\), tenemos que \(((\left\langle s_{1},s_{2},...\right\rangle )_{1},(\left\langle s_{1},s_{2},...\right\rangle )_{2},...)=(s_{1},s_{2},...)\). Es claro que lo anterior garantiza que los mapeos en cuestion son uno inversa del otro \(\Box\)

  \textbf{\underline{Lemma 13}} Para cada \(x\in \mathbf{N}\):
\(Lt(x)=0\) sii \(x=1\)
\(x=\prod\nolimits_{i=1}^{Lt(x)}pr(i)^{(x)_{i}}\)
Cabe destacar entonces que la funcion \(\lambda ix[(x)_{i}]\) tiene dominio igual a \(\mathbf{N}^{2}\) y la funcion \(\lambda ix[Lt(x)]\) tiene dominio igual a \(\mathbf{N}\).
