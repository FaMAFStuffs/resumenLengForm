\subsection{Ordenes naturales sobre \(\Sigma ^{\ast }\)}

Sea \(A\) un conjunto no vacio cualquiera. Una relacion binaria \(< \) sobre \(A\) sera llamada un orden total estricto sobre \(A\) si se cumplen las siguientes condiciones:

(1) Para todo \(a\in A\), no se da que \(a< a\)
(2) Para todo \(a,b\in A\), si \(a\neq b\), entonces \(a< b\) o \(b< a\)
(3) Para todo \(a,b,c\in A\), si \(a< b\) y \(b< c\), entonces \(a< c\).
Sea \(\Sigma \) un alfabeto no vacio y supongamos \(< \) es un orden total estricto sobre \(\Sigma \). Supongamos que \(\Sigma =\{a_{1},...,a_{n}\}\), con \( a_{1}< a_{2}< ...< a_{n}\). Podemos extender \(< \) a un orden total estricto sobre \(\Sigma ^{\ast }\), de la siguiente manera

- \(\alpha < \beta \), siempre que \(\left\vert \alpha \right\vert < \left\vert \beta \right\vert \)
- \(\alpha a_{i}\beta < \alpha a_{j}\gamma \), siempre que \(\left\vert \beta \right\vert =\left\vert \gamma \right\vert \) y \(i< j\)
Lema 2 La relacion \(< \) es un orden total estricto sobre \(\Sigma ^{\ast }\).
Prueba: Facil. \(\Box\)

Lema 3 La funcion \(s^{< }:\Sigma ^{\ast }\rightarrow \Sigma ^{\ast }\), definida recursivamente de la siguiente manera:
\(\displaystyle \begin{array}{rcl} s^{< }(\varepsilon ) & =& a_{1} \\ s^{< }(\alpha a_{i}) & =& \alpha a_{i+1}\text{, }i< n \\ s^{< }(\alpha a_{n}) & =& s^{< }(\alpha )a_{1} \end{array} \)

tiene la siguiente propiedad
\(\displaystyle s^{< }(\alpha )=\min \{\beta \in \Sigma ^{\ast }:\alpha < \beta \}\text{.} \)
Prueba: Supongamos que \(\alpha < \beta \). Probaremos entonces que \(s^{< }(\alpha )\leq \beta \).

Caso \(\left\vert \alpha \right\vert < \left\vert \beta \right\vert \).

Se puede ver facilmente que \(\left\vert \alpha \right\vert =\left\vert s^{< }(\alpha )\right\vert \) salvo en el caso en que \(\alpha \in \{a_{n}\}^{\ast }\), por lo cual solo resta ver el caso \(\alpha \in \{a_{n}\}^{\ast }.\) Supongamos \(\alpha =a_{n}^{\left\vert \alpha \right\vert }.\) Entonces \(s^{< }(\alpha )=a_{1}^{\left\vert \alpha \right\vert +1}.\) Si \( \left\vert \beta \right\vert =\left\vert \alpha \right\vert +1\) entonces es facil ver usando 2. de la definicion del orden de \(\Sigma ^{\ast }\) que \( s^{< }(\alpha )=a_{1}^{\left\vert \alpha \right\vert +1}\leq \beta .\) Si \( \left\vert \beta \right\vert >\left\vert \alpha \right\vert +1\), entonces por 1. de tal definicion tenemos que \(s^{< }(\alpha )=a_{1}^{\left\vert \alpha \right\vert +1}< \beta \)

Caso \(\left\vert \alpha \right\vert =\left\vert \beta \right\vert \).

Tenemos entonces que

\(\displaystyle \begin{array}{rcl} \alpha & =& \alpha _{1}a_{i}\gamma _{1} \\ \beta & =& \alpha _{1}a_{j}\gamma _{2} \end{array} \)

con \(i< j\) y \(\left\vert \gamma _{1}\right\vert =\left\vert \gamma _{2}\right\vert \). Si \(\gamma _{1}=\gamma _{2}=\varepsilon \) entonces es claro que \(s^{< }(\alpha )\leq \beta .\) El caso en el que \(\gamma _{1}\) termina con \(a_{l}\) para algun \(l< n\) es facil. Veamos el caso en que \(\gamma _{1}=a_{n}^{k}\) con \(k\geq 1.\) Tenemos que
\(\displaystyle \begin{array}{rcl} s^{< }(\alpha ) & = & s^{< }(\alpha _{1}a_{i}a_{n}^{k}) \\ & = & s^{< }(\alpha _{1}a_{i}a_{n}^{k-1})a_{1} \\ & \vdots & \;\;\;\;\;\;\vdots \\ & = & s^{< }(\alpha _{1}a_{i})a_{1}^{k} \\ & = & \alpha _{1}a_{i+1}a_{1}^{k} \\ & \leq & \alpha _{1}a_{j}\gamma _{2}=\beta \end{array} \)

Supongamos finalmente que \(\gamma _{1}=\rho _{1}a_{l}a_{n}^{k}\) con \(k\geq 1\) y \(l< n\). Tenemos que
\(\displaystyle \begin{array}{rcl} s^{< }(\alpha ) & = & s^{< }(\alpha _{1}a_{i}\rho _{1}a_{l}a_{n}^{k}) \\ & = & s^{< }(\alpha _{1}a_{i}\rho _{1}a_{l}a_{n}^{k-1})a_{1} \\ & \vdots & \;\;\;\;\;\;\vdots \\ & = & s^{< }(\alpha _{1}a_{i}\rho _{1}a_{l})a_{1}^{k} \\ & = & \alpha _{1}a_{i}\rho _{1}a_{l+1}a_{1}^{k} \\ & \leq & \beta . \end{array} \)

Para completar nuestra demostracion debemos probar que \(\alpha < s^{< }(\alpha )\), para cada \(\alpha \in \Sigma ^{\ast }\). Dejamos al lector como ejercicio esta prueba la cual puede ser hecha por inducion en \(\left\vert \alpha \right\vert \) usando argumentos parecidos a los usados recien. \(\Box\)
En virtud del lema anterior llamaremos a \(s^{< }\) la funcion sucesor respecto del orden \(< \) de \(\Sigma ^{\ast }\).

Corolario 4 \(s^{< }\) es inyectiva.
Prueba: Supongamos \(\alpha \neq \beta .\) Ya que el orden de \(\Sigma ^{\ast }\) es total podemos suponer sin perdida de generalidad que \(\alpha < \beta .\) Por el lema anterior tenemos que \(s^{< }(\alpha )\leq \beta < s^{< }(\beta )\) y ya que \(< \) es transitiva obtenemos que \(s^{< }(\alpha )< s^{< }(\beta )\), lo cual nos dice \(s^{< }(\alpha )\neq s^{< }(\beta )\). \(\Box\)

Lema 5 Se tiene que
(1) \(\varepsilon \neq s^{< }(\alpha )\), para cada \(\alpha \in \Sigma ^{\ast }\)
(2) Si \(\alpha \neq \varepsilon \), entonces \(\alpha =s^{< }(\beta )\) para algun \(\beta \)
(3) Si \(S\subseteq \Sigma ^{\ast }\) es no vacio, entonces existe \( \alpha \in S\) tal que \(\alpha < \beta \), para cada \(\beta \in S-\{\alpha \}\).
Prueba: (1) y (2) son dejadas al lector. Probaremos (3). Sea \(k=\min \{\left\vert \alpha \right\vert :\alpha \in S\}\). Notese que hay una cantidad finita de palabras de \(S\) con longitud igual a \(k\) y que la menor de ellas es justamente la menor palabra de \(S\). \(\Box\)

Definamos recursivamente la funcion \(\ast ^{< }:\omega \rightarrow \Sigma ^{\ast }\) de la siguiente manera

\(\displaystyle \begin{array}{rcl} \ast ^{< }(0) & =& \varepsilon \\ \ast ^{< }(x+1) & =& s^{< }(\ast ^{< }(x)) \end{array} \)

Lema 6 Tenemos que
\(\displaystyle \Sigma ^{\ast }=\{\ast ^{< }(0),\ast ^{< }(1),...\} \)
Mas aun la funcion \(\ast ^{< }\) es biyectiva.
Prueba: Supongamos \(\ast ^{< }(x)=\ast ^{< }(y)\) con \(x >y\). Note que \(y\neq 0\) ya que \( \varepsilon \) no es el sucesor de ninguna palabra. O sea que \(s^{< }(\ast ^{< }(x-1))=s^{< }(\ast ^{< }(y-1))\) lo cual ya que \(\ast ^{< }\) es inyectiva nos dice que \(\ast ^{< }(x-1)=\ast ^{< }(y-1)\). Iterando este razonamiento llegamos a que \(\ast ^{< }(z)=\ast ^{< }(0)=\varepsilon \) para algun \(z >0\), lo cual es absurdo.

Veamos que \(\ast ^{< }\) es sobre. Supongamos no lo es, es decir supongamos que \(\Sigma ^{\ast }-I_{\ast ^{< }}\neq \varnothing \). Por (3) del lema anterior \(\Sigma ^{\ast }-I_{\ast ^{< }}\) tiene un menor elemento \(\alpha \). Ya que \(\alpha \neq \varepsilon \), tenemos que \(\alpha =s^{< }(\beta )\), para algun \(\beta \). Ya que \(\beta < \alpha \) tenemos que \(\beta \notin \Sigma ^{\ast }-I_{\ast ^{< }}\), es decir que \(\beta =\ast ^{< }(x)\), para algun \( x\in \omega \). Esto nos dice que \(\alpha =s^{< }(\ast ^{< }(x))\), lo cual por la definicion de \(\ast ^{< }\) nos dice que \(\alpha =\ast ^{< }(x+1)\). Pero esto es absurdo ya que \(\alpha \notin I_{\ast ^{< }}\). \(\Box\)

Lema 7 Sea \(n\geq 1\) fijo. Entonces cada \(x\geq 1\) se escribe en forma unica de la siguiente manera:
\(\displaystyle x=i_{k}n^{k}+i_{k-1}n^{k-1}+...+i_{0}n^{0}, \)
con \(k\geq 0\) y \(1\leq i_{k},i_{k-1},...,i_{0}\leq n\).
Prueba: Primero la unicidad. Supongamos que

\(\displaystyle i_{k}n^{k}+i_{k-1}n^{k-1}+...+i_{0}n^{0}=j_{m}n^{m}+j_{m-1}n^{m-1}+...+j_{0}n^{0} \)

con \(k,m\geq 0\) y \(1\leq i_{k},i_{k-1},...,i_{0},j_{m},...,j_{0}\leq n\). Supongamos \(k< m\). Llegaremos a un absurdo. Notese que
\(\displaystyle \begin{array}{ccl} i_{k}n^{k}+i_{k-1}n^{k-1}+...+i_{0}n^{0} & \leq & n.n^{k}+n.n^{k-1}+...+n.n^{0} \\ & \leq & n^{k+1}+n^{k}+...+n^{1} \\ & < & n^{k+1}+n^{k}+...+n^{1}+n^{0} \\ & \leq & n^{m}+n^{m-1}+...+n^{0} \\ & \leq & j_{m}n^{m}+j_{m-1}n^{m-1}+...+j_{0}n^{0} \end{array} \)

lo cual contradice la primera igualdad.
Probaremos por induccion en \(x\) que

(1) Existen \(k\geq 0\) y \(i_{k},i_{k-1},...,i_{0}\in \{1,...,n\}\) tales que
\(\displaystyle x=i_{k}n^{k}+i_{k-1}n^{k-1}+...+i_{0}n^{0} \)

El caso \(x=1\) es trivial. Supongamos (1) vale para \(x\), probaremos que vale para \(x+1\). Hay varios casos:

Caso \(i_{0}< n\). Entonces

\(\displaystyle \begin{array}{ll} x+1 & =\left( i_{k}n^{k}+i_{k-1}n^{k-1}+...+i_{0}n^{0}\right) +1 \\ & =i_{k}n^{k}+i_{k-1}n^{k-1}+...+(i_{0}+1)n^{0} \end{array} \)

Caso \(i_{k}=i_{k-1}=...=i_{0}=n\). Tenemos que

\(\displaystyle \begin{array}{ll} x+1 & =\left( i_{k}n^{k}+i_{k-1}n^{k-1}+...+i_{0}n^{0}\right) +1 \\ & =\left( nn^{k}+nn^{k-1}+...+nn^{0}\right) +1 \\ & =1n^{k+1}+1n^{k}+...+1n^{1}+1n^{0} \end{array} \)

Caso \(i_{0}=i_{1}=...=i_{h}=n\), \(\;i_{h+1}\not=n\), para algun \( 0\leq h< k\). Tenemos

\(\displaystyle \begin{array}{ll} x+1 & =\left( i_{k}n^{k}+...+i_{h+2}n^{h+2}+i_{h+1}n^{h+1}+nn^{h}+...+nn^{0}\right) +1 \\ & =\left( i_{k}n^{k}+...+i_{h+2}n^{h+2}+i_{h+1}n^{h+1}+n^{h+1}+n^{h}+...+n^{1}\right) +1 \\ & =i_{k}n^{k}+...+i_{h+2}n^{h+2}+(i_{h+1}+1)n^{h+1}+1n^{h}+...+1n^{1}+1n^{0}. \end{array} \)

\(\Box\)
Notese que cada \(\alpha \in \Sigma ^{\ast }-\{\varepsilon \}\) se escribe de la forma

\(\displaystyle \alpha =a_{i_{k}}...a_{i_{0}} \)

donde \(k\geq 0\) y \(1\leq i_{k},i_{k-1},...,i_{0}\leq n\). Definamos la funcion \(\#^{< }\) de la siguiente manera
\(\displaystyle \begin{array}{rll} \#^{< }:\Sigma ^{\ast } & \rightarrow & \omega \\ \varepsilon & \rightarrow & 0 \\ a_{i_{k}}...a_{i_{0}} & \rightarrow & i_{k}n^{k}+...+i_{0}n^{0} \end{array} \)

Notese que el lema anterior nos dice que fijado \(n\geq 1\), los numeros naturales estan identificados o en correspondencia biunivoca con las sucesiones finitas de elementos del conjunto \(\{1,...,n\}\). Ya que podemos identificar cada \(a_{i}\) con \(i\) el lema anterior nos garantiza que los numero naturales estan en correspondencia biunivoca con las palabras no nulas del alfabeto \(\Sigma \). Es decir que hemos probado que
Lema 8 La funcion \(\#^{< }\) es biyectiva
Concluimos la seccion con la siguiente observacion

Lema 9 Las funciones \(\#^{< }\) y \(\ast ^{< }\) son una inversa de la otra.
Prueba: Probaremos por induccion en \(x\) que para cada \(x\in \omega \), se tiene que \( \#^{< }(\ast ^{< }(x))=x\). El caso \(x=0\) es trivial. Supongamos que \( \#^{< }(\ast ^{< }(x))=x\), veremos entonces que \(\#^{< }(\ast ^{< }(x+1))=x+1\). Sean \(k\geq 0\) y \(i_{k},...,i_{0}\) tales que \(\ast ^{< }(x)=a_{i_{0}}...a_{i_{0}}\). Ya que \(\#^{< }(\ast ^{< }(x))=x\) tenemos que \( x=i_{k}n^{k}+...+i_{0}n^{0}\). Hay varios casos

Caso \(i_{0}< n\). Entonces \(\ast ^{< }(x+1)=s^{< }(\ast ^{< }(x))=a_{i_{k}}...a_{i_{0}+1}\) por lo cual

\(\displaystyle \begin{array}{ll} \#^{< }(\ast ^{< }(x+1)) & =i_{k}n^{k}+i_{k-1}n^{k-1}+...+(i_{0}+1)n^{0} \\ & =\left( i_{k}n^{k}+i_{k-1}n^{k-1}+...+i_{0}n^{0}\right) +1 \\ & =x+1 \end{array} \)

Caso \(i_{k}=i_{k-1}=...=i_{0}=n\). Entonces \(\ast ^{< }(x+1)=s^{< }(\ast ^{< }(x))=a_{1}^{k+2}\) por lo cual

\(\displaystyle \begin{array}{ll} \#^{< }(\ast ^{< }(x+1)) & =1n^{k+1}+1n^{k}+...+1n^{1}+1n^{0} \\ & =\left( nn^{k}+nn^{k-1}+...+nn^{0}\right) +1 \\ & =x+1 \end{array} \)

Caso \(i_{0}=i_{1}=...=i_{h}=n\), \(\;i_{h+1}\not=n\), para algun \( 0\leq h< k\). Entonces \(\ast ^{< }(x+1)=s^{< }(\ast ^{< }(x))=a_{i_{k}}...a_{i_{h+2}}a_{i_{h+1}+1}a_{1}...a_{1}\) por lo cual

\(\displaystyle \begin{array}{ll} \#^{< }(\ast ^{< }(x+1)) & =i_{k}n^{k}+...+i_{h+2}n^{h+2}+(i_{h+1}+1)n^{h+1}+1n^{h}+...+1n^{1}+1n^{0} \\ & =\left( i_{k}n^{k}+...+i_{h+2}n^{h+2}+i_{h+1}n^{h+1}+n^{h+1}+n^{h}+...+n^{1}\right) +1 \\ & =\left( i_{k}n^{k}+...+i_{h+2}n^{h+2}+i_{h+1}n^{h+1}+nn^{h}+...+nn^{0}\right) +1 \\ & =x+1 \end{array} \)

\(\Box\)
