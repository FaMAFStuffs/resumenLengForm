\section{El lenguaje ${S}^{\Sigma}$}

  % Lemma 52: Sin prueba.
  \begin{lemma}
    \PN Se tiene que:

    \begin{enumerate}[a)]
      \item Si $I_{1}, \dotsc, I_{n} = J_{1}, \dotsc, J_{m}$, con $I_{1}, \dotsc, I_{n}, J_{1}, \dotsc, J_{m} \in
        \mathrm{Ins}^{\Sigma}$, entonces $n=m$ y $I_{j}=J_{j}$ para cada $j \geq 1$.
      \item Si $\mathcal{P} \in \mathrm{Pro}^{\Sigma}$, entonces existe una única sucesión de instrucciones $I_{1},
        \dotsc, I_{n}$ tal que $\mathcal{P} = I_{1} \dotsc, I_{n}$
    \end{enumerate}
  \end{lemma}

  % Theorem 53: Con prueba.
  \begin{theorem}
    \PN Si $f$ es $\Sigma$-computable, entonces $f$ es $\Sigma$-efectivamente computable.
  \end{theorem}
  \begin{proof}
    \PN Sea $f: S \subseteq \omega^{n} \times \Sigma^{\ast m} \rightarrow O$ una función computada por $\mathcal{P} \in
    \mathrm{Pro}^{\Sigma}$. El procedimiento que consiste en realizar las sucesivas instrucciones de $\mathcal{P}$,
    partiendo de $((x_{1},\dotsc,x_{n},0,0,\dotsc), \linebreak (\alpha_{1},\dotsc,\alpha_{m},\varepsilon,\varepsilon,
    \dotsc))$, terminando eventualmente en caso de que nos toque realizar la instrucción $n(\mathcal{P})+1$, y dar como
    salida el contenido de la variable $\mathrm{N}1$, es un procedimiento efectivo que computa a $f$.
  \end{proof}

  % Proposition 54: Sin prueba.
  \begin{proposition}
    \begin{enumerate}[a)]
      \item Sea $f: S \subseteq \omega^{n} \times \Sigma^{\ast m} \rightarrow \omega$ una función $\Sigma$-computable,
        entonces existe un macro:
        \[
          \left[\mathrm{V}\overline{n+1}\leftarrow f(\mathrm{V}1,\dotsc,\mathrm{V}\bar{n},\mathrm{W}1,\dotsc,\mathrm{W}
          \bar{m})\right]
        \]

      \item Sea $f: S \subseteq \omega^{n} \times \Sigma^{\ast m} \rightarrow \SIGMA$ una función $\Sigma$-computable,
        entonces existe un macro:
        \[
          \left[\mathrm{W}\overline{m+1}\leftarrow f(\mathrm{V}1,\dotsc,\mathrm{V}\bar{n},\mathrm{W}1,\dotsc,\mathrm{W}
          \bar{m})\right]
        \]
    \end{enumerate}
  \end{proposition}

  % Proposition 55: Sin prueba.
  \begin{proposition}
    \PN Sea $P: S \subseteq \omega^{n} \times \Sigma^{\ast m} \rightarrow \omega$ un predicado $\Sigma$-computable,
    entonces existe un macro:
    \[
      \left[\mathrm{IF} \; P(\mathrm{V}1,\dotsc,\mathrm{V}\bar{n},\mathrm{W}1,\dotsc,\mathrm{W}\bar{m}) \; \mathrm{GOTO}
      \; \mathrm{A}1\right]
    \]
  \end{proposition}

  % Theorem 56: Con prueba. Solo el caso 2, con \Sigma = {@, $}.
  \begin{theorem}
    \PN Si $h$ es $\Sigma$-recursiva, entonces $h$ es $\Sigma$-computable.
  \end{theorem}
  \begin{proof}
    \PN Como $h$ es $\Sigma$-R $\Rightarrow \exists k$ tal que $h \in \mathrm{R}_{k}^{\Sigma}$, probaremos entonces por
    inducción en $k$ que si $h \in \mathrm{R}_{k}^{\Sigma}$, entonces $h$ es $\Sigma$-computable.

    \vspace{3mm}
    \PN \underline{Caso Base:} \begin{tabular}{|c|} \hline $k = 0$ \\\hline \end{tabular}

    \vspace{1mm}
    \PN Luego $h \in R_{0}^{\Sigma} = PR_{0}^{\Sigma}$, es decir:
    \[
      h \in \{Suc, Pred, C_{0}^{0,0}, C_{\varepsilon}^{0,0}\} \cup \{d_{a}: a \in \Sigma\} \cup \{p_{j}^{n,m} : 1 \leq j
      \leq n+m\}
    \]

    \PN Por lo tanto, $h$ es $\Sigma$-computable, dado que existen programas que computan dichas funciones.

    \vspace{3mm}
		\PN \underline{Caso Inductivo:} \begin{tabular}{|c|} \hline $k > 0$ \\\hline \end{tabular}

    \PN Supongamos ahora que si $h \in \mathrm{R}_{k}^{\Sigma} \Rightarrow h$ es $\Sigma$-computable, veamos que $h \in
    \mathrm{R}_{k+1}^{\Sigma} \Rightarrow h$ es $\Sigma$-computable.

    \PN Sea $h \in \mathrm{R}_{k+1}^{\Sigma} - \mathrm{R}_{k}^{\Sigma}$. Supongamos $\Sigma = \{@, \$\}$. Existen varios
    casos, probaremos el caso $h=R(f,\mathcal{G})$, con:
    \begin{eqnarray*}
      f &:& S_{1 }\times \dotsc \times S_{n} \times L_{1} \times \dotsc \times L_{m} \rightarrow \SIGMA \\
      \mathcal{G}_{@} &:& S_{1} \times \dotsc \times S_{n} \times L_{1} \times \dotsc \times L_{m} \times \SIGMA \times
        \SIGMA \rightarrow \SIGMA \\
      \mathcal{G}_{\$} &:& S_{1} \times \dotsc \times S_{n} \times L_{1} \times \dotsc \times L_{m} \times \SIGMA \times
        \SIGMA \rightarrow \SIGMA
    \end{eqnarray*}

    \PN elementos de $\mathrm{R}_{k}^{\Sigma}$. Por hipótesis inductiva, las funciones $f$, $\mathcal{G}_{@}$,
    $\mathcal{G}_{\$}$, son $\Sigma$-computables y por lo tanto podemos hacer el siguiente programa utilizando macros:

    \begin{eqnarray*}
      && \qquad\;\;       \left[\mathrm{P}\overline{m+3} \leftarrow f(\mathrm{N}1,\dotsc,\mathrm{N}\bar{n},\mathrm{P}1,
                          \dotsc,\mathrm{P}\bar{m})\right] \\
      && \mathrm{L}3:\;\; \mathrm{IF}\;\mathrm{P}\overline{m+1}\; \text{BEGINS}\;@\; \mathrm{GOTO}\;\mathrm{L}1 \\
      && \qquad\;\;       \mathrm{IF}\;\mathrm{P}\overline{m+1}\; \text{BEGINS}\;\$\; \mathrm{GOTO}\;\mathrm{L}2 \\
      && \qquad\;\;       \mathrm{GOTO}\; \mathrm{L}4\\
      && \mathrm{L}1:\;\; \mathrm{P}\overline{m+1} \leftarrow \text{ }^{\curvearrowright } \mathrm{P}\overline{m+1} \\
      && \qquad\;\;       \left[ \mathrm{P}\overline{m+3}\; \leftarrow\; \mathcal{G}_{@}
                                (\mathrm{N} 1,\dotsc,\mathrm{N}\bar{n},
                                \mathrm{P}1,\dotsc,\mathrm{P}\bar{m},\mathrm{P} \overline{m+2},\mathrm{P}\overline{m+3})
                          \right] \\
      && \qquad\;\;       \mathrm{P}\overline{m+2}\leftarrow \mathrm{P}\overline{m+2}.@  \\
      && \qquad\;\;       \mathrm{GOTO}\;\mathrm{L}3 \\
      && \mathrm{L}2:\;\; \mathrm{P}\overline{m+1} \leftarrow \text{ }^{\curvearrowright } \mathrm{P}\overline{m+1} \\
      && \qquad\;\;       \left[ \mathrm{P}\overline{m+3}\; \leftarrow\; \mathcal{G}_{\$}
                                (\mathrm{N} 1,\dotsc,\mathrm{N}\bar{n},
                                \mathrm{P}1,\dotsc,\mathrm{P}\bar{m},\mathrm{P} \overline{m+2},\mathrm{P}\overline{m+3})
                          \right] \\
      && \qquad\;\;       \mathrm{P}\overline{m+2}\leftarrow \mathrm{P}\overline{m+2}.\$ \\
      && \qquad\;\;       \mathrm{GOTO}\;\mathrm{L}3 \\
      && \mathrm{L}4:\;\; \mathrm{P}1\leftarrow \mathrm{P}\overline{m+3} \\
    \end{eqnarray*}

    \PN Luego se tiene que el programa computa a $h$, entonces $h$ es $\Sigma$-computable.
  \end{proof}

  % Lemma 57: Sin prueba.
  \begin{lemma}
    \PN Sea $\Sigma$ un alfabeto cualquiera. Las funciones $S$ y \---- son $(\Sigma \cup \Sigma_{p})$-PR.
  \end{lemma}

  % Lemma 58: Sin prueba.
  \begin{lemma}
    \PN Para cada $n, x \in \omega$, tenemos que $\lvert \bar{n} \rvert \leq x \Leftrightarrow n \leq 10^{x}-1$.
  \end{lemma}

  % Lemma 59: Con prueba.
  \begin{lemma}
    \PN $\mathrm{Ins}^{\Sigma}$ es un conjunto $(\Sigma \cup \Sigma_{p})$-PR.
  \end{lemma}
  \begin{proof}
    \PN Para simplificar la prueba asumiremos que $\Sigma =\{@,\&\}$. Dado que $\mathrm{Ins}^{\Sigma}$ es unión de los
    siguientes conjuntos:
    \begin{eqnarray*}
      L_{1} &=& \left\{\mathrm{N}\bar{k}\leftarrow \mathrm{N}\bar{k}+1:k\in \mathbb{N}\right\} \\
      L_{2} &=& \left\{\mathrm{N}\bar{k}\leftarrow \mathrm{N}\bar{k}\dot{-}1:k\in \mathbb{N}\right\} \\
      L_{3} &=& \left\{\mathrm{N}\bar{k}\leftarrow \mathrm{N}\bar{n}:k,n\in \mathbb{N}\right\} \\
      L_{4} &=& \left\{\mathrm{N}\bar{k}\leftarrow 0:k\in \mathbb{N}\right\} \\
      L_{5} &=& \left\{\mathrm{IF}\;\mathrm{N}\bar{k}\neq 0\;\mathrm{GOTO}\;
                       \mathrm{L}\bar{m}:k,m\in \mathbb{N}\right\} \\
      L_{6} &=& \left\{\mathrm{P}\bar{k}\leftarrow \mathrm{P}\bar{k}.@:k\in \mathbb{N}\right\} \\
      L_{7} &=& \left\{\mathrm{P}\bar{k}\leftarrow \mathrm{P}\bar{k}.\& :k\in \mathbb{N}\right\} \\
      L_{8} &=& \left\{\mathrm{P}\bar{k}\leftarrow \text{ }^{\curvearrowright }
                       \mathrm{P}\bar{k}:k\in \mathbb{N}\right\} \\
      L_{9} &=& \left\{\mathrm{P}\bar{k}\leftarrow \mathrm{P}\bar{n}:k,n\in \mathbb{N}\right\} \\
      L_{10} &=& \left\{\mathrm{P}\bar{k}\leftarrow \varepsilon :k\in \mathbb{N} \right\} \\
      L_{11} &=& \left\{\mathrm{IF}\;\mathrm{P}\bar{k}\;\mathrm{BEGINS}\;@\; \mathrm{GOTO}\;
                        \mathrm{L}\bar{m}:k,m\in \mathbb{N}\right\} \\
      L_{12} &=& \left\{\mathrm{IF}\;\mathrm{P}\bar{k}\;\mathrm{BEGINS}\;\& \; \mathrm{GOTO}\;
                        \mathrm{L}\bar{m}:k,m\in \mathbb{N}\right\} \\
      L_{13} &=& \left\{\mathrm{GOTO}\;\mathrm{L}\bar{m}:m\in \mathbb{N}\right\} \\
      L_{14} &=& \left\{\mathrm{SKIP}\right\} \\
      L_{15} &=& \left\{\mathrm{L}\bar{k}\alpha :k\in \mathbf{N\;}
                        \text{y }\alpha \in L_{1}\cup \dotsc\cup L_{14}\right\}
    \end{eqnarray*}

    \PN solo debemos probar que $L_{1}, \dotsc, L_{15}$ son $(\Sigma \cup \Sigma_{p})$-PR. Puesto que las pruebas son
    análogas probaremos solamente que:
    \[
      L_{11} = \left\{\mathrm{IF}\;\mathrm{P}\bar{k}\;\mathrm{BEGINS}\;@\;\mathrm{GOTO}\;\mathrm{L}\bar{m}: k,m \in
      \mathbb{N}\right\}
    \]

    \PN es $(\Sigma \cup \Sigma_{p})$-PR. Primero nótese que $\alpha \in L_{11} \Leftrightarrow \exists k, m \in
    \mathbb{N}$ tales que:
    \[
      \alpha = \mathrm{IFP}\bar{k}\mathrm{BEGINS}@\mathrm{GOTOL}\bar{m}
    \]

    \PN Más formalmente tenemos que $\alpha \in L_{11}$ si y solo si:
    \[
      (\exists k \in \mathbb{N}), (\exists m \in \mathbb{N}): \alpha = \mathrm{IFP}\bar{k}\mathrm{BEGINS}@
      \mathrm{GOTOL}\bar{m}
    \]

    \PN Ya que cuando existen tales $k, m$ tenemos que $\bar{k}$ y $\bar{m}$ son subpalabras de $\alpha$, el
    \textbf{Lemma 58} nos dice que $\alpha \in L_{11}$ si y solo si:
    \[
      (\exists k \in \mathbb{N})_{k\leq 10^{\lvert \alpha \rvert}}(\exists m \in \mathbb{N})_{m \leq
      10^{\lvert \alpha \rvert}} \; \alpha = \mathrm{IFP}\bar{k}\mathrm{BEGINS}@\mathrm{GOTOL}\bar{m}
    \]

    \PN Sea:
    \[
      P = \lambda mk\alpha \left[\alpha =\mathrm{IFP}\bar{k}\mathrm{BEGINS}@\mathrm{GOTOL}\bar{m}\right]
    \]

    \PN Ya que $D_{\lambda k \left[\bar{k}\right]} = \omega$, tenemos que $D_{P} = \omega \times \omega \times
    (\Sigma \cup \Sigma_{p})^{\ast}$. Notar que:
    \[
      P = \lambda \alpha\beta \left[\alpha =\beta\right] \circ \left(p_{3}^{2,1},f\right)
    \]

    \PN donde:
    \[
      f = \lambda \alpha_{1}\alpha_{2}\alpha_{3}\alpha_{4}\left[\alpha_{1}\alpha_{2}\alpha_{3}\alpha_{4}\right] \circ
      \left(C_{\mathrm{IFP}}^{2,1},\lambda k \left[\bar{k}\right] \circ p_{2}^{2,1},C_{\mathrm{BEGINS}@
      \mathrm{GOTOL}}^{2,1},\lambda k \left[\bar{k}\right] \circ p_{1}^{2,1}\right)
    \]

    \PN lo cual nos dice que $P$ es $(\Sigma \cup \Sigma_{p})$-PR.
    \PN Nótese que:
    \[
      \chi_{L_{11}} = \lambda \alpha \left[(\exists k \in \mathbb{N})_{k\leq 10^{\lvert \alpha \rvert}}(\exists
      m \in \mathbb{N})_{m\leq 10^{\lvert \alpha \rvert}} \; P(m,k,\alpha)\right]
    \]

    \PN Utilizando dos veces el \textbf{Lemma 39} veremos que $\chi_{L_{11}}$ es $(\Sigma \cup \Sigma_{p})$-PR. Sea:
    \[
      Q = \lambda k\alpha \left[(\exists m \in \mathbb{N})_{m\leq 10^{\lvert \alpha \rvert}}\; P(m,k,\alpha)
      \right]
    \]

    \PN Por el \textbf{Lemma 39} tenemos que:
    \[
      \lambda xk\alpha \left[(\exists m \in \mathbb{N})_{m\leq x} \; P(m,k,\alpha) \right]
    \]

    \PN es $(\Sigma \cup \Sigma_{p})$-PR, lo cual nos dice que:
    \[
      Q = \lambda xk\alpha \left[(\exists m \in \mathbb{N})_{m\leq x} \; P(m,k,\alpha) \right] \circ (\lambda \alpha
      \left[10^{\lvert \alpha \rvert}\right] \circ p_{2}^{1,1},p_{1}^{1,1},p_{2}^{1,1})
    \]

    \PN lo es. Ya que:
    \[
      \chi_{L_{11}} = \lambda \alpha \left[(\exists k \in \mathbb{N})_{k \leq 10^{\lvert \alpha \rvert}} \; Q(k,\alpha)
      \right]
    \]

    \PN nuevamente, aplicando el \textbf{Lemma 39} obtenemos que $\chi_{L_{11}}$ es $(\Sigma \cup \Sigma_{p})$-PR.

    \PN En forma similar podemos probar que $L_{1}, \dotsc, L_{14}$ son $(\Sigma \cup \Sigma_{p})$-PR. Por lo tanto,
    $L_{1} \cup \dotsc \cup L_{14}$ es $(\Sigma \cup \Sigma_{p})$-PR. Nótese que $L_{1} \cup \dotsc \cup L_{14}$ es el
    conjunto de las instrucciones básicas de $\mathcal{S}^{\Sigma}$. Llamemos $\mathrm{InsBas}^{\Sigma}$ a dicho
    conjunto, resta ver que $L_{15}$ es $(\Sigma \cup \Sigma_{p})$-PR. Notemos que:
    \[
      \chi_{L_{15}}=\lambda \alpha \left[ (\exists k\in \mathbb{N})_{k\leq 10^{\lvert \alpha \rvert}}(\exists \beta \in
      \mathrm{InsBas}^{\Sigma})_{\lvert \beta \rvert \leq \lvert \alpha \rvert} \ \alpha = \mathrm{L}\bar{k}\beta\right]
    \]

    \PN lo cual nos dice que aplicando dos veces el \textbf{Lemma 39} obtenemos que $\chi_{L_{15}}$ es $(\Sigma \cup
    \Sigma_{p})$-PR. Ya que $\mathrm{Ins}^{\Sigma} = \mathrm{InsBas}^{\Sigma} \cup L_{15}$ tenemos que
    $\mathrm{Ins}^{\Sigma}$ es $(\Sigma \cup \Sigma_{p})$-PR.
  \end{proof}

  % Lemma 60: Con prueba.
  \begin{lemma}
    \PN $Bas$ y $Lab$ son funciones $(\Sigma \cup \Sigma_{p})$-PR.
  \end{lemma}
  \begin{proof}
    \PN Sean:
    \begin{itemize}
      \item $<$ un orden total estricto sobre $\Sigma \cup \Sigma_{p}$
      \item $L = \{ \mathrm{L}\bar{k}:k\in \mathbb{N}\} \cup \{\varepsilon \}$
    \end{itemize}

    \PN Veamos que $L$ es $\Sigma \cup \Sigma_{p}$-PR. Notar que:
    \[
      \alpha \in L \Leftrightarrow \alpha = \varepsilon \vee (\exists k \in \mathbb{N}) \; \alpha = \mathrm{L}\bar{k}
    \]

    \PN Dado que cuando existe $k$ tenemos que $\bar{k}$ es una subpalabra de $\alpha $, utilizando el \textbf{Lemma 58}
    tenemos que:
    \[
      \alpha \in L \Leftrightarrow \alpha = \varepsilon \vee (\exists k \in \mathbb{N})_{k \leq 10^{\lvert \alpha
      \rvert}} \; \alpha = \mathrm{L}\bar{k}
    \]

    \PN Sean:
    \begin{eqnarray*}
      Q &=& \lambda k\alpha \left[\alpha = \mathrm{L}\bar{k}\right] \\
      &=& \lambda \alpha\beta \left[\alpha = \beta\right] \circ \left(p_{2}^{1,1}, \lambda \alpha\beta \left[\alpha\beta
        \right] \circ \left(C_{L}^{1,1},\lambda k \left[\bar{k}\right] \circ p_{1}^{1,1}\right)\right) \\
      &\Rightarrow& Q \text{ es } (\Sigma \cup \Sigma_{p})-PR \\
      \\
      R &=& \lambda \alpha \left[(\exists k \in \mathbb{N})_{k \leq 10^{\lvert \alpha \rvert}} \; Q(k,\alpha)\right] \\
      &=& \lambda x\alpha \left[(\exists k \in \mathbb{N})_{k \leq x} \; Q(k,\alpha)\right] \circ \left(\lambda \alpha
        \left[10^{\lvert \alpha \rvert}\right] \circ p_{1}^{0,1}, p_{1}^{0,1}\right) \\
      &\Rightarrow& R \text{ es } (\Sigma \cup \Sigma_{p})-PR
    \end{eqnarray*}

    \PN Luego:
    \[
      \chi_{L} = \lambda \alpha \left[\alpha = \varepsilon\right] \vee R
    \]

    \PN es $(\Sigma \cup \Sigma_{p})$-PR y por lo tanto L lo es. Sea:
    \[
      P = \lambda I\alpha \left[\alpha \in \mathrm{Ins}^{\Sigma} \wedge I \in \mathrm{Ins}^{\Sigma} \wedge \lbrack
      \alpha]_{1} \neq \mathrm{L} \wedge (\exists \beta \in L)\ I = \beta\alpha\right]
    \]

    \PN Veamos que $P$ es $(\Sigma \cup \Sigma_{p})$-PR. Note que $D_{P} = (\Sigma \cup \Sigma_{p})^{\ast 2}$ y además:
    \[
      P = P_1 \wedge P_2 \wedge P_3 \wedge P_4
    \]

    \PN donde:
    \begin{eqnarray*}
      P_1 &=& \lambda I\alpha \left[\alpha \in \mathrm{Ins}^{\Sigma}\right] \\
      P_2 &=& \lambda I\alpha \left[I \in \mathrm{Ins}^{\Sigma}\right] \\
      P_3 &=& \lambda I\alpha \left[\lbrack \alpha]_{1} \neq \mathrm{L}\right] \\
      P_4 &=& \lambda I\alpha \left[ (\exists \beta \in L)\ I = \beta\alpha \right]
    \end{eqnarray*}

    \PN Es fácil ver que $P_1, P_2$ y $P_3$ son $(\Sigma \cup \Sigma_{p})$-PR. Veamos que $P_4$ es $(\Sigma \cup
    \Sigma_{p})$-PR, para ello definamos el siguiente predicado:
    \[
      T = \lambda I\alpha\beta \left[ I = \beta\alpha\right]
    \]

    \PN Tenemos que:
    \[
      P_4 = \lambda I\alpha \left[(\exists \beta \in L) \; T(I,\alpha,\beta)\right]
    \]

    \PN Notar que, como $\beta$ es una subpalabra de $I$, tenemos que $|\beta| \leq |I|$
    \[
      P_4 = \lambda I\alpha \left[ (\exists \beta \in L)_{|\beta|\leq |I|} R(I, \alpha, \beta) \right]
    \]

    \PN Utilizando el \textbf{Lemma 39} tenemos que:
    \[
      \lambda xI\alpha \left[(\exists \beta \in L)_{\lvert \beta \rvert \leq x} \; R(I,\alpha,\beta)\right]
    \]

    \PN es $(\Sigma \cup \Sigma_{p})$-PR. Lo cual nos dice que:
    \[
      P_4 = \lambda xI\alpha \left[(\exists \beta \in L)_{\lvert \beta \rvert \leq x} \; R(I,\alpha,\beta)\right] \circ
      (\lambda \alpha \left[\lvert \alpha \rvert\right] \circ p_{1}^{0,2},p_{1}^{0,2},p_{2}^{0,2})
    \]

    \PN lo es. Por lo tanto $P$ es $(\Sigma \cup \Sigma_{p})$-PR.

    \vspace{3mm}
    \PN Nótese que cuando $I \in \mathrm{Ins}^{\Sigma}$ tenemos que $P(I,\alpha)=1 \Leftrightarrow \alpha=Bas(I)$.
    Además, notar que $Bas = M^{<}\left(P\right)$, por lo que para ver que $Bas$ es $(\Sigma \cup \Sigma_{p})$-PR, solo
    resta encontrar un función cota $(\Sigma \cup \Sigma_{p})$-PR y $(\Sigma \cup \Sigma_{p})$-total para $Bas$. Notar
    que $\lvert Bas(I)\rvert \leq \lvert I \rvert = p_{1}^{0,1}(I)$, para cada $I \in \mathrm{Ins}^{\Sigma}$. Luego,
    $Bas$ $(\Sigma \cup \Sigma_{p})$-PR.

    \vspace{3mm}
    \PN Finalmente note que:
    \[
      Lab = M^{<} \left(\lambda I\alpha \left[\alpha Bas(I)=I\right]\right)
    \]

    \PN lo cual nos dice que $Lab$ es $(\Sigma \cup \Sigma_{p})$-PR.
  \end{proof}

  % Lemma 61: Sin prueba.
  \begin{lemma}
    \begin{enumerate}[a)]
      \item $\mathrm{Pro}^{\Sigma}$ es un conjunto $(\Sigma \cup \Sigma_{p})$-PR.
      \item $\lambda \mathcal{P} \left[n(\mathcal{P})\right]$ y $\lambda i\mathcal{P} \left[I_{i}^{\mathcal{P}}\right]$
        son funciones $(\Sigma \cup \Sigma_{p})$-PR.
    \end{enumerate}
  \end{lemma}

  % Lemma 62: Nada.
  \begin{lemma}
    \PN Este lemma no se evalua.
  \end{lemma}

  % Lemma 63: Nada.
  \begin{lemma}
    \PN Este lemma no se evalua.
  \end{lemma}

  % Proposition 64: Sin prueba.
  \begin{proposition}
    \PN Sean $n, m \geq 0$, las funciones $i^{n,m}, E_{\#j}^{n,m}, E_{\star j}^{n,m}$ con $j = 1,2,\dotsc$, son $\Sigma
    \cup \Sigma_{p}$-PR.
  \end{proposition}

  % Theorem 65: Con prueba.
  \begin{theorem}
    \PN Las funciones $\Phi_{\#}^{n,m}$ y $\Phi_{\ast}^{n,m}$ son $(\Sigma \cup \Sigma_{p})$-recursivas.
  \end{theorem}
  \begin{proof}
    \PN Sea $P$ el siguiente predicado $(\Sigma \cup \Sigma_{p})$-mixto:
    \[
      \lambda t\vec{x}\vec{\alpha}\mathcal{P}\left[i^{n,m}(t,x_{1},\dotsc,x_{n},\alpha_{1},\dotsc,\alpha_{m},
      \mathcal{P}) = n(\mathcal{P}) + 1\right]
    \]

    \PN Note que $D_{P} = \omega^{n+1} \times \Sigma^{\ast m} \times \mathrm{Pro}^{\Sigma}$. Ya que las funciones
    $i^{n,m}$ y $\lambda \mathcal{P} \left[n(\mathcal{P})\right]$ son $(\Sigma \cup \Sigma_{p})$-PR, $P$ lo es.

    \PN Notar que $D_{M(P)} = D_{\Phi_{\#}^{n,m}} = D_{\Phi_{\ast}^{n,m}}$. Además para $(\vec{x},\vec{\alpha},
    \mathcal{P}) \in D_{M(P)}$, tenemos que $M(P)(\vec{x},\vec{\alpha},\mathcal{P})$ es la menor cantidad de pasos
    necesarios para que $\mathcal{P}$ termine partiendo del estado $((x_{1},\dotsc,x_{n},0, \linebreak 0,\dotsc)$,
    $(\alpha_{1},\dotsc,\alpha_{m},\varepsilon,\varepsilon,\dotsc))$. Ya que $P$ es $(\Sigma \cup \Sigma_{p})$-PR,
    tenemos que $M(P)$ es $(\Sigma \cup \Sigma_{p})$-recursiva. Nótese que para $(\vec{x},\vec{\alpha},\mathcal{P})
    \in D_{M(P)} = D_{\Phi_{\#}^{n,m}} = D_{\Phi_{\ast}^{n,m}}$ tenemos que:
    \begin{eqnarray*}
      \Phi_{\#}^{n,m}(\vec{x},\vec{\alpha},\mathcal{P}) &=& E_{\#1}^{n,m}\left(M(P)(\vec{x},\vec{\alpha},\mathcal{P}),
        \vec{x},\vec{\alpha},\mathcal{P}\right) \\
      \Phi_{\ast}^{n,m}(\vec{x},\vec{\alpha},\mathcal{P}) &=& E_{\ast1}^{n,m}\left(M(P)(\vec{x},\vec{\alpha},
        \mathcal{P}),\vec{x},\vec{\alpha},\mathcal{P}\right)
    \end{eqnarray*}

    \PN es decir,
    \begin{eqnarray*}
      \Phi_{\#}^{n,m} &=& E_{\#1}^{n,m} \circ \left(M(P),p_{1}^{n,m+1},\dotsc,p_{n+m+1}^{n,m+1}\right) \\
      \Phi_{\ast}^{n,m} &=& E_{\ast1}^{n,m} \circ \left(M(P),p_{1}^{n,m+1},\dotsc,p_{n+m+1}^{n,m+1}\right)
    \end{eqnarray*}

    \PN Ya que las funciones $E_{\#1}^{n,m}, E_{\ast1}^{n,m}$ son $(\Sigma \cup \Sigma_{p})$-recursiva, lo son
    $\Phi_{\#}^{n,m}$ y $\Phi_{\ast}^{n,m}$.
  \end{proof}

  \pagebreak
  % Corollary 66: Con prueba.
  \begin{corollary}
    \PN Si $f: D_{f} \subseteq \omega^{n} \times \Sigma^{\ast m} \rightarrow O$ es $\Sigma$-computable, entonces $f$ es
    $\Sigma$-recursiva.
  \end{corollary}
  \begin{proof}
    \PN Sea $\mathcal{P}_{0}$ un programa que compute a $f$. Primero veremos que $f$ es $(\Sigma \cup
    \Sigma_{p})$-recursiva. Note que:
    \begin{eqnarray*}
      O = \omega &\Rightarrow& f = \Phi_{\#}^{n,m} \circ \left(p_{1}^{n,m},\dotsc,p_{n+m}^{n,m},
        C_{\mathcal{P}_{0}}^{n,m}\right) \\
      O = \SIGMA &\Rightarrow& f = \Phi_{\ast}^{n,m} \circ \left(p_{1}^{n,m},\dotsc,p_{n+m}^{n,m},
        C_{\mathcal{P}_{0}}^{n,m}\right)
    \end{eqnarray*}

    \PN donde cabe destacar que $p_{1}^{n,m},\dotsc,p_{n+m}^{n,m}$ son las proyecciones respecto del alfabeto $\Sigma
    \cup \Sigma_{p}$, es decir que tienen dominio $\omega^{n} \times (\Sigma \cup \Sigma_{p})^{\ast m}$. Dado que,
    $\Phi_{\#}^{n,m}, \Phi_{\ast}^{n,m}$ son $(\Sigma \cup \Sigma_{p})$-recursivas tenemos que $f$ es $(\Sigma \cup
    \Sigma_{p})$-recursiva. Luego, utilizando el \textbf{Theorem 51} tenemos:
    \begin{equation*}
		  	\left.
		  	\begin{array}{l}
          f \text{ es } \Sigma\text{-mixta} \\
          f \text{ es } (\Sigma \cup \Sigma_{p})\text{-mixta} \\
          f \text{ es } (\Sigma \cup \Sigma_{p})\text{-R}
		  	\end{array}
		  	\right\rbrace
        \Rightarrow f \text{ es } \Sigma\text{-R}
		\end{equation*}
  \end{proof}

  % Tésis de Church.
  \PN \textbf{\underline{Tésis de Church:}} Toda función $\Sigma$-efectivamente computable es $\Sigma$-recursiva.

  % Corollary 67: Nada.
  \begin{corollary}
    \PN Este corolario no se evalua.
  \end{corollary}

  % Lemma 68: Con prueba.
  \begin{lemma}
    \PN Supongamos $f_{i}: D_{f_{i}} \subseteq \omega^{n}\times \Sigma^{\ast m} \rightarrow O, i=1, \dotsc, k$, son
    funciones $\Sigma$-recursivas tales que $D_{f_{i}} \cap D_{f_{j}} = \emptyset$ para $i \neq j$, entonces la función
    $f_{1} \cup \dotsc \cup f_{k}$ es $\Sigma$-recursiva.
  \end{lemma}
  \begin{proof}
    \PN Probaremos solo el caso en que $k = 2$. Sean $\mathcal{P}_{1}$ y $ \mathcal{P}_{2}$ programas que computen las
    funciones $f_{1}$ y $f_{2}$, respectivamente. Sean:
    \begin{eqnarray*}
      P_{1} &=& \lambda t\vec{x}\vec{\alpha}\left[i^{n,m}(t,\vec{x},\vec{\alpha},\mathcal{P}_{1}) = n(\mathcal{P}_{1}) +
        1\right] \\
      P_{2} &=& \lambda t\vec{x}\vec{\alpha}\left[i^{n,m}(t,\vec{x},\vec{\alpha},\mathcal{P}_{2}) = n(\mathcal{P}_{2}) +
        1\right]
    \end{eqnarray*}

    \PN Notese que $D_{P_{1}} = D_{P_{2}} = \omega \times \omega^{n} \times \Sigma^{\ast m}$ y que $P_{1}$ y $P_{2}$ son
    $(\Sigma \cup \Sigma_{p})$-PR. Además, utilizando el \textbf{Theorem 51} tenemos que son $\Sigma$-PR. También notar
    que $D_{M(P_{1} \vee P_{2})} = D_{f_{1}} \cup D_{f_{2}}$. Definimos:

    \PN \begin{tabular}{|c|} \hline $O = \omega$ \\\hline \end{tabular}
    \begin{eqnarray*}
      g_{1} & =& \lambda \vec{x}\vec{\alpha}\left[E_{\# 1}^{n,m} (M \left(P_{1} \vee P_{2}\right)(\vec{x},
        \vec{\alpha}),\vec{x},\vec{\alpha},\mathcal{P}_{1})^{P_{1} (M (P_{1} \vee P_{2})(\vec{x},\vec{\alpha}),\vec{x},
        \vec{\alpha})}\right] \\
      g_{2} & =& \lambda \vec{x}\vec{\alpha}\left[E_{\# 1}^{n,m} (M \left(P_{1} \vee P_{2}\right)(\vec{x},
        \vec{\alpha}),\vec{x},\vec{\alpha},\mathcal{P}_{2})^{P_{2} (M (P_{1} \vee P_{2})(\vec{x},\vec{\alpha}),\vec{x},
        \vec{\alpha})}\right]
    \end{eqnarray*}

    \PN \begin{tabular}{|c|} \hline $O = \SIGMA$ \\\hline \end{tabular}
    \begin{eqnarray*}
      h_{1} & =& \lambda \vec{x}\vec{\alpha}\left[E_{\ast 1}^{n,m} (M \left(P_{1} \vee P_{2}\right)(\vec{x},
        \vec{\alpha}),\vec{x},\vec{\alpha},\mathcal{P}_{1})^{P_{1} (M (P_{1} \vee P_{2})(\vec{x},\vec{\alpha}),\vec{x},
        \vec{\alpha})}\right] \\
      h_{2} & =& \lambda \vec{x}\vec{\alpha}\left[E_{\ast 1}^{n,m} (M \left(P_{1} \vee P_{2}\right)(\vec{x},
        \vec{\alpha}),\vec{x},\vec{\alpha},\mathcal{P}_{2})^{P_{2} (M (P_{1} \vee P_{2})(\vec{x},\vec{\alpha}),\vec{x},
        \vec{\alpha})}\right]
    \end{eqnarray*}

    \PN Nótese que $g_{1}, g_{2}, h_{1}$ y $h_{2}$ son $\Sigma$-recursivas y que $D_{g_{1}} = D_{g_{2}} = D_{h_{1}} =
    D_{h_{2}} = D_{f_{1}} \cup D_{f_{2}}$. Además nótese que:
    \[
      g_{1}(\vec{x},\vec{\alpha}) = \left\{
        \begin{array}{lll}
          f_{1}(\vec{x},\vec{\alpha}) & & \text{si } (\vec{x},\vec{\alpha})\in D_{f_{1}} \\
          1 & & \text{caso contrario}
        \end{array} \right.
      \\ \qquad
      g_{2}(\vec{x},\vec{\alpha}) = \left\{
        \begin{array}{lll}
          f_{2}(\vec{x},\vec{\alpha}) & & \text{si }(\vec{x},\vec{\alpha})\in D_{f_{2}} \\
          1 & & \text{caso contrario}
        \end{array} \right.
    \]

    \[
      h_{1}(\vec{x},\vec{\alpha}) = \left\{
        \begin{array}{lll}
          f_{1}(\vec{x},\vec{\alpha}) & & \text{si } (\vec{x},\vec{\alpha})\in D_{f_{1}} \\
          \varepsilon & & \text{caso contrario}
        \end{array} \right.
      \\ \qquad
      h_{2}(\vec{x},\vec{\alpha}) = \left\{
        \begin{array}{lll}
          f_{2}(\vec{x},\vec{\alpha}) & & \text{si }(\vec{x},\vec{\alpha})\in D_{f_{2}} \\
          \varepsilon & & \text{caso contrario}
        \end{array} \right.
    \]

    \PN Osea, que:
    \[
      f_{1} \cup f_{2} = \left\{
        \begin{array}{lll}
          \lambda xy \left[x.y\right] \circ (g_{1},g_{2}) \qquad\ \text{Si } O = \omega \\
          \lambda \alpha \beta \left[ \alpha \beta \right] \circ (h_{1},h_{2}) \qquad \text{Si } O = \SIGMA
        \end{array} \right.
    \]

    \PN es $\Sigma$-recursiva.
  \end{proof}

  % Lemma 69: Con prueba.
  \begin{lemma}
    \PN Supongamos $\Sigma_{p} \subseteq \Sigma$, entonces $Halt^{\Sigma}$ es no $\Sigma$-recursivo.
  \end{lemma}
  \begin{proof}
    \PN Supongamos $Halt^{\Sigma}$ es $\Sigma$-recursivo y por lo tanto $\Sigma$-computable. Recordemos que:

    \begin{center}
    \begin{tabular}{|c|} \hline $Halt(\mathcal{P}) = 1 \Leftrightarrow \mathcal{P}$ se detiene partiendo del estado
    $\left((0,0,\dotsc), (\mathcal{P},\varepsilon,\varepsilon,\dotsc)\right)$ \\\hline \end{tabular} $(\star)$
    \end{center}

    \PN Por la \textbf{Proposition 55} tenemos que existe un macro:
    \[
      \left[\mathrm{IF} \; Halt^{\Sigma}(\mathrm{W}1)\;\mathrm{GOTO}\;\mathrm{A}1\right]
    \]

    \PN Sea $\mathcal{P}_{0}$ el siguiente programa de $\mathcal{S}^{\Sigma}$
    \[
      \mathrm{L}1\;\left[\mathrm{IF}\;Halt^{\Sigma}(\mathrm{P}1)\;\mathrm{GOTO}\;\mathrm{L}1\right]
    \]

    \PN Note que:

    \begin{center}
      $\mathcal{P}_{0}$ termina partiendo desde $\left((0,0,\dotsc), (\mathcal{P}_{0},\varepsilon,\varepsilon,\dotsc)
      \right) \Leftrightarrow Halt^{\Sigma}(\mathcal{P}_{0}) = 0$
    \end{center}

    \PN lo cual produce una contradicción si tomamos en $(\star): \mathcal{P}= \mathcal{P}_{0}$.
  \end{proof}

  % Theorem 70: Con prueba.
  \begin{theorem}
    \PN Sea $S \subseteq \omega^{n} \times \Sigma^{\ast m}$, entonces $S$ es $\Sigma$-efectivamente enumerable
    $\Leftrightarrow S$ es $\Sigma$-recursivamente enumerable.
  \end{theorem}
  \begin{proof}
    \begin{tabular}{|c|} \hline $\Rightarrow$ \\\hline \end{tabular}
    \PN Sea $\mathbb{P}$ un procedimiento efectivo que enumera a $S$, llamemos $\vec{e}$  a la salida de dicho
    procedimiento.

    \PN Definimos:
    \begin{eqnarray*}
      F_{1} &:& \text{primera coordenada del vector } \vec{e}. \\
      F_{2} &:& \text{segunda coordenada del vector } \vec{e}. \\
      && \vdotswithin{(n-1)m+1} \qquad \vdotswithin{(n-1)m+2} \\
      F_{n+m} &:& (n+m)\text{-ésima coordenada del vector } \vec{e}. \\
      F &:& \omega \rightarrow \omega^{n} \times \Sigma^{\ast m} \\
        && x \rightarrow \left(F_{1}(x), F_{2}(x), \dotsc, F_{n+m}(x)\right)
    \end{eqnarray*}

    \PN Luego, dado que $S$ es $\Sigma$-efectivamente enumerable, por el \textbf{Theorem 16}, sabemos que $S$ es
    $\Sigma$-efectivamente computable y por ende $\chi_{S}$ es $\Sigma$-efectivamente computable. Por lo tanto, existen
    procedimientos efectivos que computan cada $F_{i}$, utilizando la \textbf{Tésis de Church}, cada $F_{i}$ es
    $\Sigma$-R. Finalmente $S$ es $\Sigma$-recursivamente enumerable, dado que es la imagen de la función $F$.

    \vspace{3mm}
    \PN \begin{tabular}{|c|} \hline $\Leftarrow$ \\\hline \end{tabular}
    \vspace{2mm}
    \PN S es $\Sigma$-recursivamente enumerable, por definición, $S = \emptyset$ ó $\exists \ F: \omega \rightarrow S$,
    tal que cada $F_{i}$ es $\Sigma$-R. El caso $S = \emptyset$ es trivial, supongamos que existe $F: \omega \rightarrow
    S$. Utilizando el \textbf{Theorem 42}, tenemos que cada $F_{i}$ es $\Sigma$-efectivamente computable, es decir,
    existen:
    \begin{itemize}
      \item $\mathbb{P}_{1}$ un procedimiento efectivo que compute a $F_{1}$.
      \item $\mathbb{P}_{2}$ un procedimiento efectivo que compute a $F_{2}$.
      \item $\vdotswithin{(n-1)m+1} \qquad \vdotswithin{(n-1)m+2} \qquad \vdotswithin{(n-1)m+r}$
      \item $\mathbb{P}_{n+m}$ un procedimiento efectivo que compute a $F_{n+m}$.
    \end{itemize}

    \PN El siguiente procedimiento efectivo enumera a $S$:

    \vspace{3mm}
    \PN \textbf{Etapa 1:}
    Realizar $\mathbb{P}_{1}$ con dato de entrada $x$ para obtener de salida $o_{1}$.

    $\qquad\;\;\;\;$Realizar $\mathbb{P}_{2}$ con dato de entrada $x$ para obtener de salida $o_{2}$.

    $\qquad\;\;\;\;\vdotswithin{(n-1)m+1} \qquad \vdotswithin{(n-1)m+2} \qquad \vdotswithin{(n-1)m+r}$

    $\qquad\;\;\;\;$Realizar $\mathbb{P}_{n+m}$ con dato de entrada $x$ para obtener de salida $o_{n+m}$.

    \PN \textbf{Etapa 2:}
    Dar como dato de salida $(o_{1}, o_{2}, \dotsc, o_{n+m})$.

    \vspace{3mm}
    \PN Por lo tanto $S$ es $\Sigma$-efectivamente enumerable.
  \end{proof}

  % Theorem 71: Sin prueba.
  \begin{theorem}
    \PN Dado $S \subseteq \omega^{n} \times \Sigma^{\ast m}$, son equivalentes:
    \begin{enumerate}
      \item $S$ es $\Sigma$-recursivamente enumerable.
      \item $S = I_{F}$, para alguna $F: D_{F} \subseteq \omega^{k} \times \Sigma^{\ast l} \rightarrow \omega^{n} \times
        \Sigma^{\ast m}$ tal que cada $F_{i}$ es $\Sigma$-recursiva.
      \item $S = D_{f}$, para alguna función $\Sigma$-recursiva $f$.
      \item $S = \emptyset $ ó $S = I_{F}$, para alguna $F: \omega \rightarrow \omega^{n} \times \Sigma^{\ast m}$ tal
        que cada $F_{i}$ es $\Sigma$-PR.
    \end{enumerate}
  \end{theorem}

  % Corollary 72: Con prueba.
  \begin{corollary}
    \PN Supongamos $f: D_{f} \subseteq \omega^{n} \times \Sigma^{\ast m} \rightarrow O$ es $\Sigma$-recursiva y
    $S\subseteq D_{f}$ es $\Sigma$-RE, entonces $f \mid_{S}$ es $\Sigma$-recursiva.
  \end{corollary}
  \begin{proof}
    \PN Por el \textbf{Theorem 71} $S=D_{g}$, para alguna función $\Sigma$-recursiva $g$. Nótese que componiendo
    adecuadamente podemos suponer que:
    \begin{itemize}
      \item $I_{g} = \{ 1\}$, si $O = \omega$
      \item $I_{g} = \{\varepsilon\}$, si $O = \SIGMA$
    \end{itemize}

    \PN Luego, tenemos:
    \[
      f\mid_{S} = \left\{
        \begin{array}{lll}
          \lambda xy \left[x.y\right] \circ (f,g) \qquad \text{Si } O = \omega \\
          \lambda \alpha \beta \left[ \alpha \beta \right] \circ (f,g) \qquad \text{Si } O = \SIGMA
        \end{array} \right.
    \]
  \end{proof}

  % Corollary 73: Nada.
  \begin{corollary}
    \PN Este corolario no se evalua.
  \end{corollary}

  % Corollary 74: Con prueba.
  \begin{corollary}
    \PN Supongamos $S_{1}, S_{2} \subseteq \omega^{n} \times \Sigma^{\ast m}$ son conjuntos $\Sigma$-RE, entonces
    $S_{1} \cap S_{2}$ es $\Sigma$-RE.
  \end{corollary}
  \begin{proof}
    \PN Por el \textbf{Theorem 71} $S_{1} = D_{g_{1}}, S_{2} = D_{g_{2}}$, con $g_{1}, g_{2}$ funciones
    $\Sigma$-recursivas. Notar que:

    \begin{itemize}
      \item Si $I_{g_{1}}, I_{g_{2}} \subseteq \omega \;\; \Rightarrow S_{1} \cap S_{2} = D_{\lambda xy \left[x.y\right]
      \circ (g_{1},g_{2})}$ es $\Sigma$-RE.
      \item Si $I_{g_{1}}, I_{g_{2}} \subseteq \Sigma^{\ast} \Rightarrow S_{1} \cap S_{2} = D_{\lambda \alpha\beta
        \left[\alpha\beta\right] \circ (g_{1},g_{2})}$ es $\Sigma$-RE.
    \end{itemize}
  \end{proof}

  % Corollary 75: Con prueba.
  \begin{corollary}
    \PN Supongamos $S_{1}, S_{2} \subseteq \omega^{n} \times \Sigma^{\ast m}$ son conjuntos $\Sigma$-RE, entonces
    $S_{1} \cup S_{2}$ es $\Sigma$-RE.
  \end{corollary}
  \begin{proof}
    \PN Supongamos $S_{1}, S_{2} \neq \emptyset$. Utilizando el \textbf{Lemma 71}, tomamos $F, G: \omega \rightarrow
    \omega^{n} \times \Sigma^{\ast m}$ tales que $I_{F}=S_{1}$, $I_{G}=S_{2}$ y las funciones $F_{i} {\acute{}} s$ y
    $G_{i} {\acute{}} s$ son $\Sigma $-recursivas. Sean:

    \begin{itemize}
      \item $f = \lambda x\left[Q(x,2)\right]$
      \item $g = \lambda x\left[Q(x\dot{-}1,2)\right]$
      \item $H: \omega \rightarrow \omega^{n} \times \Sigma^{\ast m}$, dada por:
        \[
          H_{i} = (F_{i} \circ f) \mathrm{\mid}_{\{x:x\mathrm{\ es\ par}\}} \cup (G_{i}\circ g)
          \mathrm{\mid}_{\{x:x\mathrm{\ es\ impar}\}}
        \]
    \end{itemize}

    \PN Por el \textbf{Corollary 72} y el \textbf{Lemma 68}, cada $H_{i}$ es $\Sigma$-recursiva. Dado que
    $I_{H} = S_{1} \cup S_{2}$, tenemos que $S_{1} \cup S_{2}$ es $\Sigma$-RE.
  \end{proof}

  % Theorem 76: Con prueba.
  \begin{theorem}
     \PN Sea $S \subseteq \omega^{n} \times \Sigma^{\ast m}$, entonces $S$ es $\Sigma$-efectivamente computable
     $\Leftrightarrow S$ es $\Sigma$-recursivo.
  \end{theorem}
  \begin{proof}
    \begin{tabular}{|c|} \hline $\Rightarrow$ \\\hline \end{tabular}
    \begin{eqnarray*}
      \text{S es $\Sigma$-efectivamente computable} &\Rightarrow& \chi_{S}\text{ es $\Sigma$-efectivamente computable} \\
      \text{Por \textbf{Tésis de Church}} &\Rightarrow& \chi_{S} \text{ es $\Sigma$-recursivo} \\
      &\Rightarrow& \text{S es $\Sigma$-recursivo}
    \end{eqnarray*}

    \PN \begin{tabular}{|c|} \hline $\Leftarrow$ \\\hline \end{tabular}
    \begin{eqnarray*}
      \text{S es $\Sigma$-recursivo} &\Rightarrow& \chi_{S}\text{ es $\Sigma$-recursivo} \\
      \text{Por \textbf{Theorem 42}} &\Rightarrow& \chi_{S} \text{ es $\Sigma$-efectivamente computable} \\
      &\Rightarrow& \text{S es $\Sigma$-efectivamente computable}
    \end{eqnarray*}
  \end{proof}

  % Theorem 77: Con prueba.
  \begin{theorem}
    \PN Sea $S \subseteq \omega^{n} \times \Sigma^{\ast m}$, son equivalentes:
    \begin{enumerate}[a)]
      \item $S$ es $\Sigma$-recursivo.
      \item $S$ y $(\omega^{n} \times \Sigma^{\ast m}) - S$ son $\Sigma$-recursivamente enumerables.
    \end{enumerate}
  \end{theorem}
  \begin{proof}
    \begin{tabular}{|c|} \hline $(a) \Rightarrow (b)$ \\\hline \end{tabular}

    \vspace{3mm}
    \PN Notar que:
    \begin{itemize}
      \item $S = D_{Pred\ \circ\ \chi_{S}}$
      \item $(\omega^{n} \times \Sigma^{\ast m}) - S = D_{Pred\ \circ\ \chi_{(\omega^{n} \times \Sigma^{\ast m})-S}}$
    \end{itemize}

    \PN Dado que $Pred \circ \chi_{S}$ y $Pred \circ \chi_{(\omega^{n} \times \Sigma^{\ast m})-S}$ son
    $\Sigma$-recursivas, utilizando el \textbf{Theorem 71} tenemos que $S$ y $(\omega^{n} \times \Sigma^{\ast m}) - S$
    son $\Sigma$-recursivamente enumerables, donde $\chi_{(\omega^{n} \times \Sigma^{\ast m}) - S} = \lambda xy \left[x
    \dot{-}y\right] \circ\ (C_{1}^{1,0},\chi_S)$.

    \vspace{3mm}
    \PN \begin{tabular}{|c|} \hline $(b) \Rightarrow (a)$ \\\hline \end{tabular}
    \vspace{3mm}
    \PN Notar que:
    \[
      \chi_{S} = C_{1}^{n,m}\mathrm{\mid}_{S} \cup C_{0}^{n,m}\mathrm{\mid}_{\omega^{n} \times \Sigma^{\ast m} - S}
    \]

    \PN Luego, utilizando el \textbf{Theorem 72} y el \textbf{Lemma 68}, tenemos que $S$ es $\Sigma$-recursivo.
  \end{proof}

  Lemma 78: Con prueba.
  \begin{lemma}
    \PN Supongamos que $\Sigma_{p} \subseteq \Sigma$, entonces:
    \[
      A = \left\{\mathcal{P} \in \mathrm{Pro}^{\Sigma}: Halt^{\Sigma}(\mathcal{P})\right\}
    \]

    \PN es $\Sigma$-RE y no es $\Sigma$-recursivo. Más aún el conjunto:
    \[
      N = \left\{\mathcal{P} \in \mathrm{Pro}^{\Sigma}: \lnot Halt^{\Sigma}(\mathcal{P})\right\}
    \]

    \PN no es $\Sigma$-RE.
  \end{lemma}
  \begin{proof}
    \PN Sea:
    \[
      P = \lambda t\mathcal{P}\left[i^{0,1}(t,\mathcal{P},\mathcal{P}) = n(\mathcal{P}) + 1\right]
    \]

    \PN Note que $P$ es $\Sigma$-PR, por lo que $M(P)$ es $\Sigma$-recursiva. Además note que $D_{M(P)}=A$, lo cual,
    utilizando el \textbf{Theorem 71}, implica que $A$ es $\Sigma$-RE. Dado que $Halt^{\Sigma}$ es no $\Sigma$-recursivo
    por \textbf{Lemma 69}, y dado que:
    \[
      Halt^{\Sigma} = C_{1}^{0,1}\mid_{A} \cup \ C_{0}^{0,1}\mid_{N}
    \]

    \PN el \textbf{Lemma 68} y el \textbf{Theorem 71} nos dice que $N$ no es $\Sigma$-RE. Finalmente, supongamos $A$ es
    $\Sigma$-recursivo, entonces el conjunto:
    \[
      N = \left(\SIGMA-A\right) \cap \mathrm{Pro}^{\Sigma}
    \]
    \PN debería serlo, lo cual es absurdo. Por lo tanto, $A$ no es $\Sigma$-recursivo.
  \end{proof}
