\section{El lenguaje ${S}^{\Sigma}$}

\textbf{\underline{Lemma 52:}} Se tiene que:
(a) Si \(I_{1}...I_{n}=J_{1}...J_{m}\), con \( I_{1},...,I_{n},J_{1},...,J_{m}\in \mathrm{Ins}^{\Sigma }\), entonces \(n=m\) y \(I_{j}=J_{j}\) para cada \(j\geq 1\).
(b) Si \(\mathcal{P}\in \mathrm{Pro}^{\Sigma }\), entonces existe una unica sucesion de instrucciones \(I_{1},...,I_{n}\) tal que \(\mathcal{P} =I_{1}...I_{n}\)


\PROOF (a) Supongamos \(I_{n}\) es un tramo final propio de \(J_{m}.\) Notar que entonces \(n >1\). Es facil ver que entonces ya sea \(J_{m}=\mathrm{L}\bar{u} I_{n}\) para algun \(u\in \mathbf{N}\), o \(I_{n}\) es de la forma \(\mathrm{GOTO} \;\mathrm{L}\bar{n}\) y \(J_{m}\) es de la forma \(w\mathrm{IF}\;\mathrm{P}\bar{k }\;\mathrm{BEGINS}\;a\;\mathrm{GOTO}\;\mathrm{L}\bar{n}\) donde \(w\in \{ \mathrm{L}\bar{n}:n\in \mathbf{N}\}\cup \{\varepsilon \}\). El segundo caso no puede darse porque entonces el anteultimo simbolo de \(I_{n-1}\) deberia ser \(\mathrm{S}\) lo cual no sucede para ninguna instruccion. O sea que

\(\displaystyle I_{1}...I_{n}=J_{1}...J_{m-1}\mathrm{L}\bar{u}I_{n} \)

lo cual dice que
(*) \(I_{1}...I_{n-1}=J_{1}...J_{m-1}\mathrm{L}\bar{u}.\)
Es decir que \(\mathrm{L}\bar{u}\) es tramo final de \(I_{n-1}\) y por lo tanto \(\mathrm{GOTO}\;\mathrm{L}\bar{u}\) es tramo final de \(I_{n-1}.\) Por (*), \(\mathrm{GOTO}\) es tramo final de \(J_{1}...J_{m-1}\), lo cual es impossible. Hemos llegado a una contradiccion lo cual nos dice que \(I_{n}\) no es un tramo final propio de \(J_{m}.\) Por simetria tenemos que \( I_{n}=J_{m} \), lo cual usando un razonamiento inductivo nos dice que \(n=m\) y \(I_{j}=J_{j} \) para cada \(j\geq 1\).

(b) Es consecuencia directa de (a). \(\Box\)


\textbf{\underline{Theorem 53:}} Si \(f\) es \(\Sigma \)-computable, entonces \(f\) es \(\Sigma \)-efectivamente computable.


\PROOF Supongamos por ejemplo que \(f:S\subseteq \omega ^{n}\times \Sigma ^{\ast m}\rightarrow \omega \) es computada por \(\mathcal{P}\in \mathrm{Pro}^{\Sigma }\). Es claro que el procedimiento que consiste en realizar las sucesivas instrucciones de \(\mathcal{P}\) (partiendo de \(((x_{1},...,x_{n},0,0,...),( \alpha _{1},...,\alpha _{m},\varepsilon ,\varepsilon ,...))\)) y eventualmente terminar en caso de que nos toque realizar la instruccion \(n( \mathcal{P})+1\), y dar como salida el contenido de la variable \(\mathrm{N}1\) , es un procedimiento efectivo que computa a \(f\). \(\Box\)


\textbf{\underline{Proposition 54:}}

(a) Sea \(f:S\subseteq \omega ^{n}\times \Sigma ^{\ast m}\rightarrow \omega \) una funcion \(\Sigma \)-computable. Entonces hay un macro
\(\displaystyle \left[ \mathrm{V}\overline{n+1}\leftarrow f(\mathrm{V}1,...,\mathrm{V}\bar{n} ,\mathrm{W}1,...,\mathrm{W}\bar{m})\right] \)
(b) Sea \(f:S\subseteq \omega ^{n}\times \Sigma ^{\ast m}\rightarrow \Sigma ^{\ast }\) una funcion \(\Sigma \)-computable. Entonces hay un macro
\(\displaystyle \left[ \mathrm{W}\overline{m+1}\leftarrow f(\mathrm{V}1,...,\mathrm{V}\bar{n} ,\mathrm{W}1,...,\mathrm{W}\bar{m})\right] \)

\PROOF (b) Sea \(\mathcal{P}\) un programa que compute a \(f\). Tomemos un \(k\) tal que \( k\geq n,m\) y tal que todas las variables y labels de \(\mathcal{P}\) estan en el conjunto

\(\displaystyle \{\mathrm{N}1,...,\mathrm{N}\bar{k},\mathrm{P}1,...,\mathrm{P}\bar{k}, \mathrm{L}1,...,\mathrm{L}\bar{k}\}\text{.} \)

Sea \(\mathcal{P}^{\prime }\) la palabra que resulta de reemplazar en \( \mathcal{P}\):
- la variable \(\mathrm{N}\overline{j}\) por \(\mathrm{V}\overline{n+j}\) , para cada \(j=1,...,k\)
- la variable \(\mathrm{P}\overline{j}\) por \(\mathrm{W}\overline{m+j}\) , para cada \(j=1,...,k\)
- el label \(\mathrm{L}\overline{j}\) por \(\mathrm{A}\overline{j}\), para cada \(j=1,...,k\)
Notese que

\(\displaystyle \begin{array}{l} \mathrm{V}\overline{n+1}\leftarrow \mathrm{V}1 \\ \ \ \ \ \ \ \ \ \ \vdots \\ \mathrm{V}\overline{n+n}\leftarrow \mathrm{V}\overline{n} \\ \mathrm{V}\overline{n+n+1}\leftarrow 0 \\ \ \ \ \ \ \ \ \ \ \vdots \\ \mathrm{V}\overline{n+k}\leftarrow 0 \\ \mathrm{W}\overline{m+1}\leftarrow \mathrm{W}1 \\ \ \ \ \ \ \ \ \ \ \vdots \\ \mathrm{W}\overline{m+m}\leftarrow \mathrm{W}\overline{m} \\ \mathrm{W}\overline{m+m+1}\leftarrow \varepsilon \\ \ \ \ \ \ \ \ \ \ \vdots \\ \mathrm{W}\overline{m+k}\leftarrow \varepsilon \\ \mathcal{P}^{\prime } \end{array} \)

es el macro buscado, el cual tendra sus variables auxiliares y labels en la lista
\(\displaystyle \mathrm{V}\overline{n+1},...,\mathrm{V}\overline{n+k},\mathrm{W}\overline{m+2 },...,\mathrm{V}\overline{m+k},\mathrm{A}1,...,\mathrm{A}\overline{k}. \)

\(\Box\)




\textbf{\underline{Proposition 55:}} Sea \(P:S\subseteq \omega ^{n}\times \Sigma ^{\ast m}\rightarrow \omega \) un predicado \(\Sigma \)-computable. Entonces hay un macro
\(\displaystyle \left[ \mathrm{IF}\;P(\mathrm{V}1,...,\mathrm{V}\bar{n},\mathrm{W}1,..., \mathrm{W}\bar{m})\;\mathrm{GOTO}\;\mathrm{A}1\right] \)



\textbf{\underline{Theorem 56:}} Si \(h\) es \(\Sigma \)-recursiva, entonces \(h\) es \(\Sigma \) -computable.

\PROOF Probaremos por induccion en \(k\) que

(*) Si \(h\in \mathrm{R}_{k}^{\Sigma }\), entonces \(h\) es \(\Sigma \) -computable.
El caso \(k=0\) es dejado al lector. Supongamos (*) vale para \(k\), veremos que vale para \(k+1\). Sea \(h\in \mathrm{R}_{k+1}^{\Sigma }-\mathrm{R} _{k}^{\Sigma }.\) Hay varios casos

Caso 1. Supongamos \(h=M(P)\), con \(P:\omega \times \omega ^{n}\times \Sigma ^{\ast m}\rightarrow \omega \), un predicado perteneciente a \(\mathrm{R} _{k}^{\Sigma }\). Por hipotesis inductiva, \(P\) es \(\Sigma \)- computable y por lo tanto tenemos un macro

\(\displaystyle \left[ \mathrm{IF}\;P(\mathrm{V}1,...,\mathrm{V}\overline{n+1},\mathrm{W} 1,...,\mathrm{W}\bar{m})\;\mathrm{GOTO}\;\mathrm{A}1\right] \)

lo cual nos permite realizar el siguiente programa
\(\displaystyle \begin{array}{ll} \mathrm{L}2 & \mathrm{IF}\;P(\mathrm{N}\overline{n+1},\mathrm{N}1,..., \mathrm{N}\bar{n},\mathrm{P}1,...,\mathrm{P}\bar{m})\text{\ }\mathrm{GOTO}\; \mathrm{L}1 \\ & \mathrm{N}\overline{n+1}\leftarrow \mathrm{N}\overline{n+1}+1 \\ & \mathrm{GOTO}\;\mathrm{L}2 \\ \mathrm{L}1 & \mathrm{N}1\leftarrow \mathrm{N}\overline{n+1} \end{array} \)

Es facil chequear que este programa computa \(h.\)
Caso 2. Supongamos \(h=R(f,\mathcal{G})\), con

\(\displaystyle \begin{array}{rcl} f & :& S_{1}\times ...\times S_{n}\times L_{1}\times ...\times L_{m}\rightarrow \Sigma ^{\ast } \\ \mathcal{G}_{a} & :& S_{1}\times ...\times S_{n}\times L_{1}\times ...\times L_{m}\times \Sigma ^{\ast }\times \Sigma ^{\ast }\rightarrow \Sigma ^{\ast } \text{, }a\in \Sigma \end{array} \)

elementos de \(\mathrm{R}_{k}^{\Sigma }\). Sea \(\Sigma =\{a_{1},...,a_{r}\}.\) Por hipotesis inductiva, las funciones \(f\), \(\mathcal{G}_{a}\), \(a\in \Sigma \) , son \(\Sigma \)-computables y por lo tanto podemos hacer el siguiente programa via el uso de macros
\(\displaystyle \begin{array}{rl} & \left[ \mathrm{P}\overline{m+3}\leftarrow f(\mathrm{N}1,...,\mathrm{N}\bar{ n},\mathrm{P}1,...,\mathrm{P}\bar{m})\right] \\ \mathrm{L}\overline{r+1} & \mathrm{IF}\;\mathrm{P}\overline{m+1}\ \text{ {B}}\mathrm{EGINS\ }a_{1}\text{ }\mathrm{GOTO}\;\mathrm{L}1 \\ & \ \ \ \ \ \ \ \ \ \ \ \ \vdots \\ & \mathrm{IF}\;\mathrm{P}\overline{m+1}\ \mathrm{BEGINS\ }a_{r}\text{ } \mathrm{GOTO}\;\mathrm{L}\bar{r} \\ & \mathrm{GOTO}\;\mathrm{L}\overline{r+2} \\ \mathrm{L}1 & \mathrm{P}\overline{m+1}\leftarrow \text{ }^{\curvearrowright } \mathrm{P}\overline{m+1} \\ & \left[ \mathrm{P}\overline{m+3}\leftarrow \mathcal{G}_{a_{1}}(\mathrm{N} 1,...,\mathrm{N}\bar{n},\mathrm{P}1,...,\mathrm{P}\bar{m},\mathrm{P} \overline{m+2},\mathrm{P}\overline{m+3})\right] \\ & \mathrm{P}\overline{m+2}\leftarrow \mathrm{P}\overline{m+2}a_{1} \\ & \mathrm{GOTO}\;\mathrm{L}\overline{r+1} \\ & \ \ \ \ \ \ \ \ \ \ \ \ \vdots \\ \mathrm{L}\bar{r} & \mathrm{P}\overline{m+1}\leftarrow \text{ } ^{\curvearrowright }\mathrm{P}\overline{m+1} \\ & \mathrm{P}\overline{m+3}\leftarrow \mathcal{G}_{a_{r}}(\mathrm{N}1,..., \mathrm{N}\bar{n},\mathrm{P}1,...,\mathrm{P}\bar{m},\mathrm{P}\overline{m+2}, \mathrm{P}\overline{m+3}) \\ & \mathrm{P}\overline{m+2}\leftarrow \mathrm{P}\overline{m+2}a_{r} \\ & \mathrm{GOTO}\;\mathrm{L}\overline{r+1} \\ \mathrm{L}\overline{r+2} & \mathrm{P}1\leftarrow \mathrm{P}\overline{m+3} \end{array} \)

Es facil chequear que este programa computa \(h.\)
El resto de los casos son dejados al lector. \(\Box\)


\textbf{\underline{Lemma 57:}} Sea \(\Sigma \) un alfabeto cualquiera. Las funciones \(S\) y \(\overline{\ \;}\) son \((\Sigma \cup \Sigma _{p})\)-p.r..

\PROOF Use el Teorema 51. \(\Box\)

\textbf{\underline{Lemma 58:}} Para cada \(n,x\in \omega \), tenemos que \( \left\vert \bar{n}\right\vert \leq x\) si y solo si \(n\leq 10^{x}-1\)

\textbf{\underline{Lemma 59:}} \(\mathrm{Ins}^{\Sigma }\) es un conjunto \((\Sigma \cup \Sigma _{p})\)-p.r..

\PROOF Para simplificar la Proof asumiremos que \(\Sigma =\{@,\& \}\). Ya que \( \mathrm{Ins}^{\Sigma }\) es union de los siguientes conjuntos

\(\displaystyle \begin{array}{rcl} L_{1} & =& \left\{ \mathrm{N}\bar{k}\leftarrow \mathrm{N}\bar{k}+1:k\in \mathbf{N}\right\} \\ L_{2} & =& \left\{ \mathrm{N}\bar{k}\leftarrow \mathrm{N}\bar{k}\dot{-}1:k\in \mathbf{N}\right\} \\ L_{3} & =& \left\{ \mathrm{N}\bar{k}\leftarrow \mathrm{N}\bar{n}:k,n\in \mathbf{N}\right\} \\ L_{4} & =& \left\{ \mathrm{N}\bar{k}\leftarrow 0:k\in \mathbf{N}\right\} \\ L_{5} & =& \left\{ \mathrm{IF}\;\mathrm{N}\bar{k}\neq 0\;\mathrm{GOTO}\; \mathrm{L}\bar{m}:k,m\in \mathbf{N}\right\} \\ L_{6} & =& \left\{ \mathrm{P}\bar{k}\leftarrow \mathrm{P}\bar{k}.@:k\in \mathbf{N}\right\} \\ L_{7} & =& \left\{ \mathrm{P}\bar{k}\leftarrow \mathrm{P}\bar{k}.\& :k\in \mathbf{N}\right\} \\ L_{8} & =& \left\{ \mathrm{P}\bar{k}\leftarrow \text{ }^{\curvearrowright } \mathrm{P}\bar{k}:k\in \mathbf{N}\right\} \\ L_{9} & =& \left\{ \mathrm{P}\bar{k}\leftarrow \mathrm{P}\bar{n}:k,n\in \mathbf{N}\right\} \\ L_{9} & =& \left\{ \mathrm{P}\bar{k}\leftarrow \varepsilon :k\in \mathbf{N} \right\} \\ L_{10} & =& \left\{ \mathrm{IF}\;\mathrm{P}\bar{k}\;\mathrm{BEGINS}\;@\; \mathrm{GOTO}\;\mathrm{L}\bar{m}:k,m\in \mathbf{N}\right\} \\ L_{11} & =& \left\{ \mathrm{IF}\;\mathrm{P}\bar{k}\;\mathrm{BEGINS}\;\& \; \mathrm{GOTO}\;\mathrm{L}\bar{m}:k,m\in \mathbf{N}\right\} \\ L_{12} & =& \left\{ \mathrm{GOTO}\;\mathrm{L}\bar{m}:m\in \mathbf{N}\right\} \\ L_{13} & =& \left\{ \mathrm{SKIP}\right\} \\ L_{14} & =& \left\{ \mathrm{L}\bar{k}\alpha :k\in \mathbf{N\;}\text{y }\alpha \in L_{1}\cup ...\cup L_{13}\right\} \end{array} \)

solo debemos probar que \(L_{1},...,L_{14}\) son \((\Sigma \cup \Sigma _{p})\) -p.r.. Veremos primero por ejemplo que
\(\displaystyle L_{10}=\left\{ \mathrm{IFP}\bar{k}\mathrm{BEGINS}@\mathrm{GOTOL}\bar{m} :k,m\in \mathbf{N}\right\} \)

es \((\Sigma \cup \Sigma _{p})\)-p.r.. Primero notese que \(\alpha \in L_{10}\) si y solo si existen \(k,m\in \mathbf{N}\) tales que
\(\displaystyle \alpha =\mathrm{IFP}\bar{k}\mathrm{BEGINS}@\mathrm{GOTOL}\bar{m} \)

Mas formalmente tenemos que \(\alpha \in L_{10}\) si y solo si
\(\displaystyle (\exists k\in \mathbf{N})(\exists m\in \mathbf{N})\;\alpha =\mathrm{IFP}\bar{ k}\mathrm{BEGINS}@\mathrm{GOTOL}\bar{m} \)

Ya que cuando existen tales \(k,m\) tenemos que \(\bar{k}\) y \(\bar{m}\) son subpalabras de \(\alpha \), el lema anterior nos dice que \(\alpha \in L_{10}\) si y solo si
\(\displaystyle (\exists k\in \mathbf{N})_{k\leq 10^{\left\vert \alpha \right\vert }}(\exists m\in \mathbf{N})_{m\leq 10^{\left\vert \alpha \right\vert }}\;\alpha =\mathrm{IFP}\bar{k}\mathrm{BEGINS}@\mathrm{GOTOL}\bar{m} \)

Sea
\(\displaystyle P=\lambda mk\alpha \left[ \alpha =\mathrm{IFP}\bar{k}\mathrm{BEGINS}@\mathrm{ GOTOL}\bar{m}\right] \)

Ya que \(D_{\lambda k\left[ \bar{k}\right] }=\omega \), tenemos que \( D_{P}=\omega \times (\Sigma \cup \Sigma _{p})^{\ast }\times (\Sigma \cup \Sigma _{p})^{\ast }\). Notese que
\(\displaystyle P=\lambda \alpha \beta \left[ \alpha =\beta \right] \circ \left( p_{3}^{2,1},f\right) \)

donde
\(\displaystyle f=\lambda \alpha _{1}\alpha _{2}\alpha _{3}\alpha _{4}\left[ \alpha _{1}\alpha _{2}\alpha _{3}\alpha _{4}\right] \circ \left( C_{\mathrm{IFP} }^{2,1},\lambda k\left[ \bar{k}\right] \circ p_{2}^{2,1},C_{\mathrm{BEGINS}@ \mathrm{GOTOL}}^{2,1},\lambda k\left[ \bar{k}\right] \circ p_{1}^{2,1}\right) \)

lo cual nos dice que \(P\) es \((\Sigma \cup \Sigma _{p})\)-p.r..
Notese que

\(\displaystyle \chi _{L_{10}}=\lambda \alpha \left[ (\exists k\in \mathbf{N})_{k\leq 10^{\left\vert \alpha \right\vert }}(\exists m\in \mathbf{N})_{m\leq 10^{\left\vert \alpha \right\vert }}\;P(m,k,\alpha )\right] \)

Esto nos dice que podemos usar dos veces el Lema 39 para ver que \(\chi _{L_{10}}\) es \((\Sigma \cup \Sigma _{p})\)-p.r.. Veamos como. Sea
\(\displaystyle Q=\lambda k\alpha \left[ (\exists m\in \mathbf{N})_{m\leq 10^{\left\vert \alpha \right\vert }}\;P(m,k,\alpha )\right] \)

Por el Lema 39 tenemos que
\(\displaystyle \lambda xk\alpha \left[ (\exists m\in \mathbf{N})_{m\leq x}\;P(m,k,\alpha ) \right] \)

es \((\Sigma \cup \Sigma _{p})\)-p.r. lo cual nos dice que
\(\displaystyle Q=\lambda xk\alpha \left[ (\exists m\in \mathbf{N})_{m\leq x}\;P(m,k,\alpha ) \right] \circ (\lambda \alpha \left[ 10^{\left\vert \alpha \right\vert } \right] \circ p_{2}^{1,1},p_{1}^{1,1},p_{2}^{1,1}) \)

lo es. Ya que
\(\displaystyle \chi _{L_{10}}=\lambda \alpha \left[ (\exists k\in \mathbf{N})_{k\leq 10^{\left\vert \alpha \right\vert }}\;Q(k,\alpha )\right] \)

podemos en forma similar aplicar el Lema 39 y obtener finalmente que \(\chi _{L_{10}}\) es \((\Sigma \cup \Sigma _{p})\)-p.r..
En forma similar podemos probar que \(L_{1},...,L_{13}\) son \((\Sigma \cup \Sigma _{p})\)-p.r.. Esto nos dice que \(L_{1}\cup ...\cup L_{13}\) es \((\Sigma \cup \Sigma _{p})\)-p.r.. Notese que \(L_{1}\cup ...\cup L_{13}\) es el conjunto de las instrucciones basicas de \(\mathcal{S}^{\Sigma }\). Llamemos \( \mathrm{InsBas}^{\Sigma }\) a dicho conjunto. Para ver que \(L_{14}\) es \( (\Sigma \cup \Sigma _{p})\)-p.r. notemos que

\(\displaystyle \chi _{L_{14}}=\lambda \alpha \left[ (\exists k\in \mathbf{N})_{k\leq 10^{\left\vert \alpha \right\vert }}(\exists \beta \in \mathrm{InsBas} ^{\Sigma })_{\left\vert \beta \right\vert \leq \left\vert \alpha \right\vert }\;\alpha =\mathrm{L}\bar{k}\beta \right] \)

lo cual nos dice que aplicando dos veces el Lema 39 obtenemos que \(\chi _{L_{14}}\) es \((\Sigma \cup \Sigma _{p})\)-p.r.. Ya que \( \mathrm{Ins}^{\Sigma }=\mathrm{InsBas}^{\Sigma }\cup L_{14}\) tenemos que \( \mathrm{Ins}^{\Sigma }\) es \((\Sigma \cup \Sigma _{p})\)-p.r.. \(\Box\)


\textbf{\underline{Lemma 60:}} \(Bas\) y \(Lab\) son funciones \((\Sigma \cup \Sigma _{p})\)-p.r.

\PROOF Sea \(< \) un orden total estricto sobre \(\Sigma \cup \Sigma _{p}\). Sea \(L=\{ \mathrm{L}\bar{k}:k\in \mathbf{N}\}\cup \{\varepsilon \}\). Dejamos al lector probar que \(L\) es un conjunto \((\Sigma \cup \Sigma _{p})\)-p.r.. Sea

\(\displaystyle P=\lambda I\alpha \left[ \alpha \in \mathrm{Ins}^{\Sigma }\wedge I\in \mathrm{Ins}^{\Sigma }\wedge \lbrack \alpha ]_{1}\neq \mathrm{L}\wedge (\exists \beta \in L)\ I=\beta \alpha \right] \)

Note que \(D_{P}=(\Sigma \cup \Sigma _{p})^{\ast 2}\). Dejamos al lector probar que \(P\) es \((\Sigma \cup \Sigma _{p})\)-p.r.. Notese ademas que cuando \(I\in \mathrm{Ins}^{\Sigma }\) tenemos que \(P(I,\alpha )=1\) sii \(\alpha =Bas(I)\). Dejamos al lector probar que \(Bas=M^{< }\left( P\right) \) por lo que para ver que \(Bas\) es \((\Sigma \cup \Sigma _{p})\)-p.r., solo nos falta ver que la funcion \(Bas\) es acotada por alguna funcion \((\Sigma \cup \Sigma _{p})\)-p.r. y \((\Sigma \cup \Sigma _{p})\)-total. Pero esto es trivial ya que \(\left\vert Bas(I)\right\vert \leq \left\vert I\right\vert =p_{1}^{0,1}(I)\) para cada \(I\in \mathrm{Ins}^{\Sigma }\).
Finalmente note que

\(\displaystyle Lab=M^{< }\left( \lambda I\alpha \left[ \alpha Bas(I)=I\right] \right) \)

lo cual nos dice que \(Lab\) es \((\Sigma \cup \Sigma _{p})\)-p.r.. \(\Box\)
Recordemos que dado un programa \(\mathcal{P}\) habiamos definido \(I_{i}^{ \mathcal{P}}=\varepsilon \), para \(i=0\) o \(i >n(\mathcal{P}).\) O sea que la funcion \((\Sigma \cup \Sigma _{p})\)-mixta \(\lambda i\mathcal{P}\left[ I_{i}^{ \mathcal{P}}\right] \) tiene dominio igual a \(\omega \times \mathrm{Pro} ^{\Sigma }\).

\textbf{\underline{Lemma 61:}}
(a) \(\mathrm{Pro}^{\Sigma }\) es un conjunto \((\Sigma \cup \Sigma _{p}) \)-p.r.
(b) \(\lambda \mathcal{P}\left[ n(\mathcal{P})\right] \) y \(\lambda i \mathcal{P}\left[ I_{i}^{\mathcal{P}}\right] \) son funciones \((\Sigma \cup \Sigma _{p})\)-p.r..

\PROOF Ya que \(\mathrm{Pro}^{\Sigma }=D_{\lambda \mathcal{P}\left[ n(\mathcal{P}) \right] }\) tenemos que (b) implica (a). Para probar (b) sea \(< \) un orden total estricto sobre \(\Sigma \cup \Sigma _{p}\). Sea \(P\) el siguiente predicado

\(\lambda x\left[ Lt(x) >0\wedge (\forall t\in \mathbf{N})_{t\leq Lt(x)}\;\ast ^{< }((x)_{t})\in \mathrm{Ins}^{\Sigma }\wedge \right. \)
\(\ \ \ \ \ \ \ \ \ \ \ \ \ \ \ \ \ \ \ \ \ \ \ \ \ \ \ \ \ (\forall t\in \mathbf{N})_{t\leq Lt(x)}(\forall m\in \mathbf{N})\;\lnot (\mathrm{L} \bar{m}\ \)t-final \(\ast ^{< }((x)_{t}))\vee \)
\(\ \ \ \ \ \ \ \ \ \ \ \ \ \ \ \ \ \ \ \ \ \ \ \ \ \ \ \ \ \ \ \ \ \ \ \ \ \ \ \ \ \ \ \ \ \ \ \ \ \ \ \ \ \ \ \ \ \ \ \left. (\exists j\in \mathbf{ N})_{j\leq Lt(x)}(\exists \alpha \in (\Sigma \cup \Sigma _{p})-Num)\;\mathrm{ L}\bar{m}\alpha \ \text{t-inicial}\ast ^{< }((x)_{j})\right] \)
Notese que \(D_{P}=\mathbf{N}\) y que \(P(x)=1\) sii \(Lt(x) >0\), \(\ast ^{< }((x)_{t})\in \mathrm{Ins}^{\Sigma }\), para cada \(t=1,...,Lt(x)\) y ademas \(\subset _{t=1}^{t=Lt(x)}\ast ^{< }((x)_{t})\in \mathrm{Pro}^{\Sigma }\). Para ver que \(P\) es \((\Sigma \cup \Sigma _{p})\)-p.r. solo nos falta acotar el cuantificador \((\forall m\in \mathbf{N})\) de la expresion lambda que define a \(P\). Ya que nos interesan los valores de \(m\) para los cuales \(\bar{m}\) es posiblemente una subpalabra de alguna de las palabras \(\ast ^{< }((x)_{j})\), el Lema 58 nos dice que una cota posible es \( 10^{\max \{\left\vert \ast ^{< }((x)_{j})\right\vert :1\leq j\leq Lt(x)\}}-1\) . Dejamos al lector los detalles de la Proof de que \(P\) es \((\Sigma \cup \Sigma _{p})\)-p.r.. Sea

\(\displaystyle Q=\lambda x\alpha \left[ P(x)\wedge \alpha =\subset _{t=1}^{t=Lt(x)}\ast ^{< }((x)_{t})\right] \text{.} \)

Note que \(D_{Q}=\mathbf{N}\times (\Sigma \cup \Sigma _{p})^{\ast }\). Claramente \(Q\) es \((\Sigma \cup \Sigma _{p})\)-p.r.. Ademas note que \( D_{M(Q)}=\mathrm{Pro}^{\Sigma }\). Notese que para \(\mathcal{P}\in \mathrm{Pro }^{\Sigma }\), tenemos que \(M(Q)(\mathcal{P})\) es aquel numero tal que pensado como infinitupla (via mirar su secuencia de exponentes) codifica la secuencia de instrucciones que forman a \(\mathcal{P}\). Es decir
\(\displaystyle M(Q)(\mathcal{P})=\left\langle \#^{< }(I_{1}^{\mathcal{P}}),\#^{< }(I_{2}^{ \mathcal{P}}),...,\#^{< }(I_{n(\mathcal{P})}^{\mathcal{P}}),0,0,...\right \rangle \)

Por (b) del Lema 43, \(M(Q)\) es \((\Sigma \cup \Sigma _{p})\) -p.r. ya que para cada \(\mathcal{P}\in \mathrm{Pro}^{\Sigma }\) tenemos que
\(\displaystyle \begin{array}{rcl} M(Q)(\mathcal{P}) & =& \left\langle \#^{< }(I_{1}^{\mathcal{P}}),\#^{< }(I_{2}^{ \mathcal{P}}),...,\#^{< }(I_{n(\mathcal{P})}^{\mathcal{P}}),0,0,...\right \rangle \\ & =& \underset{i=1}{\overset{n(\mathcal{P})}{\Pi }}pr(i)^{\#^{< }(I_{1}^{ \mathcal{P}})} \\ & \leq & \underset{i=1}{\overset{\left\vert \mathcal{P}\right\vert }{\Pi }} pr(i)^{\#^{< }(\mathcal{P})} \end{array} \)

Ademas tenemos que
\(\displaystyle \begin{array}{rcl} \lambda \mathcal{P}\left[ n(\mathcal{P})\right] & =& \lambda x\left[ Lt(x) \right] \circ M(Q) \\ \lambda i\mathcal{P}\left[ I_{i}^{\mathcal{P}}\right] & =& \ast ^{< }\circ g\circ \left( p_{1}^{1,1},M(Q)\circ p_{2}^{1,1}\right) \end{array} \)

donde \(g=C_{0}^{1,1}\mid _{\{0\}\times \omega }\cup \lambda ix\left[ (x)_{i} \right] \), lo cual dice que \(\lambda \mathcal{P}\left[ n(\mathcal{P})\right] \) y \(\lambda i\mathcal{P}\left[ I_{i}^{\mathcal{P}}\right] \) son funciones \( (\Sigma \cup \Sigma _{p})\)-p.r.. \(\Box\)


\textbf{\underline{Lemma 62:}} Dado un orden total estricto \( < \) sobre \(\Sigma \cup \Sigma _{p}\), las funciones \(s\), \(S_{\#}\) y \(S_{\ast } \) son \((\Sigma \cup \Sigma _{p})\)-p.r..

\PROOF Necesitaremos algunas funciones \((\Sigma \cup \Sigma _{p})\)-p.r.. Dada una instruccion \(I\) en la cual al menos ocurre una variable, usaremos \(\#Var1(I)\) para denotar el numero de la primer variable que ocurre en \(I\). Por ejemplo

\(\displaystyle \#Var1\left( \mathrm{L}\bar{n}\;\mathrm{IF\;N}\bar{k}\neq 0\;\mathrm{GOTO\;L} \bar{m}\right) =k \)

Notese que \(\lambda I[\#Var1(I)]\) tiene dominio igual a \(\mathrm{Ins} ^{\Sigma }-L\), donde \(L\) es la union de los siguientes conjuntos \begin{gather*} \{\mathrm{GOTO L}\bar{m}:m\in \mathbf{N\}\cup }\{\mathrm{L}\bar{k} \mathrm{ GOTO L}\bar{m}:k,m\in \mathbf{N\}}
\left\{ \mathrm{SKIP}\right\} \mathbf{\cup }\{\mathrm{L}\bar{k} \mathrm{SKIP }:k\in \mathbf{N\}} \end{gather*} Dada una instruccion \(I\) en la cual ocurren dos variables, usaremos \( \#Var2(I)\) para denotar el numero de la segunda variable que ocurre en \(I\). Por ejemplo
\(\displaystyle \#Var2\left( \mathrm{N}\bar{k}\leftarrow \mathrm{N}\bar{m}\right) =m \)

Notese que el dominio de \(\lambda I[\#Var2(I)]\) es igual a la union de los siguientes conjuntos
\(\displaystyle \begin{array}{rcl} \{\mathrm{N}\bar{k} & \leftarrow & \mathrm{N}\bar{m}:k,m\in \mathbf{N\}\cup }\{ \mathrm{L}\bar{j}\ \mathrm{N}\bar{k}\leftarrow \mathrm{N}\bar{m}:j,k,m\in \mathbf{N\}} \\ \{\mathrm{P}\bar{k} & \leftarrow & \mathrm{P}\bar{m}:k,m\in \mathbf{N\}\cup }\{ \mathrm{L}\bar{j}\ \mathrm{P}\bar{k}\leftarrow \mathrm{P}\bar{m}:j,k,m\in \mathbf{N\}} \end{array} \)

Ademas notese que para una instruccion \(I\) tenemos que
\(\displaystyle \begin{array}{rcl} \#Var1(I) & =& \min_{k}(\mathrm{N}\bar{k}\mathrm{\leftarrow }\text{ }\mathrm{ ocu}\text{ }I\vee \mathrm{N}\bar{k}\mathrm{\neq }\text{ }\mathrm{ocu}\text{ } I\vee \mathrm{P}\bar{k}\mathrm{\leftarrow }\text{ }\mathrm{ocu}\text{ }I\vee \mathrm{P}\bar{k}\mathrm{B}\;\mathrm{ocu}\text{ }I) \\ \#Var2(I) & =& \min_{k}(\mathrm{N}\bar{k}\ \text{t-final }I\vee \mathrm{N}\bar{ k}\mathrm{+}\text{ }\mathrm{ocu}\text{ }I\vee \mathrm{N}\bar{k}\mathrm{\dot{- }}\text{ }\mathrm{ocu}\text{ }I\vee \mathrm{P}\bar{k}\ \text{t-final }I\vee \mathrm{P}\bar{k}.\text{ }\mathrm{ocu}\text{ }I) \end{array} \)

Esto nos dice que si llamamos \(P\) al predicado
\(\displaystyle \lambda k\alpha \left[ \alpha \in \mathrm{Ins}^{\Sigma }\wedge (\mathrm{N} \bar{k}\mathrm{\leftarrow }\text{ }\mathrm{ocu}\text{ }\alpha \vee \mathrm{N} \bar{k}\mathrm{\neq }\text{ }\mathrm{ocu}\text{ }\alpha \vee \mathrm{P}\bar{k }\mathrm{\leftarrow }\text{ }\mathrm{ocu}\text{ }\alpha \vee \mathrm{P}\bar{k }\mathrm{B}\;\mathrm{ocu}\text{ }\alpha )\right] \)

entonces \(\lambda I[\#Var1(I)]=M(P)\) por lo cual \(\lambda I[\#Var1(I)]\) es \( (\Sigma \cup \Sigma _{p})\)-p.r. Similarmente se puede ver que \(\lambda I[\#Var2(I)]\) es \((\Sigma \cup \Sigma _{p})\)-p.r.. Sea
\(\displaystyle \begin{array}{rll} F_{\dot{-}}:\mathbf{N}\times \mathbf{N} & \rightarrow & \omega \\ (x,j) & \rightarrow & \left\langle (x)_{1},....,(x)_{j-1},(x)_{j}\dot{-} 1,(x)_{j+1},...\right\rangle \end{array} \)

Ya que
\(\displaystyle F_{\dot{-}}(x,j)=\left\{ \begin{array}{lll} Q(x,pr(j)) & & \text{si }pr(j)\text{ divide }x \\ x & & \text{caso contrario} \end{array} \right. \)

tenemos que \(F_{\dot{-}}\) es \((\Sigma \cup \Sigma _{p})\)-p.r.. Sea
\(\displaystyle \begin{array}{rll} F_{+}:\mathbf{N}\times \mathbf{N} & \rightarrow & \omega \\ (x,j) & \rightarrow & \left\langle (x)_{1},....,(x)_{j-1},(x)_{j}+1,(x)_{j+1},...\right\rangle \end{array} \)

Ya que \(F_{+}(x,j)=x.pr(j)\) tenemos que \(F_{+}\) es \((\Sigma \cup \Sigma _{p}) \)-p.r.. Sea
\(\displaystyle \begin{array}{rll} F_{\leftarrow }:\mathbf{N}\times \mathbf{N}\times \mathbf{N} & \rightarrow & \omega \\ (x,j,k) & \rightarrow & \left\langle (x)_{1},....,(x)_{j-1},(x)_{k},(x)_{j+1},...\right\rangle \end{array} \)

Ya que \(F_{\leftarrow }(x,j,k)=Q(x,pr(j)^{(x)_{j}}).pr(j)^{(x)_{k}}\) tenemos que \(F_{\leftarrow }\) es \((\Sigma \cup \Sigma _{p})\)-p.r.. Sea
\(\displaystyle \begin{array}{rll} F_{0}:\mathbf{N}\times \mathbf{N} & \rightarrow & \omega \\ (x,j) & \rightarrow & \left\langle (x)_{1},....,(x)_{j-1},0,(x)_{j+1},...\right\rangle \end{array} \)

Es facil ver que \(F_{0}\) es \((\Sigma \cup \Sigma _{p})\)-p.r.. Para cada \( a\in \Sigma \), sea
\(\displaystyle \begin{array}{rll} F_{a}:\mathbf{N}\times \mathbf{N} & \rightarrow & \omega \\ (x,j) & \rightarrow & \left\langle (x)_{1},....,(x)_{j-1},\#^{< }(\ast ^{< }((x)_{j})a),(x)_{j+1},...\right\rangle \end{array} \)

Es facil ver que \(F_{a}\) es \((\Sigma \cup \Sigma _{p})\)-p.r.. En forma similar puede ser probado que
\(\displaystyle \begin{array}{rll} F_{\curvearrowright }:\mathbf{N}\times \mathbf{N} & \rightarrow & \omega \\ (x,j) & \rightarrow & \left\langle (x)_{1},....,(x)_{j-1},\#^{< }(^{\curvearrowright }(\ast ^{< }((x)_{j}))),(x)_{j+1},...\right\rangle \end{array} \)

es \((\Sigma \cup \Sigma _{p})\)-p.r.
Dado \((i,x,y,\mathcal{P})\in \omega \times \mathbf{N}\times \mathbf{N}\times \mathrm{Pro}^{\Sigma }\), tenemos varios casos en los cuales los valores \( s(i,x,y,\mathcal{P}),S_{\#}(i,x,y,\mathcal{P})\) y \(S_{\ast }(i,x,y,\mathcal{P })\) pueden ser obtenidos usando las funciones antes definidas:

(1) CASO \(i=0\vee i >n(\mathcal{P})\). Entonces
\(\displaystyle \begin{array}{rcl} s(i,x,y,\mathcal{P}) & =& i \\ S_{\#}(i,x,y,\mathcal{P}) & =& x \\ S_{\ast }(i,x,y,\mathcal{P}) & =& y \end{array} \)

(2) CASO \((\exists j\in \omega )\;Bas(I_{i}^{\mathcal{P}})=\mathrm{N} \bar{j}\leftarrow \mathrm{N}\bar{j}+1\). Entonces
\(\displaystyle \begin{array}{rcl} s(i,x,y,\mathcal{P}) & =& i+1 \\ S_{\#}(i,x,y,\mathcal{P}) & =& F_{+}(x,\#Var1(I_{i}^{\mathcal{P}})) \\ S_{\ast }(i,x,y,\mathcal{P}) & =& y \end{array} \)

(3) CASO \((\exists j\in \omega )\;Bas(I_{i}^{\mathcal{P}})=\mathrm{N} \bar{j}\leftarrow \mathrm{N}\bar{j}\dot{-}1\). Entonces
\(\displaystyle \begin{array}{rcl} s(i,x,y,\mathcal{P}) & =& i+1 \\ S_{\#}(i,x,y,\mathcal{P}) & =& F_{\dot{-}}(x,\#Var1(I_{i}^{\mathcal{P}})) \\ S_{\ast }(i,x,y,\mathcal{P}) & =& y \end{array} \)

(4) CASO \((\exists j,k\in \omega )\;Bas(I_{i}^{\mathcal{P}})=\mathrm{N }\bar{j}\leftarrow \mathrm{N}\bar{k}\). Entonces
\(\displaystyle \begin{array}{rcl} s(i,x,y,\mathcal{P}) & =& i+1 \\ S_{\#}(i,x,y,\mathcal{P}) & =& F_{\leftarrow }(x,\#Var1(I_{i}^{\mathcal{P} }),\#Var2(I_{i}^{\mathcal{P}})) \\ S_{\ast }(i,x,y,\mathcal{P}) & =& y \end{array} \)

(5) CASO \((\exists j,k\in \omega )\;Bas(I_{i}^{\mathcal{P}})=\mathrm{N }\bar{j}\leftarrow 0\). Entonces
\(\displaystyle \begin{array}{rcl} s(i,x,y,\mathcal{P}) & =& i+1 \\ S_{\#}(i,x,y,\mathcal{P}) & =& F_{0}(x,\#Var1(I_{i}^{\mathcal{P}})) \\ S_{\ast }(i,x,y,\mathcal{P}) & =& y \end{array} \)

(6) CASO \((\exists j,m\in \omega )\;\left( Bas(I_{i}^{\mathcal{P}})= \mathrm{IF}\;\mathrm{N}\bar{j}\neq 0\;\mathrm{GOTO}\;\mathrm{L}\bar{m}\wedge (x)_{j}=0\right) \). Entonces
\(\displaystyle \begin{array}{rcl} s(i,x,y,\mathcal{P}) & =& i+1 \\ S_{\#}(i,x,y,\mathcal{P}) & =& x \\ S_{\ast }(i,x,y,\mathcal{P}) & =& y \end{array} \)

(7) CASO \((\exists j,m\in \omega )\;\left( Bas(I_{i}^{\mathcal{P}})= \mathrm{IF}\;\mathrm{N}\bar{j}\neq 0\;\mathrm{GOTO}\;\mathrm{L}\bar{m}\wedge (x)_{j}\neq 0\right) \). Entonces
\(\displaystyle \begin{array}{rcl} s(i,x,y,\mathcal{P}) & =& \min_{l}\left( Lab(I_{l}^{\mathcal{P}})\neq \varepsilon \wedge Lab(I_{l}^{\mathcal{P}})\text{ }\mathrm{t}\text{ { -final} }I_{i}^{\mathcal{P}}\right) \\ S_{\#}(i,x,y,\mathcal{P}) & =& x \\ S_{\ast }(i,x,y,\mathcal{P}) & =& y \end{array} \)

(8) CASO \((\exists j\in \omega )\;Bas(I_{i}^{\mathcal{P}})=\mathrm{P} \bar{j}\leftarrow \mathrm{P}\bar{j}.a\). Entonces
\(\displaystyle \begin{array}{rcl} s(i,x,y,\mathcal{P}) & =& i+1 \\ S_{\#}(i,x,y,\mathcal{P}) & =& x \\ S_{\ast }(i,x,y,\mathcal{P}) & =& F_{a}(y,\#Var1(I_{i}^{\mathcal{P}})) \end{array} \)

(9) CASO \((\exists j\in \omega )\;Bas(I_{i}^{\mathcal{P}})=\mathrm{P} \bar{j}\leftarrow \) \(^{\curvearrowright }\mathrm{P}\bar{j}\). Entonces
\(\displaystyle \begin{array}{rcl} s(i,x,y,\mathcal{P}) & =& i+1 \\ S_{\#}(i,x,y,\mathcal{P}) & =& x \\ S_{\ast }(i,x,y,\mathcal{P}) & =& F_{\curvearrowright }(y,\#Var1(I_{i}^{ \mathcal{P}})) \end{array} \)

(10) CASO \((\exists j,k\in \omega )\;Bas(I_{i}^{\mathcal{P}})=\mathrm{ P}\bar{j}\leftarrow \mathrm{P}\bar{k}\). Entonces
\(\displaystyle \begin{array}{rcl} s(i,x,y,\mathcal{P}) & =& i+1 \\ S_{\#}(i,x,y,\mathcal{P}) & =& x \\ S_{\ast }(i,x,y,\mathcal{P}) & =& F_{\leftarrow }(y,\#Var1(I_{i}^{\mathcal{P} }),\#Var2(I_{i}^{\mathcal{P}})) \end{array} \)

(11) CASO \((\exists j\in \omega )\;Bas(I_{i}^{\mathcal{P}})=\mathrm{P} \bar{j}\leftarrow \varepsilon \). Entonces
\(\displaystyle \begin{array}{rcl} s(i,x,y,\mathcal{P}) & =& i+1 \\ S_{\#}(i,x,y,\mathcal{P}) & =& x \\ S_{\ast }(i,x,y,\mathcal{P}) & =& F_{0}(y,\#Var1(I_{i}^{\mathcal{P}})) \end{array} \)

(12) CASO \((\exists j,m\in \omega )(\exists a\in \Sigma )\;\left( Bas(I_{i}^{\mathcal{P}})=\mathrm{IF}\;\mathrm{P}\bar{j}\;\mathrm{BEGINS}\;a\; \mathrm{GOTO}\;\mathrm{L}\bar{m}\wedge \lbrack \ast ^{< }((y)_{j})]_{1}\neq a\right) \). Entonces
\(\displaystyle \begin{array}{rcl} s(i,x,y,\mathcal{P}) & =& i+1 \\ S_{\#}(i,x,y,\mathcal{P}) & =& x \\ S_{\ast }(i,x,y,\mathcal{P}) & =& y \end{array} \)

(13) CASO \((\exists j,m\in \omega )(\exists a\in \Sigma )\;\left( Bas(I_{i}^{\mathcal{P}})=\mathrm{IF\;P}\bar{j}\;\mathrm{BEGINS\;}a\;\mathrm{ GOTO\;L}\bar{m}\wedge \lbrack \ast ^{< }((y)_{j})]_{1}=a\right) \). Entonces
\(\displaystyle \begin{array}{rcl} s(i,x,y,\mathcal{P}) & =& \min_{l}\left( Lab(I_{l}^{\mathcal{P}})\neq \varepsilon \wedge Lab(I_{l}^{\mathcal{P}})\text{ }\mathrm{t}\text{ { -final} }I_{i}^{\mathcal{P}}\right) \\ S_{\#}(i,x,y,\mathcal{P}) & =& x \\ S_{\ast }(i,x,y,\mathcal{P}) & =& y \end{array} \)

(14) CASO \((\exists j\in \omega )\;Bas(I_{i}^{\mathcal{P}})=\mathrm{ GOTO}\) \(\mathrm{L}\bar{j}\). Entonces
\(\displaystyle \begin{array}{rcl} s(i,x,y,\mathcal{P}) & =& \min_{l}\left( Lab(I_{l}^{\mathcal{P}})\neq \varepsilon \wedge Lab(I_{l}^{\mathcal{P}})\text{ }\mathrm{t}\text{ { -final} }I_{i}^{\mathcal{P}}\right) \\ S_{\#}(i,x,y,\mathcal{P}) & =& x \\ S_{\ast }(i,x,y,\mathcal{P}) & =& y \end{array} \)

(15) CASO \(Bas(I_{i}^{\mathcal{P}})=\mathrm{SKIP}\). Entonces
\(\displaystyle \begin{array}{rcl} s(i,x,y,\mathcal{P}) & =& k+1 \\ S_{\#}(i,x,y,\mathcal{P}) & =& x \\ S_{\ast }(i,x,y,\mathcal{P}) & =& y \end{array} \)

O sea que los casos anteriores nos permiten definir conjuntos \( S_{1},...,S_{15}\), los cuales son disjuntos de a pares y cuya union da el conjunto \(\omega \times \mathbf{N}\times \mathbf{N}\times \mathrm{Pro} ^{\Sigma }\), de manera que cada una de las funciones \(s,S_{\#}\) y \(S_{\ast }\) pueden escribirse como union disjunta de funciones \((\Sigma \cup \Sigma _{p}) \)-p.r. restrinjidas respectivamente a los conjuntos \(S_{1},...,S_{15}\) . Ya que los conjuntos \(S_{1},...,S_{15}\) son \((\Sigma \cup \Sigma _{p})\) -p.r. el Lema 35 nos dice que \(s,S_{\#}\) y \(S_{\ast }\) lo son. \(\Box\)


\textbf{\underline{Lemma 63:}} Sean: ... hacer!!

\PROOF Hacer

\textbf{\underline{Proposition 64:}} Sean $n, m \leq 0$, las funciones $i^{n, m}, E_{\#j}^{n, m}, j = 1, 2, ...$ son
  $\Sigma \cup \Sigma_{p}$-PR.

\PROOF Sea $<$ un orden total estricto sobre $\Sigma \cup \Sigma_{p}$ y sean $s, S_{\#}, S_{\*}$ las funciones previamente definidas en el Lemma 62, definamos:

\(\displaystyle \begin{array}{rcl} C_{\#}^{n,m} & =& \lambda t\vec{x}\vec{\alpha}\mathcal{P}\left[ \left\langle E_{\#1}^{n,m}(t,\vec{x},\vec{\alpha},\mathcal{P}),E_{\#2}^{n,m}(t,\vec{x}, \vec{\alpha},\mathcal{P}),...\right\rangle \right] \\ C_{\ast }^{n,m} & =& \lambda t\vec{x}\vec{\alpha}\mathcal{P}\left[ \left\langle \#^{< }(E_{\ast 1}^{n,m}(t,\vec{x},\vec{\alpha},\mathcal{P} )),\#^{< }(E_{\ast 2}^{n,m}(t,\vec{x},\vec{\alpha},\mathcal{P} )),...\right\rangle \right] \end{array} \)

Notese que
\(\displaystyle \begin{array}{rcl} i^{n,m}(0,\vec{x},\vec{\alpha},\mathcal{P}) & =& 1 \\ C_{\#}^{n,m}(0,\vec{x},\vec{\alpha},\mathcal{P}) & =& \left\langle x_{1},...,x_{n}\right\rangle \\ C_{\ast }^{n,m}(0,\vec{x},\vec{\alpha},\mathcal{P}) & =& \left\langle \#^{< }(\alpha _{1}),...,\#^{< }(\alpha _{m})\right\rangle \\ i^{n,m}(t+1,\vec{x},\vec{\alpha},\mathcal{P}) & =& s(i^{n,m}(t,\vec{x},\vec{ \alpha},\mathcal{P}),C_{\#}^{n,m}(t,\vec{x},\vec{\alpha},\mathcal{P} ),C_{\ast }^{n,m}(t,\vec{x},\vec{\alpha},\mathcal{P})) \\ C_{\#}^{n,m}(t+1,\vec{x},\vec{\alpha},\mathcal{P}) & =& S_{\#}(i^{n,m}(t,\vec{x },\vec{\alpha},\mathcal{P}),C_{\#}^{n,m}(t,\vec{x},\vec{\alpha},\mathcal{P} ),C_{\ast }^{n,m}(t,\vec{x},\vec{\alpha},\mathcal{P})) \\ C_{\ast }^{n,m}(t+1,\vec{x},\vec{\alpha},\mathcal{P}) & =& S_{\ast }(i^{n,m}(t, \vec{x},\vec{\alpha},\mathcal{P}),C_{\#}^{n,m}(t,\vec{x},\vec{\alpha}, \mathcal{P}),C_{\ast }^{n,m}(t,\vec{x},\vec{\alpha},\mathcal{P})) \end{array} \)

Por el Lema 63 tenemos que \(i^{n,m}\), \( C_{\#}^{n,m}\) y \(C_{\ast }^{n,m}\) son \((\Sigma \cup \Sigma _{p})\)-p.r.. Ademas notese que
\(\displaystyle \begin{array}{rcl} E_{\#j}^{n,m} & =& \lambda t\vec{x}\vec{\alpha}\mathcal{P}\left[ (C_{\#}^{n,m}(t,\vec{x},\vec{\alpha},\mathcal{P}))_{j}\right] \\ E_{\ast j}^{n,m} & =& \lambda t\vec{x}\vec{\alpha}\mathcal{P}\left[ \ast ^{< }((C_{\ast }^{n,m}(t,\vec{x},\vec{\alpha},\mathcal{P}))_{j})\right] \end{array} \)

por lo cual las funciones \(E_{\#j}^{n,m}\), \(E_{\ast j}^{n,m}\), \(j=1,2,...\), son \((\Sigma \cup \Sigma _{p})\)-p.r. \(\Box\)


\textbf{\underline{Theorem 65:}} Las funciones \(\Phi _{\#}^{n,m}\) y \(\Phi _{\ast }^{n,m}\) son \((\Sigma \cup \Sigma _{p})\)-recursivas.

\PROOF Veremos que \(\Phi _{\#}^{n,m}\) es \((\Sigma \cup \Sigma _{p})\)-recursiva. Sea \(H\) el predicado \((\Sigma \cup \Sigma _{p})\)-mixto

\(\displaystyle \lambda t\vec{x}\vec{\alpha}\mathcal{P}\left[ i^{n,m}(t,x_{1},...,x_{n}, \alpha _{1},...,\alpha _{m},\mathcal{P})=n(\mathcal{P})+1\right] \text{.} \)

Note que \(D_{H}=\omega ^{n+1}\times \Sigma ^{\ast m}\times \mathrm{Pro} ^{\Sigma }\). Ya que the functiones \(i^{n,m}\) y \(\lambda \mathcal{P}\left[ n( \mathcal{P})\right] \) son \((\Sigma \cup \Sigma _{p})\)-p.r., \(H\) lo es. Notar que \(D_{M(H)}=D_{\Phi _{\#}^{n,m}}\). Ademas para \((\vec{x},\vec{\alpha}, \mathcal{P})\in D_{M(H)}\), tenemos que \(M(H)(\vec{x},\vec{\alpha},\mathcal{P} )\) es la menor cantidad de pasos necesarios para que \(\mathcal{P}\) termine partiendo del estado \(((x_{1},...,x_{n},0,0,...),(\alpha _{1},...,\alpha _{m},\varepsilon ,\varepsilon ,...))\). Ya que \(H\) es \((\Sigma \cup \Sigma _{p})\)-p.r., tenemos que \(M(H)\) es \((\Sigma \cup \Sigma _{p})\)-r.. Notese que para \((\vec{x},\vec{\alpha},\mathcal{P})\in D_{M(H)}=D_{\Phi _{\#}^{n,m}} \) tenemos que
\(\displaystyle \Phi _{\#}^{n,m}(\vec{x},\vec{\alpha},\mathcal{P})=E_{\#1}^{n,m}\left( M(H)( \vec{x},\vec{\alpha},\mathcal{P}),\vec{x},\vec{\alpha},\mathcal{P}\right) \)

lo cual con un poco mas de trabajo nos permite probar que
\(\displaystyle \Phi _{\#}^{n,m}=E_{\#1}^{n,m}\circ \left( M(H),p_{1}^{n,m+1},...,p_{n+m+1}^{n,m+1}\right) \)

Ya que la funcion \(E_{\#1}^{n,m}\) es \((\Sigma \cup \Sigma _{p})\)-r., lo es \( \Phi _{\#}^{n,m}\). \(\Box\)



\textbf{\underline{Corollary 66:}} Si \(f:D_{f}\subseteq \omega ^{n}\times \Sigma ^{\ast m}\rightarrow O\) es \( \Sigma \)-computable, entonces \(f\) es \(\Sigma \)-recursiva.

\PROOF Haremos el caso \(O=\Sigma ^{\ast }\). Sea \(\mathcal{P}_{0}\) un programa que compute a \(f\). Primero veremos que \(f\) es \((\Sigma \cup \Sigma _{p})\) -recursiva. Note que

\(\displaystyle f=\Phi _{\ast }^{n,m}\circ \left( p_{1}^{n,m},...,p_{n+m}^{n,m},C_{\mathcal{P }_{0}}^{n,m}\right) \)

donde cabe destacar que \(p_{1}^{n,m},...,p_{n+m}^{n,m}\) son las proyecciones respecto del alfabeto \(\Sigma \cup \Sigma _{p}\), es decir que tienen dominio \(\omega ^{n}\times (\Sigma \cup \Sigma _{p})^{\ast m}\). Ya que \(\Phi _{\ast }^{n,m}\) es \((\Sigma \cup \Sigma _{p})\)-recursiva tenemos que \(f\) lo es. O sea que el Teorema 51 nos dice que \(f\) es \(\Sigma \) -recursiva. \(\Box\)


\textbf{\underline{Tesis de Church:}} Toda funcion \(\Sigma \)-efectivamente computable es \(\Sigma \)-recursiva.


\textbf{\underline{Corollary 67:}} Si \(f:D_{f}\subseteq \omega ^{n}\times \Sigma ^{\ast m}\rightarrow O\) es \( \Sigma \)-recursiva, entonces existe un predicado \(\Sigma \)-p.r. \(P:\mathbf{N} \times \omega ^{n}\times \Sigma ^{\ast m}\rightarrow \omega \) y una funcion \( \Sigma \)-p.r. \(g:\mathbf{N}\rightarrow O\) tales que \(f=g\circ M(P).\)

\PROOF Supongamos que \(O=\Sigma ^{\ast }\). Sea \(\mathcal{P}_{0}\) un programa el cual compute a \(f\). Sea \(< \) un orden total estricto sobre \(\Sigma \). Note que podemos tomar

\(\displaystyle \begin{array}{rcl} P & =& \lambda t\vec{x}\vec{\alpha}[i^{n,m}\left( (t)_{1},\vec{x},\vec{\alpha}, \mathcal{P}_{0}\right) =n(\mathcal{P}_{0})+1\wedge (t)_{2}=\#^{< }(E_{\ast 1}^{n,m}((t)_{1},\vec{x},\vec{\alpha},\mathcal{P}_{0}))] \\ g & =& \lambda t\left[ \ast ^{< }((t)_{2})\right] \text{.} \end{array} \)

(Justifique por que \(P\) es \(\Sigma \)-p.r..) \(\Box\)


\textbf{\underline{Lemma 68:}} Supongamos \(f_{i}:D_{f_{i}}\subseteq \omega ^{n}\times \Sigma ^{\ast m}\rightarrow O\), \(i=1,...,k\), son funciones \(\Sigma \)-recursivas tales que \(D_{f_{i}}\cap D_{f_{j}}=\varnothing \) para \(i\neq j\). Entonces la funcion \(f_{1}\cup ...\cup f_{k}\) es \(\Sigma \)-recursiva.

\PROOF Probaremos el caso \(k=2\) y \(O=\Sigma ^{\ast }\). Sean \(\mathcal{P}_{1}\) y \( \mathcal{P}_{2}\) programas que computen las funciones \(f_{1}\) y \(f_{2}\), respectivamente. Sean

\(\displaystyle \begin{array}{rcl} P_{1} & =& \lambda t\vec{x}\vec{\alpha}\left[ i^{n,m}(t,\vec{x},\vec{\alpha}, \mathcal{P}_{1})=n(\mathcal{P}_{1})+1\right] \\ P_{2} & =& \lambda t\vec{x}\vec{\alpha}\left[ i^{n,m}(t,\vec{x},\vec{\alpha}, \mathcal{P}_{2})=n(\mathcal{P}_{2})+1\right] \end{array} \)

Notese que \(D_{P_{1}}=D_{P_{2}}=\omega \times \omega ^{n}\times \Sigma ^{\ast m}\) y que \(P_{1}\) y \(P_{2}\) son \((\Sigma \cup \Sigma _{p})\)-p.r.. Ya que son \(\Sigma \)-mixtos, el Teorema 51 nos dice que son \( \Sigma \)-p.r.. Tambien notese que \(D_{M((P_{1}\vee P_{2}))}=D_{f_{1}}\cup D_{f_{2}}\). Definamos
\(\displaystyle \begin{array}{rcl} g_{1} & =& \lambda \vec{x}\vec{\alpha}\left[ E_{\ast 1}^{n,m}(M\left( (P_{1}\vee P_{2})\right) (\vec{x},\vec{\alpha}),\vec{x},\vec{\alpha}, \mathcal{P}_{1})^{P_{i}(M\left( (P_{1}\vee P_{2})\right) (\vec{x},\vec{\alpha }),\vec{x},\vec{\alpha})}\right] \\ g_{2} & =& \lambda \vec{x}\vec{\alpha}\left[ E_{\ast 1}^{n,m}(M\left( (P_{1}\vee P_{2})\right) (\vec{x},\vec{\alpha}),\vec{x},\vec{\alpha}, \mathcal{P}_{2})^{P_{i}(M\left( (P_{1}\vee P_{2})\right) (\vec{x},\vec{\alpha }),\vec{x},\vec{\alpha})}\right] \end{array} \)

Notese que \(g_{1}\) y \(g_{2}\) son \(\Sigma \)-recursivas y que \( D_{g_{1}}=D_{g_{2}}=D_{f_{1}}\cup D_{f_{2}}\), Ademas notese que
\(\displaystyle g_{1}(\vec{x},\vec{\alpha})=\left\{ \begin{array}{lll} f_{1}(\vec{x},\vec{\alpha}) & & \text{si }(\vec{x},\vec{\alpha})\in D_{f_{1}} \\ \varepsilon & & \text{caso contrario} \end{array} \right. \)

\(\displaystyle g_{2}(\vec{x},\vec{\alpha})=\left\{ \begin{array}{lll} f_{2}(\vec{x},\vec{\alpha}) & & \text{si }(\vec{x},\vec{\alpha})\in D_{f_{2}} \\ \varepsilon & & \text{caso contrario} \end{array} \right. \)

O sea que \(f_{1}\cup f_{2}=\lambda \alpha \beta \left[ \alpha \beta \right] \circ (g_{1},g_{2})\) es \(\Sigma \)-recursiva. \(\Box\)


\textbf{\underline{Lemma 69:}} Supongamos \(\Sigma \supseteq \Sigma _{p}\). Entonces \( Halt^{\Sigma }\) es no \(\Sigma \)-recursivo.

\PROOF Supongamos \(Halt^{\Sigma }\) es \(\Sigma \)-recursivo y por lo tanto \(\Sigma \) -computable. Por la proposicion de existencia de macros tenemos que hay un macro

\(\displaystyle \left[ \mathrm{IF}\;Halt^{\Sigma }(\mathrm{W}1)\;\mathrm{GOTO}\;\mathrm{A}1 \right] \)

Sea \(\mathcal{P}_{0}\) el siguiente programa de \(\mathcal{S}^{\Sigma }\)
\(\displaystyle \mathrm{L}1\;\left[ \mathrm{IF}\;Halt^{\Sigma }(\mathrm{P}1)\;\mathrm{GOTO}\; \mathrm{L}1\right] \)

Note que
- \(\mathcal{P}_{0}\) termina partiendo desde \(\left( (0,0,...),( \mathcal{P}_{0},\varepsilon ,\varepsilon ,...)\right) \) sii \(Halt^{\Sigma }( \mathcal{P}_{0})=0\),
lo cual produce una contradiccion si tomamos en (*) \(\mathcal{P}= \mathcal{P}_{0}\). \(\Box\)


\textbf{\underline{Theorem 70:}} Sea \(S\subseteq \omega ^{n}\times \Sigma ^{\ast m}\). Entonces \(S\) es \(\Sigma \)-efectivamente enumerable sii \(S\) es \(\Sigma \)-recursivamente enumerable


\PROOF (\(\Rightarrow \)) Use la Tesis de Church.

(\(\Leftarrow \)) Use el Theorem 42. \(\Box\)



\textbf{\underline{Theorem 71:}} Dado \(S\subseteq \omega ^{n}\times \Sigma ^{\ast m} \), son equivalentes
(1) \(S\) es \(\Sigma \)-recursivamente enumerable
(2) \(S=I_{F}\), para alguna \(F:D_{F}\subseteq \omega ^{k}\times \Sigma ^{\ast l}\rightarrow \omega ^{n}\times \Sigma ^{\ast m}\) tal que cada \(F_{i}\) es \(\Sigma \)-recursiva.
(3) \(S=D_{f}\), para alguna funcion \(\Sigma \)-recursiva \(f\)
(4) \(S=\varnothing \) o \(S=I_{F}\), para alguna \(F:\omega \rightarrow \omega ^{n}\times \Sigma ^{\ast m}\) tal que cada \(F_{i}\) es \(\Sigma \)-p.r.


\PROOF (2)\(\Rightarrow \)(3). Para \(i=1,...,n+m\), sea \(\mathcal{P}_{i}\) un programa el cual computa a \(F_{i}\) y sea \(< \) un orden total estricto sobre \(\Sigma \). Sea \(P:\mathbf{N}\times \omega ^{n}\times \Sigma ^{\ast m}\rightarrow \omega \) dado por \(P(t,\vec{x},\vec{\alpha})=1\) sii se cumplen las siguientes condiciones

\(\displaystyle \begin{array}{rcl} i^{k,l}(\left( (t)_{k+l+1},(t)_{1},...,(t)_{k},\ast ^{< }((t)_{k+1}),...,\ast ^{< }((t)_{k+l})),\mathcal{P}_{1}\right) & =& n(\mathcal{P}_{1})+1 \\ & & \vdots \\ i\left( (t)_{k+l+1},(t)_{1}...(t)_{k},\ast ^{< }((t)_{k+1})...\ast ^{< }((t)_{k+l})),\mathcal{P}_{n+m}\right) & =& n(\mathcal{P}_{n+m})+1 \\ E_{\#1}^{k,l}((t)_{k+l+1},(t)_{1},...,(t)_{k},\ast ^{< }((t)_{k+1}),...,\ast ^{< }((t)_{k+l})),\mathcal{P}_{1}) & =& x_{1} \\ & & \vdots \\ E_{\#1}^{k,l}((t)_{k+l+1},(t)_{1},...,(t)_{k},\ast ^{< }((t)_{k+1}),...,\ast ^{< }((t)_{k+l})),\mathcal{P}_{n}) & =& x_{n} \\ E_{\ast 1}^{k,l}((t)_{k+l+1},(t)_{1},...,(t)_{k},\ast ^{< }((t)_{k+1}),...,\ast ^{< }((t)_{k+l})),\mathcal{P}_{n+1}) & =& \alpha _{1} \\ & & \vdots \\ E_{\ast 1}^{k,l}((t)_{k+l+1},(t)_{1},...,(t)_{k},\ast ^{< }((t)_{k+1}),...,\ast ^{< }((t)_{k+l})),\mathcal{P}_{n+m}) & =& \alpha _{m} \end{array} \)

Note que \(P\) es \((\Sigma \cup \Sigma _{p})\)-p.r. y por lo tanto \(P\) es \( \Sigma \)-p.r.. Pero entonces \(M(P)\) es \(\Sigma \)-r. lo cual nos dice que se cumple (3) ya que \(D_{M(P)}=I_{F}=S\).
(3)\(\Rightarrow \)(4). Supongamos \(S\neq \varnothing \). Sea \( (z_{1},...,z_{n},\gamma _{1},...,\gamma _{m})\in S\) fijo. Sea \(\mathcal{P}\) un programa el cual compute a \(f\) y sea \(< \) un orden total estricto sobre \( \Sigma \). Sea \(P:\mathbf{N}\rightarrow \omega \) dado por \(P(x)=1\) sii

\(\displaystyle i^{n,m}\left( (x)_{n+m+1},(x)_{1},...,(x)_{n},\ast ^{< }((x)_{n+1}),...,\ast ^{< }((x)_{n+m})),\mathcal{P}\right) =n(\mathcal{P})+1 \)

Es facil ver que \(P\) es \((\Sigma \cup \Sigma _{p})\)-p.r. por lo cual es \( \Sigma \)-p.r.. Sea \(\bar{P}=P\cup C_{0}^{1,0}\mid _{\{0\}}\). Para \(i=1,...,n\) , definamos \(F_{i}:\omega \rightarrow \omega \) de la siguiente manera
\(\displaystyle F_{i}(x)=\left\{ \begin{array}{ccc} (x)_{i} & \text{si} & \bar{P}(x)=1 \\ z_{i} & \text{si} & \bar{P}(x)\neq 1 \end{array} \right. \)

Para \(i=n+1,...,n+m\), definamos \(F_{i}:\omega \rightarrow \Sigma ^{\ast }\) de la siguiente manera
\(\displaystyle F_{i}(x)=\left\{ \begin{array}{lll} \ast ^{< }((x)_{i}) & \text{si} & \bar{P}(x)=1 \\ \gamma _{i-n} & \text{si} & \bar{P}(x)\neq 1 \end{array} \right. \)

Por el lema de division por casos, cada \(F_{i}\) es \(\Sigma \)-p.r.. Es facil ver que \(F=(F_{1},...,F_{n+m})\) cumple (4). \(\Box\)


\textbf{\underline{Corollary 72:}} Supongamos \(f:D_{f}\subseteq \omega ^{n}\times \Sigma ^{\ast m}\rightarrow O\) es \(\Sigma \)-recursiva y \(S\subseteq D_{f}\) es \( \Sigma \)-r.e., entonces \(f\mid _{S}\) es \(\Sigma \)-recursiva.


\PROOF Supongamos \(O=\Sigma ^{\ast }.\) Por el Theorem anterior \(S=D_{g}\), para alguna funcion \(\Sigma \)-recursiva \(g.\) Notese que componiendo adecuadamente podemos suponer que \(I_{g}=\{\varepsilon \}.\) O sea que tenemos \(f\mid _{S}=\lambda \alpha \beta \left[ \alpha \beta \right] \circ (f,g)\). \(\Box\)


\textbf{\underline{Corollary 73:}} Supongamos \(f:D_{f}\subseteq \omega ^{n}\times \Sigma ^{\ast m}\rightarrow O\) es \(\Sigma \)-recursiva y \(S\subseteq I_{f}\) es \(\Sigma \)-r.e., entonces \( f^{-1}(S)=\{(\vec{x},\vec{\alpha}):f(\vec{x},\vec{\alpha})\in S\}\) es \( \Sigma \)-r.e..

\PROOF Por el Theorem anterior \(S=D_{g}\), para alguna funcion \(\Sigma \)-recursiva \( g \). O sea que \(f^{-1}(S)=D_{g\circ f}\) es \(\Sigma \)-r.e.. \(\Box\)


\textbf{\underline{Corollary 74:}} Supongamos \(S_{1},S_{2}\subseteq \omega ^{n}\times \Sigma ^{\ast m}\) son conjuntos \(\Sigma \)-r.e.. Entonces \(S_{1}\cap S_{2}\) es \(\Sigma \)-r.e..

\PROOF Por el Theorem anterior \(S_{i}=D_{g_{i}}\), con \(g_{1},g_{2}\) funciones \( \Sigma \)-recursivas\(.\) Notese que podemos suponer que \(I_{g_{1}},I_{g_{2}} \subseteq \omega \) por lo que \(S_{1}\cap S_{2}=D_{\lambda xy\left[ xy\right] \circ (g_{1},g_{2})}\) es \(\Sigma \)-r.e.\(.\) \(\Box\)


\textbf{\underline{Corollary 75:}} Supongamos \(S_{1},S_{2}\subseteq \omega ^{n}\times \Sigma ^{\ast m}\) son conjuntos \(\Sigma \)-r.e.. Entonces \(S_{1}\cup S_{2}\) es \(\Sigma \)-r.e.

\PROOF Supongamos \(S_{1}\neq \varnothing \neq S_{2}.\) Sean \(F,G:\omega \rightarrow \omega ^{n}\times \Sigma ^{\ast m}\) tales que \(I_{F}=S_{1}\), \(I_{G}=S_{2}\) y las funciones \(F_{i} {\acute{}} s\) y \(G_{i} {\acute{}} s\) son \(\Sigma \)-recursivas. Sean \(f=\lambda x\left[ Q(x,2)\right] \) y \( g=\lambda x\left[ Q(x\dot{-}1,2)\right] .\) Sea \(H:\omega \rightarrow \omega ^{n}\times \Sigma ^{\ast m}\) dada por

\(\displaystyle H_{i}=(F_{i}\circ f)\mathrm{\mid }_{\{x:x\mathrm{\ es\ par}\}}\cup (G_{i}\circ g)\mathrm{\mid }_{\{x:x\mathrm{\ es\ impar}\}} \)

Por el Corollary 72 y el Lema 68, cada \(H_{i}\) es \( \Sigma \)-recursiva. Ya que \(I_{H}=S_{1}\cup S_{2}\).tenemos que \(S_{1}\cup S_{2}\) es \(\Sigma \)-r.e. \(\Box\)


\textbf{\underline{Theorem 76:}} Sea \(S\subseteq \omega ^{n}\times \Sigma ^{\ast m}\). Entonces \(S\) es \(\Sigma \)-efectivamente computable sii \(S\) es \(\Sigma \)-recursivo
\PROOF (\(\Rightarrow \)) Use la Tesis de Church.

(\(\Leftarrow \)) Use el Teorema 42. \(\Box\)




\textbf{\underline{Theorem 77:}} Sea \(S\subseteq \omega ^{n}\times \Sigma ^{\ast m}.\) Son equivalentes
(a) \(S\) es \(\Sigma \)-recursivo
(b) \(S\) y \((\omega ^{n}\times \Sigma ^{\ast m})-S\) son \(\Sigma \) -recursivamente enumerables


\PROOF (a)\(\Rightarrow \)(b)\(.\) Note que \(S=D_{Pred\circ \chi _{S}}.\)

(b)\(\Rightarrow \)(a). Note que \(\chi _{S}=C_{1}^{n,m}\mathrm{\mid }_{S}\cup C_{0}^{n,m}\mathrm{\mid }_{\omega ^{n}\times \Sigma ^{\ast m}-S}\). \(\Box\)

\textbf{\underline{Lemma 78:}} Supongamos que \(\Sigma \supseteq \Sigma _{p}.\) Entonces
\(\displaystyle A=\left\{ \mathcal{P}\in \mathrm{Pro}^{\Sigma }:Halt^{\Sigma }(\mathcal{P} )\right\} \)

es \(\Sigma \)-r.e. y no es \(\Sigma \)-recursivo. Mas aun el conjunto
\(\displaystyle N=\left\{ \mathcal{P}\in \mathrm{Pro}^{\Sigma }:\lnot Halt^{\Sigma }( \mathcal{P})\right\} \)
no es \(\Sigma \)-r.e.


\PROOF Sea \(P=\lambda t\mathcal{P}\left[ i^{0,1}(t,\mathcal{P},\mathcal{P})=n( \mathcal{P})+1\right] \). Note que \(P\) es \(\Sigma \)-p.r. por lo que \(M(P)\) es \(\Sigma \)-r.. Ademas note que \(D_{M(P)}=A\), lo cual implica que \(A\) es \( \Sigma \)-r.e.. Ya que \(Halt^{\Sigma }\) es no \(\Sigma \)-recursivo (Lema 69) y

\(\displaystyle Halt^{\Sigma }=C_{1}^{0,1}\mid _{A}\cup C_{0}^{0,1}\mid _{N} \)

el Lema 68 nos dice que \(N\) no es \(\Sigma \)-r.e.. Finalmente supongamos \(A\) es \(\Sigma \)-recursivo. Entonces el conjunto
\(\displaystyle N=\left( \Sigma ^{\ast }-A\right) \cap \mathrm{Pro}^{\Sigma } \)

deberia serlo, lo cual es absurdo. \(\Box\)
