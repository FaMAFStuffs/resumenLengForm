\section{El lenguaje ${S}^{\Sigma}$}

  % Lemma 52
  \begin{lemma} Se tiene que:
    \begin{itemize}
      \item[(a)]  Si $I_{1}...I_{n}=J_{1}...J_{m}$, con $ I_{1},...,I_{n},J_{1},...,J_{m}\in \mathrm{Ins}^{\Sigma }$,
                  entonces $n=m$ y $I_{j}=J_{j}$ para cada $j\geq 1$.
      \item[(b)]  Si $\mathcal{P}\in \mathrm{Pro}^{\Sigma }$, entonces existe una única sucesión de instrucciones
                  $I_{1},...,I_{n}$ tal que $\mathcal{P} =I_{1}...I_{n}$
    \end{itemize}
  \end{lemma}

  % Theorem 53
  \begin{theorem}
    Si $f$ es $\Sigma $-computable, entonces $f$ es $\Sigma $-efectivamente computable.
  \begin{proof}
    Supongamos por ejemplo que $f:S\subseteq \omega ^{n}\times \Sigma ^{\ast m}\rightarrow \omega $ es computada por
    $\mathcal{P}\in \mathrm{Pro}^{\Sigma }$. Es claro que el procedimiento que consiste en realizar las sucesivas
    instrucciones de $\mathcal{P}$ (partiendo de $((x_{1},...,x_{n},0,0,...),( \alpha _{1},...,\alpha _{m},
    \varepsilon ,\varepsilon ,...))$) y eventualmente terminar en caso de que nos toque realizar la instrucción
    $n( \mathcal{P})+1$, y dar como salida el contenido de la variable $\mathrm{N}1$ , es un procedimiento efectivo
    que computa a $f$.
  \end{proof}
  \end{theorem}

  % Proposition 54
  \begin{proposition}
    \begin{itemize}
      \item[(a)]  Sea $f:S\subseteq \omega ^{n}\times \Sigma ^{\ast m}\rightarrow \omega $ una función
                  $\Sigma $-computable. Entonces hay un macro
                  \[
                  \displaystyle \left[ \mathrm{V}\overline{n+1}\leftarrow f(\mathrm{V}1,...,\mathrm{V}\bar{n} ,
                  \mathrm{W}1,...,\mathrm{W}\bar{m})\right]
                  \]
      \item[(b)]  Sea $f:S\subseteq \omega ^{n}\times \Sigma ^{\ast m}\rightarrow \Sigma ^{\ast }$ una función
                  $\Sigma $-computable. Entonces hay un macro
                  \[
                  \displaystyle \left[ \mathrm{W}\overline{m+1}\leftarrow f(\mathrm{V}1,...,\mathrm{V}\bar{n} ,
                  \mathrm{W}1,...,\mathrm{W}\bar{m})\right]
                  \]
    \end{itemize}
  \end{proposition}

  % Proposition 55
  \begin{proposition}
    Sea $P:S\subseteq \omega ^{n}\times \Sigma ^{\ast m}\rightarrow \omega $ un predicado $\Sigma $-computable.
    Entonces hay un macro
    \[
    \displaystyle \left[ \mathrm{IF}\;P(\mathrm{V}1,...,\mathrm{V}\bar{n},\mathrm{W}1,...,
    \mathrm{W}\bar{m})\;\mathrm{GOTO}\;\mathrm{A}1\right]
    \]
  \end{proposition}

  % Theorem 56
  \begin{theorem}
    Si $h$ es $\Sigma $-recursiva, entonces $h$ es $\Sigma $ -computable.

  \begin{proof}
    Probaremos por induccion en $k$ que
    \begin{itemize}
      \item[(*)] Si $h\in \mathrm{R}_{k}^{\Sigma }$, entonces $h$ es $\Sigma $ -computable.
    \end{itemize}
    El caso $k=0$ es dejado al lector.
    Supongamos (*) vale para $k$, veremos que vale para $k+1$.
    Sea $h\in \mathrm{R}_{k+1}^{\Sigma }-\mathrm{R} _{k}^{\Sigma }.$ Hay varios casos
    Supongamos que $\Sigma = \{@, \$\}$

    \noindent Caso 2. Supongamos $h=R(f,\mathcal{G})$, con
    \[
      \displaystyle \begin{array}{rcl}
        f & :& S_{1}\times ...\times S_{n}\times L_{1}\times ...\times L_{m}\rightarrow \Sigma ^{\ast } \\
        \mathcal{G}_{@} & :& S_{1}\times ...\times S_{n}\times L_{1}\times ...\times L_{m}\times \Sigma ^{\ast }
          \times \Sigma ^{\ast }\rightarrow \Sigma ^{\ast } \\
        \mathcal{G}_{\$} & :& S_{1}\times ...\times S_{n}\times L_{1}\times ...\times L_{m}\times \Sigma ^{\ast }
          \times \Sigma ^{\ast }\rightarrow \Sigma ^{\ast }
      \end{array}
    \]
    elementos de $\mathrm{R}_{k}^{\Sigma }$.  Por hipotesis inductiva, las funciones $f$, $\mathcal{G}_{@}$,
    $\mathcal{G}_{\$}$ , son $\Sigma $-computables y por lo tanto podemos hacer el siguiente programa via el uso de
    macros
    \begin{eqnarray*}
      && \qquad\;\;       \left[ \mathrm{P}\overline{m+3} \leftarrow
                              f(\mathrm{N}1,...,\mathrm{N}\bar{ n},\mathrm{P}1,...,\mathrm{P}\bar{m})\right] \\
      && \mathrm{L}3:\;\; \mathrm{IF}\;\mathrm{P}\overline{m+1}\; \text{BEGINS}\;@\; \mathrm{GOTO}\;\mathrm{L}1 \\
      && \qquad\;\;       \mathrm{IF}\;\mathrm{P}\overline{m+1}\; \text{BEGINS}\;\ \$\; \mathrm{GOTO}\;\mathrm{L}1 \\
      && \qquad\;\;       \mathrm{GOTO}\; \mathrm{L}4\\
      && \mathrm{L}1:\;\; \mathrm{P}\overline{m+1} \leftarrow \text{ }^{\curvearrowright } \mathrm{P}\overline{m+1} \\
      && \qquad\;\;       \left[ \mathrm{P}\overline{m+3}\; \leftarrow\; \mathcal{G}_{@}
                                (\mathrm{N} 1,...,\mathrm{N}\bar{n},
                                \mathrm{P}1,...,\mathrm{P}\bar{m},\mathrm{P} \overline{m+2},\mathrm{P}\overline{m+3})
                          \right] \\
      && \qquad\;\;       \mathrm{P}\overline{m+2}\leftarrow \mathrm{P}\overline{m+2}.@  \\
      && \qquad\;\;       \mathrm{GOTO}\;\mathrm{L}3 \\
      && \mathrm{L}2:\;\; \mathrm{P}\overline{m+1} \leftarrow \text{ }^{\curvearrowright } \mathrm{P}\overline{m+1} \\
      && \qquad\;\;       \left[ \mathrm{P}\overline{m+3}\; \leftarrow\; \mathcal{G}_{\$}
                                (\mathrm{N} 1,...,\mathrm{N}\bar{n},
                                \mathrm{P}1,...,\mathrm{P}\bar{m},\mathrm{P} \overline{m+2},\mathrm{P}\overline{m+3})
                          \right] \\
      && \qquad\;\;       \mathrm{P}\overline{m+2}\leftarrow \mathrm{P}\overline{m+2}.\$ \\
      && \qquad\;\;       \mathrm{GOTO}\;\mathrm{L}3 \\
      && \mathrm{L}4:\;\; \mathrm{P}1\leftarrow \mathrm{P}\overline{m+2} \\
    \end{eqnarray*}

    Luego se tiene que el programa computa a $h$, entonces $h$ es $\Sigma$-computable
  \end{proof}
  \end{theorem}

  %Lemma 57
  \begin{lemma}
    Sea $\Sigma $ un alfabeto cualquiera. Las funciones $S$ y $\overline{\ \;}$ son
    $(\Sigma \cup \Sigma _{p})$-p.r..
  \end{lemma}

  % Lemma 58
  \begin{lemma}
    Para cada $n,x\in \omega $, tenemos que $ \left\vert \bar{n}\right\vert \leq x$ si y solo si $n\leq 10^{x}-1$
  \end{lemma}

  % Lemma 59
  \begin{lemma}
    $\mathrm{Ins}^{\Sigma }$ es un conjunto $(\Sigma \cup \Sigma _{p})$-p.r..
  \begin{proof}
    Para simplificar la prueba asumiremos que $\Sigma =\{@,\& \}$. Ya que $ \mathrm{Ins}^{\Sigma }$ es union de los
    siguientes conjuntos

    \[
      \displaystyle
      \begin{array}{rcl}
        L_{1} & =& \left\{ \mathrm{N}\bar{k}\leftarrow \mathrm{N}\bar{k}+1:k\in \mathbf{N}\right\} \\
        L_{2} & =& \left\{ \mathrm{N}\bar{k}\leftarrow \mathrm{N}\bar{k}\dot{-}1:k\in \mathbf{N}\right\} \\
        L_{3} & =& \left\{ \mathrm{N}\bar{k}\leftarrow \mathrm{N}\bar{n}:k,n\in \mathbf{N}\right\} \\
        L_{4} & =& \left\{ \mathrm{N}\bar{k}\leftarrow 0:k\in \mathbf{N}\right\} \\
        L_{5} & =& \left\{ \mathrm{IF}\;\mathrm{N}\bar{k}\neq 0\;\mathrm{GOTO}\;
                           \mathrm{L}\bar{m}:k,m\in \mathbf{N}\right\} \\
        L_{6} & =& \left\{ \mathrm{P}\bar{k}\leftarrow \mathrm{P}\bar{k}.@:k\in \mathbf{N}\right\} \\
        L_{7} & =& \left\{ \mathrm{P}\bar{k}\leftarrow \mathrm{P}\bar{k}.\& :k\in \mathbf{N}\right\} \\
        L_{8} & =& \left\{ \mathrm{P}\bar{k}\leftarrow \text{ }^{\curvearrowright }
                           \mathrm{P}\bar{k}:k\in \mathbf{N}\right\} \\
        L_{9} & =& \left\{ \mathrm{P}\bar{k}\leftarrow \mathrm{P}\bar{n}:k,n\in \mathbf{N}\right\} \\
        L_{10} & =& \left\{ \mathrm{P}\bar{k}\leftarrow \varepsilon :k\in \mathbf{N} \right\} \\
        L_{11} & =& \left\{ \mathrm{IF}\;\mathrm{P}\bar{k}\;\mathrm{BEGINS}\;@\; \mathrm{GOTO}\;
                            \mathrm{L}\bar{m}:k,m\in \mathbf{N}\right\} \\
        L_{12} & =& \left\{ \mathrm{IF}\;\mathrm{P}\bar{k}\;\mathrm{BEGINS}\;\& \; \mathrm{GOTO}\;
                            \mathrm{L}\bar{m}:k,m\in \mathbf{N}\right\} \\
        L_{13} & =& \left\{ \mathrm{GOTO}\;\mathrm{L}\bar{m}:m\in \mathbf{N}\right\} \\
        L_{14} & =& \left\{ \mathrm{SKIP}\right\} \\
        L_{15} & =& \left\{ \mathrm{L}\bar{k}\alpha :k\in \mathbf{N\;}
                            \text{y }\alpha \in L_{1}\cup ...\cup L_{14}\right\}
      \end{array}
    \]

    solo debemos probar que $L_{1},...,L_{15}$ son $(\Sigma \cup \Sigma _{p})$ -p.r.. Veremos primero por ejemplo que
    \[
      \displaystyle L_{11}=\left\{ \mathrm{IFP}\bar{k}\mathrm{BEGINS}@\mathrm{GOTOL}\bar{m} :k,m\in \mathbf{N}\right\}
    \]
    es $(\Sigma \cup \Sigma _{p})$-p.r.. Primero nótese que $\alpha \in L_{11}$ si y solo si existen
    $k,m\in \mathbf{N}$ tales que
    \[
      \displaystyle \alpha =\mathrm{IFP}\bar{k}\mathrm{BEGINS}@\mathrm{GOTOL}\bar{m}
    \]
    Mas formalmente tenemos que $\alpha \in L_{11}$ si y solo si
    \[
      \displaystyle (\exists k\in \mathbf{N})(\exists m\in \mathbf{N})\;\alpha =\mathrm{IFP}\bar{ k}\mathrm{BEGINS}@
      \mathrm{GOTOL}\bar{m}
    \]
    Ya que cuando existen tales $k,m$ tenemos que $\bar{k}$ y $\bar{m}$ son subpalabras de $\alpha $, el
    \textbf{Lemma 58} nos dice que $\alpha \in L_{10}$ si y solo si
    \[
      \displaystyle (\exists k\in \mathbf{N})_{k\leq 10^{\left\vert \alpha \right\vert }}(\exists m\in \mathbf{N})_{m
      \leq 10^{\left\vert \alpha \right\vert }}\;\alpha =\mathrm{IFP}\bar{k}\mathrm{BEGINS}@\mathrm{GOTOL}\bar{m}
    \]
    Sea
    \[
      \displaystyle P=\lambda mk\alpha \left[ \alpha =\mathrm{IFP}\bar{k}\mathrm{BEGINS}@\mathrm{ GOTOL}\bar{m}\right]
    \]
    Ya que $D_{\lambda k\left[ \bar{k}\right] }=\omega $, tenemos que $ D_{P}=\omega \times (\Sigma \cup \Sigma _{p})
    ^{\ast }\times (\Sigma \cup \Sigma _{p})^{\ast }$. Nótese que
    \[
      \displaystyle P=\lambda \alpha \beta \left[ \alpha =\beta \right] \circ \left( p_{3}^{2,1},f\right)
    \]
    donde
    \[
      \displaystyle f=\lambda \alpha _{1}\alpha _{2}\alpha _{3}\alpha _{4}\left[ \alpha _{1}\alpha _{2}\alpha _{3}\alpha
      _{4}\right] \circ \left( C_{\mathrm{IFP} }^{2,1},\lambda k\left[ \bar{k}\right] \circ p_{2}^{2,1},C_{
      \mathrm{BEGINS}@ \mathrm{GOTOL}}^{2,1},\lambda k\left[ \bar{k}\right] \circ p_{1}^{2,1}\right)
    \]
    lo cual nos dice que $P$ es $(\Sigma \cup \Sigma _{p})$-p.r..
    Nótese que
    \[
      \displaystyle \chi _{L_{11}}=\lambda \alpha \left[ (\exists k\in \mathbf{N})_{k\leq 10^{\left\vert \alpha
      \right\vert }}(\exists m\in \mathbf{N})_{m\leq 10^{\left\vert \alpha \right\vert }}\;P(m,k,\alpha )\right]
    \]
    Esto nos dice que podemos usar dos veces el \textbf{Lema 39} para ver que $\chi _{L_{11}}$ es
    $(\Sigma \cup \Sigma _{p})$-p.r.. Veamos como. Sea
    \[
      \displaystyle Q=\lambda k\alpha \left[ (\exists m\in \mathbf{N})_{m\leq 10^{\left\vert \alpha \right\vert }}\;
      P(m,k,\alpha )\right]
    \]
    Por el \textbf{Lema 39} tenemos que
    \[
      \displaystyle \lambda xk\alpha \left[ (\exists m\in \mathbf{N})_{m\leq x}\;P(m,k,\alpha ) \right]
    \]
    es $(\Sigma \cup \Sigma _{p})$-p.r. lo cual nos dice que
    \[
      \displaystyle Q=\lambda xk\alpha \left[ (\exists m\in \mathbf{N})_{m\leq x}\;P(m,k,\alpha ) \right] \circ
      (\lambda \alpha \left[ 10^{\left\vert \alpha \right\vert } \right] \circ p_{2}^{1,1},p_{1}^{1,1},p_{2}^{1,1})
    \]
    lo es. Ya que
    \[
      \displaystyle \chi _{L_{10}}=\lambda \alpha \left[ (\exists k\in \mathbf{N})_{k\leq 10^{\left\vert \alpha \right
      \vert }}\;Q(k,\alpha )\right]
    \]
    podemos en forma similar aplicar el \textbf{Lema 39} y obtener finalmente que $\chi _{L_{11}}$ es
    $(\Sigma \cup \Sigma _{p})$-p.r..
    En forma similar podemos probar que $L_{1},...,L_{14}$ son $(\Sigma \cup \Sigma _{p})$-p.r..
    Esto nos dice que $L_{1}\cup ...\cup L_{14}$ es $(\Sigma \cup \Sigma _{p})$-p.r..
    Nótese que $L_{1}\cup ...\cup L_{14}$ es el conjunto de las instrucciones básicas de $\mathcal{S}^{\Sigma }$.
    Llamemos $ \mathrm{InsBas}^{\Sigma }$ a dicho conjunto.
    Para ver que $L_{15}$ es $ (\Sigma \cup \Sigma _{p})$-p.r. notemos que
    \[
      \displaystyle \chi _{L_{15}}=\lambda \alpha \left[ (\exists k\in \mathbf{N})_{k\leq 10^{\left\vert \alpha \right
      \vert }}(\exists \beta \in \mathrm{InsBas} ^{\Sigma })_{\left\vert \beta \right\vert \leq \left\vert \alpha \right
      \vert }\;\alpha =\mathrm{L}\bar{k}\beta \right]
    \]
    lo cual nos dice que aplicando dos veces el \textbf{Lema 39} obtenemos que $\chi _{L_{15}}$ es $(\Sigma \cup \Sigma _{p})
    $-p.r.. Ya que $ \mathrm{Ins}^{\Sigma }=\mathrm{InsBas}^{\Sigma }\cup L_{15}$ tenemos que
    $ \mathrm{Ins}^{\Sigma }$ es $(\Sigma \cup \Sigma _{p})$-p.r..
  \end{proof}
  \end{lemma}

  % Lemma 60
  \begin{lemma}
    $Bas$ y $Lab$ son funciones $(\Sigma \cup \Sigma _{p})$-p.r.

  \begin{proof}
    Sea $< $ un orden total estricto sobre $\Sigma \cup \Sigma _{p}$. Sea $L=\{ \mathrm{L}\bar{k}:k\in \mathbf{N}\}
    \cup \{\varepsilon \}$. Veamos que $ L $ es $\Sigma \cup \Sigma _{p}$-p.r.
    Sea $\chi_L$
    \[
      \displaystyle \chi_L(\alpha)=\left\{\begin{array}{lll}
                                            1 & & \text{si }\alpha\in L\\
                                            0 & & \text{si }\alpha\notin L
                                          \end{array}\right.
    \]
    Primero nótese que tenemos que $\alpha \in L$ sii
    \[
      \begin{array}{lll}
        (\exists k \in \mathbf{N})\ \alpha = L\bar{k} \lor \alpha = \varepsilon
      \end{array}
    \]
    Ya que cuando existe $k$ tenemos que $\bar{k}$ es una subpalabra de $\alpha $, el \textbf{lema 58} nos dice
    que $\alpha \in L$ si y solo si
    \[
      \begin{array}{lll}
        (\exists k \in \mathbf{N})_{k\leq 10^{\ \left\vert \alpha \right \vert}}\ \alpha = L\bar{k} \lor \alpha = \varepsilon
      \end{array}
    \]
    Sea
    \[
      \begin{array}{lll}
        Q & = & \lambda \alpha\left[
                                \lbrack \alpha = \mathrm{L}\bar{k} \lor \alpha = \varepsilon
                              \right]
      \end{array}
    \]
    Ya que $D_{\lambda k\left[ \bar{k}\right] }=\omega $, tenemos que
    $ D_{Q}=\omega \times (\Sigma \cup \Sigma _{p})^{\ast } $.
    Notese que
    \[
      \begin{array}{lll}
        Q & = & Q_1 \land Q_2
      \end{array}
    \]
    donde
    \[
      \begin{array}{lll}
        \displaystyle Q_1 = \lambda \alpha \beta \left[ \alpha =\beta \right] \circ
        \left( p_{2}^{1,1},
        \lambda \alpha \beta \left[ \alpha\beta \right] \circ (C_{L}^{1,1},\lambda k\left[ \bar{k}\right]
        \circ p_{2}^{1,1})
        \right) \\
        Q_2 = \lambda \alpha \beta \left[ \alpha =\beta \right] \circ (p_{2}^{1,1}, C_{\varepsilon}^{1,1})
      \end{array}
    \]
    lo cual nos dice que $Q$ es $(\Sigma \cup \Sigma _{p})$-p.r..
    Nótese que
    \[
      \displaystyle \chi_{L}=\lambda \alpha \left[ (\exists k\in \mathbf{N})_{k\leq 10^{\left\vert \alpha
      \right\vert }}\;Q(k,\alpha )\right]
    \]
    Esto nos dice que podemos usar el \textbf{Lema 39} para ver que $\chi_L$ es $(\Sigma \cup \Sigma _{p})$-p.r..
    Por el \textbf{Lema 39} tenemos que
    \[
      \displaystyle \lambda x\alpha \left[ (\exists m\in \mathbf{N})_{m\leq x}\;Q(m,\alpha ) \right]
    \]
    es $(\Sigma \cup \Sigma _{p})$-p.r. lo cual nos dice que
    \[
      \displaystyle \chi_L=\lambda x\alpha \left[ (\exists m\in \mathbf{N})_{m\leq x}\;Q(m,\alpha ) \right] \circ
      (\lambda \alpha \left[ 10^{\left\vert \alpha \right\vert } \right] \circ p_{1}^{0,1},p_{1}^{0,1})
    \]
    lo es. Luego como $ \chi_L $ es $(\Sigma \cup \Sigma _{p})$-p.r entonces $L$ es $(\Sigma \cup \Sigma _{p})$-p.r

    \noindent Sea
    \[
      \displaystyle P=\lambda I\alpha \left[ \alpha \in \mathrm{Ins}^{\Sigma }\wedge I\in \mathrm{Ins}^{\Sigma }\wedge
      \lbrack \alpha ]_{1}\neq \mathrm{L}\wedge (\exists \beta \in L)\ I=\beta \alpha \right]
    \]
    Veamos que $P$ es $(\Sigma \cup \Sigma _{p})$-p.r.
    Note que $D_{P}=(\Sigma \cup \Sigma _{p})^{\ast 2}$ y ademas
    \[
      \displaystyle P = P_1 \land P_2 \land P_3 \land P_4
    \]
    donde
    \[
      \begin{array}{lll}
        \displaystyle
        \bigskip
        P_1 = \lambda I\alpha \left[ \alpha \in \mathrm{Ins}^{\Sigma} \right] \\
        \bigskip
        P_2 = \lambda I\alpha \left[ I \in \mathrm{Ins}^{\Sigma } \right] \\
        \bigskip
        P_3 = \neg  (\lambda \alpha\beta \left[ \alpha =\beta \right] \circ
                    (
                      \lambda i\alpha \left[ \left[\alpha\right]_i \right] \circ
                      (
                      p_{2}^{0,2},
                      C_{1}^{0,2}
                      ), C_{L}^{0,2}
                    )) \\
        \bigskip
        P_4 = \lambda I\alpha \left[ (\exists \beta \in L)  I\ = \beta\alpha \right]
      \end{array}
    \]
    Es fácil ver que $P_1$, $P_2$ y $P_3$ son $(\Sigma \cup \Sigma _{p})$-p.r.
    Veamos que $P_4$ es $(\Sigma \cup \Sigma _{p})$-p.r. Para ello definamos el siguiente predicado
    \[
      R = \lambda I\alpha\beta \left[ I = \beta\alpha\right]
    \]
    Tenemos que
    \[
      P_4 = \lambda I\alpha \left[ (\exists \beta \in L)\; R(I, \alpha, \beta) \right]
    \]
    Notar que, como $\beta$ es una subpalabra de $I$, tenemos que $|\beta| \leq |I|$
    \[
      P_4 = \lambda I\alpha \left[ (\exists \beta \in L)_{|\beta|\leq |I|} R(I, \alpha, \beta) \right]
    \]
    Por el \textbf{Lema 39} tenemos que
    \[
      \displaystyle \lambda xI\alpha \left[ (\exists \beta \in L)_{|\beta|\leq x}\;Q(m,\alpha ) \right]
    \]
    es $(\Sigma \cup \Sigma _{p})$-p.r. Lo cual nos dice que
    \[
      \displaystyle P_4 = \lambda xI\alpha \left[ (\exists \beta \in L)_{|\beta|\leq x}\;Q(m,\alpha ) \right] \circ
      (\lambda \alpha \left[ |\alpha| \right] \circ p_{1}^{0,2}, p_{1}^{0,2}, p_{1}^{0,2})
    \]
    lo es.
    Por lo tanto $P$ es $(\Sigma \cup \Sigma _{p})$-p.r.
    Nótese ademas que cuando $I\in \mathrm{Ins}^{\Sigma }$ tenemos que
    $P(I,\alpha )=1$ sii $\alpha =Bas(I)$.
    Dejamos al lector probar que $Bas=M^{< }\left( P\right) $ % TODO: DEMOSTRAR
    por lo que para ver que $Bas$ es $(\Sigma \cup \Sigma _{p})$-p.r., solo nos falta ver que la función $Bas$
    es acotada por alguna función $(\Sigma \cup \Sigma _{p})$-p.r. y $(\Sigma \cup \Sigma _{p})$-total.
    Pero esto es trivial ya que $\left\vert Bas(I)\right\vert \leq \left\vert I\right\vert =p_{1}^{0,1}(I)$
    para cada $I\in \mathrm{Ins}^{\Sigma }$.
    Finalmente note que

    \[
      \displaystyle Lab=M^{< }\left( \lambda I\alpha \left[ \alpha Bas(I)=I\right] \right)
    \]
    lo cual nos dice que $Lab$ es $(\Sigma \cup \Sigma _{p})$-p.r..
  \end{proof}
  \end{lemma}


  % Lemma 61
  \begin{lemma}
    \par
    \begin{itemize}
      \item[(a)]  $\mathrm{Pro}^{\Sigma }$ es un conjunto $(\Sigma \cup \Sigma _{p}) $-p.r.
      \item[(b)]  $\lambda \mathcal{P}\left[ n(\mathcal{P})\right] $ y $\lambda i \mathcal{P}\left[ I_{i}
                  ^{\mathcal{P}}\right] $ son funciones $(\Sigma \cup \Sigma _{p})$-p.r..
    \end{itemize}
  \end{lemma}

  % Lemma 62
  \begin{lemma}
    \par Este lemma no se evalua.
  \end{lemma}

  % Lemma 63
  \begin{lemma}
    \par Este lemma no se evalua.
  \end{lemma}

  % Proposition 64
  \begin{proposition}
    Sean $n, m \leq 0$, las funciones $i^{n, m}, E_{\#j}^{n, m}, j = 1, 2, ...$ son $\Sigma \cup \Sigma_{p}$-PR.
  \end{proposition}

  % Theorem 65
  \begin{theorem}
    Las funciones $\Phi _{\#}^{n,m}$ y $\Phi _{\ast }^{n,m}$ son $(\Sigma \cup \Sigma _{p})$-recursivas.

  \begin{proof}
    Veremos que $\Phi _{\#}^{n,m}$ es $(\Sigma \cup \Sigma _{p})$-recursiva. Sea $H$ el predicado
    $(\Sigma \cup \Sigma _{p})$-mixto

    \[
      \displaystyle \lambda t\vec{x}\vec{\alpha}\mathcal{P}\left[ i^{n,m}(t,x_{1},...,x_{n}, \alpha _{1},...,\alpha _{m},
      \mathcal{P})=n(\mathcal{P})+1\right] \text{.}
    \]
    Note que $D_{H}=\omega ^{n+1}\times \Sigma ^{\ast m}\times \mathrm{Pro} ^{\Sigma }$.
    Ya que las funciones $i^{n,m}$ y $\lambda \mathcal{P}\left[ n( \mathcal{P})\right] $ son
    $(\Sigma \cup \Sigma _{p})$-p.r., $H$ lo es. Notar que $D_{M(H)}=D_{\Phi _{\#}^{n,m}}$.
    Ademas para $(\vec{x},\vec{\alpha}, \mathcal{P})\in D_{M(H)}$, tenemos que $M(H)(\vec{x},\vec{\alpha},
    \mathcal{P} )$ es la menor cantidad de pasos necesarios para que $\mathcal{P}$ termine partiendo del estado
    $((x_{1},...,x_{n},0,0,...),(\alpha _{1},...,\alpha _{m},\varepsilon ,\varepsilon ,...))$.
    Ya que $H$ es $(\Sigma \cup \Sigma _{p})$-p.r., tenemos que $M(H)$ es $(\Sigma \cup \Sigma _{p})$-r..
    Nótese que para $(\vec{x},\vec{\alpha},\mathcal{P})\in D_{M(H)}=D_{\Phi _{\#}^{n,m}} $ tenemos que
    \[
      \displaystyle \Phi _{\#}^{n,m}(\vec{x},\vec{\alpha},\mathcal{P})=E_{\#1}^{n,m}\left( M(H)( \vec{x},\vec{\alpha},
      \mathcal{P}),\vec{x},\vec{\alpha},\mathcal{P}\right)
    \]
    lo cual con un poco mas de trabajo nos permite probar que % TODO: Hacer el poco mas de trabajo
    \[
      \displaystyle \Phi _{\#}^{n,m}=E_{\#1}^{n,m}\circ \left( M(H),p_{1}^{n,m+1},...,p_{n+m+1}^{n,m+1}\right)
    \]
    Ya que la función $E_{\#1}^{n,m}$ es $(\Sigma \cup \Sigma _{p})$-r., lo es $ \Phi _{\#}^{n,m}$.
  \end{proof}
  \end{theorem}


  % Corollary 66
  \begin{corollary}
    Si $f:D_{f}\subseteq \omega ^{n}\times \Sigma ^{\ast m}\rightarrow O$ es $ \Sigma $-computable,
    entonces $f$ es $\Sigma $-recursiva.
  \begin{proof}
    Haremos el caso $O=\Sigma ^{\ast }$. Sea $\mathcal{P}_{0}$ un programa que compute a $f$.
    Primero veremos que $f$ es $(\Sigma \cup \Sigma _{p})$-recursiva. Note que
    \[
      \displaystyle f=\Phi _{\ast }^{n,m}\circ \left( p_{1}^{n,m},...,p_{n+m}^{n,m},C_{\mathcal{P }_{0}}^{n,m}\right)
    \]
    donde cabe destacar que $p_{1}^{n,m},...,p_{n+m}^{n,m}$ son las proyecciones respecto del alfabeto
    $\Sigma \cup \Sigma _{p}$, es decir que tienen dominio $\omega ^{n}\times (\Sigma \cup \Sigma _{p})^{\ast m}$.
    Ya que $\Phi _{\ast }^{n,m}$ es $(\Sigma \cup \Sigma _{p})$-recursiva tenemos que $f$ es
    $(\Sigma \cup \Sigma _{p})$-recursiva.
    O sea que el \textbf{Teorema 51} (independencia del alfabeto) nos dice que $f$ es $\Sigma $-recursiva.
  \end{proof}
  \end{corollary}


  % Tesis de Church
  \noindent \textbf{\underline{Tesis de Church:}} Toda función $\Sigma $-efectivamente computable es $\Sigma $-recursiva.

  % Corollary 67
  \begin{corollary}
    \par Este corolario no se evalua.
  \end{corollary}

  % Lemma 68
  \begin{lemma}
    Supongamos $f_{i}:D_{f_{i}}\subseteq \omega ^{n}\times \Sigma ^{\ast m}\rightarrow O$, $i=1,...,k$,
    son funciones $\Sigma $-recursivas tales que $D_{f_{i}}\cap D_{f_{j}}=\emptyset $ para $i\neq j$.
    Entonces la función $f_{1}\cup ...\cup f_{k}$ es $\Sigma $-recursiva.
  \begin{proof}
    Probaremos el caso $k=2$ y $O=\Sigma ^{\ast }$. Sean $\mathcal{P}_{1}$ y $ \mathcal{P}_{2}$ programas que
    computen las funciones $f_{1}$ y $f_{2}$, respectivamente. Sean
    \[
      \displaystyle
      \begin{array}{rcl}
        P_{1} & =& \lambda t\vec{x}\vec{\alpha}\left[ i^{n,m}(t,\vec{x},\vec{\alpha},
          \mathcal{P}_{1})=n(\mathcal{P}_{1})+1\right] \\
        P_{2} & =& \lambda t\vec{x}\vec{\alpha}\left[ i^{n,m}(t,\vec{x},\vec{\alpha},
          \mathcal{P}_{2})=n(\mathcal{P}_{2})+1\right]
      \end{array}
    \]
    Nótese que $D_{P_{1}}=D_{P_{2}}=\omega \times \omega ^{n}\times \Sigma ^{\ast m}$ y que $P_{1}$ y $P_{2}$
    son $(\Sigma \cup \Sigma _{p})$-p.r.. Ya que son $\Sigma $-mixtos, el \textbf{Teorema 51}
    (independencia del alfabeto) nos dice que son
    $\Sigma $-p.r.. También nótese que $D_{M((P_{1}\vee P_{2}))}=D_{f_{1}}\cup D_{f_{2}}$. Definamos
    \[
      \displaystyle
        \begin{array}{rcl}
          g_{1} & =& \lambda \vec{x}\vec{\alpha}\left[ E_{\ast 1}^{n,m}(M\left( (P_{1}\vee P_{2})\right)
            (\vec{x},\vec{\alpha}),\vec{x},\vec{\alpha}, \mathcal{P}_{1})^{P_{1}(M\left( (P_{1}\vee P_{2})\right)
            (\vec{x},\vec{\alpha }),\vec{x},\vec{\alpha})}\right] \\
          g_{2} & =& \lambda \vec{x}\vec{\alpha}\left[ E_{\ast 1}^{n,m}(M\left( (P_{1}\vee P_{2})\right)
            (\vec{x},\vec{\alpha}),\vec{x},\vec{\alpha}, \mathcal{P}_{2})^{P_{2}(M\left( (P_{1}\vee P_{2})\right)
            (\vec{x},\vec{\alpha }),\vec{x},\vec{\alpha})}\right]
        \end{array}
    \]
    Nótese que $g_{1}$ y $g_{2}$ son $\Sigma $-recursivas y que $ D_{g_{1}}=D_{g_{2}}=D_{f_{1}}\cup D_{f_{2}}$,
    Ademas nótese que
    \[
      \displaystyle
        g_{1}(\vec{x},\vec{\alpha})=\left\{
          \begin{array}{lll}
            f_{1}(\vec{x},\vec{\alpha}) & & \text{si } (\vec{x},\vec{\alpha})\in D_{f_{1}} \\
            \varepsilon & & \text{caso contrario}
          \end{array} \right.
    \]

    \[
      \displaystyle
        g_{2}(\vec{x},\vec{\alpha})=\left\{
          \begin{array}{lll}
            f_{2}(\vec{x},\vec{\alpha}) & & \text{si }(\vec{x},\vec{\alpha})\in D_{f_{2}} \\
            \varepsilon & & \text{caso contrario}
          \end{array} \right.
    \]
    O sea que $f_{1}\cup f_{2}=\lambda \alpha \beta \left[ \alpha \beta \right] \circ (g_{1},g_{2})$ es
    $\Sigma $-recursiva.
  \end{proof}
  \end{lemma}

  % Lemma 69
  \begin{lemma}
    Supongamos $\Sigma \supseteq \Sigma _{p}$. Entonces $ Halt^{\Sigma }$ es no $\Sigma $-recursivo.
  \begin{proof}
    Supongamos $Halt^{\Sigma }$ es $\Sigma $-recursivo y por lo tanto $\Sigma $ -computable.
    Por la proposición de existencia de macros tenemos que hay un macro
    \[
      \displaystyle \left[ \mathrm{IF}\;Halt^{\Sigma }(\mathrm{W}1)\;\mathrm{GOTO}\;\mathrm{A}1 \right]
    \]
    Sea $\mathcal{P}_{0}$ el siguiente programa de $\mathcal{S}^{\Sigma }$
    \[
      \displaystyle \mathrm{L}1\;\left[ \mathrm{IF}\;Halt^{\Sigma }(\mathrm{P}1)\;\mathrm{GOTO}\; \mathrm{L}1\right]
    \]
    Note que

    $\mathcal{P}_{0}$ termina partiendo desde $\left( (0,0,...),( \mathcal{P}_{0},\varepsilon ,\varepsilon ,...)
    \right) $ sii $Halt^{\Sigma }( \mathcal{P}_{0})=0$,

    \noindent lo cual produce una contradicción si tomamos en (*) $\mathcal{P}= \mathcal{P}_{0}$.
  \end{proof}
  \end{lemma}


  % Theorem 70
  \begin{theorem} % TODO: Hacer bien la ida y la vuelta
    Sea $S\subseteq \omega ^{n}\times \Sigma ^{\ast m}$. Entonces $S$ es $\Sigma $-efectivamente enumerable
    sii $S$ es $\Sigma $-recursivamente enumerable
  \begin{proof}
    ($\Rightarrow $) Use la Tesis de Church.

    ($\Leftarrow $) Use el Theorem 42.
  \end{proof}
  \end{theorem}

  % Theorem 71
  \begin{theorem} Dado $S\subseteq \omega ^{n}\times \Sigma ^{\ast m} $, son equivalentes
    \begin{enumerate}
      \item $S$ es $\Sigma $-recursivamente enumerable
      \item $S=I_{F}$, para alguna $F:D_{F}\subseteq \omega ^{k}\times \Sigma ^{\ast l}\rightarrow \omega ^{n}\times
            \Sigma ^{\ast m}$ tal que cada $F_{i}$ es $\Sigma $-recursiva.
      \item $S=D_{f}$, para alguna función $\Sigma $-recursiva $f$
      \item $S=\varnothing $ o $S=I_{F}$, para alguna $F:\omega \rightarrow \omega ^{n}\times \Sigma ^{\ast m}$
            tal que cada $F_{i}$ es $\Sigma $-p.r.
    \end{enumerate}
  \end{theorem}


  % Corollary 72
  \begin{corollary}
    Supongamos $f:D_{f}\subseteq \omega ^{n}\times \Sigma ^{\ast m}\rightarrow O$ es $\Sigma $-recursiva y
    $S\subseteq D_{f}$ es $ \Sigma $-r.e., entonces $f\mid _{S}$ es $\Sigma $-recursiva.
  \begin{proof}
    Supongamos $O=\Sigma ^{\ast }.$ Por el \textbf{Teorema 71} $S=D_{g}$, para alguna función
    $\Sigma $-recursiva $g.$ Nótese que componiendo adecuadamente podemos suponer que $I_{g}=\{\varepsilon \}.$
    O sea que tenemos $f\mid _{S}=\lambda \alpha \beta \left[ \alpha \beta \right] \circ (f,g)$.
  \end{proof}
  \end{corollary}


  % Corollary 73
  \begin{corollary}
    \par Este corolario no se evalua.
  \end{corollary}

  % Corollary 74
  \begin{corollary}
    Supongamos $S_{1},S_{2}\subseteq \omega ^{n}\times \Sigma ^{\ast m}$ son conjuntos $\Sigma $-r.e..
    Entonces $S_{1}\cap S_{2}$ es $\Sigma $-r.e..
  \begin{proof}
    Por el \textbf{Teorema 71} $S_{i}=D_{g_{i}}$, con $g_{1},g_{2}$ funciones $ \Sigma $-recursivas$.$
    Nótese que podemos suponer que $I_{g_{1}},I_{g_{2}} \subseteq \Sigma^{\ast} $ por lo que
    $S_{1}\cap S_{2}=D_{\lambda \alpha\beta \left[ \alpha\beta\right] \circ (g_{1},g_{2})}$ es $\Sigma $-r.e.$.$
  \end{proof}
  \end{corollary}

  % Corollary 75
  \begin{corollary}
    Supongamos $S_{1},S_{2}\subseteq \omega ^{n}\times \Sigma ^{\ast m}$ son conjuntos $\Sigma $-r.e..
    Entonces $S_{1}\cup S_{2}$ es $\Sigma $-r.e.
  \begin{proof}
    Supongamos $S_{1}\neq \emptyset \neq S_{2}.$ Sean $F,G:\omega \rightarrow \omega ^{n}\times \Sigma ^{\ast m}$
    tales que $I_{F}=S_{1}$, $I_{G}=S_{2}$ y las funciones $F_{i} {\acute{}} s$ y $G_{i} {\acute{}} s$ son
    $\Sigma $-recursivas. Sean $f=\lambda x\left[ Q(x,2)\right] $ y $ g=\lambda x\left[ Q(x\dot{-}1,2)\right] .$
    Sea $H:\omega \rightarrow \omega ^{n}\times \Sigma ^{\ast m}$ dada por
    \[
      \displaystyle H_{i}=(F_{i}\circ f)\mathrm{\mid }_{\{x:x\mathrm{\ es\ par}\}}\cup (G_{i}\circ g)
      \mathrm{\mid }_{\{x:x\mathrm{\ es\ impar}\}}
    \]

    Por el \textbf{Corollary 72} (restriccion de una funcion) y el \textbf{Lema 68} (union de funciones),
    cada $H_{i}$ es $ \Sigma $-recursiva. Ya que $I_{H}=S_{1}\cup S_{2}$.
    tenemos que $S_{1}\cup S_{2}$ es $\Sigma $-r.e.
  \end{proof}
  \end{corollary}

  % Theorem 76
  \begin{theorem}
     Sea $S\subseteq \omega ^{n}\times \Sigma ^{\ast m}$. Entonces $S$ es $\Sigma $-efectivamente computable sii
     $S$ es $\Sigma $-recursivo
  \begin{proof}
    ($\Rightarrow $) Use la Tesis de Church.

    ($\Leftarrow $) Use el Teorema 42. $\Box$
  \end{proof}
  \end{theorem}

  % Theorem 77
  \begin{theorem} Sea $S\subseteq \omega ^{n}\times \Sigma ^{\ast m}.$ Son equivalentes
    \begin{itemize}
      \item[(a)] $S$ es $\Sigma $-recursivo
      \item[(b)] $S$ y $(\omega ^{n}\times \Sigma ^{\ast m})-S$ son $\Sigma $ -recursivamente enumerables
    \end{itemize}
  \begin{proof}
    (a)$\Rightarrow $(b)$.$ Note que $S=D_{Pred\ \circ\ \chi _{S}}.$
    Luego, por \textbf{Teorema 71} $S$ es $\Sigma $-recursivamente enumerable. De igual manera podemos ver que
    $(\omega ^{n}\times \Sigma ^{\ast m})-S=D_{Pred\ \circ\ \chi _{(\omega ^{n}\times \Sigma ^{\ast m})-S}}$
    es $\Sigma $-recursivamente enumerable.
    Donde $\chi_{(\omega ^{n}\times \Sigma ^{\ast m})-S} = \lambda xy \left[x \dot{-}y\right] \circ\ (C_{1}^{1,0},
    \chi_S)$

    (b)$\Rightarrow $(a). Note que $\chi _{S}=C_{1}^{n,m}\mathrm{\mid }_{S}\cup C_{0}^{n,m}\mathrm{\mid }_{\omega
    ^{n}\times \Sigma ^{\ast m}-S}$.
  \end{proof}
  \end{theorem}

  % Lemma 78
  \begin{lemma} Supongamos que $\Sigma \supseteq \Sigma _{p}.$ Entonces
    \[
      \displaystyle A=\left\{ \mathcal{P}\in \mathrm{Pro}^{\Sigma }:Halt^{\Sigma }(\mathcal{P} )\right\}
    \]
    es $\Sigma $-r.e. y no es $\Sigma $-recursivo. Mas aun el conjunto
    \[
      \displaystyle N=\left\{ \mathcal{P}\in \mathrm{Pro}^{\Sigma }:\lnot Halt^{\Sigma }( \mathcal{P})\right\}
    \]
    no es $\Sigma $-r.e.
  \begin{proof}
    Sea $P=\lambda t\mathcal{P}\left[ i^{0,1}(t,\mathcal{P},\mathcal{P})=n( \mathcal{P})+1\right] $.
    Note que $P$ es $\Sigma $-p.r. por lo que $M(P)$ es $\Sigma $-r.. Ademas note que $D_{M(P)}=A$,
    lo cual implica que $A$ es $ \Sigma $-r.e.. Ya que $Halt^{\Sigma }$ es no $\Sigma $-recursivo por \textbf{Lema 69} y
    \[
      \displaystyle Halt^{\Sigma }=C_{1}^{0,1}\mid _{A}\cup C_{0}^{0,1}\mid _{N}
    \]
    el Lema 68 nos dice que $N$ no es $\Sigma $-r.e.. Finalmente supongamos $A$ es $\Sigma $-recursivo. Entonces el conjunto
    \[
      \displaystyle N=\left( \Sigma ^{\ast }-A\right) \cap \mathrm{Pro}^{\Sigma }
    \]
    debería serlo, lo cual es absurdo.
  \end{proof}
  \end{lemma}
