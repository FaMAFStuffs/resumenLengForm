\section{Funciones $\Sigma$-recursivas}
  % Resta terminar los teoremas: 19, 27


  % Lemma 18: Con prueba.
  \begin{lemma}
    \par Si $f, f_{1}, \dotsc, f_{n+m}$ son $\Sigma$-efectivamente computables, entonces $f \circ (f_{1}, \dotsc,
    f_{n+m})$ lo es.
  \end{lemma}
  \begin{proof}
    Sean:

    \begin{itemize}
      \item $\mathbb{P}$ un procedimiento efectivo que compute a $f$.
      \item $\mathbb{P}_{1}$ un procedimiento efectivo que compute a $f_{1}$.
      \item $\mathbb{P}_{2}$ un procedimiento efectivo que compute a $f_{2}$.
      \item $\vdotswithin{(n-1)m+1} \qquad \vdotswithin{(n-1)m+2} \qquad \vdotswithin{(n-1)m+r}$
      \item $\mathbb{P}_{n+m}$ un procedimiento efectivo que compute a $f_{n+m}$.
    \end{itemize}

    \par El siguiente porsedimiento $\mathbb{P}$ computa $f \circ (f_{1}, \dotsc, f_{n+m})$:

    \vspace{3mm}
    \textbf{Etapa 1:}
    Realizar $\mathbb{P}_{1}$ con $(\vec{x}, \vec{\alpha})$ de entrada para obtener como salida $o_{1}$

    $\qquad\qquad\;\;$Realizar $\mathbb{P}_{2}$ con $(\vec{x}, \vec{\alpha})$ de entrada para obtener como salida $o_{2}$

    $\qquad\qquad\;\;\vdotswithin{(n-1)m+1} \qquad \vdotswithin{(n-1)m+2}$

    $\qquad\qquad\;\;$Realizar $\mathbb{P}_{n+m}$ con $(\vec{x}, \vec{\alpha})$ de entrada para obtener como salida $o_{n+m}$


    \textbf{Etapa 2:}
    Dar como dato de salida el resultado de $\mathbb{P}$ con $(o_{1}, o_{2}, \dotsc, o_{n+m})$ de entrada.
  \end{proof}

  % Lemma 19: Con prueba. HACER!
  \begin{lemma}
    \par Si $f$ y $g$ son $\Sigma $-efectivamente computables, entonces $R(f,g)$ lo es.
  \end{lemma}
  \begin{proof}
    HACER!!!!!!!
  \end{proof}

  % Lemma 20: Sin prueba.
  \begin{lemma}
    \par Si $f$ y cada $\mathcal{G}_{a}$ son $\Sigma$-efectivamente computables, entonces $R(f, \mathcal{G})$ lo es.
  \end{lemma}

  % Theorem 21: Con prueba.
  \begin{theorem}
    \par Si $f \in \mathrm{PR}^{\Sigma}$, entonces $f$ es $\Sigma$-efectivamente computable.
  \end{theorem}
  \begin{proof}
    \par Recordemos que $PR^{\Sigma} = \bigcup\limits_{k \geq 0} PR_{k}^{\Sigma}$. Supongamos que $f \in
    PR_{k}^{\Sigma}$, probaremos este teorema por inducción en $k$.

    \vspace{3mm}
    \underline{Caso Base:} \begin{tabular}{|c|} \hline $k = 0$ \\\hline \end{tabular}

    \par Luego $f \in PR_{0}^{\Sigma}$, es decir $f \in \{Suc, Pred, C_{0}^{0,0}, C_{\varepsilon}^{0,0}\} \cup \{d_{a}:
    a \in \Sigma\} \cup \{p_{j}^{n,m} : 1 \leq j \geq n+m\}$. Por lo tanto, $f$ es $\Sigma$-efectivamente computable.

    \vspace{3mm}
		\underline{Caso Inductivo:} \begin{tabular}{|c|} \hline $k > 0$ \\\hline \end{tabular}

    \par Supongamos ahora que si $f \in \mathrm{PR}_{k}^{\Sigma} \Rightarrow f$ es
    $\Sigma$-efectivamente computable, veamos que $f \in \mathrm{PR}_{k+1}^{\Sigma} \Rightarrow f$ es
    $\Sigma$-efectivamente computable.

    \par Dado que las funciones de $PR_{k}^{\Sigma}$ son $\Sigma$-efectivamente computable por hipótesis inductiva, y
    que $PR_{k+1}^{\Sigma}$ se contruye a partir de las mismas, a través de recursiones y/o composiciones, las cuales
    probamos son $\Sigma$-efectivamente computables en el \textbf{Lemma 18} y \textbf{Lemma 19}, entonces concluimos que
    $f$ es $\Sigma$-efectivamente computable.
  \end{proof}

  % Lemma 22: Con prueba.
  \begin{lemma}
    \begin{enumerate}
      \item $\emptyset \in \mathrm{PR}^{\emptyset}$.
      \item $\lambda xy \left[x+y\right] \in \mathrm{PR}^{\emptyset}$.
      \item $\lambda xy\left[x.y\right] \in \mathrm{PR}^{\emptyset}$.
      \item $\lambda x\left[x!\right] \in \mathrm{PR}^{\emptyset}$.
    \end{enumerate}
  \end{lemma}
  \begin{proof}
    \begin{enumerate}
      \item Notese que $\emptyset = Pred \circ C_{0}^{0,0} \in \mathrm{PR}_{1}^{\emptyset}$, entonces $\emptyset \in
        \mathrm{PR}^{\emptyset}$.
      \item Notar que:
        \begin{eqnarray}
          \nonumber \lambda xy \left[x+y\right](0, x_{1}) &=& x_{1} = p_{1}^{1,0}(x_{1}) \\
          \nonumber \lambda xy \left[x+y\right](t+1, x_{1}) &=& \lambda xy \left[x+y\right](t, x_{1}) + 1 \\
          \nonumber & =& (Suc \circ p_{1}^{3,0})(\lambda xy \left[x+y\right](t,x_{1}), t, x_{1})
        \end{eqnarray}

        \par lo cual implica que $\lambda xy \left[x+y\right] = R(p_{1}^{1,0}, Suc \circ p_{1}^{3,0}) \in
        \mathrm{PR}_{2}^{\emptyset}$, entonces $\lambda xy \left[x+y\right] \in \mathrm{PR}^{\emptyset}$.
      \item Primero note que:
        \begin{eqnarray}
          \nonumber C_{0}^{1,0}(0) &=& C_{0}^{0,0}(\Diamond) \\
          \nonumber C_{0}^{1,0}(t+1) &=& C_{0}^{1,0}(t)
        \end{eqnarray}

        \par lo cual implica que $C_{0}^{1,0} = R(C_{0}^{0,0}, p_{1}^{2,0}) \in \mathrm{PR}_{1}^{\emptyset}$.
        \par También note que:

        \begin{eqnarray}
          \nonumber \lambda xy \left[x.y\right](0, x_{1}) &=& x_{1} = p_{1}^{1,0}(x_{1}) \\
          \nonumber \lambda xy \left[x.y\right](t+1, x_{1}) &=& \lambda xy \left[x.y\right](t, x_{1}) + x_{1} \\
          \nonumber &=& \lambda xy \left[x+y\right] \circ (p_{1}^{3,0}, p_{3}^{3,0})
        \end{eqnarray}

        \par lo cual implica que $\lambda xy \left[x.y\right] = R(C_{0}^{1,0}, \lambda xy \left[x+y\right] \circ
        (p_{1}^{3,0}, p_{3}^{3,0}))$, lo cual por (1) implica que $\lambda xy \left[x.y\right] \in
        \mathrm{PR}_{4}^{\emptyset}$, entonces $\lambda xy \left[x.y\right] \in \mathrm{PR}^{\emptyset}$.
      \item Notar que:
        \begin{eqnarray}
          \nonumber \lambda x \left[x!\right](0) &=& 1 = C_{1}^{0,0}(\Diamond) \\
          \nonumber \lambda x \left[x!\right](t+1) &=& \lambda x \left[x!\right](t).(t+1)
        \end{eqnarray}

        \par lo cual implica que: $\lambda x \left[x!\right] = R(C_{1}^{0, 0}, \lambda xy \left[x.y\right] \circ
        (p_{1}^{2,0}, Suc \circ p_{2}^{2,0}))$. Ya que $C_{1}^{0,0} = Suc \circ C_{0}^{0,0}$, tenemos que $C_{1}^{0,0}
        \in \mathrm{PR}_{1}^{\emptyset}$. Por (2), tenemos que $\lambda xy \left[x.y\right] \circ (p_{1}^{2,0}, Suc
        \circ p_{2}^{2,0}) \in \mathrm{PR}_{4}^{\emptyset}$, obteniendo que $\lambda x \left[x!\right] \in
        \mathrm{PR}_{5}^{\emptyset}$, entonces $\lambda x \left[x!\right] \in \mathrm{PR}^{\emptyset}$.
    \end{enumerate}
  \end{proof}

  % Lemma 23: Con prueba.
  \begin{lemma}
    \par Supongamos $\Sigma \neq \emptyset$.

    \begin{enumerate}[a)]
      \item $\lambda \alpha \beta \left[\alpha\beta\right] \in \mathrm{PR}^{\Sigma}$.
      \item $\lambda \alpha \left[\lvert\alpha \rvert\right] \in \mathrm{PR}^{\Sigma}$.
    \end{enumerate}
  \end{lemma}
  \begin{proof}
    \begin{enumerate}[a)]
      \item Ya que:
        \begin{eqnarray}
          \nonumber \lambda \alpha\beta \left[\alpha \beta\right](\alpha_{1}, \varepsilon) &=& \alpha_{1} =
          p_{1}^{0,1}(\alpha_{1}) \\
          \nonumber \lambda \alpha\beta \left[\alpha \beta\right](\alpha_{1}, \alpha a) &=& d_{a}(\lambda \alpha\beta
          \left[\alpha\beta\right](\alpha_{1}, \alpha), \qquad \text{para a} \in \Sigma
        \end{eqnarray}

        \par tenemos que $\lambda \alpha\beta \left[\alpha \beta\right] = R(p_{1}^{0, 1}, \mathcal{G})$, donde
        $\mathcal{G}_{a} = d_{a} \circ p_{3}^{0,3}$, para cada $a \in \Sigma$. Luego, $\lambda \alpha\beta
        \left[\alpha\beta\right] \in \mathrm{PR}^{\Sigma}$.
      \item Ya que:
        \begin{eqnarray}
          \nonumber \lambda \alpha \left[\lvert\alpha \rvert\right](\varepsilon) &=& 0 = C_{0}^{0,0}(\Diamond) \\
          \nonumber \lambda \alpha \left[\lvert\alpha \rvert\right](\alpha a) &=& \lambda \alpha \left[\lvert\alpha
            \rvert\right](\alpha) + 1
        \end{eqnarray}

        \par tenemos que $\lambda \alpha \left[\lvert\alpha \rvert\right] = R(C_{0}^{0, 0}, \mathcal{G})$, donde
        $\mathcal{G}_{a} = Suc \circ p_{1}^{1, 1}$, para cada $a \in \Sigma$. Luego, $\lambda \alpha
        \left[\lvert\alpha \rvert\right] \in \mathrm{PR}^{\Sigma}$.
    \end{enumerate}
  \end{proof}

  % Lemma 24: Sin prueba.
  \begin{lemma}
    \begin{enumerate}[a)]
      \item $C_{k}^{n,m}, C_{\alpha}^{n,m} \in \mathrm{PR}^{\Sigma}$, para $n, m, k \geq 0$, y $\alpha \in \SIGMA$.
      \item $C_{k}^{n,0} \in \mathrm{PR}^{\emptyset}$, para $n, k \geq 0$.
    \end{enumerate}
  \end{lemma}

  % Lemma 25: Con prueba.
  \begin{lemma}
    \begin{enumerate}
      \item $\lambda xy \left[x^{y}\right] \in \mathrm{PR}^{\emptyset}$.
      \item $\lambda t\alpha \left[\alpha^{t}\right] \in \mathrm{PR}^{\Sigma}$.
    \end{enumerate}
  \end{lemma}
  \begin{proof}
    \begin{enumerate}[a)]
      \item Notar que:

        \begin{eqnarray}
          \nonumber \lambda tx \left[x^{t}\right](0, x_{1}) &=& 0 = C_{0}^{1,0}(x_{1}) \\
          \nonumber \lambda tx \left[x^{t}\right](t+1, x_{1}) &=& \lambda tx \left[x^{t}\right](t, x_{1}) . x_{1} \\
          \nonumber &=& \lambda xy \left[x.y\right] \circ (p_{1}^{3,0}, p_{3}^{3,0})
        \end{eqnarray}
        \par Osea que $\lambda xy \left[x^{y}\right] = \lambda tx \left[x^{t}\right] \circ (p_{2}^{2, 0}, p_{1}^{2, 0})
        \in \mathrm{PR}^{\emptyset}$.

      \item Notar que:

      \begin{eqnarray}
        \nonumber \lambda t\alpha \left[\alpha^{t}\right](t, \varepsilon) &=& \varepsilon = C_{\varepsilon}^{0,1}(t) \\
        \nonumber \lambda t\alpha \left[\alpha^{t}\right](t, \alpha a) &=& \lambda t\alpha \left[\alpha^{t}\right](t,
          \alpha) \alpha \\
        \nonumber &=& \lambda \alpha\beta \left[\alpha\beta \right] \circ \left(p_{3}^{1,2}, p_{2}^{1,2}\right)
      \end{eqnarray}

      \par Por lo tanto, $\lambda t\alpha \left[\alpha^{t}\right] \in \mathrm{PR}^{\Sigma}$.
    \end{enumerate}
  \end{proof}

  % Lemma 26: Con prueba.
  \begin{lemma}
    \par Si $<$ es un orden total estricto sobre un alfabeto no vacío $\Sigma$, entonces:

    \begin{enumerate}[a)]
      \item $s^{<} \in \mathrm{PR}^{\Sigma}$.
      \item $\#^{<} \in \mathrm{PR}^{\Sigma}$.
      \item $\ast^{<} \in \mathrm{PR}^{\Sigma}$.
    \end{enumerate}
  \end{lemma}
  \begin{proof}
    \par Supongamos $\Sigma = \{a_{1}, \dotsc, a_{k}\}$ y $<$ dado por $a_{1} < \dotsc < a_{k}$.

    \begin{enumerate}[a)]
      \item Ya que:
        \begin{eqnarray}
          \nonumber s^{<}(\varepsilon) &=& a_{1} \\
          \nonumber s^{<}(\alpha a_{i}) &=& \alpha a_{i+1} \text{, para } i < k \\
          \nonumber s^{<}(\alpha a_{k}) &=& s^{<}(\alpha) a_{1}
        \end{eqnarray}

        \par tenemos que $s^{<} = R(C_{a_{1}}^{0, 0}, \mathcal{G})$, donde $\mathcal{G} = \{\left(a_{i}, d_{a_{i+1}}
        \circ p_{1}^{0,2} \right), \left(a_{k}, d_{a_{1}} \circ p_{2}^{0,2} \right)\}$. Luego, $s^{<} \in
        \mathrm{PR}^{\Sigma}$.

      \item Ya que:
        \begin{eqnarray}
          \nonumber \ast^{<}(0) &=& \varepsilon \\
          \nonumber \ast^{<}(t+1) &=& s^{<}(\ast^{<}(t))
        \end{eqnarray}

        \par tenemos que $\ast^{<} = R(C_{\varepsilon}^{0,0}, s^{<} \circ p_{1}^{2,0})$. Luego, $\ast^{<} \in
        \mathrm{PR}^{\Sigma}$.

      \item Ya que:
        \begin{eqnarray}
          \nonumber \#^{<}(\varepsilon) &=& 0 \\
          \nonumber \#^{<}(\alpha a_{i}) &=& \#^{<}(\alpha). k + i \\
          \nonumber \text{para } i &=& 1, \dotsc, k
        \end{eqnarray}

        \par tenemos que $\#^{<} = R(C_{0}^{0, 0}, \mathcal{G})$, donde $\mathcal{G}_{a_{i}} = \lambda xy
        \left[x+y\right] \circ (\lambda xy \left[x.y\right] \circ (p_{1}^{1, 1}, C_{k}^{1, 1}), C_{i}^{1, 1})
        \text{, para } i = 1, \dotsc, k$. Luego, $\#^{<} \in \mathrm{PR}^{\Sigma}$.
    \end{enumerate}
  \end{proof}

  % Lemma 27: Con prueba. HACER!
  \begin{lemma}
    \begin{enumerate}[a)]
      \item $\lambda xy \left[x \dot{-}y\right] \in \mathrm{PR}^{\emptyset}$.
      \item $\lambda xy \left[\max (x,y)\right] \in \mathrm{PR}^{\emptyset}$.
      \item $\lambda xy \left[x=y\right] \in \mathrm{PR}^{\emptyset}$.
      \item $\lambda xy \left[x \leq y\right] \in \mathrm{PR}^{\emptyset}$.
      \item Si $\Sigma \neq \emptyset \Rightarrow \lambda \alpha \beta \left[\alpha = \beta\right] \in
        \mathrm{PR}^{\Sigma}$.
    \end{enumerate}
  \end{lemma}
  \begin{proof}
    \begin{enumerate}[a)]
      \item Primero notar que:

        \begin{eqnarray}
          \nonumber \lambda x \left[x \dot{-}1\right](0) &=& 0 = C_{0}^{0,0} \\
          \nonumber \lambda x \left[x \dot{-}1\right](t+1) &=& t \\
          \nonumber &=& p_{2}^{2,0}
        \end{eqnarray}

        \par es decir $\lambda x \left[x \dot{-}1\right] = R(C_{0}^{0,0}, p_{2}^{2,0}) \in \mathrm{PR}^{\emptyset}$.
        \par También notar que:

        \begin{eqnarray}
          \nonumber \lambda tx \left[x \dot{-}t\right](0, x_{1}) &=& x_{1} = p_{1}^{1,0}(x_{1}) \\
          \nonumber \lambda tx \left[x \dot{-}t\right](t+1, x_{1}) &=& \lambda tx \left[x \dot{-}t\right](t, x_{1})
            \dot{-} 1\\
          \nonumber &=& \lambda x \left[x \dot{-}1\right] \circ p_{1}^{3,0}
        \end{eqnarray}

        \par es decir, $\lambda tx \left[x \dot{-}t\right] = R(p_{1}^{1, 0}, \lambda x \left[x \dot{-}1\right] \circ
        p_{1}^{3, 0}) \in \mathrm{PR}^{\emptyset}$. Por lo tanto, $\lambda xy \left[x \dot{-}y\right] = \lambda tx
        \left[x \dot{-}t\right] \circ (p_{2}^{2, 0}, p_{1}^{2, 0}) \in \mathrm{PR}^{\emptyset}$.
      \item Notar que:


       $\lambda xy \left[\max (x,y)\right] = \lambda xy \left[(x + (y \dot{-}x)\right]$.
      \item Note que $\lambda xy \left[x = y\right] = \lambda xy \left[1 \dot{-}((x \dot{-} y) + (y \dot{-} x))\right]$.
      \item Note que $\lambda xy \left[x \leq y\right] = \lambda xy\left[1 \dot{-}(x \dot{-}y)\right]$.
      \item Sea $<$ un orden total estricto sobre $\Sigma$. Ya que $\alpha = \beta \text{ sii } \#^{<}(\alpha) =
        \#^{<}(\beta)$, tenemos que $\lambda \alpha \beta \left[\alpha = \beta\right] = \lambda xy \left[x = y\right]
        \circ (\#^{<} \circ p_{1}^{0, 2}, \#^{<} \circ p_{2}^{0, 2})$.

        \par Osea que podemos aplicar (c) y \textbf{Lema 28} implica que $\chi_{S}$ es $\Sigma$-PR.
    \end{enumerate}
  \end{proof}

  % Lemma 28: Con prueba.
  \begin{lemma}
    \par Si $P: S \subseteq \omega^{n} \times \Sigma^{\ast m} \rightarrow \omega$ y $Q: S \subseteq \omega^{n} \times
    \Sigma^{\ast m} \rightarrow \omega$ son predicados $\Sigma$-PR, entonces $(P \vee Q), (P \wedge Q)$ y $\neg P$ lo
    son también.
  \end{lemma}
  \begin{proof}
    \par Notar que:

    \begin{eqnarray}
      \nonumber \neg P &=& \lambda xy \left[x \dot{-} y\right] \circ (C_{1}^{n, m}, P) \\
      \nonumber (P \wedge Q) &=& \lambda xt \left[x.y\right] \circ (P, Q) \\
      \nonumber (P \vee Q) &=& \neg (\neg P \wedge \neg Q)
    \end{eqnarray}
  \end{proof}

  % Lemma 29: Con prueba.
  \begin{lemma}
    \par Si $S_{1}, S_{2} \subseteq \omega^{n} \times \Sigma^{\ast m}$ son $\Sigma$-PR, entonces $S_{1} \cup S_{2}$,
    $S_{1} \cap S_{2}$ y $S_{1} - S_{2}$ lo son.
  \end{lemma}
  \begin{proof}
    \par Notar que:

    \begin{eqnarray}
      \nonumber \chi_{S_{1} \cup S_{2}} &=& (\chi_{S_{1}} \vee \chi_{S_{2}}) \\
      \nonumber \chi_{S_{1} \cap S_{2}} &=& (\chi_{S_{1}} \wedge \chi_{S_{2}}) \\
      \nonumber \chi_{S_{1} - S_{2}} &=& \lambda \left[x \dot{-} y\right] \circ (\chi_{S_{1}}, \chi_{S_{2}})
    \end{eqnarray}
  \end{proof}

  % Corollary 30: Con prueba. Solo el caso n = m = 1.
  \begin{corollary}
    \par Si $S \subseteq \omega^{n} \times \Sigma^{\ast m}$ es finito, entonces $S$ es $\Sigma$-PR.
  \end{corollary}
  \begin{proof}
    \par Se probará el caso $n = m = 1$, es decir, $S \subseteq \omega \times \SIGMA$. Supongamos, sin pérdida de
    generalidad, utilizando el \textbf{Lemma 29} que:

    \[
      S = \{(z, \gamma)\}
    \]

    \par Notar que $\chi_{S}$ es el siguiente predicado:

    \[
      \left(\chi_{z} \circ p_{1}^{1,1} \wedge \chi_{\gamma} \circ p_{2}^{1,1}\right)
    \]

    \par Ya que los predicados:

    \begin{eqnarray}
      \nonumber \chi_{z} = \lambda xy \left[x = y\right] \circ \left(p_{1}^{1,0}, C_{z}^{1,0}\right) \\
      \nonumber \chi_{\gamma} = \lambda \alpha\beta \left[\alpha = \beta\right] \circ \left(p_{1}^{0,1},
      C_{\gamma}^{0,1}\right)
    \end{eqnarray}

    \par son $\Sigma$-PR, el \textbf{Lema 28} implica que $\chi_{S}$ es $\Sigma$-PR, por lo tanto $S$ es $\Sigma$-PR.
  \end{proof}

  % Lemma 31: Con prueba. Solo el caso n = m = 1.
  \begin{lemma}
    \par Supongamos $S_{1}, \dotsc, S_{n} \subseteq \omega, L_{1}, \dotsc, L_{m} \subseteq \SIGMA$ son conjuntos no
    vacíos, entonces $S_{1} \times \dotsc \times S_{n} \times L_{1} \times \dotsc \times L_{m}$ es $\Sigma$-PR sii
    $S_{1}, \dotsc, S_{n}, L_{1}, \dotsc, L_{m}$ son $\Sigma$-PR.
  \end{lemma}
  \begin{proof}
    \par Se probará el caso $n = m = 1$, es decir, $S \subseteq \omega$, $L \subseteq \SIGMA$.

    \vspace{3mm}
    \begin{tabular}{|c|} \hline $\Rightarrow$ \\\hline \end{tabular} Veremos que $L_{1}, S_{1}$ es $\Sigma$-PR. Sea
    $(z_{1}, \gamma_{1})$ un elemento fijo de $S_{1} \times L_{1}$. Notar que:

    \begin{eqnarray}
      \nonumber x \in S_{1} \text{ sii } (x, \gamma_{1}) \in S_{1} \times L_{1} \\
      \nonumber \alpha \in L_{1} \text{ sii } (z_{1}, \alpha) \in S_{1} \times L_{1}
    \end{eqnarray}

    \par lo cual implica que:

    \begin{eqnarray}
      \nonumber \chi_{S_{1}} = \chi_{S_{1} \times L_{1}} \circ \left(p_{1}^{1,0}, C_{\gamma_{1}}^{0,1}\right) \\
      \nonumber \chi_{L_{1}} = \chi_{S_{1} \times L_{1}} \circ \left(C_{z_{1}}^{0,1}, p_{1}^{0,1}\right)
    \end{eqnarray}

    \par por lo tanto, $L_{1}, S_{1}$ es $\Sigma$-PR.

    \vspace{3mm}
    \begin{tabular}{|c|} \hline $\Leftarrow$\\\hline \end{tabular} Notar que:

    \[
      \chi_{S_{1} \times L_{1}} = \left(\chi_{S_{1}} \circ p_{1}^{1,1} \wedge \chi_{L_{1}} \circ p_{2}^{1,1} \right)
    \]

    \par luego, por el \textbf{Lemma 28}, $S_{1} \times L_{1}$ son $\Sigma$-PR.
  \end{proof}

  % Lemma 32: Con prueba.
  \begin{lemma}
    \par Supongamos $f: D_{f} \subseteq \omega^{n} \times \Sigma^{\ast m} \rightarrow O$ es $\Sigma$-PR, donde $O \in
    \{\omega, \SIGMA\}$. Si $S \subseteq D_{f}$ es $\Sigma$-PR, entonces $f \mid_{S}$ es $\Sigma$-PR.
  \end{lemma}
  \begin{proof}
    \begin{tabular}{|c|} \hline $O = \SIGMA$ \\\hline \end{tabular} Notar que:

    \[
      f \mid_{S} = \lambda x\alpha \left[\alpha^{x}\right] \circ (Suc \circ Pred \circ \chi_{S}, f)
    \]

    \par luego $f$ es $\Sigma$-PR.

    \begin{tabular}{|c|} \hline $O = \omega$ \\\hline \end{tabular} Notar que:

    \[
      f \mid_{S} = \lambda xy \left[x^{y}\right] \circ (f, Suc \circ Pred \circ \chi_{S})
    \]

    \par luego $f$ es $\Sigma$-PR.

    \vspace{3mm}
    \par Notar que \begin{tabular}{|c|} \hline $Suc \circ Pred \circ \chi_{S}$ \\\hline \end{tabular} funciona como un
    interruptor que evalua $f$, si el elemento pertenece a $S$, y que no evalua en caso contrario.
  \end{proof}

  % Lemma 33: Sin prueba.
  \begin{lemma}
    \par Si $f: D_{f} \subseteq \omega^{n} \times \SIGMA \rightarrow O$ es $\Sigma$-PR, entonces existe una función
    $\Sigma$-PR $\bar{f}: \omega^{n} \times \Sigma^{\ast m} \rightarrow O$, tal que $f = \bar{f} \mid_{D_{f}}$.
  \end{lemma}

  % Proposition 34: Con prueba.
  \begin{proposition}
    \par Un conjunto $S$ es $\Sigma$-PR sii $S$ es el dominio de una función $\Sigma$-PR.
  \end{proposition}
  \begin{proof}
    \begin{tabular}{|c|} \hline $\Rightarrow$ \\\hline \end{tabular} Notar que $S = D_{Pred \circ \chi_{S}}$.

    \begin{tabular}{|c|} \hline $\Leftarrow$ \\\hline \end{tabular} Probaremos por inducción en $k$ que $D_{F}$ es
    $\Sigma$-PR para cada $F \in PR_{k}^{\Sigma}$.

    \vspace{3mm}
    \underline{Caso Base:} \begin{tabular}{|c|} \hline $k = 0$ \\\hline \end{tabular} es decir, $F \in PR_{0}^{\Sigma}$.
    Luego:

    \begin{eqnarray}
      \nonumber F &\in& \{Suc, Pred, C_{0}^{0,0}, C_{\varepsilon}^{0,0}\} \cup \{d_{a}: a \in \Sigma\} \cup \\
      \nonumber && \{p_{j}^{n,m}: 1 \leq j \geq n+m\} \\
      \nonumber D_{F} &\in& \{\omega, \mathbb{N}\}
    \end{eqnarray}

    \par luego, $S$ es $\Sigma$-PR.

    \vspace{3mm}
    \underline{Caso Inductivo:} Supongamos el resultado vale para un $k$ fijo y supongamos $F \in
    \mathrm{PR}_{k+1}^{\Sigma}$, veremos entonces que $D_{F}$ es $\Sigma$-PR. Existen varios casos, analizaremos cada
    uno por separado.

    \begin{enumerate}
      \item \begin{tabular}{|c|} \hline $F = R(f, g)$ \\\hline \end{tabular}
        \begin{itemize}
          \item Recursión primitiva sobre variable numérica.
          \begin{enumerate}
            \item Caso 1:
            \begin{eqnarray}
              \nonumber f &:& S_{1} \times \dotsc \times S_{n} \times L_{1} \times \dotsc \times L_{m} \rightarrow
                \omega \\
              \nonumber g &:& \omega \times \omega \times S_{1} \times \dotsc \times S_{n} \times L_{1} \times \dotsc
                \times L_{m} \rightarrow \omega \\
              \nonumber F &=& \omega \times S_{1} \times \dotsc \times S_{n} \times L_{1} \times \dotsc \times L_{m}
                \rightarrow \omega
            \end{eqnarray}
            \item Caso 2:
            \begin{eqnarray}
              \nonumber f &:& S_{1} \times \dotsc \times S_{n} \times L_{1} \times \dotsc \times L_{m} \rightarrow
                \SIGMA \\
              \nonumber g &:& \omega \times S_{1} \times \dotsc \times S_{n} \times L_{1} \times \dotsc \times L_{m}
                \times \SIGMA \rightarrow \SIGMA \\
              \nonumber F &=& \omega \times S_{1} \times \dotsc \times S_{n} \times L_{1} \times \dotsc \times L_{m}
                \rightarrow \SIGMA
            \end{eqnarray}
          \end{enumerate}

          \item Recursión primitiva sobre variable alfabética.
          \begin{enumerate}
            \item Caso 1:
            \begin{eqnarray}
              \nonumber f &:& S_{1} \times \dotsc \times S_{n} \times L_{1} \times \dotsc \times L_{m} \rightarrow
                \omega \\
              \nonumber \mathcal{G}_{a} &:& \omega \times S_{1} \times \dotsc \times S_{n} \times L_{1} \times \dotsc
              \times L_{m} \times \SIGMA \rightarrow \omega \\
              \nonumber F &=& S_{1} \times \dotsc \times S_{n} \times L_{1} \times \dotsc \times L_{m} \times \SIGMA
                \rightarrow \omega
            \end{eqnarray}
            \item Caso 2:
            \begin{eqnarray}
              \nonumber f &:& S_{1} \times \dotsc \times S_{n} \times L_{1} \times \dotsc \times L_{m} \rightarrow
                \SIGMA \\
              \nonumber \mathcal{G}_{a} &:& S_{1} \times \dotsc \times S_{n} \times L_{1} \times \dotsc \times L_{m}
                \times \SIGMA \times \SIGMA \rightarrow \SIGMA \\
              \nonumber F &=& S_{1} \times \dotsc \times S_{n} \times L_{1} \times \dotsc \times L_{m} \times \SIGMA
                \rightarrow \SIGMA
            \end{eqnarray}
          \end{enumerate}
        \end{itemize}

        \par con $S_{1}, \dotsc, S_{n} \subseteq \omega$ y $L_{1}, \dotsc, L_{m} \subseteq \SIGMA$ conjuntos no vacíos
        y $f, g \in \mathrm{PR}_{k}^{\Sigma}$, para todos los casos anteriores.

        \par Por hipótesis inductiva tenemos que $D_{f} = S_{1} \times \dotsc \times S_{n} \times L_{1} \times \dotsc
        \times L_{m}$ es $\Sigma$-PR, lo cual por el \textbf{Lemma 31} nos dice que los conjuntos $S_{1}, \dotsc, S_{n},
        L_{1}, \dotsc, L_{m}$ son $\Sigma$-PR. Ya que $\omega$ es $\Sigma$-PR, el \textbf{Lemma 31} nos dice que $D_{F}$
        es $\Sigma$-PR.

      \item \begin{tabular}{|c|} \hline $F = g \circ (g_{1}, \dotsc, g_{n+m})$ \\\hline \end{tabular} donde:

        \begin{eqnarray}
          \nonumber g &:& D_{g} \subseteq \omega^{n} \times \Sigma^{\ast m} \rightarrow O \\
          \nonumber g_{i} &:& D_{g_{i}} \subseteq \omega^{k} \times \Sigma^{\ast l} \rightarrow \omega \qquad \;\; i = 1,
          \dotsc, n \\
          \nonumber g_{i} &:& D_{g_{i}} \subseteq \omega^{k} \times \Sigma^{\ast l} \rightarrow \SIGMA \qquad i = n + 1,
          \dotsc, n + m
        \end{eqnarray}

        \par están en $\mathrm{PR}_{k}^{\Sigma}$. Por \textbf{Lemma 33}, hay funciones $\Sigma$-PR $\bar{g}_{1}, \dotsc,
        \bar{g}_{n+m}$ las cuales son $\Sigma$-totales y cumplen:

        \begin{eqnarray}
          \nonumber g_{i} &=& \bar{g}_{i} \mid_{D_{g_{i}}} \\
          \nonumber \text{para } i &=& 1, \dotsc, n + m
        \end{eqnarray}

        \par Por hipótesis inductiva, los conjuntos $D_{g}$, $D_{g_{i}}$, para $i = 1, \dotsc, n + m$, son $\Sigma$-PR y
        por lo tanto:

        \[
          S = \bigcap_{i=1}^{n+m} D_{g_{i}}
        \]

        \par lo es. Notese además, que:

        \[
          \chi_{D_{F}} = \left((\chi_{D_{g}} \circ (\bar{g}_{1}, \dotsc, \bar{g}_{n+m})) \wedge \chi_{S}\right)
        \]

        \par lo cual nos dice que $D_{F}$ es $\Sigma$-PR.
    \end{enumerate}
  \end{proof}
