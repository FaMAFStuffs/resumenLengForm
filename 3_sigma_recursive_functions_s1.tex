\section{Funciones $\Sigma$-recursivas}

  % Lemma 18
  \begin{lemma}
    \par Si $f, f_{1}, \dotsc, f_{n+m}$ son $\Sigma$-efectivamente computables, entonces $f \circ (f_{1}, \dotsc,
    f_{n+m})$ lo es.
  \end{lemma}
  \begin{proof}
    Sean $\mathbb{P}, \mathbb{P}_{1}, \dotsc, \mathbb{P}_{n+m}$ procedimientos efectivos los cuales computen las
    funciones $f, f_{1}, \dotsc, f_{n+m}$, respectivamente. Usando estos procedimientos es fácil definir un
    procedimiento efectivo el cual compute a $f \circ (f_{1},\dotsc,f_{n+m})$.
  \end{proof}

  % Lemma 19
  \begin{lemma}
    \par Si $f$ y $g$ son $\Sigma $-efectivamente computables, entonces $R(f,g)$ lo es.
  \end{lemma}
  \begin{proof}
    HACER!!!!!!!
  \end{proof}

  % Lemma 20
  \begin{lemma}
    \par Si $f$ y cada $\mathcal{G}_{a}$ son $\Sigma$-efectivamente computables, entonces $R(f, \mathcal{G})$ lo es.
  \end{lemma}

  % Theorem 21
  \begin{theorem}
    \par Si $f \in \mathrm{PR}^{\Sigma}$, entonces $f$ es $\Sigma$-efectivamente computable.
  \end{theorem}
  \begin{proof}
    \par Dejamos al lector la prueba por inducción en $k$ de que si $f \in \mathrm{PR}_{k}^{\Sigma}$, entonces $f$ es
    $\Sigma$-efectivamente computable, la cual sale en forma directa usando los lemas anteriores que garantizan que los
    constructores de composición y recursión primitiva preservan la computabilidad efectiva.
  \end{proof}

  % Lemma 22
  \begin{lemma}
    \begin{enumerate}
      \item $\emptyset \in \mathrm{PR}^{\emptyset}$.
      \item $\lambda xy \left[x+y\right] \in \mathrm{PR}^{\emptyset}$.
      \item $\lambda xy\left[x.y\right] \in \mathrm{PR}^{\emptyset}$.
      \item $\lambda x\left[x!\right] \in \mathrm{PR}^{\emptyset}$.
    \end{enumerate}
  \end{lemma}
  \begin{proof}
    \begin{enumerate}
      \item Notese que $\emptyset = Pred \circ C_{0}^{0,0} \in \mathrm{PR}_{1}^{\emptyset}$
      \item Notar que:
        \begin{eqnarray}
          \nonumber \lambda xy\left[x+y\right](0, x_{1}) &=& x_{1} = p_{1}^{1,0}(x_{1}) \\
          \nonumber \lambda xy\left[x+y\right](t+1, x_{1}) &=& \lambda xy\left[x+y\right](t, x_{1}) + 1 \\
          \nonumber & =& (Suc \circ p_{1}^{3,0}) (\lambda xy\left[x+y\right](t,x_{1}), t, x_{1})
        \end{eqnarray}

        \par lo cual implica que $\lambda xy\left[x+y\right] = R (p_{1}^{1, 0}, Suc \circ p_{1}^{3,0}) \in
        \mathrm{PR}_{2}^{\emptyset}.$
      \item Primero note que:
        \begin{eqnarray}
          \nonumber C_{0}^{1,0}(0) &=& C_{0}^{0,0}(\Diamond) \\
          \nonumber C_{0}^{1,0}(t+1) &=& C_{0}^{1,0}(t)
        \end{eqnarray}

        \par lo cual implica que $C_{0}^{1,0} = R (C_{0}^{0,0}, p_{1}^{2, 0}) \in \mathrm{PR}_{1}^{\emptyset}$.
        También note que $\lambda tx \left[t.x\right] = R (C_{0}^{1,0}, \lambda xy\left[x+y\right] \circ (p_{1}^{3, 0},
        p_{3}^{3,0}))$, lo cual por (1) implica que $\lambda tx\left[t.x\right] \in \mathrm{PR}_{3}^{\emptyset}$.
      \item Note que:
        \begin{eqnarray}
          \nonumber \lambda x\left[x!\right](0) &=& 1 = C_{1}^{0,0}(\Diamond) \\
          \nonumber \lambda x\left[x!\right](t+1) &=& \lambda x\left[x!\right](t).(t+1)
        \end{eqnarray}

        \par lo cual implica que: $\lambda x \left[x!\right] = R (C_{1}^{0, 0}, \lambda xy\left[x.y\right] \circ
        (p_{1}^{2, 0}, Suc \circ p_{2}^{2,0}))$.

        \par Ya que $C_{1}^{0, 0} = Suc \circ C_{0}^{0,0}$, tenemos que $C_{1}^{0, 0} \in \mathrm{PR}_{1}^{\emptyset}$.
        Por (2), tenemos que $\lambda xy \left[x.y\right] \circ (p_{1}^{2, 0}, Suc \circ p_{2}^{2,0}) \in
        \mathrm{PR}_{4}^{\emptyset}$, obteniendo que $\lambda x \left[x!\right] \in \mathrm{PR}_{5}^{\emptyset}$.
    \end{enumerate}
  \end{proof}

  % Lemma 23
  \begin{lemma}
    \par Supongamos $\Sigma \neq \emptyset$.

    \begin{enumerate}[a)]
      \item $\lambda \alpha \beta \left[\alpha\beta\right] \in \mathrm{PR}^{\Sigma}$.
      \item $\lambda \alpha \left[\left\vert\alpha \right\vert\right] \in \mathrm{PR}^{\Sigma}$.
    \end{enumerate}
  \end{lemma}
  \begin{proof}
    \begin{enumerate}[a)]
      \item Ya que:
        \begin{eqnarray}
          \nonumber \lambda \alpha \beta \left[\alpha \beta\right](\alpha_{1}, \varepsilon) &=& \alpha_{1} =
          p_{1}^{0,1}(\alpha_{1}) \\
          \nonumber \lambda \alpha \beta \left[\alpha \beta\right](\alpha_{1}, \alpha a) &=& d_{a}(\lambda \alpha \beta
          \left[\alpha \beta\right](\alpha_{1}, \alpha), a \in \Sigma
        \end{eqnarray}

        \par tenemos que $\lambda \alpha \beta \left[\alpha \beta\right] = R(p_{1}^{0, 1}, \mathcal{G})$, donde
        $\mathcal{G}_{a} = d_{a} \circ p_{3}^{0,3}$, para cada $a \in \Sigma$.
      \item Ya que:
        \begin{eqnarray}
          \nonumber \lambda \alpha \left[\left\vert \alpha \right\vert\right](\varepsilon) &=& 0 =
          C_{0}^{0,0}(\Diamond) \\
          \nonumber \lambda \alpha \left[\left\vert \alpha \right\vert\right](\alpha a) &=& \lambda \alpha
          \left[\left\vert \alpha \right\vert\right](\alpha) + 1
        \end{eqnarray}

        \par tenemos que $\lambda \alpha \left[\left\vert \alpha \vert \right] = R(C_{0}^{0, 0}, \mathcal{G}\right)$,
        donde $\mathcal{G}_{a} = Suc \circ p_{1}^{1, 1}$, para cada $a \in \Sigma$.
    \end{enumerate}
  \end{proof}

  % Lemma 24
  \begin{lemma}
    \begin{enumerate}[a)]
      \item $C_{k}^{n, m}, C_{\alpha}^{n, m} \in \mathrm{PR}^{\Sigma}$, para $n, m, k \geq 0, \alpha \in \SIGMA$.
      \item $C_{k}^{n, 0} \in \mathrm{PR}^{\emptyset}$, para $n, k \geq 0$.
    \end{enumerate}
  \end{lemma}

  % Lemma 25
  \begin{lemma}
    \begin{enumerate}
      \item $\lambda xy \left[x^{y}\right] \in \mathrm{PR}^{\emptyset}$.
      \item $\lambda t \alpha \left[\alpha^{t}\right] \in \mathrm{PR}^{\Sigma}$.
    \end{enumerate}
  \end{lemma}
  \begin{proof}
    \begin{enumerate}[a)]
      \item Note que: $\lambda tx \left[x^{t}\right] = R(C_{1}^{1, 0}, \lambda xy \left[x.y\right] \circ (p_{1}^{3, 0}, p_{3}^{3, 0})) \in \mathrm{PR}^{\emptyset}$.

        \par Osea que $\lambda xy \left[x^{y}\right] = \lambda tx \left[x^{t}\right] \circ (p_{2}^{2, 0}, p_{1}^{2, 0}) \in \mathrm{PR}^{\emptyset}$.
      \item Note que: $\lambda t \alpha \left[\alpha^{t}\right] = R(C_{\varepsilon}^{0, 1}, \lambda \alpha \beta \left[\alpha \beta\right] \circ (p_{3}^{1, 2}, p_{2}^{1, 2})) \in \mathrm{PR}^{\Sigma}$.
    \end{enumerate}
  \end{proof}

  % Lemma 26
  \begin{lemma}
    \par Si $<$ es un orden total estricto sobre un alfabeto no vacío $\Sigma$, entonces $s^{<}, \#^{<}$ y $\ast^{<}$
    pertenecen a $\mathrm{PR}^{\Sigma}$.
  \end{lemma}
  \begin{proof}
    \par Supongamos $\Sigma = \{a_{1}, \dotsc, a_{k}\}$ y $<$ dado por $a_{1} < \dotsc < a_{k}$. Ya que:

    \begin{eqnarray}
      \nonumber s^{<}(\varepsilon) &=& a_{1} \\
      \nonumber s^{<}(\alpha a_{i}) &=& \alpha a_{i+1} \text{, para } i < k \\
      \nonumber s^{<}(\alpha a_{k}) &=& s^{<}(\alpha) a_{1}
    \end{eqnarray}

    \par tenemos que $s^{<} = R(C_{a_{1}}^{0, 0}, \mathcal{G})$, donde $\mathcal{G}_{a_{i}} = d_{a_{i+1}} \circ
    p_{1}^{0, 2}$, para $i = 1, \dotsc, k-1$ y $\mathcal{G}_{a_{k}} = d_{a_{1}} \circ p_{2}^{0,2}.$ Osea que $s^{<} \in
    \mathrm{PR}^{\Sigma}$. Ya que:

    \begin{eqnarray}
      \nonumber \ast^{<}(0) &=& \varepsilon \\
      \nonumber \ast^{<}(t+1) &=& s^{<}(\ast^{<}(t))
    \end{eqnarray}

    \par podemos ver que $\ast^{<} \in \mathrm{PR}^{\Sigma}$. Ya que:

    \begin{eqnarray}
      \nonumber \#^{<}(\varepsilon) &=& 0 \\
      \nonumber \#^{<}(\alpha a_{i}) &=& \#^{<}(\alpha). k + i \text{, para } i = 1, \dotsc, k
    \end{eqnarray}

    \par tenemos que $\#^{<} = R(C_{0}^{0, 0}, \mathcal{G})$, donde $\mathcal{G}_{a_{i}} = \lambda xy \left[x+y\right]
    \circ (\lambda xy \left[x.y\right] \circ (p_{1}^{1, 1}, C_{k}^{1, 1}), C_{i}^{1, 1})
    \text{, para } i = 1, \dotsc, k$.

    \par Osea que $\#^{<} \in \mathrm{PR}^{\Sigma}$.
  \end{proof}

  % Lemma 27
  \begin{lemma}
    \begin{enumerate}[a)]
      \item $\lambda xy \left[x \dot{-}y\right] \in \mathrm{PR}^{\emptyset}$.
      \item $\lambda xy \left[\max (x,y)\right] \in \mathrm{PR}^{\emptyset}$.
      \item $\lambda xy \left[x=y\right] \in \mathrm{PR}^{\emptyset}$.
      \item $\lambda xy \left[x \leq y\right] \in \mathrm{PR}^{\emptyset}$.
      \item Si $\Sigma \neq \emptyset \Rightarrow \lambda \alpha \beta \left[\alpha = \beta\right] \in
        \mathrm{PR}^{\Sigma}$.
    \end{enumerate}
  \end{lemma}
  \begin{proof}
    \begin{enumerate}[a)]
      \item Primero notar que $\lambda x \left[x \dot{-}1\right] = R(C_{0}^{0, 0}, p_{2}^{2, 0}) \in
        \mathrm{PR}^{\emptyset}$. También note que $\lambda tx \left[x \dot{-}t\right] = R(p_{1}^{1, 0}, \lambda x
        \left[x \dot{-}1\right] \circ p_{1}^{3, 0}) \in \mathrm{PR}^{\emptyset}$.

        \par Osea que $\lambda xy \left[x \dot{-}y\right] = \lambda tx \left[x \dot{-}t\right] \circ (p_{2}^{2, 0},
        p_{1}^{2, 0}) \in \mathrm{PR}^{\emptyset}$.
      \item Note que $\lambda xy \left[\max (x,y)\right] = \lambda xy \left[(x + (y \dot{-}x)\right]$.
      \item Note que $\lambda xy \left[x = y\right] = \lambda xy \left[1 \dot{-}((x \dot{-} y) + (y \dot{-} x))\right]$.
      \item Note que $\lambda xy \left[x \leq y\right] = \lambda xy\left[1 \dot{-}(x \dot{-}y)\right]$.
      \item Sea $<$ un orden total estricto sobre $\Sigma$. Ya que $\alpha = \beta \text{ sii } \#^{<}(\alpha) =
        \#^{<}(\beta)$, tenemos que $\lambda \alpha \beta \left[\alpha = \beta\right] = \lambda xy \left[x = y\right]
        \circ (\#^{<} \circ p_{1}^{0, 2}, \#^{<} \circ p_{2}^{0, 2})$.

        \par Osea que podemos aplicar (c) y \textbf{Lema 28} implica que $\chi_{S}$ es $\Sigma$-PR.
    \end{enumerate}
  \end{proof}

  % Lemma 28
  \begin{lemma}
    \par Si $P: S \subseteq \omega^{n} \times \Sigma^{\ast m} \rightarrow \omega$ y $Q: S \subseteq \omega^{n} \times
    \Sigma^{\ast m} \rightarrow \omega$ son predicados $\Sigma$-PR, entonces $(P \vee Q), (P \wedge Q)$ y $not P$ lo son
    también.
  \end{lemma}
  \begin{proof}
    \par Note que:

    \begin{eqnarray}
      \nonumber \neg P &=& \lambda xy \left[x \dot{-} y\right] \circ (C_{1}^{n, m}, P) \\
      \nonumber (P \wedge Q) &=& \lambda xt \left[x.y\right] \circ (P, Q) \\
      \nonumber (P \vee Q) &=& \neg (\neg P \wedge \neg Q)
    \end{eqnarray}
  \end{proof}

  % Lemma 29
  \begin{lemma}
    \par Si $S_{1}, S_{2} \subseteq \omega^{n} \times \Sigma^{\ast m}$ son $\Sigma$-PR, entonces $S_{1} \cup S_{2}$,
    $S_{1} \cap S_{2}$ y $S_{1} - S_{2}$ lo son.
  \end{lemma}
  \begin{proof}
    \par Note que:

    \begin{eqnarray}
      \nonumber \chi_{S_{1} \cup S_{2}} &=& (\chi_{S_{1}} \vee \chi_{S_{2}}) \\
      \nonumber \chi_{S_{1} \cap S_{2}} &=& (\chi_{S_{1}} \wedge \chi_{S_{2}}) \\
      \nonumber \chi_{S_{1} - S_{2}} &=& \lambda \left[x \dot{-} y\right] \circ (\chi_{S_{1}}, \chi_{S_{2}})
    \end{eqnarray}
  \end{proof}

  % Corollary 30
  \begin{corollary}
    \par Si $S \subseteq \omega^{n} \times \Sigma^{\ast m}$ es finito, entonces $S$ es $\Sigma$-PR.
  \end{corollary}
  \begin{proof}
    \par Por el lema anterior, podemos suponer que:

    \[
      S = \{ (z_{1}, \dotsc, z_{n}, \gamma_{1}, \dotsc, \gamma_{m}) \}
    \]

    \par Noque $\chi_{S}$ es el siguiente predicado:

    \[
      \left( \chi_{z_{1}} \circ p_{1}^{n, m} \wedge \dotsc \wedge \chi_{z_{n}} \circ p_{n}^{n, m} \wedge
      \chi_{z_{1}} \circ p_{n+1}^{n, m} \wedge \dotsc \wedge \chi_{\gamma_{m}} \circ p_{n+m}^{n, m} \right)
    \]

    \par Ya que los predicados:

    \begin{eqnarray}
      \nonumber \chi_{z_{i}} = \lambda xy \left[x = y\right] \circ \left(p_{1}^{1, 0}, C_{z_{i}}^{1, 0} \right)
      \nonumber \chi_{\gamma_{i}} = \lambda \alpha \beta \left[\alpha = \beta\right] \circ \left(p_{1}^{0, 1},
      C_{\gamma_{i}}^{0, 1} \right)
    \end{eqnarray}

    \par son $\Sigma$-PR, el \textbf{Lema 28} implica que $\chi_{S}$ es $\Sigma$-PR.
  \end{proof}

  % Lemma 31
  \begin{lemma}
    \par Supongamos $S_{1}, \dotsc, S_{n} \subseteq \omega, L_{1}, \dotsc, L_{m} \subseteq \SIGMA$ son conjuntos no
    vacíos, entonces $S_{1} \times \dotsc \times S_{n} \times L_{1} \times \dotsc \times L_{m}$ es $\Sigma$-PR sii
    $S_{1}, \dotsc, S_{n}, L_{1}, \dotsc, L_{m}$ son $\Sigma$-PR.
  \end{lemma}
  \begin{proof}
    \begin{tabular}{|c|} \hline $\Rightarrow$\\\hline \end{tabular} Veremos por ejemplo que $L_{1}$ es $\Sigma$-PR. Sea
      $(z_{1}, \dotsc, z_{n}, z_{1}, \dotsc, z_{m})$ un elemento fijo de $S_{1} \times \dotsc \times S_{n}
      \times L_{1} \times \dotsc \times L_{m}$. Note que: $\alpha \in L_{1} \text{ sii } (z_{1}, \dotsc, z_{n}, \alpha,
      z_{2}, \dotsc, z_{m}) \in S_{1} \times \dotsc \times S_{n} \times L_{1} \times \dotsc \times L_{m}$,
      lo cual implica que $\chi_{L_{1}} = \chi_{S_{1} \times \dotsc \times S_{n} \times L_{1} \times \dotsc \times
      L_{m}} \circ (C_{z_{1}}^{0, 1}, \dotsc, C_{z_{n}}^{0, 1}, p_{1}^{0, 1}, C_{z_{2}}^{0, 1}, \dotsc,
      C_{z_{m}}^{0, 1})$.

    \begin{tabular}{|c|} \hline $\Leftarrow$\\\hline \end{tabular} Note que $\chi_{S_{1} \times \dotsc \times S_{n}
      \times L_{1} \times \dotsc \times L_{m}}$ es el predicado $(\chi_{S_{1}} \circ p_{1}^{n, m} \wedge \dotsc \wedge
      \chi_{S_{n}} \circ p_{n}^{n, m} \wedge \chi_{L_{1}} \circ p_{n+1}^{n, m} \wedge \dotsc \wedge \chi_{L_{m}} \circ
      p_{n+m}^{n, m})$.
  \end{proof}

  % Lemma 32
  \begin{lemma}
    \par Supongamos $f: D_{f} \subseteq \omega^{n} \times \Sigma^{\ast m} \rightarrow O$ es $\Sigma$-PR, donde $O \in
    \{\omega, \SIGMA\}$. Si $S \subseteq D_{f}$ es $\Sigma$-PR, entonces $f \mid_{S}$ es $\Sigma$-PR.
  \end{lemma}
  \begin{proof}
    \par Supongamos $O = \SIGMA$, entonces $f \mid_{S} = \lambda x \alpha \left[\alpha^{x}\right] \circ (Suc \circ Pred
    \circ \chi_{S}, f)$ es $\Sigma$-PR. El caso $O = \omega$ es similar usando $\lambda xy \left[x^{y}\right]$ en lugar
    de $\lambda x \alpha \left[\alpha^{x}\right]$.
  \end{proof}

  % Lemma 33
  \begin{lemma}
    \par Si $f: D_{f} \subseteq \omega^{n} \times \SIGMA \rightarrow O$ es $\Sigma$-PR, entonces existe una función
    $\Sigma$PR. $\bar{f}: \omega^{n} \times \Sigma^{\ast m} \rightarrow O$, tal que $f = \bar{f} \mid_{D_{f}}$.
  \end{lemma}
  \begin{proof}
    \par Es fácil ver por inducción en $k$ que el enunciado se cumple para cada $f \in \mathrm{PR}_{k}^{\Sigma}$.
  \end{proof}

  % Proposition 34
  \begin{proposition}
    \par Un conjunto $S$ es $\Sigma$-PR sii $S$ es el dominio de una función $\Sigma$-PR.
  \end{proposition}
  \begin{proof}
    \begin{tabular}{|c|} \hline $\Rightarrow$\\\hline \end{tabular} Note que $S = D_{Pred \circ \chi_{S}}$.

    \begin{tabular}{|c|} \hline $\Leftarrow$\\\hline \end{tabular} Probaremos por inducción en $k$ que $D_{F}$ es
      $\Sigma$-PR, para cada $F \in \mathrm{PR}_{k}^{\Sigma}$. El caso $k=0$ es fácil. Supongamos el resultado vale para
      un $k$ fijo y supongamos $F \in \mathrm{PR}_{k+1}^{\Sigma}$. Veremos entonces que $D_{F}$ es $\Sigma$-PR. Hay
      varios casos.

    \begin{enumerate}
      \item \begin{tabular}{|c|} \hline $F = R(f, g)$\\\hline \end{tabular}
        \begin{eqnarray}
          \nonumber f &:& S_{1} \times \dotsc \times S_{n} \times L_{1} \times \dotsc \times L_{m} \rightarrow \SIGMA \\
          \nonumber g &:& \omega \times S_{1} \times \dotsc \times S_{n} \times L_{1} \times \dotsc \times L_{m} \times
          \SIGMA \rightarrow \SIGMA
        \end{eqnarray}

        \par con $S_{1}, \dotsc, S_{n} \subseteq \omega$ y $L_{1}, \dotsc, L_{m} \subseteq \SIGMA$ conjuntos no vacíos
        y $f, g \in \mathrm{PR}_{k}^{\Sigma}$. Notese que por definición de $R(f, g)$, tenemos que $D_{F} = \omega
        \times S_{1} \times \dotsc \times S_{n} \times L_{1} \times \dotsc \times L_{m}$.

        \par Por HI tenemos que $D_{f} = S_{1} \times \dotsc \times S_{n} \times L_{1} \times \dotsc \times L_{m}$ es
        $\Sigma$-PR, lo cual por el \textbf{Lema 31} nos dice que los conjuntos $S_{1}, \dotsc, S_{n}, L_{1}, \dotsc,
        L_{m}$ son $\Sigma$-PR. Ya que $\omega$ es $\Sigma$-PR, el \textbf{Lema 31} nos dice que $D_{F}$ es $\Sigma$-PR.

        \par Los otros casos de recursión primitiva son dejados al lector.

      \item \begin{tabular}{|c|} \hline $F = g \circ (g_{1}, \dotsc, g_{n+m})$\\\hline \end{tabular}
        \begin{eqnarray}
          \nonumber g &:& D_{g} \subseteq \omega^{n} \times \Sigma^{\ast m} \rightarrow O \\
          \nonumber g_{i} &:& D_{g_{i}} \subseteq \omega^{k} \times \Sigma^{\ast l} \rightarrow \omega \text{, } i = 1,
          \dotsc, n \\
          \nonumber g_{i} &:& D_{g_{i}} \subseteq \omega^{k} \times \Sigma^{\ast l} \rightarrow \SIGMA, i = n + 1,
          \dotsc, n + m
        \end{eqnarray}

        \par están en $\mathrm{PR}_{k}^{\Sigma}$. Por \textbf{Lema 33}, hay funciones $\Sigma$-PR $\bar{g}_{1}, \dotsc,
        \bar{g}_{n+m}$ las cuales son $\Sigma$-totales y cumplen $g_{i} = \bar{g}_{i} \mid_{D_{g_{i}}} \text{, para }
        i = 1, \dotsc, n + m$.

        \par Por HI los conjuntos $D_{g}$, $D_{g_{i}}$, $i = 1, \dotsc, n + m$, son $\Sigma$-PR y por lo tanto
        $S = \bigcap_{i=1}^{n+m} D_{g_{i}}$ lo es. Notese que $\chi_{D_{F}} = (\chi_{D_{g}} \circ (\bar{g}_{1}, \dotsc,
        \bar{g}_{n+m}) \wedge \chi_{S})$ lo cual nos dice que $D_{F}$ es $\Sigma$-PR.
    \end{enumerate}
  \end{proof}
