\section{Funciones $\Sigma$-recursivas}

  \textbf{\underline{Lemma 18:}} Si \(f,f_{1},...,f_{n+m}\) son \(\Sigma \)-efectivamente computables, entonces \( f\circ (f_{1},...,f_{n+m})\) lo es.

  \PROOF Sean \(\mathbb{P},\mathbb{P}_{1},...,\mathbb{P}_{n+m}\) procedimientos efectivos los cuales computen las funciones \(f,f_{1},...,f_{n+m}\), respectivamente. Usando estos procedimientos es facil definir un procedimiento efectivo el cual compute a \(f\circ (f_{1},...,f_{n+m})\). \(\Box\)

  \textbf{\underline{Lemma 19:}} Si \(f\) y \(g\) son \(\Sigma \)-efectivamente computables, entonces \(R(f,g)\) lo es.

  \PROOF La Proof es dejada al lector. \(\Box\)

  \textbf{\underline{Lemma 20:}} Si \(f\) y cada \(\mathcal{G}_{a}\) son \(\Sigma \)-efectivamente computables, entonces \(R(f,\mathcal{G})\) lo es.

  \PROOF Es dejada al lector con la recomendacion de que haga la Proof para el caso \( \Sigma =\{@,\& \}\) \(\Box\)

  \textbf{\underline{Theorem 21:}} Si \(f\in \mathrm{PR}^{\Sigma }\), entonces \(f\) es \(\Sigma \)-efectivamente computable.

  \PROOF Dejamos al lector la Proof por induccion en \(k\) de que si \(f\in \mathrm{PR} _{k}^{\Sigma }\), entonces \(f\) es \(\Sigma \)-efectivamente computable, la cual sale en forma directa usando los lemas anteriores que garantizan que los constructores de composicion y recursion primitiva preservan la computabilidad efectiva \(\Box\)

  \textbf{\underline{Lemma 22:}}
  (1) \(\varnothing \in \mathrm{PR}^{\varnothing }\).
  (2) \(\lambda xy\left[ x+y\right] \in \mathrm{PR}^{\varnothing }\).
  (3) \(\lambda xy\left[ x.y\right] \in \mathrm{PR}^{\varnothing }\).
  (4) \(\lambda x\left[ x!\right] \in \mathrm{PR}^{\varnothing }\).

  \PROOF (1) Notese que \(\varnothing =Pred\circ C_{0}^{0,0}\in \mathrm{PR} _{1}^{\varnothing }\)

  (2) Notar que

  \(\displaystyle \begin{array}{rcl} \lambda xy\left[ x+y\right] (0,x_{1}) & =& x_{1}=p_{1}^{1,0}(x_{1}) \\ \lambda xy\left[ x+y\right] (t+1,x_{1}) & =& \lambda xy\left[ x+y\right] (t,x_{1})+1 \\ & =& \left( Suc\circ p_{1}^{3,0}\right) \left( \lambda xy\left[ x+y\right] (t,x_{1}),t,x_{1}\right) \end{array} \)

  lo cual implica que \(\lambda xy\left[ x+y\right] =R\left( p_{1}^{1,0},Suc\circ p_{1}^{3,0}\right) \in \mathrm{PR}_{2}^{\varnothing }.\)
  (3) Primero note que

  \(\displaystyle \begin{array}{rcl} C_{0}^{1,0}(0) & =& C_{0}^{0,0}(\Diamond ) \\ C_{0}^{1,0}(t+1) & =& C_{0}^{1,0}(t) \end{array} \)

  lo cual implica que \(C_{0}^{1,0}=R\left( C_{0}^{0,0},p_{1}^{2,0}\right) \in \mathrm{PR}_{1}^{\varnothing }.\) Tambien note que
  \(\displaystyle \lambda tx\left[ t.x\right] =R\left( C_{0}^{1,0},\lambda xy\left[ x+y\right] \circ \left( p_{1}^{3,0},p_{3}^{3,0}\right) \right) , \)

  lo cual por (1) implica que \(\lambda tx\left[ t.x\right] \in \mathrm{PR} _{3}^{\varnothing }\).
  (4) Note que

  \(\displaystyle \begin{array}{rcl} \lambda x\left[ x!\right] (0) & =& 1=C_{1}^{0,0}(\Diamond ) \\ \lambda x\left[ x!\right] (t+1) & =& \lambda x\left[ x!\right] (t).(t+1), \end{array} \)

  lo cual implica que
  \(\displaystyle \lambda x\left[ x!\right] =R\left( C_{1}^{0,0},\lambda xy\left[ x.y\right] \circ \left( p_{1}^{2,0},Suc\circ p_{2}^{2,0}\right) \right) . \)

  Ya que \(C_{1}^{0,0}=\) \(Suc\circ C_{0}^{0,0}\), tenemos que \(C_{1}^{0,0}\in \mathrm{PR}_{1}^{\varnothing }\). Por (2), tenemos que
  \(\displaystyle \lambda xy\left[ x.y\right] \circ \left( p_{1}^{2,0},Suc\circ p_{2}^{2,0}\right) \in \mathrm{PR}_{4}^{\varnothing }, \)

  obteniendo que \(\lambda x\left[ x!\right] \in \mathrm{PR}_{5}^{\varnothing }\). \(\Box\)


  \textbf{\underline{Lemma 23:}} Supongamos \(\Sigma \) es no vacio.
  (a) \(\lambda \alpha \beta \left[ \alpha \beta \right] \in \mathrm{PR} ^{\Sigma }\)
  (b) \(\lambda \alpha \left[ \left\vert \alpha \right\vert \right] \in \mathrm{PR}^{\Sigma }\)

  \PROOF (a) Ya que

  \(\displaystyle \begin{array}{rcl} \lambda \alpha \beta \left[ \alpha \beta \right] (\alpha _{1},\varepsilon ) & =& \alpha _{1}=p_{1}^{0,1}(\alpha _{1}) \\ \lambda \alpha \beta \left[ \alpha \beta \right] (\alpha _{1},\alpha a) & =& d_{a}(\lambda \alpha \beta \left[ \alpha \beta \right] (\alpha _{1},\alpha )),a\in \Sigma \end{array} \)

  tenemos que \(\lambda \alpha \beta \left[ \alpha \beta \right] =R\left( p_{1}^{0,1},\mathcal{G}\right) \), donde \(\mathcal{G}_{a}=d_{a}\circ p_{3}^{0,3}\), para cada \(a\in \Sigma \).
  (b) Ya que

  \(\displaystyle \begin{array}{rcl} \lambda \alpha \left[ \left\vert \alpha \right\vert \right] (\varepsilon ) & =& 0=C_{0}^{0,0}(\Diamond ) \\ \lambda \alpha \left[ \left\vert \alpha \right\vert \right] (\alpha a) & =& \lambda \alpha \left[ \left\vert \alpha \right\vert \right] (\alpha )+1 \end{array} \)

  tenemos que \(\lambda \alpha \left[ \left\vert \alpha \right\vert \right] =R\left( C_{0}^{0,0},\mathcal{G}\right) \), donde \(\mathcal{G}_{a}=\) \( Suc\circ p_{1}^{1,1}\), para cada \(a\in \Sigma .\). \(\Box\)


  \textbf{\underline{Lemma 24:}}
  (a) \(C_{k}^{n,m},C_{\alpha }^{n,m}\in \mathrm{PR}^{\Sigma }\), para \( n,m,k\geq 0\), \(\alpha \in \Sigma ^{\ast }\).
  (b) \(C_{k}^{n,0}\in \mathrm{PR}^{\varnothing }\), para \(n,k\geq 0\).


  \PROOF (a) Note que \(C_{k+1}^{0,0}=\) \(Suc\circ C_{k}^{0,0}\), lo cual implica \( C_{k}^{0,0}\in \mathrm{PR}_{k}^{\Sigma }\), para \(k\geq 0\). Tambien note que \( C_{\alpha a}^{0,0}=d_{a}\circ C_{\alpha }^{0,0}\), lo cual dice que \( C_{\alpha }^{0,0}\in \mathrm{PR}^{\Sigma }\), para \(\alpha \in \Sigma ^{\ast } \). Para ver que \(C_{k}^{0,1}\in \mathrm{PR}^{\Sigma }\) notar que

  \(\displaystyle \begin{array}{rcl} C_{k}^{0,1}(\varepsilon ) & =& k=C_{k}^{0,0}(\Diamond ) \\ C_{k}^{0,1}(\alpha a) & =& C_{k}^{0,1}(\alpha )=p_{1}^{1,1}\left( C_{k}^{0,1}(\alpha ),\alpha \right) \end{array} \)

  lo cual implica que \(C_{k}^{0,1}=R\left( C_{k}^{0,0},\mathcal{G}\right) \), con \(\mathcal{G}_{a}=p_{1}^{1,1}\), \(a\in \Sigma \). En forma similar podemos ver que \(C_{k}^{1,0},C_{\alpha }^{1,0},C_{\alpha }^{0,1}\in \mathrm{PR} ^{\Sigma }\). Supongamos ahora que \(m >0\). Entonces
  \(\displaystyle \begin{array}{rcl} C_{k}^{n,m} & =& C_{k}^{0,1}\circ p_{n+1}^{n,m} \\ C_{\alpha }^{n,m} & =& C_{\alpha }^{0,1}\circ p_{n+1}^{n,m} \end{array} \)

  de lo cual obtenemos que \(C_{k}^{n,m},C_{\alpha }^{n,m}\in \mathrm{PR} ^{\Sigma }\). El caso \(n >0\) es similar
  (b) Use argumentos similares a los usados en la Proof de (a). \(\Box\)


  \textbf{\underline{Lemma 25:}}
  (a) \(\lambda xy\left[ x^{y}\right] \in \mathrm{PR}^{\varnothing }\).
  (b) \(\lambda t\alpha \left[ \alpha ^{t}\right] \in \mathrm{PR} ^{\Sigma }\).


  \PROOF (a) Note que

  \(\displaystyle \lambda tx\left[ x^{t}\right] =R\left( C_{1}^{1,0},\lambda xy\left[ x.y \right] \circ \left( p_{1}^{3,0},p_{3}^{3,0}\right) \right) \in \mathrm{PR} ^{\varnothing }. \)

  O sea que \(\lambda xy\left[ x^{y}\right] =\lambda tx\left[ x^{t}\right] \circ \left( p_{2}^{2,0},p_{1}^{2,0}\right) \in \mathrm{PR}^{\varnothing }\).
  (b) Note que

  \(\displaystyle \lambda t\alpha \left[ \alpha ^{t}\right] =R\left( C_{\varepsilon }^{0,1},\lambda \alpha \beta \left[ \alpha \beta \right] \circ \left( p_{3}^{1,2},p_{2}^{1,2}\right) \right) \in \mathrm{PR}^{\Sigma }. \)

  \(\Box\)


  \textbf{\underline{Lemma 26:}} Si \(< \) es un orden total estricto sobre un alfabeto no vacio \( \Sigma \), entonces \(s^{< }\), \(\#^{< }\) y \(\ast ^{< }\) pertenecen a \(\mathrm{PR} ^{\Sigma }\)

  \PROOF Supongamos \(\Sigma =\{a_{1},...,a_{k}\}\) y \(< \) dado por \(a_{1}< ...< a_{k}\). Ya que

  \(\displaystyle \begin{array}{rcl} s^{< }(\varepsilon ) & =& a_{1} \\ s^{< }(\alpha a_{i}) & =& \alpha a_{i+1}\text{, para }i< k \\ s^{< }(\alpha a_{k}) & =& s^{< }(\alpha )a_{1} \end{array} \)

  tenemos que \(s^{< }=R\left( C_{a_{1}}^{0,0},\mathcal{G}\right) \), donde \( \mathcal{G}_{a_{i}}=d_{a_{i+1}}\circ p_{1}^{0,2}\), para \(i=1,...,k-1\) y \( \mathcal{G}_{a_{k}}=d_{a_{1}}\circ p_{2}^{0,2}.\) O sea que \(s^{< }\in \mathrm{ PR}^{\Sigma }.\) Ya que
  \(\displaystyle \begin{array}{rcl} \ast ^{< }(0) & =& \varepsilon \\ \ast ^{< }(t+1) & =& s^{< }(\ast ^{< }(t)) \end{array} \)

  podemos ver que \(\ast ^{< }\in \mathrm{PR}^{\Sigma }.\) Ya que
  \(\displaystyle \begin{array}{rcl} \#^{< }(\varepsilon ) & =& 0 \\ \#^{< }(\alpha a_{i}) & =& \#^{< }(\alpha ).k+i\text{, para }i=1,...,k, \end{array} \)

  tenemos que \(\#^{< }=R\left( C_{0}^{0,0},\mathcal{G}\right) \), donde
  \(\displaystyle \mathcal{G}_{a_{i}}=\lambda xy\left[ x+y\right] \circ \left( \lambda xy\left[ x.y\right] \circ \left( p_{1}^{1,1},C_{k}^{1,1}\right) ,C_{i}^{1,1}\right) \text{, para }i=1,...,k\text{.} \)

  O sea que \(\#^{< }\in \mathrm{PR}^{\Sigma }\). \(\Box\)

  \textbf{\underline{Lemma 27:}}
  (a) \(\lambda xy\left[ x\dot{-}y\right] \in \mathrm{PR}^{\varnothing }.\)
  (b) \(\lambda xy\left[ \max (x,y)\right] \in \mathrm{PR}^{\varnothing }.\)
  (c) \(\lambda xy\left[ x=y\right] \in \mathrm{PR}^{\varnothing }.\)
  (d) \(\lambda xy\left[ x\leq y\right] \in \mathrm{PR}^{\varnothing }.\)
  (e) Si \(\Sigma \) es no vacio, entonces \(\lambda \alpha \beta \left[ \alpha =\beta \right] \in \mathrm{PR}^{\Sigma }\)


  \PROOF (a) Primero notar que \(\lambda x\left[ x\dot{-}1\right] =R\left( C_{0}^{0,0},p_{2}^{2,0}\right) \in \mathrm{PR}^{\varnothing }.\) Tambien note que

  \(\displaystyle \lambda tx\left[ x\dot{-}t\right] =R\left( p_{1}^{1,0},\lambda x\left[ x\dot{ -}1\right] \circ p_{1}^{3,0}\right) \in \mathrm{PR}^{\varnothing }. \)

  O sea que \(\lambda xy\left[ x\dot{-}y\right] =\lambda tx\left[ x\dot{-}t \right] \circ \left( p_{2}^{2,0},p_{1}^{2,0}\right) \in \mathrm{PR} ^{\varnothing }.\)
  (b) Note que \(\lambda xy\left[ \max (x,y)\right] =\lambda xy\left[ (x+(y\dot{ -}x)\right] .\)

  (c) Note que \(\lambda xy\left[ x=y\right] =\lambda xy\left[ 1\dot{-}((x\dot{- }y)+(y\dot{-}x))\right] .\)

  (d) Note que \(\lambda xy\left[ x\leq y\right] =\lambda xy\left[ 1\dot{-}(x \dot{-}y)\right] .\)

  (e) Sea \(< \) un orden total estricto sobre \(\Sigma .\) Ya que

  \(\displaystyle \alpha =\beta \text{ sii }\#^{< }(\alpha )=\#^{< }(\beta ) \)

  tenemos que
  \(\displaystyle \lambda \alpha \beta \left[ \alpha =\beta \right] =\lambda xy\left[ x=y \right] \circ \left( \#^{< }\circ p_{1}^{0,2},\#^{< }\circ p_{2}^{0,2}\right) . \)

  O sea que podemos aplicar (c) y Lema 28 implica que \(\chi _{S}\) es \( \Sigma \)-p.r.. \(\Box\)


  \textbf{\underline{Lemma 28:}}
    \textbf{Hacer}
  \PROOF

  \textbf{\underline{Lemma 29:}}
    \textbf{Hacer}

  \PROOF

  \textbf{\underline{Corollary 30:}}
    \textbf{Hacer}

  \PROOF

  %% =========== FALTA UNA PARTE ===========%%

  \textbf{\underline{Lemma 31:}} Supongamos \(S_{1},...,S_{n}\subseteq \omega \), \( L_{1},...,L_{m}\subseteq \Sigma ^{\ast }\) son conjuntos no vacios. Entonces \( S_{1}\times ...\times S_{n}\times L_{1}\times ...\times L_{m}\) es \(\Sigma \) -p.r. sii \(S_{1},...,S_{n},L_{1},...,L_{m}\) son \(\Sigma \)-p.r.

  \PROOF (\(\Rightarrow \)) Veremos por ejemplo que \(L_{1}\) es \(\Sigma \)-p.r.. Sea \( (z_{1},...,z_{n},\zeta _{1},...,\zeta _{m})\) un elemento fijo de \( S_{1}\times ...\times S_{n}\times L_{1}\times ...\times L_{m}.\) Note que

  \(\displaystyle \alpha \in L_{1}\text{ sii }(z_{1},...,z_{n},\alpha ,\zeta _{2},...,\zeta _{m})\in S_{1}\times ...\times S_{n}\times L_{1}\times ...\times L_{m}, \)

  lo cual implica que
  \(\displaystyle \chi _{L_{1}}=\chi _{S_{1}\times ...\times S_{n}\times L_{1}\times ...\times L_{m}}\circ \left( C_{z_{1}}^{0,1},...,C_{z_{n}}^{0,1},p_{1}^{0,1},C_{\zeta _{2}}^{0,1},...,C_{\zeta _{m}}^{0,1}\right) . \)

  (\(\Leftarrow \)) Note que \(\chi _{S_{1}\times ...\times S_{n}\times L_{1}\times ...\times L_{m}}\) es el predicado
  \(\displaystyle \left( \chi _{S_{1}}\circ p_{1}^{n,m}\wedge ...\wedge \chi _{S_{n}}\circ p_{n}^{n,m}\wedge \chi _{L_{1}}\circ p_{n+1}^{n,m}\wedge ...\wedge \chi _{L_{m}}\circ p_{n+m}^{n,m}\right) . \)

  \(\Box\)


  \textbf{\underline{Lemma 32:}} Supongamos \(f:D_{f}\subseteq \omega ^{n}\times \Sigma ^{\ast m}\rightarrow O\) es \(\Sigma \)-p.r., donde \(O\in \{\omega ,\Sigma ^{\ast }\}.\) Si \(S\subseteq D_{f}\) es \(\Sigma \)-p.r., entonces \(f\mid _{S}\) es \(\Sigma \)-p.r..

  \PROOF Supongamos \(O=\Sigma ^{\ast }\). Entonces

  \(\displaystyle f\mid _{S}=\lambda x\alpha \left[ \alpha ^{x}\right] \circ \left( Suc\circ Pred\circ \chi _{S},f\right) \)

  es \(\Sigma \)-p.r.. El caso \(O=\omega \) es similar usando \(\lambda xy\left[ x^{y}\right] \) en lugar de \(\lambda x\alpha \left[ \alpha ^{x}\right] \). \(\Box\)


  \textbf{\underline{Lemma 33:}} Si \(f:D_{f}\subseteq \omega ^{n}\times \Sigma ^{\ast m}\rightarrow O\) es \(\Sigma \)-p.r., entonces existe una funcion \(\Sigma \) -p.r. \(\bar{f}:\omega ^{n}\times \Sigma ^{\ast m}\rightarrow O\), tal que \(f= \bar{f}\mid _{D_{f}}\).

  \PROOF Es facil ver por induccion en \(k\) que el enunciado se cumple para cada \(f\in \mathrm{PR}_{k}^{\Sigma }\) \(\Box\)


  \textbf{\underline{Proposition 34:}} Un conjunto \(S\) es \(\Sigma \)-p.r. sii \(S\) es el dominio de una funcion \(\Sigma \)-p.r.\(.\)

  \PROOF (\(\Rightarrow \)) Note que \(S=D_{Pred\circ \chi _{S}}.\)

  (\(\Leftarrow \)) Probaremos por induccion en \(k\) que \(D_{F}\) es \(\Sigma \) -p.r., para cada \(F\in \mathrm{PR}_{k}^{\Sigma }.\) El caso \(k=0\) es facil\(.\) Supongamos el resultado vale para un \(k\) fijo y supongamos \(F\in \mathrm{PR} _{k+1}^{\Sigma }.\) Veremos entonces que \(D_{F}\) es \(\Sigma \)-p.r.. Hay varios casos. Consideremos primero el caso en que \(F=R(f,g)\), donde

  \(\displaystyle \begin{array}{rcl} f & :& S_{1}\times ...\times S_{n}\times L_{1}\times ...\times L_{m}\rightarrow \Sigma ^{\ast } \\ g & :& \omega \times S_{1}\times ...\times S_{n}\times L_{1}\times ...\times L_{m}\times \Sigma ^{\ast }\rightarrow \Sigma ^{\ast }, \end{array} \)

  con \(S_{1},...,S_{n}\subseteq \omega \) y \(L_{1},...,L_{m}\subseteq \Sigma ^{\ast }\) conjuntos no vacios y \(f,g\in \mathrm{PR}_{k}^{\Sigma }\). Notese que por definicion de \(R(f,g)\), tenemos que
  \(\displaystyle D_{F}=\omega \times S_{1}\times ...\times S_{n}\times L_{1}\times ...\times L_{m}. \)

  Por hipotesis inductiva tenemos que \(D_{f}=S_{1}\times ...\times S_{n}\times L_{1}\times ...\times L_{m}\) es \(\Sigma \)-p.r., lo cual por el Lema 31 nos dice que los conjuntos \(S_{1},...,S_{n}\), \( L_{1},...,L_{m}\) son \(\Sigma \)-p.r.. Ya que \(\omega \) es \(\Sigma \)-p.r., el Lema 31 nos dice que \(D_{F}\) es \(\Sigma \)-p.r..
  Los otros casos de recursion primitiva son dejados al lector.

  Supongamos ahora que \(F=g\circ (g_{1},...,g_{n+m})\), donde

  \(\displaystyle \begin{array}{rcl} g & :& D_{g}\subseteq \omega ^{n}\times \Sigma ^{\ast m}\rightarrow O \\ g_{i} & :& D_{g_{i}}\subseteq \omega ^{k}\times \Sigma ^{\ast l}\rightarrow \omega \text{, }i=1,...,n \\ g_{i} & :& D_{g_{i}}\subseteq \omega ^{k}\times \Sigma ^{\ast l}\rightarrow \Sigma ^{\ast },i=n+1,...,n+m \end{array} \)

  estan en \(\mathrm{PR}_{k}^{\Sigma }.\) Por Lema 33, hay funciones \(\Sigma \)-p.r. \(\bar{g}_{1},...,\bar{g}_{n+m}\) las cuales son \( \Sigma \)-totales y cumplen
  \(\displaystyle g_{i}=\bar{g}_{i}\mid _{D_{g_{i}}}\text{, para }i=1,...,n+m. \)

  Por hipotesis inductiva los conjuntos \(D_{g}\), \(D_{g_{i}}\), \(i=1,...,n+m\), son \(\Sigma \)-p.r. y por lo tanto
  \(\displaystyle S=\bigcap_{i=1}^{n+m}D_{g_{i}} \)

  lo es. Notese que
  \(\displaystyle \chi _{D_{F}}=(\chi _{D_{g}}\circ \left( \bar{g}_{1},...,\bar{g} _{n+m}\right) \wedge \chi _{S}) \)

  lo cual nos dice que \(D_{F}\) es \(\Sigma \)-p.r.. \(\Box\)
