\section{Funciones $\Sigma$-recursivas}

  % Lemma 18: Con prueba.
  \begin{lemma}
    \PN Si $f, f_{1}, \dotsc, f_{n+m}$ son $\Sigma$-efectivamente computables, entonces $f \circ (f_{1}, \dotsc,
    f_{n+m})$ lo es.
  \end{lemma}
  \begin{proof}
    \PN Sean:

    \begin{itemize}
      \item $\mathbb{P}$ un procedimiento efectivo que compute a $f$.
      \item $\mathbb{P}_{1}$ un procedimiento efectivo que compute a $f_{1}$.
      \item $\mathbb{P}_{2}$ un procedimiento efectivo que compute a $f_{2}$.
      \item $\vdotswithin{(n-1)m+1} \qquad \vdotswithin{(n-1)m+2} \qquad \vdotswithin{(n-1)m+r}$
      \item $\mathbb{P}_{n+m}$ un procedimiento efectivo que compute a $f_{n+m}$.
    \end{itemize}

    \PN El siguiente procedimiento $\mathbb{P}$ computa $f \circ (f_{1}, \dotsc, f_{n+m})$:

    \vspace{3mm}
    \PN \textbf{Etapa 1:}
    Realizar $\mathbb{P}_{1}$ con dato de entrada $(\vec{x}, \vec{\alpha})$ para obtener de salida $o_{1}$.

    $\qquad\;\;\;\;$Realizar $\mathbb{P}_{2}$ con dato de entrada $(\vec{x}, \vec{\alpha})$ para obtener de salida
    $o_{2}$.

    $\qquad\;\;\;\;\vdotswithin{(n-1)m+1} \qquad \vdotswithin{(n-1)m+2} \qquad \vdotswithin{(n-1)m+r}$

    $\qquad\;\;\;\;$Realizar $\mathbb{P}_{n+m}$ con dato de entrada $(\vec{x}, \vec{\alpha})$ para obtener de salida
    $o_{n+m}$.

    \PN \textbf{Etapa 2:}
    Dar como dato de salida el resultado de $\mathbb{P}$ con dato de entrada $(o_{1}, o_{2}, \dotsc, o_{n+m})$.
  \end{proof}

  % Lemma 19: Con prueba.
  \begin{lemma}
    \PN Si $f$ y $g$ son $\Sigma $-efectivamente computables, entonces $R(f,g)$ lo es.
  \end{lemma}
  \begin{proof}
    \PN Sean:

    \begin{itemize}
      \item $\mathbb{P}_{1}$ un procedimiento efectivo que compute a $f$.
      \item $\mathbb{P}_{2}$ un procedimiento efectivo que compute a $g$.
    \end{itemize}

    \PN El siguiente procedimiento computa la función $R(f,g)$:

    \vspace{3mm}
    \PN \textbf{Etapa 1:}
    Darle a la variable $T$ el valor $0$.

    \PN \textbf{Etapa 2:}
    Realizar $\mathbb{P}_{1}$ con los valores $(\vec{x}, \vec{\alpha})$ como entrada para obtener de salida $A$.

    \PN \textbf{Etapa 3:}
    \textbf{Si $T = t$:} entonces detenerse y dar como dato de salida el valor de $A$.

    $\qquad\;\;\;\;$\textbf{Si $T \neq t$:} aumentar en $1$ el valor de la variable $T$.

    \PN \textbf{Etapa 4:}
    \textbf{Si $Im(f), Im(g) \subseteq \omega$:} Realizar $\mathbb{P}_{2}$ con los valores $(A, T, \vec{x},
    \vec{\alpha})$ como entrada y dirijirse

    $\qquad\;\;\;\;\;$a la Etapa 3.

    $\qquad\;\;\;\;$\textbf{Si $Im(f), Im(g) \subseteq \SIGMA$:} Realizar $\mathbb{P}_{2}$ con los valores $(T,
    \vec{x}, \vec{\alpha}, A)$ como entrada y

    $\qquad\;\;\;\;\;$dirijirse a la Etapa 3.
  \end{proof}

  % Lemma 20: Sin prueba.
  \begin{lemma}
    \PN Si $f$ y cada $\mathcal{G}_{a}$ son $\Sigma$-efectivamente computables, entonces $R(f,\mathcal{G})$ lo es.
  \end{lemma}

  % Theorem 21: Con prueba.
  \begin{theorem}
    \PN Si $f \in \mathrm{PR}^{\Sigma}$, entonces $f$ es $\Sigma$-efectivamente computable.
  \end{theorem}
  \begin{proof}
    \PN Recordemos que $PR^{\Sigma} = \bigcup\limits_{k \geq 0} PR_{k}^{\Sigma}$. Supongamos que $f \in
    PR_{k}^{\Sigma}$, probaremos este teorema por inducción en $k$.

    \vspace{3mm}
    \PN \underline{Caso Base:} \begin{tabular}{|c|} \hline $k = 0$ \\\hline \end{tabular}

    \PN Luego $f \in PR_{0}^{\Sigma}$, es decir $f \in \{Suc, Pred, C_{0}^{0,0}, C_{\varepsilon}^{0,0}\} \cup \{d_{a}:
    a \in \Sigma\} \cup \{p_{j}^{n,m} : 1 \leq j \leq n+m\}$. Por lo tanto, $f$ es $\Sigma$-efectivamente computable.

    \vspace{3mm}
		\PN \underline{Caso Inductivo:} \begin{tabular}{|c|} \hline $k > 0$ \\\hline \end{tabular}

    \PN Supongamos ahora que si $f \in \mathrm{PR}_{k}^{\Sigma} \Rightarrow f$ es
    $\Sigma$-efectivamente computable, veamos que $f \in \mathrm{PR}_{k+1}^{\Sigma} \Rightarrow f$ es
    $\Sigma$-efectivamente computable.

    \PN Dado que las funciones de $PR_{k}^{\Sigma}$ son $\Sigma$-efectivamente computable por hipótesis inductiva, y
    que el conjunto $PR_{k+1}^{\Sigma}$ se contruye a partir de dichas funciones, a través de recursiones o
    composiciones, las cuales probamos son $\Sigma$-efectivamente computables en el \textbf{Lemma 18} y
    \textbf{Lemma 19}, concluimos entonces que $f$ es $\Sigma$-efectivamente computable.
  \end{proof}

  % Lemma 22: Con prueba.
  \begin{lemma}
    \begin{enumerate}[a)]
      \item $\emptyset \in \mathrm{PR}^{\emptyset}$.
      \item $\lambda xy \left[x+y\right] \in \mathrm{PR}^{\emptyset}$.
      \item $\lambda xy\left[x.y\right] \in \mathrm{PR}^{\emptyset}$.
      \item $\lambda x\left[x!\right] \in \mathrm{PR}^{\emptyset}$.
    \end{enumerate}
  \end{lemma}
  \begin{proof}
    \begin{enumerate}[a)]
      \item Notese que $\emptyset = Pred \circ C_{0}^{0,0} \in \mathrm{PR}_{1}^{\emptyset}$, entonces $\emptyset \in
        \mathrm{PR}^{\emptyset}$.

      \item Notar que:
        \begin{eqnarray*}
          \lambda xy \left[x+y\right](0, x_{1}) &=& x_{1} = p_{1}^{1,0}(x_{1}) \\
          \lambda xy \left[x+y\right](t+1, x_{1}) &=& \lambda xy \left[x+y\right](t, x_{1}) + 1 \\
          & =& Suc \circ p_{1}^{3,0}
        \end{eqnarray*}

        \PN lo cual implica que $\lambda xy \left[x+y\right] = R(p_{1}^{1,0}, Suc \circ p_{1}^{3,0}) \in
        \mathrm{PR}_{2}^{\emptyset}$, entonces $\lambda xy \left[x+y\right] \in \mathrm{PR}^{\emptyset}$.

      \item Primero note que:
        \begin{eqnarray*}
          C_{0}^{1,0}(0) &=& C_{0}^{0,0}(\Diamond) \\
          C_{0}^{1,0}(t+1) &=& C_{0}^{1,0}(t)
        \end{eqnarray*}

        \PN lo cual implica que $C_{0}^{1,0} = R(C_{0}^{0,0}, p_{1}^{2,0}) \in \mathrm{PR}_{1}^{\emptyset}$.
        \PN También note que:

        \begin{eqnarray*}
          \lambda xy \left[x.y\right](0, x_{1}) &=& 0 = C_{0}^{1,0}(x_{1}) \\
          \lambda xy \left[x.y\right](t+1, x_{1}) &=& \lambda xy \left[x.y\right](t, x_{1}) + x_{1} \\
          &=& \lambda xy \left[x+y\right] \circ (p_{1}^{3,0}, p_{3}^{3,0})
        \end{eqnarray*}

        \PN lo cual implica que $\lambda xy \left[x.y\right] = R(C_{0}^{1,0}, \lambda xy \left[x+y\right] \circ
        (p_{1}^{3,0}, p_{3}^{3,0})) \in \mathrm{PR}_{4}^{\emptyset}$, entonces $\lambda xy \left[x.y\right] \in
        \mathrm{PR}^{\emptyset}$.

      \item Notar que:
        \begin{eqnarray*}
          \lambda x \left[x!\right](0) &=& 1 = C_{1}^{0,0}(\Diamond) \\
          \lambda x \left[x!\right](t+1) &=& \lambda x \left[x!\right](t).(t+1) \\
          &=& \lambda \left[x.y\right] \circ (p_{1}^{2,0}, Suc \circ p_{2}^{2,0})
        \end{eqnarray*}

        \PN lo cual implica que: $\lambda x \left[x!\right] = R(C_{1}^{0, 0}, \lambda xy \left[x.y\right] \circ
        (p_{1}^{2,0}, Suc \circ p_{2}^{2,0}))$. Ya que $C_{1}^{0,0} = Suc \circ C_{0}^{0,0}$, tenemos que $C_{1}^{0,0}
        \in \mathrm{PR}_{1}^{\emptyset}$. Por (c), tenemos que $\lambda xy \left[x.y\right] \circ (p_{1}^{2,0}, Suc
        \circ p_{2}^{2,0}) \in \mathrm{PR}_{5}^{\emptyset}$, obteniendo que $\lambda x \left[x!\right] \in
        \mathrm{PR}_{6}^{\emptyset}$, entonces $\lambda x \left[x!\right] \in \mathrm{PR}^{\emptyset}$.
    \end{enumerate}
  \end{proof}

  % Lemma 23: Con prueba.
  \begin{lemma}
    \PN Supongamos $\Sigma \neq \emptyset$, entonces:

    \begin{enumerate}[a)]
      \item $\lambda \alpha \beta \left[\alpha\beta\right] \in \mathrm{PR}^{\Sigma}$.
      \item $\lambda \alpha \left[\lvert\alpha \rvert\right] \in \mathrm{PR}^{\Sigma}$.
    \end{enumerate}
  \end{lemma}
  \begin{proof}
    \begin{enumerate}[a)]
      \item Ya que:
        \begin{eqnarray*}
          \lambda \alpha\beta \left[\alpha \beta\right](\alpha_{1}, \varepsilon) &=& \alpha_{1} = p_{1}^{0,1}
            (\alpha_{1}) \\
          \lambda \alpha\beta \left[\alpha \beta\right](\alpha_{1}, \alpha a) &=& d_{a}(\lambda \alpha\beta \left[\alpha
            \beta\right](\alpha_{1}, \alpha)) \qquad \text{para a} \in \Sigma
        \end{eqnarray*}

        \PN tenemos que $\lambda \alpha\beta \left[\alpha \beta\right] = R(p_{1}^{0, 1}, \mathcal{G})$, donde
        $\mathcal{G}_{a} = d_{a} \circ p_{3}^{0,3}$, para cada $a \in \Sigma$.
        \PN Luego, $\lambda \alpha\beta \left[\alpha\beta\right] \in \mathrm{PR}^{\Sigma}$.

      \item Ya que:
        \begin{eqnarray*}
          \lambda \alpha \left[\lvert\alpha \rvert\right](\varepsilon) &=& 0 = C_{0}^{0,0}(\Diamond) \\
          \lambda \alpha \left[\lvert\alpha \rvert\right](\alpha a) &=& \lambda \alpha \left[\lvert\alpha \rvert\right]
            (\alpha) + 1
        \end{eqnarray*}

        \PN tenemos que $\lambda \alpha \left[\lvert\alpha \rvert\right] = R(C_{0}^{0, 0}, \mathcal{G})$, donde
        $\mathcal{G}_{a} = Suc \circ p_{1}^{0,1}$, para cada $a \in \Sigma$.
        \PN Luego, $\lambda \alpha \left[\lvert\alpha \rvert\right] \in \mathrm{PR}^{\Sigma}$.
    \end{enumerate}
  \end{proof}

  % Lemma 24: Sin prueba.
  \begin{lemma}
    \begin{enumerate}[a)]
      \item $C_{k}^{n,m}, C_{\alpha}^{n,m} \in \mathrm{PR}^{\Sigma}$, para $n, m, k \geq 0$, y $\alpha \in \SIGMA$.
      \item $C_{k}^{n,0} \in \mathrm{PR}^{\emptyset}$, para $n, k \geq 0$.
    \end{enumerate}
  \end{lemma}

  % Lemma 25: Con prueba.
  \begin{lemma}
    \begin{enumerate}[a)]
      \item $\lambda xy \left[x^{y}\right] \in \mathrm{PR}^{\emptyset}$.
      \item $\lambda t\alpha \left[\alpha^{t}\right] \in \mathrm{PR}^{\Sigma}$.
    \end{enumerate}
  \end{lemma}
  \begin{proof}
    \begin{enumerate}[a)]
      \item Notar que:

        \begin{eqnarray*}
          \lambda tx \left[x^{t}\right](0, x_{1}) &=& 0 = C_{0}^{1,0}(x_{1}) \\
          \lambda tx \left[x^{t}\right](t+1, x_{1}) &=& \lambda tx \left[x^{t}\right](t, x_{1}) . x_{1} \\
          &=& \lambda xy \left[x.y\right] \circ (p_{1}^{3,0}, p_{3}^{3,0})
        \end{eqnarray*}
        \PN Osea que $\lambda tx \left[x^{t}\right] = R\left(C_{1}^{1,0}, \lambda tx \left[x^{t}\right] \circ
        (p_{2}^{2,0}, p_{1}^{2,0}\right) \in \mathrm{PR}^{\emptyset}$.
        \PN Por lo tanto, $\lambda xy \left[x^{y}\right] = \lambda tx \left[x^{t}\right] \circ (p_{2}^{2,0},
        p_{1}^{2,0})$.

      \item Notar que:

      \begin{eqnarray*}
        \lambda t\alpha \left[\alpha^{t}\right](0, \alpha) &=& \varepsilon = C_{\varepsilon}^{0,1}(\alpha) \\
        \lambda t\alpha \left[\alpha^{t}\right](t+1, \alpha) &=& \lambda t\alpha \left[\alpha^{t}\right](t,
          \alpha) \alpha \\
        &=& \lambda \alpha\beta \left[\alpha\beta \right] \circ \left(p_{3}^{1,2}, p_{2}^{1,2}\right)
      \end{eqnarray*}

      \PN Por lo tanto, $\lambda t\alpha \left[\alpha^{t}\right] \in \mathrm{PR}^{\Sigma}$.
    \end{enumerate}
  \end{proof}

  % Lemma 26: Con prueba.
  \begin{lemma}
    \PN Si $<$ es un orden total estricto sobre un alfabeto no vacío $\Sigma$, entonces:

    \begin{enumerate}[a)]
      \item $s^{<} \in \mathrm{PR}^{\Sigma}$.
      \item $\#^{<} \in \mathrm{PR}^{\Sigma}$.
      \item $\ast^{<} \in \mathrm{PR}^{\Sigma}$.
    \end{enumerate}
  \end{lemma}
  \begin{proof}
    \PN Supongamos $\Sigma = \{a_{1}, \dotsc, a_{k}\}$ y $<$ dado por $a_{1} < \dotsc < a_{k}$.

    \begin{enumerate}[a)]
      \item Ya que:
        \begin{eqnarray*}
          s^{<}(\varepsilon) &=& a_{1} \\
          s^{<}(\alpha a_{i}) &=& \alpha a_{i+1} \qquad \text{para } i < k \\
          s^{<}(\alpha a_{k}) &=& s^{<}(\alpha) a_{1}
        \end{eqnarray*}

        \PN tenemos que $s^{<} = R(C_{a_{1}}^{0, 0}, \mathcal{G})$, donde $\mathcal{G} = \{\left(a_{i}, d_{a_{i+1}}
        \circ p_{1}^{0,2} \right), \left(a_{k}, d_{a_{1}} \circ p_{2}^{0,2} \right)\}$.
        \PN Luego, $s^{<} \in \mathrm{PR}^{\Sigma}$.

      \item Ya que:
        \begin{eqnarray*}
          \ast^{<}(0) &=& \varepsilon \\
          \ast^{<}(t+1) &=& s^{<}(\ast^{<}(t))
        \end{eqnarray*}

        \PN tenemos que $\ast^{<} = R(C_{\varepsilon}^{0,0}, s^{<} \circ p_{2}^{1,1})$. Luego, $\ast^{<} \in
        \mathrm{PR}^{\Sigma}$.

      \item Ya que:
        \begin{eqnarray*}
          \#^{<}(\varepsilon) &=& 0 \\
          \#^{<}(\alpha a_{i}) &=& \#^{<}(\alpha). k + i \\
          \text{para } i &=& 1, \dotsc, k
        \end{eqnarray*}

        \PN tenemos que $\#^{<} = R(C_{0}^{0, 0}, \mathcal{G})$, donde $\mathcal{G}_{a_{i}} = \lambda xy
        \left[x+y\right] \circ (\lambda xy \left[x.y\right] \circ (p_{1}^{1, 1}, C_{k}^{1, 1}), C_{i}^{1, 1})
        \text{, para } i = 1, \dotsc, k$. Luego, $\#^{<} \in \mathrm{PR}^{\Sigma}$.
    \end{enumerate}
  \end{proof}

  % Lemma 27: Con prueba.
  \begin{lemma}
    \begin{enumerate}[a)]
      \item $\lambda xy \left[x \dot{-}y\right] \in \mathrm{PR}^{\emptyset}$.
      \item $\lambda xy \left[\max (x,y)\right] \in \mathrm{PR}^{\emptyset}$.
      \item $\lambda xy \left[x=y\right] \in \mathrm{PR}^{\emptyset}$.
      \item $\lambda xy \left[x \leq y\right] \in \mathrm{PR}^{\emptyset}$.
      \item Si $\Sigma \neq \emptyset \Rightarrow \lambda \alpha \beta \left[\alpha = \beta\right] \in
        \mathrm{PR}^{\Sigma}$.
    \end{enumerate}
  \end{lemma}
  \begin{proof}
    \begin{enumerate}[a)]
      \item Primero notar que:

        \begin{eqnarray*}
          \lambda x \left[x \dot{-}1\right](0) &=& 0 = C_{0}^{0,0}(\Diamond) \\
          \lambda x \left[x \dot{-}1\right](t+1) &=& t = p_{2}^{2,0}
        \end{eqnarray*}

        \PN es decir $\lambda x \left[x \dot{-}1\right] = R(C_{0}^{0,0}, p_{2}^{2,0}) \in \mathrm{PR}^{\emptyset}$.
        \PN También notar que:

        \begin{eqnarray*}
          \lambda tx \left[x \dot{-}t\right](0, x_{1}) &=& x_{1} = p_{1}^{1,0}(x_{1}) \\
          \lambda tx \left[x \dot{-}t\right](t+1, x_{1}) &=& \lambda tx \left[x \dot{-}t\right](t, x_{1})
            \dot{-} 1\\
          &=& \lambda x \left[x \dot{-}1\right] \circ p_{1}^{3,0}
        \end{eqnarray*}

        \PN es decir, $\lambda tx \left[x \dot{-}t\right] = R(p_{1}^{1,0}, \lambda x \left[x \dot{-}1\right] \circ
        p_{1}^{3,0}) \in \mathrm{PR}^{\emptyset}$.
        \PN Por lo tanto, $\lambda xy \left[x \dot{-}y\right] = \lambda tx \left[x \dot{-}t\right] \circ (p_{2}^{2,0},
        p_{1}^{2,0}) \in \mathrm{PR}^{\emptyset}$.

      \item Notar que:

        \begin{eqnarray*}
          \lambda xy \left[\max (x,y)\right] &=& \lambda xy \left[(x + (y \dot{-}x)\right] \\
          &=& \lambda xy \left[x+y\right] \circ \left(p_{1}^{2,0}, \lambda xy \left[x \dot{-}y\right] \circ
            (p_{2}^{2,0}, p_{1}^{2,0})\right)
        \end{eqnarray*}

        \PN Por lo tanto, $\lambda xy \left[\max (x,y)\right] \in \mathrm{PR}^{\emptyset}$.

      \item Notar que:

        \begin{eqnarray*}
          \lambda xy \left[x=y\right] &=& \lambda xy \left[1 \dot{-}((x \dot{-} y) + (y \dot{-} x))\right] \\
          &=& \lambda xy \left[x \dot{-}y\right] \circ (C_{1}^{2,0}, \lambda xy \left[x+y\right] \circ
            (\lambda xy \left[x \dot{-}y\right] \circ (p_{1}^{2,0}, p_{2}^{2,0}), \lambda xy \left[x \dot{-}y\right]
            \circ (p_{2}^{2,0}, p_{1}^{2,0})))
        \end{eqnarray*}

        \PN Por lo tanto, $\lambda xy \left[x=y\right] \in \mathrm{PR}^{\emptyset}$.

      \item Notar que:

        \begin{eqnarray*}
          \lambda xy \left[x \leq y\right] &=& \lambda xy\left[1 \dot{-}(x \dot{-}y)\right] \\
          &=& \lambda xy \left[x \dot{-}y\right] \circ (C_{1}^{2,0}, \lambda xy \left[x \dot{-}y\right] \circ
            p_{1}^{2,0}, p_{2}^{2,0}))
        \end{eqnarray*}

        \PN Por lo tanto, $\lambda xy \left[x \leq y\right] \in \mathrm{PR}^{\emptyset}$.

      \item Sea $<$ un orden total estricto sobre $\Sigma$. Ya que:

        \[
          \alpha = \beta \Leftrightarrow \#^{<}(\alpha) = \#^{<}(\beta)
        \]

        \PN tenemos que:

        \[
          \lambda \alpha\beta \left[\alpha=\beta\right] = \lambda xy \left[x=y\right] \circ (\#^{<} \circ p_{1}^{0,2},
          \#^{<} \circ p_{2}^{0,2})
        \]

        \PN Luego, utilizando el inciso (c) y el \textbf{Lemma 28} obtenemos que $\lambda \alpha\beta
        \left[\alpha=\beta\right] \in \mathrm{PR}^{\Sigma}$.
    \end{enumerate}
  \end{proof}

  % Lemma 28: Con prueba.
  \begin{lemma}
    \PN Si $P: S \subseteq \omega^{n} \times \Sigma^{\ast m} \rightarrow \omega$ y $Q: S \subseteq \omega^{n} \times
    \Sigma^{\ast m} \rightarrow \omega$ son predicados $\Sigma$-PR, entonces $(P \vee Q), (P \wedge Q)$ y $\neg P$ lo
    son también.
  \end{lemma}
  \begin{proof}
    \PN Notar que:

    \begin{eqnarray*}
      \neg P &=& \lambda xy \left[x \dot{-} y\right] \circ (C_{1}^{n, m}, P) \\
      (P \wedge Q) &=& \lambda xt \left[x.y\right] \circ (P, Q) \\
      (P \vee Q) &=& \neg (\neg P \wedge \neg Q)
    \end{eqnarray*}
  \end{proof}

  % Lemma 29: Con prueba.
  \begin{lemma}
    \PN Si $S_{1}, S_{2} \subseteq \omega^{n} \times \Sigma^{\ast m}$ son $\Sigma$-PR, entonces $S_{1} \cup S_{2}$,
    $S_{1} \cap S_{2}$ y $S_{1} - S_{2}$ lo son.
  \end{lemma}
  \begin{proof}
    \PN Notar que:

    \begin{eqnarray*}
      \chi_{S_{1} \cup S_{2}} &=& (\chi_{S_{1}} \vee \chi_{S_{2}}) \\
      \chi_{S_{1} \cap S_{2}} &=& (\chi_{S_{1}} \wedge \chi_{S_{2}}) \\
      \chi_{S_{1} - S_{2}} &=& \lambda \left[x \dot{-} y\right] \circ (\chi_{S_{1}}, \chi_{S_{2}})
    \end{eqnarray*}
  \end{proof}

  % Corollary 30: Con prueba. Solo el caso n = m = 1.
  \begin{corollary}
    \PN Si $S \subseteq \omega^{n} \times \Sigma^{\ast m}$ es finito, entonces $S$ es $\Sigma$-PR.
  \end{corollary}
  \begin{proof}
    \PN Se probará el caso $n = m = 1$, es decir, $S \subseteq \omega \times \SIGMA$. Supongamos, sin pérdida de
    generalidad, utilizando el \textbf{Lemma 29} que:

    \[
      S = \{(z, \gamma)\}
    \]

    \PN Notar que $\chi_{S}$ es el siguiente predicado:

    \[
      \left(\chi_{z} \circ p_{1}^{1,1} \wedge \chi_{\gamma} \circ p_{2}^{1,1}\right)
    \]

    \PN Ya que los predicados:

    \begin{eqnarray*}
      \chi_{z} = \lambda xy \left[x = y\right] \circ \left(p_{1}^{1,0}, C_{z}^{1,0}\right) \\
      \chi_{\gamma} = \lambda \alpha\beta \left[\alpha = \beta\right] \circ \left(p_{1}^{0,1},
      C_{\gamma}^{0,1}\right)
    \end{eqnarray*}

    \PN son $\Sigma$-PR, el \textbf{Lema 28} implica que $\chi_{S}$ es $\Sigma$-PR, por lo tanto $S$ es $\Sigma$-PR.
  \end{proof}

  % Lemma 31: Con prueba. Solo el caso n = m = 1.
  \begin{lemma}
    \PN Supongamos $S_{1}, \dotsc, S_{n} \subseteq \omega, L_{1}, \dotsc, L_{m} \subseteq \SIGMA$ son conjuntos no
    vacíos, entonces $S_{1} \times \dotsc \times S_{n} \times L_{1} \times \dotsc \times L_{m}$ es $\Sigma$-PR
    $\Leftrightarrow$ $S_{1}, \dotsc, S_{n}, L_{1}, \dotsc, L_{m}$ son $\Sigma$-PR.
  \end{lemma}
  \begin{proof}
    \PN Se probará el caso $n = m = 1$, es decir, $S_{1} \subseteq \omega$, $L_{1} \subseteq \SIGMA$.

    \vspace{3mm}
    \PN \begin{tabular}{|c|} \hline $\Rightarrow$ \\\hline \end{tabular} Sea $(z_{1}, \gamma_{1})$ un elemento fijo de
    $S_{1} \times L_{1}$. Notar que:

    \begin{eqnarray*}
      x \in S_{1} \Leftrightarrow (x, \gamma_{1}) \in S_{1} \times L_{1} \\
      \alpha \in L_{1} \Leftrightarrow (z_{1}, \alpha) \in S_{1} \times L_{1}
    \end{eqnarray*}

    \PN lo cual implica que:

    \begin{eqnarray*}
      \chi_{S_{1}} = \chi_{S_{1} \times L_{1}} \circ \left(p_{1}^{1,0}, C_{\gamma_{1}}^{1,0}\right) \\
      \chi_{L_{1}} = \chi_{S_{1} \times L_{1}} \circ \left(C_{z_{1}}^{0,1}, p_{1}^{0,1}\right)
    \end{eqnarray*}

    \PN por lo tanto, $L_{1}, S_{1}$ son $\Sigma$-PR.

    \vspace{3mm}
    \PN \begin{tabular}{|c|} \hline $\Leftarrow$\\\hline \end{tabular} Notar que:

    \[
      \chi_{S_{1} \times L_{1}} = \left(\chi_{S_{1}} \circ p_{1}^{1,1} \wedge \chi_{L_{1}} \circ p_{2}^{1,1} \right)
    \]

    \PN luego, por el \textbf{Lemma 28}, $S_{1} \times L_{1}$ es $\Sigma$-PR.
  \end{proof}

  % Lemma 32: Con prueba.
  \begin{lemma}
    \PN Supongamos $f: D_{f} \subseteq \omega^{n} \times \Sigma^{\ast m} \rightarrow O$ es $\Sigma$-PR, donde $O \in
    \{\omega, \SIGMA\}$. Si $S \subseteq D_{f}$ es $\Sigma$-PR, entonces $f \mid_{S}$ es $\Sigma$-PR.
  \end{lemma}
  \begin{proof}
    \begin{tabular}{|c|} \hline $O = \SIGMA$ \\\hline \end{tabular} Notar que:

    \[
      f \mid_{S} \; = \lambda x\alpha \left[\alpha^{x}\right] \circ (Suc \circ Pred \circ \chi_{S}, f)
    \]

    \PN luego $f$ es $\Sigma$-PR.

    \PN \begin{tabular}{|c|} \hline $O = \omega$ \\\hline \end{tabular} Notar que:

    \[
      f \mid_{S} \; = \lambda xy \left[x^{y}\right] \circ (f, Suc \circ Pred \circ \chi_{S})
    \]

    \PN luego $f$ es $\Sigma$-PR.

    \vspace{3mm}
    \PN Notar que \begin{tabular}{|c|} \hline $Suc \circ Pred \circ \chi_{S}$ \\\hline \end{tabular} funciona como un
    interruptor que evalua $f$, si el elemento pertenece a $S$, y que no evalua en caso contrario.
  \end{proof}

  % Lemma 33: Sin prueba.
  \begin{lemma}
    \PN Si $f: D_{f} \subseteq \omega^{n} \times \Sigma^{\ast m} \rightarrow O$ es $\Sigma$-PR, entonces existe una
    función $\Sigma$-PR
    \PN $\bar{f}: \omega^{n} \times \Sigma^{\ast m} \rightarrow O$, tal que $f = \bar{f} \mid_{D_{f}}$.
  \end{lemma}

  % Proposition 34: Con prueba.
  \begin{proposition}
    \PN Un conjunto $S$ es $\Sigma$-PR $\Leftrightarrow S$ es el dominio de una función $\Sigma$-PR.
  \end{proposition}
  \begin{proof}
    \begin{tabular}{|c|} \hline $\Rightarrow$ \\\hline \end{tabular} Notar que $S = D_{Pred \circ \chi_{S}}$.

    \PN \begin{tabular}{|c|} \hline $\Leftarrow$ \\\hline \end{tabular} Probaremos por inducción en $k$ que $D_{F}$ es
    $\Sigma$-PR para cada $F \in PR_{k}^{\Sigma}$.

    \vspace{3mm}
    \PN \underline{Caso Base:} \begin{tabular}{|c|} \hline $k = 0$ \\\hline \end{tabular} es decir, $F
    \in PR_{0}^{\Sigma}$. Luego:

    \begin{eqnarray*}
      F &\in& \{Suc, Pred, C_{0}^{0,0}, C_{\varepsilon}^{0,0}\} \cup \{d_{a}: a \in \Sigma\} \cup \{p_{j}^{n,m}: 1 \leq
      j \leq n+m\} \\
      D_{F} &\in& \{\omega, \mathbb{N}, \SIGMA, \omega^{n} \times \Sigma^{\ast m}, \{\Diamond\}\}
    \end{eqnarray*}

    \PN dado que $\omega, \mathbb{N}, \SIGMA, \omega^{n} \times \Sigma^{\ast m}, \{\Diamond\}$ son conjuntos
    $\Sigma$-PR, entonces $D_{F}$ es $\Sigma$-PR.

    \vspace{3mm}
    \PN \underline{Caso Inductivo:} Supongamos el resultado vale para un $k$ fijo y supongamos $F \in
    \mathrm{PR}_{k+1}^{\Sigma}$, veremos entonces que $D_{F}$ es $\Sigma$-PR. Existen varios casos, analizaremos cada
    uno por separado.

    \begin{enumerate}
      \item \begin{tabular}{|c|} \hline $F = R(f, g)$ \\\hline \end{tabular}
        \begin{itemize}
          \item Recursión primitiva sobre variable numérica.
          \begin{enumerate}
            \item Caso 1:
            \begin{eqnarray*}
              f &:& S_{1} \times \dotsc \times S_{n} \times L_{1} \times \dotsc \times L_{m} \rightarrow \omega \\
              g &:& \omega \times \omega \times S_{1} \times \dotsc \times S_{n} \times L_{1} \times \dotsc \times L_{m}
                \rightarrow \omega \\
              F &=& \omega \times S_{1} \times \dotsc \times S_{n} \times L_{1} \times \dotsc \times L_{m} \rightarrow
                \omega
            \end{eqnarray*}
            \item Caso 2:
            \begin{eqnarray*}
              f &:& S_{1} \times \dotsc \times S_{n} \times L_{1} \times \dotsc \times L_{m} \rightarrow \SIGMA \\
              g &:& \omega \times S_{1} \times \dotsc \times S_{n} \times L_{1} \times \dotsc \times L_{m} \times \SIGMA
                \rightarrow \SIGMA \\
              F &=& \omega \times S_{1} \times \dotsc \times S_{n} \times L_{1} \times \dotsc \times L_{m} \rightarrow
                \SIGMA
            \end{eqnarray*}
          \end{enumerate}

          \item Recursión primitiva sobre variable alfabética.
          \begin{enumerate}
            \item Caso 1:
            \begin{eqnarray*}
              f &:& S_{1} \times \dotsc \times S_{n} \times L_{1} \times \dotsc \times L_{m} \rightarrow \omega \\
              \mathcal{G}_{a} &:& \omega \times S_{1} \times \dotsc \times S_{n} \times L_{1} \times \dotsc \times L_{m}
                \times \SIGMA \rightarrow \omega \\
              F &=& S_{1} \times \dotsc \times S_{n} \times L_{1} \times \dotsc \times L_{m} \times \SIGMA \rightarrow
                \omega
            \end{eqnarray*}
            \item Caso 2:
            \begin{eqnarray*}
              f &:& S_{1} \times \dotsc \times S_{n} \times L_{1} \times \dotsc \times L_{m} \rightarrow \SIGMA \\
              \mathcal{G}_{a} &:& S_{1} \times \dotsc \times S_{n} \times L_{1} \times \dotsc \times L_{m} \times \SIGMA
                \times \SIGMA \rightarrow \SIGMA \\
              F &=& S_{1} \times \dotsc \times S_{n} \times L_{1} \times \dotsc \times L_{m} \times \SIGMA \rightarrow
                \SIGMA
            \end{eqnarray*}
          \end{enumerate}
        \end{itemize}

        \PN con $S_{1}, \dotsc, S_{n} \subseteq \omega$ y $L_{1}, \dotsc, L_{m} \subseteq \SIGMA$ conjuntos no vacíos
        y $f, g \in \mathrm{PR}_{k}^{\Sigma}$, para todos los casos anteriores.

        \PN Por hipótesis inductiva tenemos que $D_{f} = S_{1} \times \dotsc \times S_{n} \times L_{1} \times \dotsc
        \times L_{m}$ es $\Sigma$-PR, lo cual por el \textbf{Lemma 31} nos dice que los conjuntos $S_{1}, \dotsc, S_{n},
        L_{1}, \dotsc, L_{m}$ son $\Sigma$-PR. Ya que $\omega, \SIGMA$ son $\Sigma$-PR, el \textbf{Lemma 31} nos dice
        que $D_{F}$ es $\Sigma$-PR.

      \item \begin{tabular}{|c|} \hline $F = g \circ (g_{1}, \dotsc, g_{n+m})$ \\\hline \end{tabular} donde:

        \begin{eqnarray*}
          g &:& D_{g} \subseteq \omega^{n} \times \Sigma^{\ast m} \rightarrow O \\
          g_{i} &:& D_{g_{i}} \subseteq \omega^{k} \times \Sigma^{\ast l} \rightarrow \omega \qquad \;\; i = 1, \dotsc,
            n \\
          g_{i} &:& D_{g_{i}} \subseteq \omega^{k} \times \Sigma^{\ast l} \rightarrow \SIGMA \qquad i = n + 1, \dotsc,
            n + m
        \end{eqnarray*}

        \PN están en $\mathrm{PR}_{k}^{\Sigma}$. Por \textbf{Lemma 33}, existen funciones $\Sigma$-PR $\bar{g}_{1},
        \dotsc, \bar{g}_{n+m}$ las cuales son $\Sigma$-totales y cumplen:

        \begin{eqnarray*}
          g_{i} &=& \bar{g}_{i} \mid_{D_{g_{i}}} \\
          \text{para } i &=& 1, \dotsc, n + m
        \end{eqnarray*}

        \PN Por hipótesis inductiva, los conjuntos $D_{g}$, $D_{g_{i}}$, para $i = 1, \dotsc, n + m$, son $\Sigma$-PR y
        por lo tanto:

        \[
          S = \bigcap_{i=1}^{n+m} D_{g_{i}}
        \]

        \PN lo es. Notese además, que:

        \[
          \chi_{D_{F}} = \left((\chi_{D_{g}} \circ (\bar{g}_{1}, \dotsc, \bar{g}_{n+m})) \wedge \chi_{S}\right)
        \]

        \PN lo cual nos dice que $D_{F}$ es $\Sigma$-PR.
    \end{enumerate}
  \end{proof}
