\section{Máquinas de Turing}

  %Lemma 79
  \begin{lemma}
    \par Este lemma no se evalua.
  \end{lemma}

	%Lemma 80
  \begin{lemma}
  	El predicado $\lambda ndd^{\prime} \left[d\vdash d^{\prime}\right]$ es $(\Gamma \cup Q)$-PR.
	\begin{proof}
	  Note que $D_{\lambda dd^{\prime}\left[d\ \vdash d^{\prime }\right] }=Des\times Des$.
    También nótese que los predicados

    \begin{eqnarray*}
      \lambda q\sigma p\gamma \left[ (q,\sigma ,L)\in \delta (p,\gamma )\right] \\
      \lambda q\sigma p\gamma \left[ (q,\sigma ,R)\in \delta (p,\gamma )\right] \\
      \lambda q\sigma p\gamma \left[ (q,\sigma ,K)\in \delta (p,\gamma )\right]
    \end{eqnarray*}
    son $(\Gamma \cup Q)$-PR. ya que los tres tienen dominio igual a $Q\times \Gamma \times Q\times \Gamma $ el cual es
    finito por \textbf{Corolario 36}.
    Sea $P_{R}:Des\times Des\times\Gamma\times\Gamma ^{\ast }\times \Gamma ^{\ast }\times Q\times Q\rightarrow \omega $
    definido por $P_{R}(d,d^{\prime },\sigma ,\alpha ,\beta ,p,q)=1$ sii
    \begin{eqnarray*}
      d=\alpha p\beta \wedge (q,\sigma ,R)\in \delta \left( p,\left[ \beta B\right] _{1}\right) \wedge
      d^{\prime }=\alpha \sigma q^{\curvearrowright }\beta
    \end{eqnarray*}
    Sea $P_{L}:Des\times Des\times\Gamma\times\Gamma ^{\ast }\times \Gamma ^{\ast }\times Q\times Q\rightarrow \omega$
    definido por $P_{L}(d,d^{\prime },\sigma ,\alpha ,\beta ,p,q)=1$ sii
    \begin{eqnarray*}
      d=\alpha p\beta \wedge
      (q,\sigma ,L)\in \delta \left( p,\left[ \beta B\right] _{1}\right) \wedge
      \alpha \neq \varepsilon \wedge
      d^{\prime }=\left\lfloor \alpha ^{\curvearrowleft }q\left[ \alpha \right]
      _{\left\vert \alpha \right\vert }\sigma ^{\curvearrowright }\beta \right\rfloor
    \end{eqnarray*}
    Sea $P_{K}:Des\times Des\times\Gamma\times \Gamma ^{\ast }\times \Gamma ^{\ast }\times Q\times Q\rightarrow \omega$
    definido por $P_{K}(d,d^{\prime },\sigma ,\alpha ,\beta ,p,q)=1$ sii
    \begin{eqnarray*}
      d=\alpha p\beta \wedge
      (q,\sigma ,K)\in \delta \left( p,\left[ \beta B\right] _{1}\right) \wedge
      d^{\prime }=\left\lfloor \alpha q\sigma ^{\curvearrowright }\beta \right\rfloor
    \end{eqnarray*}
    \par \noindent Veamos que $P_L$ es $(\Gamma \cup Q)$-PR. Notar que
    \[
      P_L = P_1 \land P_2 \land P_3 \land P_4
    \]
    donde $P_1, P_2,  P_3, P_4$ son los siguientes predicados
    \begin{align*}
      P_1 &=  \lambda d d^{\prime}\sigma\alpha\beta p q \left[ d=\alpha p\beta \right] \\
          &=  \lambda \alpha\beta\left[ \alpha=\beta \right]
                \circ (
                  p_{1}^{0,7},
                  \lambda \alpha_1 \alpha_2 \alpha_3 \left[ \alpha_1\alpha_2\alpha_3 \right]
                    \circ (
                      p_{4}^{0,7},
                      p_{6}^{0,7},
                      p_{5}^{0,7}
                    )
                  ) \\[10pt]
      P_2 &=  \lambda d d^{\prime}\sigma\alpha\beta p q
                \left[ (q,\sigma ,L)\in \delta \left( p,\left[ \beta B\right] _{1}\right) \right] \\
          &=  \lambda q\sigma p\gamma \left[ (q,\sigma ,L)\in \delta (p,\gamma )\right]
                \circ (
                  p_{7}^{0,7},
                  p_{3}^{0,7},
                  p_{6}^{0,7},
                  % \lambda \alpha \left[ \left[ \alpha B\right]_{1}\right]
                  \lambda i\alpha \left[ \left[\alpha\right]_i \right]
                  \circ (
                    C_{1}^{0,7},
                    \lambda \alpha \beta \left[ \alpha\beta \right] \circ (p_{5}^{0,7}, C_{B}^{0,7})
                  )
                ) \\[10pt]
      P_3 &=  \lambda d d^{\prime}\sigma\alpha\beta p q \left[ \alpha \neq \varepsilon \right] \\
          &=  \neg \lambda \alpha\beta \left[ \alpha = \beta \right]
              \circ (
                p_{4}^{0,7},
                C_{\varepsilon}^{0,7}
              ) \\[10pt]
      P_4 &=  \lambda d d^{\prime}\sigma\alpha\beta p q
                \left[
                  d^{\prime } = \left\lfloor \alpha ^{\curvearrowleft }q\left[ \alpha \right]
                  _{\left\vert \alpha \right\vert }\sigma ^{\curvearrowright }\beta \right\rfloor
                \right] \\
          &=  \lambda \alpha\beta\left[ \alpha=\beta \right]
                \circ (
                  p_{2}^{0,7},
                  \lambda \alpha \left[ \lfloor \alpha \rfloor \right] \circ f
                ) \\
    \end{align*}
    donde
    \begin{align*}
      f =  \lambda\alpha_1\alpha_2\alpha_3\alpha_4\alpha_5 \left[ \alpha_1\alpha_2\alpha_3\alpha_4\alpha_5 \right]
            \circ (
              \lambda \alpha \left[ \alpha ^{\curvearrowleft } \right] \circ p_{4}^{0,7},
              p_{7}^{0,7},
              \lambda i\alpha \left[ \left[\alpha\right]_i \right]
                \circ (
                  \lambda \alpha \left[ |\alpha| \right] \circ p_{4}^{0,7},
                  p_{4}^{0,7}
              ),
              p_{3}^{0,7},
              \lambda \alpha \left[ ^{\curvearrowright}\alpha \right] \circ p_{5}^{0,7}
            )
    \end{align*}
    Luego, es facil ver que $P_1, P_2 , P_3, P_4 $ son $(\Gamma \cup Q)$-PR, por lo tanto $P_L$ es $(\Gamma \cup Q)$-PR.
    De manera similar, podemos ver que $P_K$ y $P_R$ son $(\Gamma \cup Q)$-PR.

    \noindent Tomemos el siguiente predicado
    \[
      P = (P_{R}\vee P_{L}\vee P_{K})
    \]
    Tenemos que $P$ es $(\Gamma \cup Q)$-PR.
    \noindent Nótese que $\lambda dd^{\prime }\left[ d\vdash d^{\prime }\right] $ es igual al predicado
    \begin{eqnarray}
      \nonumber \lambda dd^{\prime }\left[ (\exists \sigma \in \Gamma )(\exists \alpha ,\beta \in \Gamma ^{\ast })
                (\exists p,q\in Q)(P_{R}\vee P_{L}\vee P_{K})(d,d^{\prime },\sigma ,\alpha ,\beta ,p,q)\right]
    \end{eqnarray}
    lo cual aplicando 5 veces el \textbf{Lema 39} nos dice que $\lambda dd^{\prime } \left[ d\vdash d^{\prime }\right]$
    es $(\Gamma \cup Q)$-PR. Veamos como
    El \textbf{Lema 39} nos dice que
    \[
      L_1 = \lambda x d d^{\prime }\sigma\alpha\beta p
              \left[
                (\exists q\in Q)_{|q| \leq x}\; P(d,d^{\prime },\sigma ,\alpha ,\beta ,p,q)
              \right]
    \]
    es $(\Gamma \cup Q)$-PR. Como $\beta, \alpha, \sigma, p, q$ son subpalabras de $d$ y $d^{\prime}$ respectivamente
    tenemos que $ |\beta|, |\alpha|, |\sigma|, |p|, |q| \leq |d| + |d^{\prime}|$. Lo cual nos dice que el predicado
    $Q_1$ es $(\Gamma \cup Q)$-PR.
    \begin{align*}
      Q_1 &=  \lambda d d^{\prime }\sigma\alpha\beta p
                \left[
                  (\exists q\in Q)_{|q| \leq |d| + |d^{\prime}|}\; P(d,d^{\prime },\sigma ,\alpha ,\beta ,p,q)
                \right] \\[10pt]
          &= L_1  \circ (
                    \lambda x y \left[ x + y \right]
                      \circ(
                        \lambda \alpha \left[ |\alpha|\right] \circ (p_{1}^{0,6}),
                        \lambda \alpha \left[ |\alpha|\right] \circ (p_{2}^{0,6})
                    ),
                    p_{1}^{0,6},
                    p_{2}^{0,6},
                    p_{3}^{0,6},
                    p_{4}^{0,6},
                    p_{5}^{0,6},
                    p_{6}^{0,6}
                  )
    \end{align*}
    De misma manera podemos que el predicado $Q_2$ es $(\Gamma \cup Q)$-PR.
    \begin{align*}
      L_2 &=  \lambda x d d^{\prime }\sigma\alpha\beta
                \left[
                  (\exists p\in Q)_{|p| \leq x}\; Q_1(d,d^{\prime },\sigma ,\alpha ,\beta , p)
                \right] \\[10pt]
      Q_2 &=  \lambda d d^{\prime }\sigma\alpha\beta
                \left[
                  (\exists p\in Q)_{|p| \leq |d| + |d^{\prime}|}\; Q_1(d,d^{\prime },\sigma ,\alpha ,\beta , p)
                \right] \\
          &= L_2  \circ (
                    \lambda x y \left[ x + y \right]
                      \circ(
                        \lambda \alpha \left[ |\alpha|\right] \circ (p_{1}^{0,5}),
                        \lambda \alpha \left[ |\alpha|\right] \circ (p_{2}^{0,5})
                    ),
                    p_{1}^{0,5},
                    p_{2}^{0,5},
                    p_{3}^{0,5},
                    p_{4}^{0,5},
                    p_{5}^{0,5}
                  )
    \end{align*}
    finalmente, tenemos que $Q_3, Q_4, Q_5$ es $(\Gamma \cup Q)$-PR
    \begin{align*}
      L_3 &=  \lambda x d d^{\prime }\sigma\alpha
                \left[
                  (\exists \beta \in \Gamma ^{\ast })_{|\beta| \leq x}\; Q_2(d,d^{\prime },\sigma ,\alpha ,\beta)
                \right] \\[10pt]
      Q_3 &=  \lambda d d^{\prime }\sigma\alpha
                \left[
                  (\exists \beta \in \Gamma ^{\ast })_{|\beta| \leq |d| + |d^{\prime}|}\;
                  Q_2(d,d^{\prime },\sigma ,\alpha ,\beta)
                \right] \\
          &= L_3  \circ (
                    \lambda x y \left[ x + y \right]
                      \circ(
                        \lambda \alpha \left[ |\alpha|\right] \circ (p_{1}^{0,4}),
                        \lambda \alpha \left[ |\alpha|\right] \circ (p_{2}^{0,4})
                    ),
                    p_{1}^{0,4},
                    p_{2}^{0,4},
                    p_{3}^{0,4},
                    p_{4}^{0,4}
                  ) \\[20pt]
      L_4 &=  \lambda x d d^{\prime }\sigma
                \left[
                  (\exists \alpha \in \Gamma ^{\ast })_{|\alpha| \leq x}\; Q_3(d,d^{\prime },\sigma ,\alpha)
                \right] \\[10pt]
      Q_4 &=  \lambda d d^{\prime }\sigma
                \left[
                  (\exists \alpha \in \Gamma ^{\ast })_{|\alpha| \leq |d| + |d^{\prime}|}\;
                  Q_3(d,d^{\prime },\sigma ,\alpha)
                \right] \\
          &= L_4  \circ (
                    \lambda x y \left[ x + y \right]
                      \circ(
                        \lambda \alpha \left[ |\alpha|\right] \circ (p_{1}^{0,3}),
                        \lambda \alpha \left[ |\alpha|\right] \circ (p_{2}^{0,3})
                    ),
                    p_{1}^{0,3},
                    p_{2}^{0,3},
                    p_{3}^{0,3}
                  )\\[20pt]
      L_5 &=  \lambda x d d^{\prime }
                \left[
                  (\exists \sigma \in \Gamma)_{|\sigma| \leq x}\; Q_4(d,d^{\prime },\sigma)
                \right] \\[10pt]
      Q_5 &=  \lambda d d^{\prime }
                \left[
                  (\exists \sigma \in \Gamma)_{|\sigma| \leq |d| + |d^{\prime}|}\; Q_4(d,d^{\prime },\sigma)
                \right] \\
          &= L_5  \circ (
                    \lambda x y \left[ x + y \right]
                      \circ(
                        \lambda \alpha \left[ |\alpha|\right] \circ (p_{1}^{0,2}),
                        \lambda \alpha \left[ |\alpha|\right] \circ (p_{2}^{0,2})
                    ),
                    p_{1}^{0,2},
                    p_{2}^{0,2},
                  )
    \end{align*}
    Notar que $Q_5 = \lambda dd^{\prime } \left[ d\vdash d^{\prime }\right]$. Por lo tanto,
    $\lambda dd^{\prime } \left[ d\vdash d^{\prime }\right]$ es $(\Gamma \cup Q)$-PR.
  \end{proof}

  \end{lemma}

	%Proposition 81
	\begin{proposition}
		$\lambda ndd^{\prime }\left[ d\overset{n}{\vdash }d^{\prime }\right] $ es $ (\Gamma \cup Q)$-PR.
	\end{proposition}


  %Theorem 82:
  \begin{theorem}
  	Sea $M=\left( Q,\Sigma ,\Gamma ,\delta ,q_{0},B,F\right) $ una máquina de Turing.
    Entonces $L(M)$ es $\Sigma $-recursivamente enumerable.

  \begin{proof}
     Sea $P$ el siguiente predicado $(\Gamma \cup Q)$-mixto

    \[
      P = \lambda n\alpha \left[ (\exists d\in Des)\;\left\lfloor q_{0}B\alpha \right\rfloor \overset{n}{\vdash }d\wedge
        St(d)\in F\right]
    \]
    Nótese que $D_{P}=\omega \times \Gamma ^{\ast }$. Veamos que $P$ es $(\Gamma \cup Q)$-PR. Para ello definamos
    \[
      P = P_1 \land P_2 \\[20pt]
    \]
    donde $P_1$ y $P_2$
    \begin{align*}
      P_1 &=  \lambda n\alpha \left[ (\exists d\in Des)\; Q(n, \alpha, d)\right] \\[10pt]
        Q &=  \lambda n \alpha d \left[ \left\lfloor q_{0}B\alpha \right\rfloor \overset{n}{\vdash }d \right] \\
          &=  \lambda n d d^{\prime} \left[ d \overset{n}{\vdash}d \right]
                \circ (
                  \lambda \alpha \left[ \left\lfloor \alpha \right\lfloor \right]
                    \circ(
                      \lambda \alpha \beta \left[ \alpha \beta \right]
                        \circ (
                          C_{q_0 B}^{1,2},
                          p_{2}^{1,2}
                        )
                  ),
                  p_{1}^{1,2}
                ) \\[30pt]
      P_2 &=  \lambda n\alpha \left[ St(d)\in F \right]
    \end{align*}
    Sabemos que el conjunto $F$ es finito, por \textbf{Corollary 30} (finito es PR) , $F$ es $(\Gamma \cup Q)$-PR.

    \noindent Tambien sabemos que $\chi_F$ es $(\Gamma \cup Q)$-PR, por lo tanto el predicado $P_2$ es
    $(\Gamma \cup Q)$-PR.

    \noindent Por \textbf{Lema 39} sabemos que
    \[
      L = \lambda x n\alpha \left[ (\exists d\in Des)_{|d| \leq x}\; Q(n, \alpha, d)\right] \\[10pt]
    \]
    es $(\Gamma \cup Q)$-PR.
    Es decir que solo nos falta acotar el cuantificador existencial, para poder aplicar el \textbf{Lema 39} de
    cuantificacion acotada. Ya que cuando \( d_{1},...,d_{n+1}\in Des\) son tales que

    \noindent \(d_{1}\vdash d_{2}\vdash ...\vdash d_{n+1}\) tenemos que
    \[
      \displaystyle \left\vert d_{i}\right\vert \leq \left\vert d_{1}\right\vert +n\text{, para } i=1,...,n
    \]
    una posible cota para dicho cuantificador es
    \[
      |d| \leq |\lfloor q_{0}B\alpha \rfloor| + n
    \]
    O sea que, por el \textbf{Lema 39} de cuantificacion acotada, tenemos que el predicado $P_1$ es
    $(\Gamma \cup Q)$-PR. En definitiva $P$ es $(\Gamma \cup Q)$-PR.
    Sea
    \[
      P^{\prime }=P\mid _{\omega \times \Sigma ^{\ast }}.
    \]
    Nótese que $P^{\prime }(n,\alpha )=1$ sii
    $\alpha \in L(M) $ atestiguado por una computación de longitud $n$.

    \noindent Por \textbf{Corollary 72} (restriccion de una funcion) $P^{\prime }$ es $(\Gamma \cup Q)$-PR,
    y ademas es $\Sigma $-mixto.

    \noindent El \textbf{Teorema 51} (independencia del alfabeto) nos dice que $P^{\prime }$ es $\Sigma $-PR.

    \noindent Ya que $ L(M)=D_{M(P^{\prime })}$, el \textbf{Teorema 71} nos dice que $ L(M)$ es $\Sigma $-r.e.
  \end{proof}
  \end{theorem}

	%Theorem 83
	\begin{theorem}
		Supongamos $f:S\subseteq \omega ^{n}\times \Sigma ^{\ast }{}^{m}\rightarrow O $ es
    $\Sigma $-Turing computable. Entonces $f$ es $\Sigma $-recursiva.
	\begin{proof}
    Supongamos $O=\Sigma ^{\ast }$ y sea $M=\left( Q,\Sigma ,\Gamma ,\delta ,q_{0},B,\shortmid ,F\right) $
    una máquina de Turing determinística con unit la cual compute a $f$. Sea $< $ un orden total estricto sobre
    $\Gamma \cup Q$ . Sea $P:\mathbf{N}\times \omega ^{n}\times \Sigma ^{\ast m}\rightarrow \omega $ dado por
    $P(x,\vec{x},\vec{\alpha})=1$ sii

    \begin{eqnarray*}
      (\exists q\in Q)\;
        \left\lfloor q_{0}B\shortmid ^{x_{1}}...B\shortmid ^{x_{n}}B\alpha _{1}...B\alpha _{m}\right\rfloor
        \overset{(x)_{1}}{\vdash }
        \left\lfloor qB\ast ^{< }((x)_{2})\right\rfloor \wedge \\
       (\forall d\in Des)
         _{\left\vert d\right\vert \leq \left\vert \ast ^{< }((x)_{2})\right\vert +2}\;
         \left\lfloor qB\ast ^{< }((x)_{2})\right\rfloor \nvdash d
    \end{eqnarray*}

    Es fácil ver que $P$ es $(\Gamma \cup Q)$-PR.
    Tomemos predicados $P_1$ y $P_2$ tales que
    \begin{align*}
      P   &=  P_1 \land P_2 \\[20pt]
      P_1 &=  \lambda x \vec{x} \vec{\alpha}
              \left[
                (\exists q\in Q)\;
                \left\lfloor q_{0}B\shortmid ^{x_{1}}...B\shortmid ^{x_{n}}B\alpha _{1}...B\alpha _{m}\right\rfloor
                \overset{(x)_{1}}{\vdash }
                \left\lfloor qB\ast ^{< }((x)_{2})\right\rfloor
              \right] \\
      P_2 &=  \lambda x \vec{x} \vec{\alpha}
              \left[
                (\forall d\in Des)
                _{\left\vert d\right\vert \leq \left\vert \ast ^{< }((x)_{2})\right\vert +2}\;
                \left\lfloor qB\ast ^{< }((x)_{2})\right\rfloor \nvdash d
              \right]
    \end{align*}
    Si tomamos $Q_1$ y $Q_2$ como
    \begin{align*}
      Q_1 &=  \lambda x \vec{x} \vec{\alpha} q
              \left[
                \left\lfloor q_{0}B\shortmid ^{x_{1}}...B\shortmid ^{x_{n}}B\alpha _{1}...B\alpha _{m}\right\rfloor
                \overset{(x)_{1}}{\vdash }
                \left\lfloor qB\ast ^{< }((x)_{2})\right\rfloor
              \right] \\
      Q_2 &=  \lambda x \vec{x} \vec{\alpha} d
              \left[
                \left\lfloor qB\ast ^{< }((x)_{2})\right\rfloor \nvdash d
              \right]
    \end{align*}
    Tenemos que
    \begin{align*}
      P_1 &=  \lambda x \vec{x} \vec{\alpha}
              \left[
                (\exists q\in Q)\;
                Q_1(x, \vec{x}, \vec{\alpha}, q)
              \right] \\
      P_2 &=  \lambda x \vec{x} \vec{\alpha}
              \left[
                (\forall d\in Des)
                _{\left\vert d\right\vert \leq \left\vert \ast ^{< }((x)_{2})\right\vert +2}\;
                Q_2(x, \vec{x}, \vec{\alpha}, d)
              \right]
    \end{align*}
    Es facil ver que $Q_1$ y $Q_2$ son $(\Gamma \cup Q)$-PR.

    \noindent Aplicando el \textbf{Lema 39} de cuantificacion acotada tenemos que $P_1$ y $P_2$ son
    $(\Gamma \cup Q)$-PR.

    \noindent Finalmente, $P$ es $(\Gamma \cup Q)$-PR.

    \noindent Ya que es $\Sigma $-mixto, el \textbf{Teorema 51} (independencia del alfabeto) nos dice que es
    $\Sigma $-PR. Nótese que
    \[
      f=\lambda \vec{x}\vec{\alpha}\left[ \left( \min_{x}P(x,\vec{x},\vec{\alpha} )\right) _{2}\right] \text{,}
    \]
    lo cual nos dice que $f$ es $\Sigma $-recursiva.
  \end{proof}
  \end{theorem}

	%Lema 84
	\begin{lemma}
		Sea $\mathcal{P}\in \mathrm{Pro}^{\Sigma }$ y sea $k$ tal que las variables que ocurren
    en $\mathcal{P}$ están todas en la lista $\mathrm{N}1,..., \mathrm{N}\bar{k},\mathrm{P}1,...,\mathrm{P}\bar{k}.$
    Para cada $a\in \Sigma \cup \{\shortmid \}$, sea $\tilde{a}$ un nuevo símbolo. Sea $\Gamma =\Sigma \cup \{B,
    \shortmid \}\cup \{\tilde{a}:a\in \Sigma \cup \{\shortmid \}\}$. Entonces hay una máquina de Turing determinística
    con unit $M=\left( Q,\Gamma ,\Sigma ,\delta ,q_{0},B,\shortmid ,\{q_{f}\}\right) $ la cual satisface
    \begin{enumerate}[\qquad(1)]
      \item $\delta (q_{f},\sigma )=\emptyset $, para cada $\sigma \in \Gamma $.
      \item Cualesquiera sean $x_{1},...,x_{k}\in \omega $ y $\alpha _{1},...,\alpha _{k}\in \Sigma ^{\ast }$, el
            programa $\mathcal{P}$ se detiene partiendo del estado
            \[
              \left( (x_{1},...,x_{k},0,...),(\alpha _{1},...,\alpha _{k},\varepsilon ,...)\right)
            \]
            sii $M$ se detiene partiendo de la descripción instantánea
            \[
              \left\lfloor q_{0}B\shortmid ^{x_{1}}B...B\shortmid ^{x_{k}}B\alpha _{1}B...B\alpha _{k}B\right\rfloor
            \]
      \item Si $x_{1},...,x_{k}\in \omega $ y $\alpha _{1},...,\alpha _{k}\in \Sigma ^{\ast }$ son tales que
            $\mathcal{P}$ se detiene partiendo del estado
            \[
              \left( (x_{1},...,x_{k},0,...),(\alpha _{1},...,\alpha _{k},\varepsilon ,...)\right)
            \]
            y llega al estado
            \[
              \left( (y_{1},...,y_{k},0,...),(\beta _{1},...,\beta _{k},\varepsilon ,...)\right)
            \]
            entonces
            \[
              \left\lfloor q_{0}B\shortmid ^{x_{1}}B...B\shortmid ^{x_{k}}B\alpha _{1}B...B\alpha _{k}B\right\rfloor \overset
              {\ast }{\vdash }\left\lfloor q_{f}B\shortmid ^{y_{1}}B...B\shortmid ^{y_{k}}B\beta _{1}B...B\beta _{k}B\right
              \rfloor
            \]
    \end{enumerate}


  \begin{proof}
    Dado un estado $((x_{1},...,x_{k},0,...),(\alpha _{1},...,\alpha _{k},\varepsilon ,...))$ lo representaremos en la
    cinta de la siguiente manera
    \[
      B\shortmid ^{x_{1}}...B\shortmid ^{x_{k}}B\alpha _{1}...B\alpha _{k}BBBB....
    \]

    A continuación describiremos una serie de maquinas las cuales simularan, vía la representación anterior, las
    distintas clases de instrucciones que pueden ocurrir en $\mathcal{P}$. Todas las maquinas definidas tendrán a
    $\shortmid $ como unit y a $B$ como blanco, tendrán a $\Sigma $ como su alfabeto terminal y su alfabeto mayor sera
    $\Gamma =\Sigma \cup \{B,\shortmid \}\cup \{\tilde{a }:a\in \Sigma \cup \{\shortmid \}\}$. Ademas tendrán uno o dos
    estados finales con la propiedad de que si $q$ es un estado final, entonces $\delta (q,\sigma )=\emptyset $, para cada $\sigma \in \Gamma $. Esta propiedad es importante ya que nos permitirá concatenar pares de dichas maquinas identificando algún estado final de la primera con el inicial de la segunda. \\

		\noindent Para $1\leq i\leq k$, sea $M_{i}^{+}$ una máquina tal que

		\bigskip

    $\displaystyle \begin{array}{lcl} B\shortmid ^{x_{1}}...B\shortmid ^{x_{k}}B\alpha _{1}...B\alpha _{k} &
    \overset{\ast }{\vdash } & B\shortmid ^{x_{1}}...B\shortmid ^{x_{i-1}}B\shortmid ^{x_{i}+1}B\shortmid
    ^{x_{i+1}}...B\shortmid ^{x_{k}}B\alpha _{1}...B\alpha _{k} \\ \uparrow & & \uparrow \\ q_{0} & & q_{f} \end{array} $

    Es claro que la máquina $M_{i}^{+}$ simula la instrucción $\mathrm{N}\bar{ \imath}\leftarrow \mathrm{N}\bar{\imath}+1$.

    \noindent Para $1\leq i\leq k$, sea $M_{i}^{\dot{-}}$ una máquina tal que

		\bigskip

    $\displaystyle \begin{array}{lcl} B\shortmid ^{x_{1}}...B\shortmid ^{x_{k}}B\alpha _{1}...B\alpha _{k} &
    \overset{\ast }{\vdash } & B\shortmid ^{x_{1}}...B\shortmid ^{x_{i-1}}B\shortmid ^{x_{i}\dot{-}1}B\shortmid
    ^{x_{i+1}}...B\shortmid ^{x_{k}}B\alpha _{1}...B\alpha _{k} \\ \uparrow & & \uparrow \\ q_{0} & & q_{f} \end{array} $

		\bigskip


    \noindent Para $1\leq i\leq k$ y $a\in \Sigma $, sea $M_{i}^{a}$ una máquina tal que

		\bigskip


    $\displaystyle \begin{array}{lcl} B\shortmid ^{x_{1}}...B\shortmid ^{x_{k}}B\alpha _{1}...B\alpha _{k} &
    \overset{\ast }{\vdash } & B\shortmid ^{x_{1}}...B\shortmid ^{x_{k}}B\alpha _{1}...B\alpha _{i-1}B\alpha
    _{i}aB\alpha _{i+1}...B\alpha _{k} \\ \uparrow & & \uparrow \\ q_{0} & & q_{f} \end{array} $

		\bigskip

    \noindent Para $1\leq i\leq k$, sea $M_{i}^{\curvearrowright }$ una máquina tal que

    $\displaystyle \begin{array}{lcl} B\shortmid ^{x_{1}}...B\shortmid ^{x_{k}}B\alpha _{1}...B\alpha _{k} &
    \overset{\ast }{\vdash } & B\shortmid ^{x_{1}}...B\shortmid ^{x_{k}}B\alpha _{1}...B\alpha _{i-1}B
    ^{\curvearrowright }\alpha _{i}B\alpha _{i+1}...B\alpha _{k} \\ \uparrow & & \uparrow \\ q_{0} & & q_{f} \end{array} $

		\bigskip

    \noindent Para $j=1,...,k$, y $a\in \Sigma $, sea $IF_{j}^{a}$ una máquina con dos estados finales $q_{si}$ y
    $q_{no}$ tal que si $\alpha _{j}$ comienza con $a$ , entonces

		\bigskip

    $\displaystyle \begin{array}{lcl} B\shortmid ^{x_{1}}...B\shortmid ^{x_{k}}B\alpha _{1}...B\alpha _{k} &
    \overset{\ast }{\vdash } & B\shortmid ^{x_{1}}...B\shortmid ^{x_{k}}B\alpha _{1}...B\alpha _{k} \\
    \uparrow & & \uparrow \\ q_{0} & & q_{si} \end{array} $

    \noindent y en caso contrario

		\bigskip

    $\displaystyle \begin{array}{lcl} B\shortmid ^{x_{1}}...B\shortmid ^{x_{k}}B\alpha _{1}...B\alpha _{k} &
    \overset{\ast }{\vdash } & B\shortmid ^{x_{1}}...B\shortmid ^{x_{k}}B\alpha _{1}...B\alpha _{k} \\
    \uparrow & & \uparrow \\ q_{0} & & q_{no} \end{array} $

		\bigskip

	 	\noindent Análogamente para $j=1,...,k$, sea $IF_{j}$ una máquina tal que si $ x_{j}\neq 0$, entonces

		\bigskip

    $\displaystyle \begin{array}{lcl} B\shortmid ^{x_{1}}...B\shortmid ^{x_{k}}B\alpha _{1}...B\alpha _{k} &
    \overset{\ast }{\vdash } & B\shortmid ^{x_{1}}...B\shortmid ^{x_{k}}B\alpha _{1}...B\alpha _{k} \\
    \uparrow & & \uparrow \\ q_{0} & & q_{si} \end{array} $

		\bigskip

	  \noindent y si $x_{j}=0$, entonces

		\bigskip

    $\displaystyle \begin{array}{lcl} B\shortmid ^{x_{1}}...B\shortmid ^{x_{k}}B\alpha _{1}...B\alpha _{k} &
    \overset{\ast }{\vdash } & B\shortmid ^{x_{1}}...B\shortmid ^{x_{k}}B\alpha _{1}...B\alpha _{k} \\
    \uparrow & & \uparrow \\ q_{0} & & q_{no} \end{array} $

		\bigskip

	  \noindent Para $1\leq i,j\leq k$, sea $M_{i\leftarrow j}^{\ast }$ una máquina tal que

		\bigskip

    $\displaystyle \begin{array}{lcl} B\shortmid ^{x_{1}}...B\shortmid ^{x_{k}}B\alpha _{1}...B\alpha _{k} &
    \overset{\ast }{\vdash } & B\shortmid ^{x_{1}}...B\shortmid
    ^{x_{k}}B\alpha _{1}...B\alpha _{i-1}B\alpha _{j}B\alpha _{i+1}...B\alpha _{k} \\
    \uparrow & & \uparrow \\ q_{0} & & q_{f} \end{array} $

		\bigskip

    \noindent Para $1\leq i,j\leq k$, sea $M_{i\leftarrow j}^{\#}$ una máquina tal que

		\bigskip

    $\displaystyle \begin{array}{lcl} B\shortmid ^{x_{1}}...B\shortmid ^{x_{k}}B\alpha _{1}...B\alpha _{k} &
    \overset{\ast }{\vdash } & B\shortmid ^{x_{1}}...B\shortmid ^{x_{i-1}}B\shortmid ^{x_{j}}B\shortmid
    ^{x_{i+1}}...B\shortmid ^{x_{k}}B\alpha _{1}...B\alpha _{k} \\ \uparrow & & \uparrow \\ q_{0} & & q_{f} \end{array} $

		\bigskip

    \noindent Para $1\leq i\leq k$, sea $M_{i\leftarrow 0}$ una máquina tal que

		\bigskip

    $\displaystyle \begin{array}{lcl} B\shortmid ^{x_{1}}...B\shortmid ^{x_{k}}B\alpha _{1}...B\alpha _{k} &
    \overset{\ast }{\vdash } & B\shortmid ^{x_{1}}...B\shortmid ^{x_{i-1}}BB\shortmid ^{x_{i+1}}...B\shortmid
    ^{x_{k}}B\alpha _{1}...B\alpha _{k} \\ \uparrow & & \uparrow \\ q_{0} & & q_{f} \end{array} $

		\bigskip

    \noindent Para $1\leq i\leq k$, sea $M_{i\leftarrow \varepsilon }$ una máquina tal que

		\bigskip

    $\displaystyle \begin{array}{lcl} B\shortmid ^{x_{1}}...B\shortmid ^{x_{k}}B\alpha _{1}...B\alpha _{k} &
    \overset{\ast }{\vdash } & B\shortmid ^{x_{1}}...B\shortmid
    ^{x_{k}}B\alpha _{1}...B\alpha _{i-1}BB\alpha _{i+1}...B\alpha _{k} \\
    \uparrow & & \uparrow \\ q_{0} & & q_{f} \end{array} $

		\bigskip

    \noindent Sea

    $\displaystyle M_{\mathrm{SKIP}}=\left( \{q_{0},q_{f}\},\Gamma ,\Sigma ,\delta ,q_{0},B,\shortmid ,\{q_{f}\}\right) ,$

		\noindent con $\delta (q_{0},B)=\{(q_{f},B,K)\}$ y $\delta =\emptyset $ en cualquier otro caso.

		\bigskip

    \noindent Finalmente sea

		\[
    \displaystyle M_{\mathrm{GOTO}}=\left( \{q_{0},q_{si},q_{no}\},\Gamma ,\Sigma ,\delta ,q_{0},B,\shortmid ,\{q_{si},q_{no}\}\right) ,
    \]

    \noindent con $\delta (q_{0},B)=\{(q_{si},B,K)\}$ y $\delta =\emptyset $ en cualquier otro caso.

		Para poder hacer concretamente las maquinas recién descriptas deberemos diseñar antes algunas máquinas auxiliares.
		Para cada $j\geq 1$, sea $D_{j}$ la máquina descripta en la Figura 1. Notese que

		\[
		\displaystyle \begin{array}{lcr} \alpha B\beta _{1}B\beta _{2}B...B\beta _{j}B\gamma & \overset{\ast }{\vdash } & \alpha B\beta _{1}B\beta _{2}B...B\beta _{j}B\gamma \\
    \ \ \uparrow & & \uparrow \ \ \\ \ \ q_{0} & & q_{f}\ \ \end{array}
		\]

		\bigskip

    \noindent siempre que $\alpha ,\gamma \in \Gamma ^{\ast }$, $\beta _{1},...,\beta _{j}\in (\Gamma -\{B\})^{\ast }$.
    Analogamente tenemos definidas las maquinas $I_{j}.$

    Para $j\geq 1$, sea $TD_{j}$ una máquina con un solo estado final $q_{f}$ y tal que

		\[
    \displaystyle \begin{array}{ccc} \alpha B\gamma & \overset{\ast }{\vdash } & \alpha BB\gamma \\
    \uparrow & & \uparrow \ \ \\ q_{0} & & q_{f}\ \ \end{array}
    \]

		\bigskip

    cada vez que $\alpha ,\gamma \in \Gamma ^{\ast }$ y $\gamma $ tiene exactamente $j$ ocurrencias de $B$.
    Es decir la máquina $TD_{j}$ corre un espacio a la derecha todo el bloque $\gamma $ y agrega un blanco en el espacio
    que se genera a la izquierda de dicho bloque. Por ejemplo, para el caso de $\Sigma =\{\& \}$ podemos tomar $TD_{3}$
    igual a la máquina de la Figura 3.

    Análogamente, para $j\geq 1$, sea $TI_{j}$ una máquina tal que

		\[
	  \displaystyle \begin{array}{ccc} \alpha B\sigma \gamma & \overset{\ast }{\vdash } & \alpha B\gamma \\
    \uparrow \ & & \uparrow \\ q_{0}\ \ & & q_{f} \end{array}
	  \]

    cada vez que $\alpha \in \Gamma ^{\ast }$, $\sigma \in \Gamma $ y $\gamma $ tiene exactamente $j$ ocurrencias de
    $B$. Es decir la máquina $TI_{j}$ corre un espacio a la izquierda todo el bloque $\gamma $ (por lo cual en el lugar
    de $\sigma $ queda el primer símbolo de $\gamma $).
    Teniendo las maquinas auxiliares antes definidas podemos combinarlas para obtener las maquinas simuladoras de
    instrucciones. Por ejemplo $M_{i}^{a}$ puede ser la máquina descripta en la Figura 4. En la Figura 2 tenemos una
    posible forma de diseñar la máquina $IF_{i}^{a}$. En la Figura 7 tenemos una posible forma de diseñar la máquina
    $M_{i\leftarrow j}^{\ast }$ para el caso $\Sigma =\{a,b\}$ y $i< j$.

    Supongamos ahora que $\mathcal{P}=I_{1}...I_{n}$. Para cada $i=1,...,n$, definiremos una máquina $M_{i}$ que
    simulara la instrucción $I_{i}$. Luego uniremos adecuadamente estas maquinas para formar la máquina que simulara a
    $ \mathcal{P}$

		\begin{itemize}
			\item Si $Bas(I_{i})=\mathrm{N}\bar{j}\leftarrow \mathrm{N}\bar{j}+1$ tomaremos $M_{i}=M_{j}^{+}$
	    \item Si $Bas(I_{i})=\mathrm{N}\bar{j}\leftarrow \mathrm{N}\bar{j}\dot{-} 1 $ tomaremos $M_{i}=M_{j}^{\dot{-}}$
	    \item Si $Bas(I_{i})=\mathrm{N}\bar{j}\leftarrow 0$ tomaremos $ M_{i}=M_{j\leftarrow 0}$.
	    \item Si $Bas(I_{i})=\mathrm{N}\bar{j}\leftarrow \mathrm{N}\bar{m}$ tomaremos $M_{i}=M_{j\leftarrow m}^{\#}$.
	    \item Si $Bas(I_{i})=\mathrm{IF}\;\mathrm{N}\bar{j}\not=0$ $\mathrm{GOTO} \;\mathrm{L}\bar{m}$ tomaremos $M_{i}=IF_{j}.$
	    \item Si $Bas(I_{i})=\mathrm{P}\bar{j}\leftarrow \mathrm{P}\bar{j}.a$ tomaremos $M_{i}=M_{j}^{a}$.
	    \item Si $Bas(I_{i})=\mathrm{P}\bar{j}\leftarrow \ ^{\curvearrowright } \mathrm{P}\bar{j}$ tomaremos $M_{i}=M_{j}^{\curvearrowright }$.
	    \item $Bas(I_{i})=\mathrm{P}\bar{j}\leftarrow \varepsilon $ tomaremos $ M_{i}=M_{j\leftarrow \varepsilon }$.
	    \item $Bas(I_{i})=\mathrm{P}\bar{j}\leftarrow \mathrm{P}\bar{m}$ tomaremos $M_{i}=M_{j\leftarrow m}^{\ast }$.
	    \item $Bas(I_{i})=\mathrm{IF}\;\mathrm{P}\bar{j}\;\mathrm{BEGINS}\;a\; \mathrm{GOTO}\;\mathrm{L}\bar{m}$ tomaremos $M_{i}=IF_{j}^{a}$.
	    \item $Bas(I_{i})=\mathrm{SKIP}$ tomaremos $M_{i}=M_{\mathrm{SKIP}}$.
	    \item $Bas(I_{i})=\mathrm{GOTO}\;\mathrm{L}\bar{m}$ tomaremos $ M_{i}=M_{\mathrm{GOTO}}$.
		\end{itemize}

    Ya que la máquina $M_{i}$ puede tener uno o dos estados finales, la representaremos como se muestra en la Figura 5,
    entendiendo que en el caso en que $M_{i}$ tiene un solo estado final, este esta representado por el circulo de
    abajo a la izquierda y en el caso en que $M_{i}$ tiene dos estados finales, el estado final representado con lineas
     punteadas corresponde al estado $q_{si}$ y el otro al estado $q_{no}$.

    Para armar la máquina que simulara a $\mathcal{P}$ hacemos lo siguiente. Primero unimos las maquinas
    $M_{1},...,M_{n}$ como lo muestra la Figura 6. Luego para cada $i$ tal que $Bas(I_{i})$ es de la forma
    $\alpha \mathrm{GOTO} \;\mathrm{L}\bar{m}$, ligamos con una flecha de la forma

		\[
    \displaystyle \underrightarrow{\;\;\;\;\;\;B,B,K\;\;\;\;\;\;}
		\]
    el estado final $q_{si}$ de la $M_{i}$ con el estado inicial de la $M_{h}$, donde $h$ es tal que $I_{h}$ es la
    primer instrucción que tiene label $ \mathrm{L}\bar{m}$.
    Es intuitivamente claro que la máquina así obtenida cumple con lo requerido aunque una Proof formal de esto puede
    resultar extremadamente tediosa.
	\end{proof}
	\end{lemma}

	%Theorem 85
	\begin{theorem}
		Si $f:D_{f}\subseteq \omega ^{n}\times \Sigma ^{\ast m}\rightarrow O$ es $ \Sigma $-recursiva, entonces $f$ es
    $\Sigma $-Turing computable.

  \begin{proof} Supongamos $O=\Sigma ^{\ast }.$ Ya que $f$ es $\Sigma $-computable, existe
    $ \mathcal{P}\in \mathrm{Pro}^{\Sigma }$ el cual computa $f$. Note que podemos suponer que $\mathcal{P}$ tiene la
    propiedad de que cuando $\mathcal{P}$ termina, en el estado alcanzado las variables numéricas tienen todas el valor
    $0$ y las alfabéticas distintas de $\mathrm{P}1$ todas el valor $ \varepsilon $. Sea $M$ la máquina de Turing con
    unit dada por el lema anterior, donde elegimos el numero $k$ con la propiedad adicional de ser mayor que $n$ y $m$.
    Sea $M_{1}$ una máquina tal que para cada $(\vec{x}, \vec{\alpha})\in \omega ^{n}\times \Sigma ^{\ast m}$,

  	\[
    \displaystyle \left\lfloor q_{0}B\shortmid ^{x_{1}}B...B\shortmid
    ^{x_{n}}B\alpha _{1}B...B\alpha _{n}B\right\rfloor \overset{\ast }{\vdash }\left\lfloor qB\shortmid
    ^{x_{1}}B...B\shortmid ^{x_{n}}B^{k-n}B\alpha _{1}B...B\alpha _{m}B\right\rfloor
  	\]

    donde $q_{0}$ es el estado inicial de $M_{1}$ y $q$ es un estado tal que $ \delta (q,\sigma )=\emptyset $, para cada
    $\sigma .$ Sea $M_{2}$ una máquina tal que para cada $\alpha \in \Sigma ^{\ast }$,

  	\[
    \displaystyle \left\lfloor q_{0}B^{k+1}\alpha \right\rfloor \overset{\ast }{\vdash } \left\lfloor qB\alpha \right\rfloor
  	\]

    donde $q_{0}$ es el estado inicial de $M_{2}$ y $q$ es un estado tal que $ \delta (q,\sigma )=\emptyset $, para cada
    $\sigma $. Note que la concatenación de $M_{1}$, $M$ y $M_{2}$ (en ese orden) produce una máquina de Turing la cual
    computa $f$.
	\end{proof}
	\end{theorem}

  % Theorem  86
  \begin{theorem}
    \par Este teorema no se evalua.
  \end{theorem}
