\section{Notación y conceptos basicos}

Usaremos \(\mathbf{N}\) para denotar el conjunto de los números naturales y \( \omega \) para denotar al conjunto
\(\mathbf{N}\cup \{0\}\). Dados conjuntos \( A_{1},...,A_{n}\) usaremos \(A_{1}\times ...\times A_{n}\) para denotar el
producto Cartesiano de \(A_{1},...,A_{n}\), es decir el conjunto formado por todas las \(n\)-uplas \((a_{1},...,a_{n})\)
tales que \(a_{1}\in A_{1},...,a_{n}\in A_{n}\). Si \(A_{1}=...=A_{n}=A\), entonces escribiremos \( A^{n}\) en lugar de
\(A_{1}\times ...\times A_{n}.\) Usaremos \(\Diamond \) para denotar la unica \(0\)-upla. O sea que \(A^{0}=\{\Diamond
\}\). Si \( A_{1},A_{2},...\) es una sucesion infinita de conjuntos, entonces usaremos \( \bigcup\nolimits_{i=1}^{\infty
}A_{i}\) o \(\bigcup\nolimits_{i\geq 1}A_{i}\) para denotar al conjunto

\(\displaystyle \{a:a\in A_{i}\mathrm{,\ para\ algun\ }i\in \mathbf{N}\} \)

Una funcion es un conjunto \(f\) de pares ordenados con la siguiente propiedad

- Si \((x,y)\in f\) y \((x,z)\in f\), entonces \(y=z\).
Dada una funcion \(f\), definamos

\(\displaystyle \begin{array}{rcl} D_{f} & =& \text{ dominio de }f=\{x:(x,y)\in f\text{ para algun }y\} \\ I_{f} & =& \text{ imagen de }f=\{y:(x,y)\in f\text{ para algun }x\} \end{array} \)

Como es usual dado \(x\in D_{f}\), usaremos \(f(x)\) para denotar al unico \(y\in I_{f}\) tal que \((x,y)\in f\). Notese que \(\varnothing \) es una funcion.
Escribiremos \(f:S\subseteq A\rightarrow B\) para expresar que \(f\) es una funcion tal que \(D_{f}=S\subseteq A\) y \(I_{f}\subseteq B\). Tambien escribiremos \(f:A\rightarrow B\) para expresar que \(f\) es una funcion tal que \(D_{f}=A\) y \(I_{f}\subseteq B\). Una funcion \(f\) es inyectiva cuando no se da que \(f(a)=f(b)\) para agun par de lementos distintos \(a,b\). Dada una funcion \(f:A\rightarrow B\) diremos que \(f\) es suryectiva cuando \( I_{f}=B\). Debe notarse que el concepto de suryectividad depende de un conjunto previamente fijado, \(B\) el cual contenga a \(I_{f}\), no tiene sentido hablar de la suryectividad de una funcion \(f\) si no decimos respecto de que conjunto lo es. Dada una funcion \(f:A\rightarrow B\) diremos que \(f\) es biyectiva cuando \(f\) sea inyectiva y suryectiva.

Un alfabeto es un conjunto finito de simbolos. Notese que \( \varnothing \) es un alfabeto. Si \(\Sigma \) es un alfabeto no vacio, entonces \( \Sigma ^{\ast }\) denotara el conjunto de todas las palabras formadas con simbolos de \(\Sigma \). Por convension definiremos \(\varnothing ^{\ast }=\varnothing \). Usaremos \(\left\vert \alpha \right\vert \) para denotar la longitud de la palabra \(\alpha \). La unica palabra de longitud \(0\) es denotada con \(\varepsilon \). Notese que funciones, \(n\)-uplas y palabras son objetos de distinta naturaleza por lo cual \(\varnothing \), \(\Diamond \) y \( \varepsilon \) son tres objetos matematicos diferentes.

Si \(\alpha _{1},...,\alpha _{n}\in \Sigma ^{\ast }\), usaremos \(\alpha _{1}...\alpha _{n}\) para denotar la concatenacion de las palabras \( \alpha _{1},...,\alpha _{n}\). Si \(\alpha _{1}=...=\alpha _{n}=\alpha \), entonces escribiremos \(\alpha ^{n}\) en lugar de \(\alpha _{1}...\alpha _{n}\). Por convencion \(\alpha ^{0}=\varepsilon \). Diremos que \(\beta \) es un tramo inicial (propio) de \(\alpha \) si hay una palabra \(\gamma \) tal que \(\alpha =\beta \gamma \) (y \(\beta \notin \{\varepsilon ,\alpha \}\)). En forma similar se define tramo final (propio).

Dados \(i\in \omega \) y \(\alpha \in \Sigma ^{\ast }\) definamos

\(\displaystyle \left[ \alpha \right] _{i}=\left\{ \begin{array}{lll} i\text{-esimo elemento de }\alpha & & \text{si }1\leq i\leq \left\vert \alpha \right\vert \\ \varepsilon & & \text{caso contrario} \end{array} \right. \)

Para \(x,y\in \omega \), definamos
\(\displaystyle x\dot{-}y=\left\{ \begin{array}{lll} x-y & & \text{si }x\geq y \\ 0 & & \text{caso contrario} \end{array} \right. \)

Dados \(x,y\in \omega \) diremos que \(x\) divide a \(y\) cuando haya un \( z\in \omega \) tal que \(y=z.x\). Escribiremos \(x\mid y\) para expresar que \(x\) divide a \(y\).
Usaremos \(\omega ^{n}\times \Sigma ^{\ast m}\) para abreviar la expresion

\(\displaystyle \overset{n}{\overbrace{\omega \times ...\times \omega }}\times \overset{m}{ \overbrace{\Sigma ^{\ast }\times ...\times \Sigma ^{\ast }}} \)

Por ejemplo, cuando \(n=m=0\), tenemos que \(\omega ^{n}\times \Sigma ^{\ast m}\) denota el conjunto \(\{\Diamond \}\) y si \(m=0\), entonces \(\omega ^{n}\times \Sigma ^{\ast m}\) denota el conjunto \(\omega ^{n}\). Escribiremos \((\vec{x}, \vec{\alpha})\) en lugar de \((x_{1},...,x_{n},\alpha _{1},...,\alpha _{m})\). Notese que cuando \(\Sigma =\varnothing \), entonces \(\omega ^{n}\times \Sigma ^{\ast m}=\varnothing \), si \(m\geq 1\).

\subsection{Funciones \(\Sigma \)-mixtas}

Una funcion \(f\) es llamada \(\Sigma \)-mixta si existen \(n,m\geq 0\), tales que \(D_{f}\subseteq \omega ^{n}\times \Sigma ^{\ast m}\) y ya sea \( I_{f}\subseteq \omega \) o \(I_{f}\subseteq \Sigma ^{\ast }\). Dada una funcion \(\Sigma \)-mixta \(f:S\subseteq \omega ^{n}\times \Sigma ^{\ast m}\rightarrow O \), con \(O\in \{\omega ,\Sigma ^{\ast }\}\), diremos que \(f\) es \(\Sigma \)- total cuando \(D_{f}=\omega ^{n}\times \Sigma ^{\ast m}\). Cabe destacar que si \(\Sigma \subseteq \Gamma \), entonces toda función \(\Sigma \) -mixta es \(\Gamma \)-mixta. Sin embargo una función puede ser \(\Sigma \)-total y no ser \(\Gamma \)-total, cuando \(\Sigma \subseteq \Gamma \). Dado que \( \varnothing ^{\ast }=\varnothing \), una función \(f\) es \(\varnothing \)-mixta si y solo si existe \(n\geq 0\), tal que \(D_{f}\subseteq \omega ^{n}\) y \( I_{f}\subseteq \omega \).

A continuación daremos algunas funciones \(\Sigma \)-mixtas básicas las cuales serán frecuentemente usadas.

La función sucesor es definida por

\(\displaystyle \begin{array}{rll} Suc:\omega & \rightarrow & \omega \\ n & \rightarrow & n+1 \end{array} \)

La función predecesor es definida por

\(\scriptstyle \begin{array}{rll} Pred:\mathbf{N} & \rightarrow & \omega \\ n & \rightarrow & n-1 \end{array} \)

Para cada \(a\in \Sigma \), definamos

\(\displaystyle \begin{array}{rll} d_{a}:\Sigma ^{\ast } & \rightarrow & \Sigma ^{\ast } \\ \alpha & \rightarrow & \alpha a \end{array} \)

Para \(n,m\in \omega \) y \(n\geq i\geq 1\), definamos

\(\displaystyle \begin{array}{rll} p_{i}^{n,m}:\omega ^{n}\times \Sigma ^{\ast m} & \rightarrow & \omega \\ (\vec{x},\vec{\alpha}) & \rightarrow & x_{i} \end{array} \)

Para \(n,m\in \omega \) y \(n+m\geq i\geq n+1\), definamos

\(\displaystyle \begin{array}{rll} p_{i}^{n,m}:\omega ^{n}\times \Sigma ^{\ast m} & \rightarrow & \Sigma ^{\ast } \\ (\vec{x},\vec{\alpha}) & \rightarrow & \alpha _{i-n} \end{array} \)

Las funciones \(p_{i}^{n,m}\) son llamadas proyecciones. Para \( n,m,k\in \omega \), y \(\alpha \in \Sigma ^{\ast }\), definamos


\(\displaystyle \begin{array}{rll} C_{k}^{n,m}:\omega ^{n}\times \Sigma ^{\ast m} & \rightarrow & \omega \\ (\vec{x},\vec{\alpha}) & \rightarrow & k \end{array} \ \ \ \ \ \ \ \ \ \ \ \ \ \ \ \ \ \ \ \ \ \ \ \ \begin{array}{rll} C_{\alpha }^{n,m}:\omega ^{n}\times \Sigma ^{\ast m} & \rightarrow & \Sigma ^{\ast } \\ (\vec{x},\vec{\alpha}) & \rightarrow & \alpha \end{array} \)

Notese que \(C_{k}^{0,0}:\{\Diamond \}\rightarrow \{k\}\) y que \( p_{i}^{n,0},C_{k}^{n,0}:\omega ^{n}\rightarrow \omega \).
\par Definamos


\(\displaystyle \begin{array}{rll} pr:\mathbf{N} & \rightarrow & \omega \\ n & \rightarrow & n\text{-esimo numero primo} \end{array} \)

Notese que \(pr(1)=2\), \(pr(2)=3\), \(pr(3)=5\), etc.

\subsection{Predicados \(\Sigma \)-mixtos}

Un predicado \(\Sigma \)-mixto es una funcion \(f\) la cual es \(\Sigma \)-mixta y ademas cumple que \(I_{f}\subseteq \{0,1\}\). Por ejemplo

\(\displaystyle \begin{array}{rll} \omega \times \omega & \rightarrow & \omega \\ (x,y) & \rightarrow & \left\{ \begin{array}{l} 1\text{ si }x=y \\ 0\text{ si }x\neq y \end{array} \right. \end{array} \ \ \ \ \ \ \ \ \ \ \ \begin{array}{rll} \omega \times \Sigma ^{\ast } & \rightarrow & \omega \\ (x,\alpha ) & \rightarrow & \left\{ \begin{array}{l} 1\text{ si }x=\left\vert \alpha \right\vert \\ 0\text{ si }x\neq \left\vert \alpha \right\vert \end{array} \right. \end{array} \)

\subsubsection{Operaciones logicas entre predicados}

Dados predicados \(P:S\subseteq \omega ^{n}\times \Sigma ^{\ast m}\rightarrow \{0,1\}\) y \(Q:S\subseteq \omega ^{n}\times \Sigma ^{\ast m}\rightarrow \{0,1\}\), con el mismo dominio, definamos nuevos predicados \((P\vee Q)\), \( (P\wedge Q)\) y \(\lnot P\) de la siguiente manera

\(\displaystyle \begin{array}{rll} (P\vee Q):S & \rightarrow & \omega \\ (\vec{x},\vec{\alpha}) & \rightarrow & \left\{ \begin{array}{lll} 1 & & \text{si }P(\vec{x},\vec{\alpha})=1\text{ o }Q(\vec{x},\vec{\alpha})=1 \\ 0 & & \text{caso contrario} \end{array} \right. \end{array} \)

\(\displaystyle \begin{array}{rll} (P\wedge Q):S & \rightarrow & \omega \\ (\vec{x},\vec{\alpha}) & \rightarrow & \left\{ \begin{array}{lll} 1 & & \text{si }P(\vec{x},\vec{\alpha})=1\text{ y }Q(\vec{x},\vec{\alpha})=1 \\ 0 & & \text{caso contrario} \end{array} \right. \end{array} \)

\(\displaystyle \begin{array}{rll} \lnot P:S & \rightarrow & \omega \\ (\vec{x},\vec{\alpha}) & \rightarrow & \left\{ \begin{array}{lll} 1 & & \text{si }P(\vec{x},\vec{\alpha})=0 \\ 0 & & \text{si }P(\vec{x},\vec{\alpha})=1 \end{array} \right. \end{array} \)

\subsection{Conjuntos \(\Sigma \)-mixtos}

Un conjunto \(S\) es llamado \(\Sigma \)-mixto si \(S\subseteq \omega ^{n}\times \Sigma ^{\ast m}\), con \(n,m\geq 0\). Notese que \(S\) es \(\varnothing \) -mixto sii existe un \(n\geq 0\), tal que \(S\subseteq \omega ^{n}\).

Dado \(S\subseteq \omega ^{n}\times \Sigma ^{\ast m}\) usaremos \(\chi _{S}\) para denotar la funcion

\(\displaystyle \begin{array}{rcl} \chi _{S}:\omega ^{n}\times \Sigma ^{\ast m} & \rightarrow & \omega \\ (\vec{x},\vec{\alpha}) & \rightarrow & \left\{ \begin{array}{c} 1\text{ si }(\vec{x},\vec{\alpha})\in S \\ 0\text{ si }(\vec{x},\vec{\alpha})\notin S \end{array} \right. \end{array} \)

Llamaremos a \(\chi _{S}\) la funcion caracteristica de \(S\).
Un conjunto \(S\subseteq \omega ^{n}\times \Sigma ^{\ast m}\) es llamado rectangular si es de la forma

\(\displaystyle S_{1}\times ...\times S_{n}\times L_{1}\times ...\times L_{m}, \)

con \(n,m\geq 0\), cada \(S_{i}\subseteq \omega \) y cada \(L_{i}\subseteq \Sigma ^{\ast }\). El concepto de conjunto rectangular es muy importante en nuestro enfoque. Aunque en general no habra restricciones acerca del dominio de las funciones y predicados, nuestra filosofia sera tratar en lo posible que los dominios de las funciones que utilicemos para hacer nuestro analisis de recursividad de distintos paradigmas, sean rectangulares. Aunque en principio puede pareser que todos los conjuntos son rectangulares, el siguiente lema mostrara cuan ingenua es esta vision.
Lema 1 Sea \(S\subseteq \omega \times \Sigma ^{\ast }\). Entonces \(S\) es rectangular si y solo si se cumple la siguiente propiedad:
R Si \((x,\alpha ),(y,\beta )\in S\), entonces \((x,\beta )\in S\)
Prueba: Ejercicio. \(\Box\)

Supongamos \(\Sigma =\{\#,\& ,\%\}\). Notese que podemos usar el lema anterior para probar por ejemplo que los siguientes conjuntos no son rectangulares

- \(\{(0,\#\#),(1,\%\%\%)\}\)
- \(\{(x,\alpha ):\left\vert \alpha \right\vert =x\}\)
Dejamos como ejercicio para el lector enunciar un lema analogo al anterior pero que caracterice cuando \(S\subseteq \omega ^{2}\times \Sigma ^{\ast 3}\) es rectangular.

\subsection{Notacion lambda}

Usaremos la notacion lambda de Church en la forma que se explica a continuacion. Supongamos ya hemos fijado un alfabeto \(\Sigma \) y supongamos \( E\) es una expresion la cual involucra variables numericas y variables alfabeticas y la cual puede estar definida o no dependiendo de que valores concretos le damos a cada una de dichas variables. Supongamos tambien que cuando \(E\) esta definida, produce un valor ya sea numerico o alfabetico. Sea \(x_{1},...,x_{n},\alpha _{1},...,\alpha _{m}\) una lista de variables tal que las variables numericas que ocurren en \(E\) estan todas contenidas en la lista \(x_{1},...,x_{n}\) y que las alfabeticas lo estan en la lista \(\alpha _{1},...,\alpha _{m}.\) Entonces

\(\displaystyle \lambda x_{1}...x_{n}\alpha _{1}...\alpha _{m}\left[ E\right] \)

denotara la funcion definida por:
El dominio de \(\lambda x_{1}...x_{n}\alpha _{1}...\alpha _{m}\left[ E \right] \) es el conjunto de las \((n+m)\)-uplas \((k_{1},...,k_{n},\beta _{1},...,\beta _{m})\in \omega ^{n}\times \Sigma ^{\ast m}\) tales que \(E\) esta definida cuando le asignamos a cada \(x_{i}\) el valor \(k_{i}\) y a cada \( \alpha _{i}\) el valor \(\beta _{i}.\)
\(\lambda x_{1}...x_{n}\alpha _{1}...\alpha _{m}\left[ E\right] (k_{1},...,k_{n},\beta _{1},...,\beta _{m})=\) valor que asume \(E\) cuando le asignamos a cada \(x_{i}\) el valor \(k_{i}\) y a cada \(\alpha _{i}\) el valor \( \beta _{i}.\)
Ejemplos:

(a) \(\lambda \alpha \beta \left[ \alpha \beta \right] \) es la funcion
\(\displaystyle \begin{array}{rll} \Sigma ^{\ast }\times \Sigma ^{\ast } & \rightarrow & \Sigma ^{\ast } \\ (\alpha ,\beta ) & \rightarrow & \alpha \beta \end{array} \)

(b) \(\lambda \beta \alpha \left[ \alpha \beta \right] \) es la funcion
\(\displaystyle \begin{array}{rll} \Sigma ^{\ast }\times \Sigma ^{\ast } & \rightarrow & \Sigma ^{\ast } \\ (\alpha ,\beta ) & \rightarrow & \beta \alpha \end{array} \)

(c) \(\lambda xy\alpha \beta \left[ Pred(\left\vert \alpha \right\vert )+Pred(y)\right] \) es la funcion
\(\displaystyle \begin{array}{rll} \left\{ (x,y,\alpha ,\beta )\in \omega ^{2}\times \Sigma ^{\ast 2}:\left\vert \alpha \right\vert \geq 1\text{ y }y\geq 1\right\} & \rightarrow & \omega \\ (x,y,\alpha ,\beta ) & \rightarrow & Pred(\left\vert \alpha \right\vert )+Pred(y) \end{array} \)

(d) Si la expresion \(E\), cuando esta definida, toma valores Booleanos \(0\) o \(1\), entonces la funcion \(\lambda x_{1}...x_{n}\alpha _{1}...\alpha _{m}\left[ E\right] \) es un predicado. Por ejemplo \(\lambda xy\left[ x=y \right] \) es el predicado
\(\displaystyle \begin{array}{rll} \omega \times \omega & \rightarrow & \omega \\ (x,y) & \rightarrow & \left\{ \begin{array}{l} 1\text{ si }x=y \\ 0\text{ si }x\neq y \end{array} \right. \end{array} \)

y \(\lambda x\alpha \left[ Pred(x)=\left\vert \alpha \right\vert \right] \) es el predicado
\(\displaystyle \begin{array}{rll} \mathbf{N}\times \Sigma ^{\ast } & \rightarrow & \omega \\ (x,\alpha ) & \rightarrow & \left\{ \begin{array}{l} 1\text{ si }Pred(x)=\left\vert \alpha \right\vert \\ 0\text{ si }Pred(x)\neq \left\vert \alpha \right\vert \end{array} \right. \end{array} \)

(e) No toda variable de la lista \(x_{1},...,x_{n},\alpha _{1},...,\alpha _{m}\) debe ocurrir en \(E.\) Por ejemplo \(\lambda xy\alpha \left[ Pred(y)\right] \) es la funcion
\(\displaystyle \begin{array}{rll} \left\{ (x,y,\alpha )\in \omega ^{2}\times \Sigma ^{\ast }:y\geq 1\right\} & \rightarrow & \omega \\ (x,y,\alpha ) & \rightarrow & Pred(y) \end{array} \)

(f) Notar que para \(S\subseteq \omega ^{n}\times \Sigma ^{\ast m}\) se tiene que \(\chi _{S}=\lambda x_{1}...x_{n}\alpha _{1}...\alpha _{m}\left[ ( \vec{x},\vec{\alpha})\in S\right] .\)
Notar que la notacion \(\lambda \) depende del alfabeto \(\Sigma \) previamente fijado. Cuando haya varios alfabetos bajo consideracion, en general resultara claro del contexto con respecto a cual de ellos se usa la notacion \(\lambda \).
