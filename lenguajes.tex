\documentclass[12pt,a4paper]{report}
\usepackage[utf8]{inputenc}
\usepackage[spanish]{babel}
\usepackage{amsmath}
\usepackage{amsfonts}
\usepackage{amssymb}
\usepackage{lmodern}
\usepackage{amsmath}
\usepackage{enumerate}
\usepackage[left=2cm,right=2cm,top=2cm,bottom=2cm]{geometry}
\usepackage{graphicx}
\usepackage{mathtools}
\usepackage{stackrel}
\renewcommand{\theequation}{\arabic{equation}}
\newcounter{neq}
\providecommand{\abs}[1]{\lvert#1\rvert}
\newcommand{\QED}{\hfill \textit{\textbf{Q.E.D.}}}
\author{Agustín Curto, agucurto95@gmail.com \\
			 Francisco Nievas, frannievas@gmail.com}
\title{Resumen de teoremas para el final \\ de Lenguajes Formales y Computabilidad}
\date{2017}

\begin{document}
\maketitle
\tableofcontents

\section{Conjuntos $\Sigma$-mixtos}

  \textbf{\underline{Lemma 1:}} Sea $S\subseteq \omega \times \Sigma ^{\ast }$, entonces $S$ es rectangular si y
    solo si se cumple la siguiente propiedad:

    \[
      \textup{Si } (x, \alpha), (y, \beta) \in S \Rightarrow (x, \beta) \in S
    \]

  \textbf{\underline{Proof:}} Ejercicio.

  \QED

\section{Ordenes naturales sobre $\Sigma^{\ast}$}

  

\section{Codificacion de sucesiones infinitas de numeros}

  \textbf{\underline{Lemma 10:}} Si \(p,p_{1},...,p_{n}\) son numeros primos y \(p\) divide a \(p_{1}...p_{n}\), entonces \(p=p_{i}\), para algun \(i\).
Ahora la version del Teorema Fundamental de la Aritmetica.

  \textbf{\underline{Theorem 11:}} Para cada \(x\in \mathbf{N}\), hay una unica sucesion \((s_{1},s_{2},...)\in \omega ^{\left[ \mathbf{N}\right] }\) tal que
\(\displaystyle x=\underset{i=1}{\overset{\infty }{\Pi }}pr(i)^{s_{i}} \)
(Notese que \(\underset{i=1}{\overset{\infty }{\Pi }}pr(i)^{s_{i}}\) tiene sentido ya que es un producto que solo tiene una cantidad finita de factores no iguales a \(1\). )

\textbf{\underline{Proof:}} Primero probaremos la existencia por induccion en \(x\). Claramente \(1= \underset{i=1}{\overset{\infty }{\Pi }}pr(i)^{0}\), con lo cual el caso \(x=1\) esta probado. Supongamos la existencia vale para cada \(y\) menor que \(x\), veremos que entonces vale para \(x\). Si \(x\) es primo, entonces \(x=pr(i_{0})\) para algun \(i_{0}\) por lo cual tenemos que \(x=\underset{i=1}{\overset{\infty }{\Pi }}pr(i)^{s_{i}}\), tomando \(s_{i}=0\) si \(i\neq i_{0}\) y \(s_{i_{0}}=1\). Si \(x\) no es primo, entonces \(x=y_{1}.y_{2}\) con \(y_{1},y_{2}< x\). Por hipotesis inductiva tenemos que hay \((s_{1},s_{2},...),(t_{1},t_{2},...)\in \omega ^{\left[ \mathbf{N}\right] }\) tales que \(y_{1}=\underset{i=1}{\overset {\infty }{\Pi }}pr(i)^{s_{i}}\) y \(y_{2}=\underset{i=1}{\overset{\infty }{\Pi }}pr(i)^{t_{i}}\). Tenemos entonces que \(x=\underset{i=1}{\overset{\infty }{ \Pi }}pr(i)^{s_{i}+t_{i}}\) lo cual concluye la prueba de la existencia.

Veamos ahora la unicidad. Suponganos que

\(\displaystyle \underset{i=1}{\overset{\infty }{\Pi }}pr(i)^{s_{i}}=\underset{i=1}{\overset{ \infty }{\Pi }}pr(i)^{t_{i}} \)

Si \(s_{i} >t_{i}\) entonces dividiendo ambos miembros por \(pr(i)^{t_{i}}\) obtenemos que \(pr(i)\) divide a un producto de primos todos distintos de el, lo cual es absurdo por el lema anterior. Analogamente llegamos a un absurdo si suponemos que \(t_{i} >s_{i}\), lo cual nos dice que \(s_{i}=t_{i}\), para cada \(i\in \mathbf{N}\) \(\Box\)

  \textbf{\underline{Lemma 12:}} Las funciones
\(\displaystyle \begin{array}{lll} \mathbf{N} & \rightarrow & \omega ^{\left[ \mathbf{N}\right] } \\ x & \rightarrow & ((x)_{1},(x)_{2},...) \end{array} \ \ \ \ \ \ \ \ \ \ \ \ \ \ \ \ \ \ \begin{array}{rll} \omega ^{\left[ \mathbf{N}\right] } & \rightarrow & \mathbf{N} \\ (s_{1},s_{2},...) & \rightarrow & \left\langle s_{1},s_{2},...\right\rangle \end{array} \)
son biyecciones una inversa de la otra.
\textbf{\underline{Proof:}} Notese que para cada \(x\in \mathbf{N}\), tenemos que \(\left\langle (x)_{1},(x)_{2},...\right\rangle =x\). Ademas para cada \((s_{1},s_{2},...)\in \omega ^{\left[ \mathbf{N}\right] }\), tenemos que \(((\left\langle s_{1},s_{2},...\right\rangle )_{1},(\left\langle s_{1},s_{2},...\right\rangle )_{2},...)=(s_{1},s_{2},...)\). Es claro que lo anterior garantiza que los mapeos en cuestion son uno inversa del otro \(\Box\)

  \textbf{\underline{Lemma 13}} Para cada \(x\in \mathbf{N}\):
\(Lt(x)=0\) sii \(x=1\)
\(x=\prod\nolimits_{i=1}^{Lt(x)}pr(i)^{(x)_{i}}\)
Cabe destacar entonces que la funcion \(\lambda ix[(x)_{i}]\) tiene dominio igual a \(\mathbf{N}^{2}\) y la funcion \(\lambda ix[Lt(x)]\) tiene dominio igual a \(\mathbf{N}\).

\section{Conjuntos $\Sigma$-efectivamente enumerables}

  \textbf{\underline{Lemma 14:}} Sean \(S_{1},S_{2}\subseteq \omega ^{n}\times \Sigma ^{\ast m}\) conjuntos \( \Sigma \)-efectivamente enumerables. Entonces \(S_{1}\cup S_{2}\) y \(S_{1}\cap S_{2}\) son \(\Sigma \)-efectivamente enumerables.

  \textbf{\underline{Proof:}} El caso en el que alguno de los conjuntos es vacio es trivial. Supongamos que ambos conjuntos son no vacios y sean \(\mathbb{P}_{1}\) y \(\mathbb{P}_{2}\) procedimientos que enumeran a \(S_{1}\) y \(S_{2}\). El siguiente procedimiento enumera al conjunto \(S_{1}\cup S_{2}\):

- Si \(x\) es par realizar \(\mathbb{P}_{1}\) partiendo de \(x/2\) y dar el elemento de \(S_{1}\) obtenido como salida. Si \(x\) es impar realizar \(\mathbb{P }_{2}\) partiendo de \((x-1)/2\) y dar el elemento de \(S_{2}\) obtenido como salida.
Veamos ahora que \(S_{1}\cap S_{2}\) es \(\Sigma \)-efectivamente enumerable. Si \(S_{1}\cap S_{2}=\varnothing \) entonces no hay nada que probar. Supongamos entonces que \(S_{1}\cap S_{2}\) en no vacio. Sea \(z_{0}\) un elemento fijo de \( S_{1}\cap S_{2}.\) Sea \(\mathbb{P}\) un procedimiento efectivo el cual enumere a \(\omega \times \omega \) (ver el ejemplo de mas arriba). Un procedimiento que enumera a \(S_{1}\cap S_{2}\) es el siguiente

Etapa 1

Realizar \(\mathbb{P}\) con dato de entrada \(x\), para obtener un par \((x_{1},x_{2})\in \omega \times \omega \).

Etapa 2

Realizar \(\mathbb{P}_{1}\) con dato de entrada \(x_{1}\) para obtener un elemento \(z_{1}\in S_{1}\)

Etapa 3

Realizar \(\mathbb{P}_{2}\) con dato de entrada \(x_{2}\) para obtener un elemento \(z_{2}\in S_{2}\)

Etapa 4

Si \(z_{1}=z_{2}\), entonces dar como dato de salida \(z_{1}.\) En caso contrario dar como dato de salida \(z_{0}\). \(\Box\)

\section{Conjuntos $\Sigma$-efectivamente computables}

  \textbf{\underline{Lemma 15:}} Si \(S\subseteq \omega ^{n}\times \Sigma ^{\ast m}\) es \(\Sigma \) -efectivamente computable entonces \(S\) es \(\Sigma \)-efectivamente enumerable.

  \textbf{\underline{Proof:}} Supongamos \(S\neq \varnothing \). Sea \((\vec{z},\gamma )\in S\), fijo. Sea \( \mathbb{P}\) un procedimiento efectivo que compute a \(\chi _{S}\). Ya vimos en el ejemplo anterior que \(\omega ^{2}\times \Sigma ^{\ast 3}\) es \(\Sigma \) -efectivamente enumerable. En forma similar se puede ver que \(\omega ^{n}\times \Sigma ^{\ast m}\) lo es. Sea \(\mathbb{P}_{1}\) un procedimiento efectivo que enumere a \(\omega ^{n}\times \Sigma ^{\ast m}\). Entonces el siguiente procedimiento enumera a \(S\):

Etapa 1

Realizar \(\mathbb{P}_{1}\) con \(x\) de entrada para obtener \((\vec{x} ,\vec{\alpha})\in \omega ^{n}\times \Sigma ^{\ast m}.\)

Etapa 2

Realizar \(\mathbb{P}\) con \((\vec{x},\vec{\alpha})\) de entrada para obtener el valor Booleano \(e\) de salida\(.\)

Etapa 3

Si \(e=1\) dar como dato de salida \((\vec{x},\vec{\alpha}).\) Si \(e=0\) dar como dato de salida \((\vec{z},\gamma )\). \(\Box\)


  \textbf{\underline{Theorem 16:}} Sea \(S\subseteq \omega ^{n}\times \Sigma ^{\ast m}\). Son equivalentes
  (a) \(S\) es \(\Sigma \)-efectivamente computable
  (b) \(S\) y \((\omega ^{n}\times \Sigma ^{\ast m})-S\) son \(\Sigma \) -efectivamente enumerables

  \textbf{\underline{Proof:}} (a)\(\Rightarrow \)(b). Por el lema anterior tenemos que \(S\) es \(\Sigma \) -efectivamente enumerable. Notese ademas que, dado que \(S\) es \(\Sigma \) -efectivamente computable, \((\omega ^{n}\times \Sigma ^{\ast m})-S\) tambien lo es (por que?). Es decir que aplicando nuevamente el lema anterior tenemos que \((\omega ^{n}\times \Sigma ^{\ast m})-S\) es \(\Sigma \)-efectivamente enumerable.

(b)\(\Rightarrow \)(a). Sea \(\mathbb{P}_{1}\) un procedimiento efectivo que enumere a \(S\) y sea \(\mathbb{P}_{2}\) un procedimiento efectivo que enumere a \((\omega ^{n}\times \Sigma ^{\ast m})-S\). Es facil ver que el siguiente procedimiento computa el predicado \(\chi _{S}\):

Etapa 1

Darle a la variable \(T\) el valor \(0\).

Etapa 2

Realizar \(\mathbb{P}_{1}\) con el valor de \(T\) como entrada para obtener de salida la upla \((\vec{y},\vec{\beta})\).

Etapa 3

Realizar \(\mathbb{P}_{2}\) con el valor de \(T\) como entrada para obtener de salida la upla \((\vec{z},\vec{\gamma})\).

Etapa 4

Si \((\vec{y},\vec{\beta})=(\vec{x},\vec{\alpha})\), entonces detenerse y dar como dato de salida el valor \(1\). Si \((\vec{z},\vec{\gamma} )=(\vec{x},\vec{\alpha})\), entonces detenerse y dar como dato de salida el valor \(0.\) Si no suceden ninguna de las dos posibilidades antes mensionadas, aumentar en \(1\) el valor de la variable \(T\) y dirijirse a la Etapa 2. \(\Box\)

  \textbf{\underline{Theorem 17:}} Dado \(S\subseteq \omega ^{n}\times \Sigma ^{\ast m}\), son equivalentes
(1) \(S\) es \(\Sigma \)-efectivamente enumerable
(2) \(S=\varnothing \) o \(S=I_{F}\), para alguna \(F:\omega \rightarrow \omega ^{n}\times \Sigma ^{\ast m}\) tal que cada \(F_{i}\) es \(\Sigma \) -efectivamente computable.
(3) \(S=I_{F}\), para alguna \(F:D_{F}\subseteq \omega ^{k}\times \Sigma ^{\ast l}\rightarrow \omega ^{n}\times \Sigma ^{\ast m}\) tal que cada \(F_{i}\) es \(\Sigma \)-efectivamente computable.
(4) \(S=D_{f}\), para alguna funcion \(f\) la cual es \(\Sigma \) -efectivamente computable.

  \textbf{\underline{Proof:}} (1)\(\Rightarrow \)(2) y (2)\(\Rightarrow \)(1) son muy naturales y son dejadas al lector. (2)\(\Rightarrow \)(3) es trivial.

(3)\(\Rightarrow \)(4). Para \(i=1,...,n+m\), sea \(\mathbb{P}_{i}\) un procedimiento el cual computa a \(F_{i}\) y sea \(\mathbb{P}\) un procedimiento el cual enumere a \(\omega \times \omega ^{k}\times \Sigma ^{\ast l}.\) El siguiente procedimiento computa la funcion \(f:I_{F}\rightarrow \{1\}\):

Etapa 1

Darle a la variable \(T\) el valor 0.

Etapa 2

Hacer correr \(\mathbb{P}\) con dato de entrada \(T\) y obtener \( (t,z_{1},...,z_{k},\gamma _{1},...,\gamma _{l})\) como dato de salida.

Etapa 3

Para cada \(i=1,...,n+m\), hacer correr \(\mathbb{P}_{i}\) durante \(t\) pasos, con dato de entrada \((z_{1},...,z_{k},\gamma _{1},...,\gamma _{l}).\) Si cada procedimiento \(\mathbb{P}_{i}\) al cabo de los \(t\) pasos termino y dio como resultado el valor \(o_{i}\), entonces comparar \((\vec{x},\vec{\alpha} )\) con \((o_{1},...,o_{n+m})\) y en caso de que sean iguales detenerse y dar como dato de salida el valor \(1\). En el caso en que no son iguales, aumentar en \(1\) el valor de la variable \(T\) y dirijirse a la Etapa 2. Si algun procedimiento \(\mathbb{P}_{i}\) al cabo de los \(t\) pasos no termino, entonces aumentar en \(1\) el valor de la variable \(T\) y dirijirse a la Etapa 2.

(4)\(\Rightarrow \)(1). Supongamos \(S\neq \varnothing .\) Sea \((\vec{z},\vec{ \gamma})\) un elemento fijo de \(S.\) Sea \(\mathbb{P}\) un procedimiento el cual compute a \(f\). Sea \(\mathbb{P}_{1}\) un procedimiento el cual enumere a \( \omega \times \omega ^{n}\times \Sigma ^{\ast m}.\) Dejamos al lector el dise \~{n}o de un procedimiento efectivo el cual enumere \(D_{f}\). \(\Box\)

\section{Funciones $\Sigma$-recursivas primitivas}

  \textbf{\underline{Lemma 18:}} Si \(f,f_{1},...,f_{n+m}\) son \(\Sigma \)-efectivamente computables, entonces \( f\circ (f_{1},...,f_{n+m})\) lo es.

  \textbf{\underline{Proof:}} Sean \(\mathbb{P},\mathbb{P}_{1},...,\mathbb{P}_{n+m}\) procedimientos efectivos los cuales computen las funciones \(f,f_{1},...,f_{n+m}\), respectivamente. Usando estos procedimientos es facil definir un procedimiento efectivo el cual compute a \(f\circ (f_{1},...,f_{n+m})\). \(\Box\)

  \textbf{\underline{Lemma 19:}} Si \(f\) y \(g\) son \(\Sigma \)-efectivamente computables, entonces \(R(f,g)\) lo es.

  \textbf{\underline{Proof:}} La Proof es dejada al lector. \(\Box\)

  \textbf{\underline{Lemma 20:}} Si \(f\) y cada \(\mathcal{G}_{a}\) son \(\Sigma \)-efectivamente computables, entonces \(R(f,\mathcal{G})\) lo es.

  \textbf{\underline{Proof:}} Es dejada al lector con la recomendacion de que haga la Proof para el caso \( \Sigma =\{@,\& \}\) \(\Box\)

  \textbf{\underline{Theorem 21:}} Si \(f\in \mathrm{PR}^{\Sigma }\), entonces \(f\) es \(\Sigma \)-efectivamente computable.

  \textbf{\underline{Proof:}} Dejamos al lector la Proof por induccion en \(k\) de que si \(f\in \mathrm{PR} _{k}^{\Sigma }\), entonces \(f\) es \(\Sigma \)-efectivamente computable, la cual sale en forma directa usando los lemas anteriores que garantizan que los constructores de composicion y recursion primitiva preservan la computabilidad efectiva \(\Box\)

  \textbf{\underline{Lemma 22:}}
  (1) \(\varnothing \in \mathrm{PR}^{\varnothing }\).
  (2) \(\lambda xy\left[ x+y\right] \in \mathrm{PR}^{\varnothing }\).
  (3) \(\lambda xy\left[ x.y\right] \in \mathrm{PR}^{\varnothing }\).
  (4) \(\lambda x\left[ x!\right] \in \mathrm{PR}^{\varnothing }\).

  \textbf{\underline{Proof:}} (1) Notese que \(\varnothing =Pred\circ C_{0}^{0,0}\in \mathrm{PR} _{1}^{\varnothing }\)

  (2) Notar que

  \(\displaystyle \begin{array}{rcl} \lambda xy\left[ x+y\right] (0,x_{1}) & =& x_{1}=p_{1}^{1,0}(x_{1}) \\ \lambda xy\left[ x+y\right] (t+1,x_{1}) & =& \lambda xy\left[ x+y\right] (t,x_{1})+1 \\ & =& \left( Suc\circ p_{1}^{3,0}\right) \left( \lambda xy\left[ x+y\right] (t,x_{1}),t,x_{1}\right) \end{array} \)

  lo cual implica que \(\lambda xy\left[ x+y\right] =R\left( p_{1}^{1,0},Suc\circ p_{1}^{3,0}\right) \in \mathrm{PR}_{2}^{\varnothing }.\)
  (3) Primero note que

  \(\displaystyle \begin{array}{rcl} C_{0}^{1,0}(0) & =& C_{0}^{0,0}(\Diamond ) \\ C_{0}^{1,0}(t+1) & =& C_{0}^{1,0}(t) \end{array} \)

  lo cual implica que \(C_{0}^{1,0}=R\left( C_{0}^{0,0},p_{1}^{2,0}\right) \in \mathrm{PR}_{1}^{\varnothing }.\) Tambien note que
  \(\displaystyle \lambda tx\left[ t.x\right] =R\left( C_{0}^{1,0},\lambda xy\left[ x+y\right] \circ \left( p_{1}^{3,0},p_{3}^{3,0}\right) \right) , \)

  lo cual por (1) implica que \(\lambda tx\left[ t.x\right] \in \mathrm{PR} _{3}^{\varnothing }\).
  (4) Note que

  \(\displaystyle \begin{array}{rcl} \lambda x\left[ x!\right] (0) & =& 1=C_{1}^{0,0}(\Diamond ) \\ \lambda x\left[ x!\right] (t+1) & =& \lambda x\left[ x!\right] (t).(t+1), \end{array} \)

  lo cual implica que
  \(\displaystyle \lambda x\left[ x!\right] =R\left( C_{1}^{0,0},\lambda xy\left[ x.y\right] \circ \left( p_{1}^{2,0},Suc\circ p_{2}^{2,0}\right) \right) . \)

  Ya que \(C_{1}^{0,0}=\) \(Suc\circ C_{0}^{0,0}\), tenemos que \(C_{1}^{0,0}\in \mathrm{PR}_{1}^{\varnothing }\). Por (2), tenemos que
  \(\displaystyle \lambda xy\left[ x.y\right] \circ \left( p_{1}^{2,0},Suc\circ p_{2}^{2,0}\right) \in \mathrm{PR}_{4}^{\varnothing }, \)

  obteniendo que \(\lambda x\left[ x!\right] \in \mathrm{PR}_{5}^{\varnothing }\). \(\Box\)


  \textbf{\underline{Lemma 23:}} Supongamos \(\Sigma \) es no vacio.
  (a) \(\lambda \alpha \beta \left[ \alpha \beta \right] \in \mathrm{PR} ^{\Sigma }\)
  (b) \(\lambda \alpha \left[ \left\vert \alpha \right\vert \right] \in \mathrm{PR}^{\Sigma }\)

  \textbf{\underline{Proof:}} (a) Ya que

  \(\displaystyle \begin{array}{rcl} \lambda \alpha \beta \left[ \alpha \beta \right] (\alpha _{1},\varepsilon ) & =& \alpha _{1}=p_{1}^{0,1}(\alpha _{1}) \\ \lambda \alpha \beta \left[ \alpha \beta \right] (\alpha _{1},\alpha a) & =& d_{a}(\lambda \alpha \beta \left[ \alpha \beta \right] (\alpha _{1},\alpha )),a\in \Sigma \end{array} \)

  tenemos que \(\lambda \alpha \beta \left[ \alpha \beta \right] =R\left( p_{1}^{0,1},\mathcal{G}\right) \), donde \(\mathcal{G}_{a}=d_{a}\circ p_{3}^{0,3}\), para cada \(a\in \Sigma \).
  (b) Ya que

  \(\displaystyle \begin{array}{rcl} \lambda \alpha \left[ \left\vert \alpha \right\vert \right] (\varepsilon ) & =& 0=C_{0}^{0,0}(\Diamond ) \\ \lambda \alpha \left[ \left\vert \alpha \right\vert \right] (\alpha a) & =& \lambda \alpha \left[ \left\vert \alpha \right\vert \right] (\alpha )+1 \end{array} \)

  tenemos que \(\lambda \alpha \left[ \left\vert \alpha \right\vert \right] =R\left( C_{0}^{0,0},\mathcal{G}\right) \), donde \(\mathcal{G}_{a}=\) \( Suc\circ p_{1}^{1,1}\), para cada \(a\in \Sigma .\). \(\Box\)


  \textbf{\underline{Lemma 24:}}
  (a) \(C_{k}^{n,m},C_{\alpha }^{n,m}\in \mathrm{PR}^{\Sigma }\), para \( n,m,k\geq 0\), \(\alpha \in \Sigma ^{\ast }\).
  (b) \(C_{k}^{n,0}\in \mathrm{PR}^{\varnothing }\), para \(n,k\geq 0\).


  \textbf{\underline{Proof:}} (a) Note que \(C_{k+1}^{0,0}=\) \(Suc\circ C_{k}^{0,0}\), lo cual implica \( C_{k}^{0,0}\in \mathrm{PR}_{k}^{\Sigma }\), para \(k\geq 0\). Tambien note que \( C_{\alpha a}^{0,0}=d_{a}\circ C_{\alpha }^{0,0}\), lo cual dice que \( C_{\alpha }^{0,0}\in \mathrm{PR}^{\Sigma }\), para \(\alpha \in \Sigma ^{\ast } \). Para ver que \(C_{k}^{0,1}\in \mathrm{PR}^{\Sigma }\) notar que

  \(\displaystyle \begin{array}{rcl} C_{k}^{0,1}(\varepsilon ) & =& k=C_{k}^{0,0}(\Diamond ) \\ C_{k}^{0,1}(\alpha a) & =& C_{k}^{0,1}(\alpha )=p_{1}^{1,1}\left( C_{k}^{0,1}(\alpha ),\alpha \right) \end{array} \)

  lo cual implica que \(C_{k}^{0,1}=R\left( C_{k}^{0,0},\mathcal{G}\right) \), con \(\mathcal{G}_{a}=p_{1}^{1,1}\), \(a\in \Sigma \). En forma similar podemos ver que \(C_{k}^{1,0},C_{\alpha }^{1,0},C_{\alpha }^{0,1}\in \mathrm{PR} ^{\Sigma }\). Supongamos ahora que \(m >0\). Entonces
  \(\displaystyle \begin{array}{rcl} C_{k}^{n,m} & =& C_{k}^{0,1}\circ p_{n+1}^{n,m} \\ C_{\alpha }^{n,m} & =& C_{\alpha }^{0,1}\circ p_{n+1}^{n,m} \end{array} \)

  de lo cual obtenemos que \(C_{k}^{n,m},C_{\alpha }^{n,m}\in \mathrm{PR} ^{\Sigma }\). El caso \(n >0\) es similar
  (b) Use argumentos similares a los usados en la Proof de (a). \(\Box\)


  \textbf{\underline{Lemma 25:}}
  (a) \(\lambda xy\left[ x^{y}\right] \in \mathrm{PR}^{\varnothing }\).
  (b) \(\lambda t\alpha \left[ \alpha ^{t}\right] \in \mathrm{PR} ^{\Sigma }\).


  \textbf{\underline{Proof:}} (a) Note que

  \(\displaystyle \lambda tx\left[ x^{t}\right] =R\left( C_{1}^{1,0},\lambda xy\left[ x.y \right] \circ \left( p_{1}^{3,0},p_{3}^{3,0}\right) \right) \in \mathrm{PR} ^{\varnothing }. \)

  O sea que \(\lambda xy\left[ x^{y}\right] =\lambda tx\left[ x^{t}\right] \circ \left( p_{2}^{2,0},p_{1}^{2,0}\right) \in \mathrm{PR}^{\varnothing }\).
  (b) Note que

  \(\displaystyle \lambda t\alpha \left[ \alpha ^{t}\right] =R\left( C_{\varepsilon }^{0,1},\lambda \alpha \beta \left[ \alpha \beta \right] \circ \left( p_{3}^{1,2},p_{2}^{1,2}\right) \right) \in \mathrm{PR}^{\Sigma }. \)

  \(\Box\)


  \textbf{\underline{Lemma 26:}} Si \(< \) es un orden total estricto sobre un alfabeto no vacio \( \Sigma \), entonces \(s^{< }\), \(\#^{< }\) y \(\ast ^{< }\) pertenecen a \(\mathrm{PR} ^{\Sigma }\)

  \textbf{\underline{Proof:}} Supongamos \(\Sigma =\{a_{1},...,a_{k}\}\) y \(< \) dado por \(a_{1}< ...< a_{k}\). Ya que

  \(\displaystyle \begin{array}{rcl} s^{< }(\varepsilon ) & =& a_{1} \\ s^{< }(\alpha a_{i}) & =& \alpha a_{i+1}\text{, para }i< k \\ s^{< }(\alpha a_{k}) & =& s^{< }(\alpha )a_{1} \end{array} \)

  tenemos que \(s^{< }=R\left( C_{a_{1}}^{0,0},\mathcal{G}\right) \), donde \( \mathcal{G}_{a_{i}}=d_{a_{i+1}}\circ p_{1}^{0,2}\), para \(i=1,...,k-1\) y \( \mathcal{G}_{a_{k}}=d_{a_{1}}\circ p_{2}^{0,2}.\) O sea que \(s^{< }\in \mathrm{ PR}^{\Sigma }.\) Ya que
  \(\displaystyle \begin{array}{rcl} \ast ^{< }(0) & =& \varepsilon \\ \ast ^{< }(t+1) & =& s^{< }(\ast ^{< }(t)) \end{array} \)

  podemos ver que \(\ast ^{< }\in \mathrm{PR}^{\Sigma }.\) Ya que
  \(\displaystyle \begin{array}{rcl} \#^{< }(\varepsilon ) & =& 0 \\ \#^{< }(\alpha a_{i}) & =& \#^{< }(\alpha ).k+i\text{, para }i=1,...,k, \end{array} \)

  tenemos que \(\#^{< }=R\left( C_{0}^{0,0},\mathcal{G}\right) \), donde
  \(\displaystyle \mathcal{G}_{a_{i}}=\lambda xy\left[ x+y\right] \circ \left( \lambda xy\left[ x.y\right] \circ \left( p_{1}^{1,1},C_{k}^{1,1}\right) ,C_{i}^{1,1}\right) \text{, para }i=1,...,k\text{.} \)

  O sea que \(\#^{< }\in \mathrm{PR}^{\Sigma }\). \(\Box\)

  \textbf{\underline{Lemma 27:}}
  (a) \(\lambda xy\left[ x\dot{-}y\right] \in \mathrm{PR}^{\varnothing }.\)
  (b) \(\lambda xy\left[ \max (x,y)\right] \in \mathrm{PR}^{\varnothing }.\)
  (c) \(\lambda xy\left[ x=y\right] \in \mathrm{PR}^{\varnothing }.\)
  (d) \(\lambda xy\left[ x\leq y\right] \in \mathrm{PR}^{\varnothing }.\)
  (e) Si \(\Sigma \) es no vacio, entonces \(\lambda \alpha \beta \left[ \alpha =\beta \right] \in \mathrm{PR}^{\Sigma }\)


  \textbf{\underline{Proof:}} (a) Primero notar que \(\lambda x\left[ x\dot{-}1\right] =R\left( C_{0}^{0,0},p_{2}^{2,0}\right) \in \mathrm{PR}^{\varnothing }.\) Tambien note que

  \(\displaystyle \lambda tx\left[ x\dot{-}t\right] =R\left( p_{1}^{1,0},\lambda x\left[ x\dot{ -}1\right] \circ p_{1}^{3,0}\right) \in \mathrm{PR}^{\varnothing }. \)

  O sea que \(\lambda xy\left[ x\dot{-}y\right] =\lambda tx\left[ x\dot{-}t \right] \circ \left( p_{2}^{2,0},p_{1}^{2,0}\right) \in \mathrm{PR} ^{\varnothing }.\)
  (b) Note que \(\lambda xy\left[ \max (x,y)\right] =\lambda xy\left[ (x+(y\dot{ -}x)\right] .\)

  (c) Note que \(\lambda xy\left[ x=y\right] =\lambda xy\left[ 1\dot{-}((x\dot{- }y)+(y\dot{-}x))\right] .\)

  (d) Note que \(\lambda xy\left[ x\leq y\right] =\lambda xy\left[ 1\dot{-}(x \dot{-}y)\right] .\)

  (e) Sea \(< \) un orden total estricto sobre \(\Sigma .\) Ya que

  \(\displaystyle \alpha =\beta \text{ sii }\#^{< }(\alpha )=\#^{< }(\beta ) \)

  tenemos que
  \(\displaystyle \lambda \alpha \beta \left[ \alpha =\beta \right] =\lambda xy\left[ x=y \right] \circ \left( \#^{< }\circ p_{1}^{0,2},\#^{< }\circ p_{2}^{0,2}\right) . \)

  O sea que podemos aplicar (c) y Lema 28 implica que \(\chi _{S}\) es \( \Sigma \)-p.r.. \(\Box\)


  \textbf{\underline{Lemma 28:}}
    \textbf{Hacer}
  \textbf{\underline{Proof:}}

  \textbf{\underline{Lemma 29:}}
    \textbf{Hacer}

  \textbf{\underline{Proof:}}

  \textbf{\underline{Corollary 30:}}
    \textbf{Hacer}

  \textbf{\underline{Proof:}}

  %% =========== FALTA UNA PARTE ===========%%

  \textbf{\underline{Lemma 31:}} Supongamos \(S_{1},...,S_{n}\subseteq \omega \), \( L_{1},...,L_{m}\subseteq \Sigma ^{\ast }\) son conjuntos no vacios. Entonces \( S_{1}\times ...\times S_{n}\times L_{1}\times ...\times L_{m}\) es \(\Sigma \) -p.r. sii \(S_{1},...,S_{n},L_{1},...,L_{m}\) son \(\Sigma \)-p.r.

  \textbf{\underline{Proof:}} (\(\Rightarrow \)) Veremos por ejemplo que \(L_{1}\) es \(\Sigma \)-p.r.. Sea \( (z_{1},...,z_{n},\zeta _{1},...,\zeta _{m})\) un elemento fijo de \( S_{1}\times ...\times S_{n}\times L_{1}\times ...\times L_{m}.\) Note que

  \(\displaystyle \alpha \in L_{1}\text{ sii }(z_{1},...,z_{n},\alpha ,\zeta _{2},...,\zeta _{m})\in S_{1}\times ...\times S_{n}\times L_{1}\times ...\times L_{m}, \)

  lo cual implica que
  \(\displaystyle \chi _{L_{1}}=\chi _{S_{1}\times ...\times S_{n}\times L_{1}\times ...\times L_{m}}\circ \left( C_{z_{1}}^{0,1},...,C_{z_{n}}^{0,1},p_{1}^{0,1},C_{\zeta _{2}}^{0,1},...,C_{\zeta _{m}}^{0,1}\right) . \)

  (\(\Leftarrow \)) Note que \(\chi _{S_{1}\times ...\times S_{n}\times L_{1}\times ...\times L_{m}}\) es el predicado
  \(\displaystyle \left( \chi _{S_{1}}\circ p_{1}^{n,m}\wedge ...\wedge \chi _{S_{n}}\circ p_{n}^{n,m}\wedge \chi _{L_{1}}\circ p_{n+1}^{n,m}\wedge ...\wedge \chi _{L_{m}}\circ p_{n+m}^{n,m}\right) . \)

  \(\Box\)


  \textbf{\underline{Lemma 32:}} Supongamos \(f:D_{f}\subseteq \omega ^{n}\times \Sigma ^{\ast m}\rightarrow O\) es \(\Sigma \)-p.r., donde \(O\in \{\omega ,\Sigma ^{\ast }\}.\) Si \(S\subseteq D_{f}\) es \(\Sigma \)-p.r., entonces \(f\mid _{S}\) es \(\Sigma \)-p.r..

  \textbf{\underline{Proof:}} Supongamos \(O=\Sigma ^{\ast }\). Entonces

  \(\displaystyle f\mid _{S}=\lambda x\alpha \left[ \alpha ^{x}\right] \circ \left( Suc\circ Pred\circ \chi _{S},f\right) \)

  es \(\Sigma \)-p.r.. El caso \(O=\omega \) es similar usando \(\lambda xy\left[ x^{y}\right] \) en lugar de \(\lambda x\alpha \left[ \alpha ^{x}\right] \). \(\Box\)


  \textbf{\underline{Lemma 33:}} Si \(f:D_{f}\subseteq \omega ^{n}\times \Sigma ^{\ast m}\rightarrow O\) es \(\Sigma \)-p.r., entonces existe una funcion \(\Sigma \) -p.r. \(\bar{f}:\omega ^{n}\times \Sigma ^{\ast m}\rightarrow O\), tal que \(f= \bar{f}\mid _{D_{f}}\).

  \textbf{\underline{Proof:}} Es facil ver por induccion en \(k\) que el enunciado se cumple para cada \(f\in \mathrm{PR}_{k}^{\Sigma }\) \(\Box\)


  \textbf{\underline{Proposition 34:}} Un conjunto \(S\) es \(\Sigma \)-p.r. sii \(S\) es el dominio de una funcion \(\Sigma \)-p.r.\(.\)

  \textbf{\underline{Proof:}} (\(\Rightarrow \)) Note que \(S=D_{Pred\circ \chi _{S}}.\)

  (\(\Leftarrow \)) Probaremos por induccion en \(k\) que \(D_{F}\) es \(\Sigma \) -p.r., para cada \(F\in \mathrm{PR}_{k}^{\Sigma }.\) El caso \(k=0\) es facil\(.\) Supongamos el resultado vale para un \(k\) fijo y supongamos \(F\in \mathrm{PR} _{k+1}^{\Sigma }.\) Veremos entonces que \(D_{F}\) es \(\Sigma \)-p.r.. Hay varios casos. Consideremos primero el caso en que \(F=R(f,g)\), donde

  \(\displaystyle \begin{array}{rcl} f & :& S_{1}\times ...\times S_{n}\times L_{1}\times ...\times L_{m}\rightarrow \Sigma ^{\ast } \\ g & :& \omega \times S_{1}\times ...\times S_{n}\times L_{1}\times ...\times L_{m}\times \Sigma ^{\ast }\rightarrow \Sigma ^{\ast }, \end{array} \)

  con \(S_{1},...,S_{n}\subseteq \omega \) y \(L_{1},...,L_{m}\subseteq \Sigma ^{\ast }\) conjuntos no vacios y \(f,g\in \mathrm{PR}_{k}^{\Sigma }\). Notese que por definicion de \(R(f,g)\), tenemos que
  \(\displaystyle D_{F}=\omega \times S_{1}\times ...\times S_{n}\times L_{1}\times ...\times L_{m}. \)

  Por hipotesis inductiva tenemos que \(D_{f}=S_{1}\times ...\times S_{n}\times L_{1}\times ...\times L_{m}\) es \(\Sigma \)-p.r., lo cual por el Lema 31 nos dice que los conjuntos \(S_{1},...,S_{n}\), \( L_{1},...,L_{m}\) son \(\Sigma \)-p.r.. Ya que \(\omega \) es \(\Sigma \)-p.r., el Lema 31 nos dice que \(D_{F}\) es \(\Sigma \)-p.r..
  Los otros casos de recursion primitiva son dejados al lector.

  Supongamos ahora que \(F=g\circ (g_{1},...,g_{n+m})\), donde

  \(\displaystyle \begin{array}{rcl} g & :& D_{g}\subseteq \omega ^{n}\times \Sigma ^{\ast m}\rightarrow O \\ g_{i} & :& D_{g_{i}}\subseteq \omega ^{k}\times \Sigma ^{\ast l}\rightarrow \omega \text{, }i=1,...,n \\ g_{i} & :& D_{g_{i}}\subseteq \omega ^{k}\times \Sigma ^{\ast l}\rightarrow \Sigma ^{\ast },i=n+1,...,n+m \end{array} \)

  estan en \(\mathrm{PR}_{k}^{\Sigma }.\) Por Lema 33, hay funciones \(\Sigma \)-p.r. \(\bar{g}_{1},...,\bar{g}_{n+m}\) las cuales son \( \Sigma \)-totales y cumplen
  \(\displaystyle g_{i}=\bar{g}_{i}\mid _{D_{g_{i}}}\text{, para }i=1,...,n+m. \)

  Por hipotesis inductiva los conjuntos \(D_{g}\), \(D_{g_{i}}\), \(i=1,...,n+m\), son \(\Sigma \)-p.r. y por lo tanto
  \(\displaystyle S=\bigcap_{i=1}^{n+m}D_{g_{i}} \)

  lo es. Notese que
  \(\displaystyle \chi _{D_{F}}=(\chi _{D_{g}}\circ \left( \bar{g}_{1},...,\bar{g} _{n+m}\right) \wedge \chi _{S}) \)

  lo cual nos dice que \(D_{F}\) es \(\Sigma \)-p.r.. \(\Box\)


  \textbf{\underline{Lemma 35:}} Supongamos \(f_{i}:D_{f_{i}}\subseteq \omega ^{n}\times \Sigma ^{\ast m}\rightarrow O\), \(i=1,...,k\), son funciones \(\Sigma \)-p.r. tales que \(D_{f_{i}}\cap D_{f_{j}}=\varnothing \) para \(i\neq j.\) Entonces \(f_{1}\cup ...\cup f_{k}\) es \(\Sigma \)-p.r..

  \textbf{\underline{Proof:}} Supongamos \(O=\Sigma ^{\ast }\) y \(k=2.\) Sean

  \(\displaystyle \bar{f}_{i}:\omega ^{n}\times \Sigma ^{\ast m}\rightarrow \Sigma ^{\ast },i=1,2, \)

  funciones \(\Sigma \)-p.r. tales que \(\bar{f}_{i}\mid _{D_{f_{i}}}=f_{i}\), \( i=1,2\) (Lema 33)\(.\) Por Lema 34 los conjuntos \(D_{f_{1}}\) y \(D_{f_{2}}\) son \(\Sigma \)-p.r. y por lo tanto lo es \( D_{f_{1}}\cup D_{f_{2}}\). Ya que
  \(\displaystyle f_{1}\cup f_{2}=\left( \lambda \alpha \beta \left[ \alpha \beta \right] \circ (\lambda x\alpha \left[ \alpha ^{x}\right] \circ (\chi _{D_{f_{1}}}, \bar{f}_{1}),\lambda x\alpha \left[ \alpha ^{x}\right] \circ (\chi _{D_{f_{2}}},\bar{f}_{2}))\right) \mid _{D_{f_{1}}\cup D_{f_{2}}} \)

  tenemos que \(f_{1}\cup f_{2}\) es \(\Sigma \)-p.r..
  El caso \(k >2\) puede probarse por induccion ya que

  \(\displaystyle f_{1}\cup ...\cup f_{k}=(f_{1}\cup ...\cup f_{k-1})\cup f_{k}. \)

  \(\Box\)


  \textbf{\underline{Corollary 36:}} Supongamos \(f\) es una funcion \(\Sigma \)-mixta cuyo dominio es finito. Entonces \(f\) es \(\Sigma \)-p.r..

  \textbf{\underline{Proof:}} Supongamos \(f:D_{f}\subseteq \omega ^{n}\times \Sigma ^{\ast m}\rightarrow O\) , con \(D_{f}=\{e_{1},...,e_{k}\}\). Por el Corolario 30, cada \( \{e_{i}\}\) es \(\Sigma \)-p.r. por lo cual el Lema 32 nos dice que \(C_{f(e_{i})}^{n,m}\mid _{\{e_{1}\}}\) es \(\Sigma \)-p.r.. O sea que

  \(\displaystyle f=C_{f(e_{1})}^{n,m}\mid _{\{e_{1}\}}\cup ...\cup C_{f(e_{k})}^{n,m}\mid _{\{e_{k}\}} \)

  es \(\Sigma \)-p.r.. \(\Box\)


  \textbf{\underline{Lemma 37:}} \(\lambda i\alpha \left[ \lbrack \alpha ]_{i}\right] \) es \(\Sigma \)-p.r..

  \textbf{\underline{Proof:}} Note que

  \(\displaystyle \begin{array}{rcl} \lbrack \varepsilon ]_{i} & =& \varepsilon \\ \lbrack \alpha a]_{i} & =& \left\{ \begin{array}{lll} \lbrack \alpha ]_{i} & & \text{si }i\neq \left\vert \alpha \right\vert +1 \\ a & & \text{si }i=\left\vert \alpha \right\vert +1 \end{array} \right. \end{array} \)

  lo cual dice que \(\lambda i\alpha \left[ \lbrack \alpha ]_{i}\right] =R\left( C_{\varepsilon }^{1,0},\mathcal{G}\right) \), donde \(\mathcal{G} _{a}:\omega \times \Sigma ^{\ast }\times \Sigma ^{\ast }\rightarrow \Sigma ^{\ast }\) es dada por
  \(\displaystyle \mathcal{G}_{a}(i,\alpha ,\zeta )=\left\{ \begin{array}{lll} \zeta & & \text{si }i\neq \left\vert \alpha \right\vert +1 \\ a & & \text{si }i=\left\vert \alpha \right\vert +1 \end{array} \right. \)

  O sea que solo resta probar que cada \(\mathcal{G}_{a}\) es \(\Sigma \)-p.r.. Primero note que los conjuntos
  \(\displaystyle \begin{array}{rcl} S_{1} & =& \left\{ (i,\alpha ,\zeta )\in \omega \times \Sigma ^{\ast }\times \Sigma ^{\ast }:i\neq \left\vert \alpha \right\vert +1\right\} \\ S_{2} & =& \left\{ (i,\alpha ,\zeta )\in \omega \times \Sigma ^{\ast }\times \Sigma ^{\ast }:i=\left\vert \alpha \right\vert +1\right\} \end{array} \)

  son \(\Sigma \)-p.r. ya que
  \(\displaystyle \begin{array}{rcl} \chi _{S_{1}} & =& \lambda xy\left[ x\neq y\right] \circ \left( p_{1}^{1,2},Suc\circ \lambda \alpha \left[ \left\vert \alpha \right\vert \right] \circ p_{2}^{1,2}\right) \\ \chi _{S_{2}} & =& \lambda xy\left[ x=y\right] \circ \left( p_{1}^{1,2},Suc\circ \lambda \alpha \left[ \left\vert \alpha \right\vert \right] \circ p_{2}^{1,2}\right) . \end{array} \)

  Ya que
  \(\displaystyle \mathcal{G}_{a}=p_{3}^{1,2}\mid _{S_{1}}\cup C_{a}^{1,2}\mid _{S_{2}}, \)

  el Lema 35 nos dice que \(\mathcal{G}_{a}\) es \(\Sigma \)-p.r., para cada \(a\in \Sigma \). \(\Box\)


  \textbf{\underline{Lemma 38:}} Sean \(n,m\geq 0\).
  (a) Si \(f:\omega \times S_{1}\times ...\times S_{n}\times L_{1}\times ...\times L_{m}\rightarrow \omega \) es \(\Sigma \)-p.r., con \( S_{1},...,S_{n}\subseteq \omega \) y \(L_{1},...,L_{m}\subseteq \Sigma ^{\ast } \) no vacios, entonces lo son las funciones \(\lambda xy\vec{x}\vec{\alpha} \left[ \sum_{t=x}^{t=y}f(t,\vec{x},\vec{\alpha})\right] \) y \(\lambda xy\vec{x }\vec{\alpha}\left[ \prod_{t=x}^{t=y}f(t,\vec{x},\vec{\alpha})\right] \).
  (b) Si \(f:\omega \times S_{1}\times ...\times S_{n}\times L_{1}\times ...\times L_{m}\rightarrow \Sigma ^{\ast }\) es \(\Sigma \)-p.r., con \( S_{1},...,S_{n}\subseteq \omega \) y \(L_{1},...,L_{m}\subseteq \Sigma ^{\ast } \) no vacios, entonces lo es la funcion \(\lambda xy\vec{x}\vec{\alpha}\left[ \subset _{t=x}^{t=y}f(t,\vec{x},\vec{\alpha})\right] \)

  \textbf{\underline{Proof:}} (a) Sea \(G=\lambda tx\vec{x}\vec{\alpha}\left[ \sum_{i=x}^{i=t}f(i,\vec{x}, \vec{\alpha})\right] \). Ya que

  \(\displaystyle \lambda xy\vec{x}\vec{\alpha}\left[ \sum_{i=x}^{i=y}f(i,\vec{x},\vec{\alpha}) \right] =G\circ \left( p_{2}^{n+2,m},p_{1}^{n+2,m},p_{3}^{n+2,m},...,p_{n+m+2}^{n+2,m}\right) \)

  solo tenemos que probar que \(G\) es \(\Sigma \)-p.r.. Primero note que
  \(\displaystyle \begin{array}{rcl} G(0,x,\vec{x},\vec{\alpha}) & =& \left\{ \begin{array}{lll} 0 & & \text{si }x >0 \\ f(0,\vec{x},\vec{\alpha}) & & \text{si }x=0 \end{array} \right. \\ G(t+1,x,\vec{x},\vec{\alpha}) & =& \left\{ \begin{array}{lll} 0 & & \text{si }x >t+1 \\ G(t,x,\vec{x},\vec{\alpha})+f(t+1,\vec{x},\vec{\alpha}) & & \text{si }x\leq t+1 \end{array} \right. \end{array} \)

  Sean
  \(\displaystyle \begin{array}{rcl} D_{1} & =& \left\{ (x,\vec{x},\vec{\alpha})\in \omega \times S_{1}\times ...\times S_{n}\times L_{1}\times ...\times L_{m}:x >0\right\} \\ D_{2} & =& \left\{ (x,\vec{x},\vec{\alpha})\in \omega \times S_{1}\times ...\times S_{n}\times L_{1}\times ...\times L_{m}:x=0\right\} \\ H_{1} & =& \left\{ (z,t,x,\vec{x},\vec{\alpha})\in \omega ^{3}\times S_{1}\times ...\times S_{n}\times L_{1}\times ...\times L_{m}:x >t+1\right\} \\ H_{2} & =& \left\{ (z,t,x,\vec{x},\vec{\alpha})\in \omega ^{3}\times S_{1}\times ...\times S_{n}\times L_{1}\times ...\times L_{m}:x\leq t+1\right\} . \end{array} \)

  Es facil de chequear que estos conjuntos son \(\Sigma \)-p.r.. Veamos que por ejemplo \(H_{1}\) lo es. Es decir debemos ver que \(\chi _{H_{1}}\) es \(\Sigma \) -p.r.. Ya que \(f\) es \(\Sigma \)-p.r. tenemos que \(D_{f}=\omega \times S_{1}\times ...\times S_{n}\times L_{1}\times ...\times L_{m}\) es \(\Sigma \) -p.r., lo cual por el Lema 31 nos dice que los conjuntos \( S_{1},...,S_{n}\), \(L_{1},...,L_{m}\) son \(\Sigma \)-p.r.. Ya que \(\omega \) es \( \Sigma \)-p.r., el Lema 31 nos dice que \(R=\omega ^{3}\times S_{1}\times ...\times S_{n}\times L_{1}\times ...\times L_{m}\) es \(\Sigma \)-p.r.. Notese que \(\chi _{H_{1}}=(\chi _{R}\wedge \lambda ztx\vec{x} \vec{\alpha}\left[ x >t+1\right] )\) por cual \(\chi _{H_{1}}\) es \(\Sigma \) -p.r. ya que es la conjuncion de dos predicados \(\Sigma \)-p.r.
  Ademas note que \(G=R(h,g)\), donde

  \(\displaystyle \begin{array}{rcl} h & =& C_{0}^{n+1,m}\mid _{D_{1}}\cup \lambda x\vec{x}\vec{\alpha}\left[ f(0, \vec{x},\vec{\alpha})\right] \mid _{D_{2}} \\ g & =& C_{0}^{n+3,m}\mid _{H_{1}}\cup \lambda ztx\vec{x}\vec{\alpha}\left[ z+f(t+1,\vec{x},\vec{\alpha})\right] )\mid _{H_{2}} \end{array} \)

  O sea que los Lemas 35 y 32 garantizan que \(G\) es \( \Sigma \)-p.r.. \(\Box\)


  \textbf{\underline{Lemma 39:}} Sean \(n,m\geq 0\).
  (a) Sea \(P:S\times S_{1}\times ...\times S_{n}\times L_{1}\times ...\times L_{m}\rightarrow \omega \) un predicado \(\Sigma \)-p.r. y supongamos \(\bar{S}\subseteq S\) es \(\Sigma \)-p.r.. Entonces \(\lambda x\vec{x}\vec{\alpha }\left[ (\forall t\in \bar{S})_{t\leq x}\;P(t,\vec{x},\vec{\alpha})\right] \) y \(\lambda x\vec{x}\vec{\alpha}\left[ (\exists t\in \bar{S})_{t\leq x}\;P(t, \vec{x},\vec{\alpha})\right] \) son predicados \(\Sigma \)-p.r.. (Note que el dominio de estos predicados es \(\omega \times S_{1}\times ...\times S_{n}\times L_{1}\times ...\times L_{m}\))
  (b) Sea \(P:S_{1}\times ...\times S_{n}\times L_{1}\times ...\times L_{m}\times L\rightarrow \omega \) un predicado \(\Sigma \)-p.r. y supongamos \( \bar{L}\subseteq L\) es \(\Sigma \)-p.r.. Entonces \(\lambda x\vec{x}\vec{\alpha} \left[ (\forall \alpha \in \bar{L})_{\left\vert \alpha \right\vert \leq x}\;P(\vec{x},\vec{\alpha},\alpha )\right] \) y \(\lambda x\vec{x}\vec{\alpha} \left[ (\exists \alpha \in \bar{L})_{\left\vert \alpha \right\vert \leq x}\;P(\vec{x},\vec{\alpha},\alpha )\right] \) son predicados \(\Sigma \)-p.r..

  \textbf{\underline{Proof:}} (a) Sea

  \(\displaystyle \bar{P}=P\mid _{\bar{S}\times S_{1}\times ...\times S_{n}\times L_{1}\times ...\times L_{m}}\cup C_{1}^{1+n,m}\mid _{(\omega -\bar{S})\times S_{1}\times ...\times S_{n}\times L_{1}\times ...\times L_{m}} \)

  Notese que \(\bar{P}\) es \(\Sigma \)-p.r.. Ya que
  \(\displaystyle \begin{array}{rcl} \lambda x\vec{x}\vec{\alpha}\left[ (\forall t\in \bar{S})_{t\leq x}P(t,\vec{x },\vec{\alpha})\right] & =& \lambda x\vec{x}\vec{\alpha}\left[ \prod\limits_{t=0}^{t=x}\bar{P}(t,\vec{x},\vec{\alpha})\right] \\ & =& \lambda xy\vec{x}\vec{\alpha}\left[ \prod\limits_{t=x}^{t=y}\bar{P}(t, \vec{x},\vec{\alpha})\right] \circ \left( C_{0}^{1+n,m},p_{1}^{1+n,m},...,p_{1+n+m}^{1+n,m}\right) \end{array} \)

  el Lema 38 implica que \(\lambda x\vec{x}\vec{\alpha}\left[ (\forall t\in \bar{S})_{t\leq x}\;P(t,\vec{x},\vec{\alpha})\right] \) es \( \Sigma \)-p.r..
  Finalmente note que

  \(\displaystyle \lambda x\vec{x}\vec{\alpha}\left[ (\exists t\in \bar{S})_{t\leq x}\;P(t, \vec{x},\vec{\alpha})\right] =\lnot \lambda x\vec{x}\vec{\alpha}\left[ (\forall t\in \bar{S})_{t\leq x}\;\lnot P(t,\vec{x},\vec{\alpha})\right] \)

  es \(\Sigma \)-p.r..
  (b) Sea \(< \) un orden total estricto sobre \(\Sigma .\) Sea \(k\) el cardinal de \( \Sigma \). Ya que

  \(\displaystyle \left\vert \alpha \right\vert \leq x\text{ sii }\#^{< }(\alpha )\leq \sum_{\iota =1}^{i=x}k^{i}, \)

  (ejercicio) tenemos que
  \(\displaystyle \lambda x\vec{x}\vec{\alpha}\left[ (\forall \alpha \in \bar{L})_{\left\vert \alpha \right\vert \leq x}P(\vec{x},\vec{\alpha},\alpha )\right] =\lambda x \vec{x}\vec{\alpha}\left[ (\forall t\in \#^{< }(\bar{L}))_{t\leq \sum_{\iota =1}^{i=x}k^{i}}P(\vec{x},\vec{\alpha},\ast ^{< }(t))\right] \)

  Sea \(H=\lambda t\vec{x}\vec{\alpha}\left[ P(\vec{x},\vec{\alpha},\ast ^{< }(t))\right] .\) Notese que \(H\) es \(\Sigma \)-p.r. y
  \(\displaystyle D_{H}=\#^{< }(L)\times S_{1}\times ...\times S_{n}\times L_{1}\times ...\times L_{m} \)

  Ademas note que \(\#^{< }(\bar{L})\) es \(\Sigma \)-p.r. (ejercicio), lo cual por (a) implica que
  \(\displaystyle Q=\lambda x\vec{x}\vec{\alpha}\left[ (\forall t\in \#^{< }(\bar{L}))_{t\leq x}H(t,\vec{x},\vec{\alpha})\right] \)

  es \(\Sigma \)-p.r.. O sea que
  \(\displaystyle \lambda x\vec{x}\vec{\alpha}\left[ (\forall \alpha \in \bar{L})_{\left\vert \alpha \right\vert \leq x}\;P(\vec{x},\vec{\alpha},\alpha )\right] =Q\circ \left( \lambda x\vec{x}\vec{\alpha}\left[ \sum\limits_{\iota =1}^{i=x}k^{i} \right] ,p_{1}^{1+n,m},...,p_{1+n+m}^{1+n,m}\right) \)

  es \(\Sigma \)-p.r.. \(\Box\)


  \textbf{\underline{Lemma 40:}}
  (a) El predicado \(\lambda xy\left[ x\text{ divide }y\right] \) es \( \varnothing \)-p.r..
  (b) El predicado \(\lambda x\left[ x\text{ es primo}\right] \) es \( \varnothing \)-p.r..
  (c) El predicado \(\lambda \alpha \beta \left[ \alpha \text{\ }\mathrm{ inicial}\ \beta \right] \) es \(\Sigma \)-p.r..

  \textbf{\underline{Proof:}} (a) Si tomamos \(P=\lambda tx_{1}x_{2}\left[ x_{2}=t.x_{1}\right] \in \mathrm{PR}^{\varnothing }\), tenemos que

  \(\displaystyle \begin{array}{rcl} \lambda x_{1}x_{2}\left[ x_{1}\text{ divide }x_{2}\right] & =& \lambda x_{1}x_{2}\left[ (\exists t\in \omega )_{t\leq x_{2}}\;P(t,x_{1},x_{2}) \right] \\ & =& \lambda xx_{1}x_{2}\left[ (\exists t\in \omega )_{t\leq x}\;P(t,x_{1},x_{2})\right] \circ \left( p_{2}^{2,0},p_{1}^{2,0},p_{2}^{2,0}\right) \end{array} \)

  por lo que podemos aplicar el lema anterior.
  (b) Ya que

  \(\displaystyle x\text{ es primo sii }x >1\wedge \left( (\forall t\in \omega )_{t\leq x}\;t=1\vee t=x\vee \lnot (t\text{ divide }x)\right) \)

  podemos usar un argumento similar al de la prueba de (a).
  (c) es dejado al lector. \(\Box\)

\section{Minimización y funciones $\Sigma$-recursivas}

\textbf{\underline{Lemma 41:}} Si \(P:D_{P}\subseteq \omega \times \omega ^{n}\times \Sigma ^{\ast m}\rightarrow \omega \) es un predicado \(\Sigma \)-efectivamente computable y \( D_{P}\) es \(\Sigma \)-efectivamente computable, entonces la funcion \(M(P)\) es \( \Sigma \)-efectivamente computable.

\textbf{\underline{Proof:}} Ejercicio \(\Box\)

\textbf{\underline{Theorem 42:}} Si \(f\in \mathrm{R} ^{\Sigma }\), entonces \(f\) es \(\Sigma \)-efectivamente computable.

\textbf{\underline{Proof:}} Dejamos al lector la prueba por induccion en \(k\) de que si \(f\in \mathrm{R} _{k}^{\Sigma }\), entonces \(f\) es \(\Sigma \)-efectivamente computable. \(\Box\)

\textbf{\underline{Lemma 43:}} Sean \(n,m\geq 0\). Sea \(P:D_{P}\subseteq \omega \times \omega ^{n}\times \Sigma ^{\ast m}\rightarrow \omega \) un predicado \(\Sigma \) -p.r.. Entonces
(a) \(M(P)\) es \(\Sigma \)-recursiva.
(b) Si hay una funcion \(\Sigma \)-p.r. \(f:\omega ^{n}\times \Sigma ^{\ast m}\rightarrow \omega \) tal que
\(\displaystyle M(P)(\vec{x},\vec{\alpha})=\min_{t}P(t,\vec{x},\vec{\alpha})\leq f(\vec{x}, \vec{\alpha})\text{, para cada }(\vec{x},\vec{\alpha})\in D_{M(P)}\text{,} \)
entonces \(M(P)\) es \(\Sigma \)-p.r..

\textbf{\underline{Proof:}} (a) Sea \(\bar{P}=P\mid _{D_{P}}\cup C_{0}^{n+1,m}\mid _{(\omega ^{n+1}\times \Sigma ^{\ast m})-D_{P}}\). Dejamos al lector verificar cuidadosamente que \( M(P)=M(\bar{P})\). Veremos entonces que \(M(\bar{P})\) es \(\Sigma \)-recursiva. Note que \(\bar{P}\) es \(\Sigma \)-p.r. (por que?). Sea \(k\) tal que \(\bar{P}\in \mathrm{PR}_{k}^{\Sigma }\). Ya que \(\bar{P}\) es \(\Sigma \)-total y \(\bar{P} \in \mathrm{PR}_{k}^{\Sigma }\subseteq \mathrm{R}_{k}^{\Sigma }\), tenemos que \(M(\bar{P})\in \mathrm{R}_{k+1}^{\Sigma }\) y por lo tanto \(M(\bar{P})\in \mathrm{R}^{\Sigma }\).

(b) Primero veremos que \(D_{M(\bar{P})}\) es un conjunto \(\Sigma \)-p.r.. Notese que

\(\displaystyle \chi _{D_{M(\bar{P})}}=\lambda \vec{x}\vec{\alpha}\left[ (\exists t\in \omega )_{t\leq f(\vec{x},\vec{\alpha})}\;\bar{P}(t,\vec{x},\vec{\alpha}) \right] \)

lo cual nos dice que
\(\displaystyle \chi _{D_{M(\bar{P})}}=\lambda x\vec{x}\vec{\alpha}\left[ (\exists t\in \omega )_{t\leq x}\;\bar{P}(t,\vec{x},\vec{\alpha})\right] \circ (f,p_{1}^{n,m},...,p_{n+m}^{n,m}) \)

Pero el Lema 39 nos dice que \(\lambda x\vec{x}\vec{\alpha} \left[ (\exists t\in \omega )_{t\leq x}\;\bar{P}(t,\vec{x},\vec{\alpha}) \right] \) es \(\Sigma \)-p.r. por lo cual tenemos que \(\chi _{D_{M(\bar{P})}}\) lo es.
Sea

\(\displaystyle P_{1}=\lambda t\vec{x}\vec{\alpha}\left[ \bar{P}(t,\vec{x},\vec{\alpha} )\wedge (\forall j\in \omega )_{j\leq t}\;j=t\vee \lnot \bar{P}(j,\vec{x}, \vec{\alpha})\right] \)

Note que \(P_{1}\) es \(\Sigma \)-total. Dejamos al lector usando lemas anteriores probar que \(P_{1}\) es \(\Sigma \)-p.r.. Ademas notese que para cada \((\vec{x},\vec{\alpha})\in \omega ^{n}\times \Sigma ^{\ast m}\) tenemos que
\(\displaystyle P_{1}(t,\vec{x},\vec{\alpha})=1\text{ si y solo si }t=M(\bar{P})(\vec{x}, \vec{\alpha}) \)

Esto nos dice que
\(\displaystyle M(\bar{P})=\left( \lambda \vec{x}\vec{\alpha}\left[ \prod_{t=0}^{f(\vec{x}, \vec{\alpha})}t^{P_{1}(t,\vec{x},\vec{\alpha})}\right] \right) \mid _{D_{M( \bar{P})}} \)

por lo cual para probar que \(M(\bar{P})\) es \(\Sigma \)-p.r. solo nos resta probar que
\(\displaystyle F=\lambda \vec{x}\vec{\alpha}\left[ \prod_{t=0}^{f(\vec{x},\vec{\alpha} )}t^{P_{1}(t,\vec{x},\vec{\alpha})}\right] \)

lo es. Pero
\(\displaystyle F=\lambda xy\vec{x}\vec{\alpha}\left[ \prod_{t=x}^{y}t^{P_{1}(t,\vec{x},\vec{ \alpha})}\right] \circ (C_{0}^{n,m},f,p_{1}^{n,m},...,p_{n+m}^{n,m}) \)

y por lo tanto el Lema 38 nos dice que \(F\) es \(\Sigma \)-p.r.. De esta manera hemos probado que \(M(\bar{P})\) es \(\Sigma \)-p.r. y por lo tanto \(M(P)\) lo es. \(\Box\)

\textbf{\underline{Lemma 44:}} Las siguientes funciones son \(\varnothing \)-p.r.:
(a) \( \begin{array}{rll} Q:\omega \times \mathbf{N} & \rightarrow & \omega \\ (x,y) & \rightarrow & \text{cociente de la division de }x\text{ por }y \end{array} \)
(b) \( \begin{array}{rll} R:\omega \times \mathbf{N} & \rightarrow & \omega \\ (x,y) & \rightarrow & \text{resto de la division de }x\text{ por }y \end{array} \)
(c) \( \begin{array}{rll} pr:\mathbf{N} & \rightarrow & \omega \\ n & \rightarrow & n\text{-esimo numero primo} \end{array} \)


\textbf{\underline{Proof:}} (a) Veamos primero veamos que \(Q=M(P)\), donde \(P=\lambda txy\left[ (t+1).y >x \right] \). Notar que

\(\displaystyle \begin{array}{rcl} D_{M(P)} & =& \{(x,y):(\exists t\in \omega )\;P(t,x,y)=1\} \\ & =& \{(x,y):(\exists t\in \omega )\;(t+1).y >x\} \\ & =& \omega \times \mathbf{N} \\ & =& D_{Q} \end{array} \)

Dejamos al lector la facil verificacion de que para cada \((x,y)\in \omega \times \mathbf{N}\), se tiene que
\(\displaystyle Q(x,y)=M(P)(x,y)=\min_{t}(t+1).y >x \)

Esto prueba que \(Q=M(P)\). Ya que \(P\) es \(\varnothing \)-p.r. y
\(\displaystyle Q(x,y)\leq p_{1}^{2,0}(x,y),\text{para cada }(x,y)\in \omega \times \mathbf{N } \)

(b) del Lema 43 implica que \(Q\in \mathrm{PR}^{\varnothing }\).
(b) Notese que

\(\displaystyle R=\lambda xy\left[ x\dot{-}Q(x,y).y\right] \)

y por lo tanto \(R\in \mathrm{PR}^{\varnothing }\).
(c) Para ver que \(pr\) es \(\varnothing \)-p.r., veremos que la extension \( h:\omega \rightarrow \omega \), dada por \(h(0)=0\) y \(h(n)=pr(n)\), \(n\geq 1\), es \(\varnothing \)-p.r.. Primero note que

\(\displaystyle \begin{array}{rcl} h(0) & =& 0 \\ h(x+1) & =& \min\nolimits_{t}\left( t\text{ es primo}\wedge t >h(x)\right) \end{array} \)

O sea que \(h=R\left( C_{0}^{0,0},M(P)\right) \), donde
\(\displaystyle P=\lambda tzx\left[ t\text{ es primo}\wedge t >z\right] \)

Es decir que solo nos resta ver que \(M(P)\) es \(\varnothing \)-p.r.. Claramente \( P\) es \(\varnothing \)-p.r.. Veamos que para cada \((z,x)\in \omega ^{2}\), tenemos que
\(\displaystyle M(P)(z,x)=\min\nolimits_{t}\left( t\text{ es primo}\wedge t >z\right) \leq z!+1 \)

Sea \(p\) primo tal que \(p\) divide a \(z!+1\). Es facil ver que entonces \(p >z\). Pero esto claramente nos dice que
\(\displaystyle \min\nolimits_{t}\left( t\text{ es primo}\wedge t >z\right) \leq p\leq z!+1 \)

O sea que (b) del Lema 43 implica que \(M(P)\) es \(\varnothing \) -p.r. ya que podemos tomar \(f=\lambda zx\left[ z!+1\right] \). \(\Box\)


\textbf{\underline{Lemma 45:}} Las funciones \(\lambda xi\left[ (x)_{i}\right] \) y \(\lambda x\left[ Lt(x) \right] \) son \(\varnothing \)-p.r.

\textbf{\underline{Proof:}} Note que \(D_{\lambda xi\left[ (x)_{i}\right] }=\mathbf{N}\times \mathbf{N}\). Sea

\(\displaystyle P=\lambda txi\left[ \lnot (pr(i)^{t+1}\ \text{divide }x)\right] \)

Note que \(P\) es \(\varnothing \)-p.r. y que \(D_{P}=\omega \times \omega \times \mathbf{N}\). Dejamos al lector la prueba de que \(\lambda xi\left[ (x)_{i} \right] =M(P)\). Ya que \((x)_{i}\leq x\), para todo \(x\in \mathbf{N}\), (b) del Lema 43 implica que \(\lambda xi\left[ (x)_{i}\right] \) es \( \varnothing \)-p.r..
Veamos que \(\lambda x\left[ Lt(x)\right] \) es \(\varnothing \)-p.r.. Sea

\(\displaystyle Q=\lambda tx\left[ (\forall i\in \mathbf{N})_{i\leq x}\;(i\leq t\vee (x)_{i}=0)\right] \)

Notese que \(D_{Q}=\omega \times \mathbf{N}\) y que ademas por el Lema 39 tenemos que \(Q\) es \(\varnothing \)-p.r. (dejamos al lector explicar como se aplica tal lema en este caso). Ademas notese que \(\lambda x \left[ Lt(x)\right] =M(Q)\) y que
\(\displaystyle Lt(x)\leq x,\text{para todo }x\in \mathbf{N} \)

lo cual por (b) del Lema 43 nos dice que \(\lambda x\left[ Lt(x)\right] \) es \(\varnothing \)-p.r.. \(\Box\)
Para \(x_{1},...,x_{n}\in \omega \), escribiremos \(\left\langle x_{1},...,x_{n}\right\rangle \) en lugar de \(\left\langle x_{1},...,x_{n},0,...\right\rangle \).



\textbf{\underline{Lemma 46:}} Sea \(n\geq 1\). La funcion \(\lambda x_{1}...x_{n}\left[ \left\langle x_{1},...,x_{n}\right\rangle \right] \) es \(\varnothing \)-p.r.

\textbf{\underline{Proof:}} Sea \(f_{n}=\lambda x_{1}...x_{n}\left[ \left\langle x_{1},...,x_{n}\right\rangle \right] \). Claramente \(f_{1}\) es \(\varnothing \) -p.r.. Ademas note que para cada \(n\geq 1\), tenemos

\(\displaystyle f_{n+1}=\lambda x_{1}...x_{n+1}\left[ \left( f_{n}(x_{1},...,x_{n})pr(n+1)^{x_{n+1}}\right) \right] \text{.} \)

O sea que podemos aplicar un argumento inductivo. \(\Box\)


\textbf{\underline{Lemma 47:}} Supongamos que \(\Sigma \neq \varnothing \). Sea \(< \) un orden total estricto sobre \(\Sigma \), sean \(n,m\geq 0\) y sea \( P:D_{P}\subseteq \omega ^{n}\times \Sigma ^{\ast m}\times \Sigma ^{\ast }\rightarrow \omega \) un predicado \(\Sigma \)-p.r.. Entonces
(a) \(M^{< }(P)\) es \(\Sigma \)-recursiva.
(b) Si existe una funcion \(\Sigma \)-p.r. \(f:\omega ^{n}\times \Sigma ^{\ast m}\rightarrow \omega \) tal que
\(\displaystyle \left\vert M^{< }(P)(\vec{x},\vec{\alpha})\right\vert =\left\vert \min\nolimits_{\alpha }^{< }P(\vec{x},\vec{\alpha},\alpha )\right\vert \leq f( \vec{x},\vec{\alpha})\text{, para cada }(\vec{x},\vec{\alpha})\in D_{M^{< }(P)}\text{,} \)
entonces \(M^{< }(P)\) es \(\Sigma \)-p.r..

\textbf{\underline{Proof:}} Sea \(Q=P\circ \left( p_{2}^{1+n,m},...,p_{1+n+m}^{1+n,m},\ast ^{< }\circ p_{1}^{1+n,m}\right) \). Note que

\(\displaystyle M^{< }(P)=\ast ^{< }\circ M(Q) \)

lo cual por (a) del Lema 43 implica que \(M^{< }(P)\) es \( \Sigma \)-recursiva.
Sea \(k\) el cardinal de \(\Sigma \). Ya que

\(\displaystyle \left\vert \ast ^{< }(M(Q)(\vec{x},\vec{\alpha}))\right\vert =\left\vert M^{< }(P)(\vec{x},\vec{\alpha})\right\vert \leq f(\vec{x},\vec{\alpha})\text{, } \)

para todo \((\vec{x},\vec{\alpha})\in D_{M^{< }(P)}=D_{M(Q)}\), tenemos que
\(\displaystyle M(Q)(\vec{x},\vec{\alpha}))\leq \sum_{\iota =1}^{i=f(\vec{x},\vec{\alpha} )}k^{i}\text{, para cada }(\vec{x},\vec{\alpha})\in D_{M(Q)}\text{.} \)

O sea que por (a) del Lema 43, \(M(Q)\) es \(\Sigma \)-p.r. y por lo tanto \(M^{< }(P)\) lo es. \(\Box\)

\subsection{Recursion primitiva sobre valores anteriores}

\textbf{\underline{Lemma 48:}} Supongamos
\(\displaystyle \begin{array}{rcl} f & :& U\subseteq \omega ^{n}\times \Sigma ^{\ast m}\rightarrow \omega \\ g & :& \omega \times \omega \times U\rightarrow \omega \\ h & :& \omega \times U\rightarrow \omega \end{array} \)

son funciones tales que
\(\displaystyle \begin{array}{rcl} h(0,\vec{x},\vec{\alpha}) & =& f(\vec{x},\vec{\alpha})\text{, para cada }(\vec{ x},\vec{\alpha})\in U \\ h(x+1,\vec{x},\vec{\alpha}) & =& g(h^{\downarrow }(x,\vec{x},\vec{\alpha}),x, \vec{x},\vec{\alpha})\text{, para cada }x\in \omega \text{ y }(\vec{x},\vec{ \alpha})\in U\text{.} \end{array} \)
Entonces \(h\) es \(\Sigma \)-p.r. si \(f\) y \(g\) lo son.

\textbf{\underline{Proof:}} Supongamos \(f,g\) son \(\Sigma \)-p.r.. Primero veremos que \(h^{\downarrow }\) es \(\Sigma \)-p.r.. Notese que

\(\displaystyle \begin{array}{rcl} h^{\downarrow }(0,\vec{x},\vec{\alpha}) & =& \left\langle h(0,\vec{x},\vec{ \alpha})\right\rangle \\ & =& \left\langle f(\vec{x},\vec{\alpha})\right\rangle \\ & =& 2^{f(\vec{x},\vec{\alpha})} \\ h^{\downarrow }(x+1,\vec{x},\vec{\alpha}) & =& h^{\downarrow }(x,\vec{x},\vec{ \alpha})pr(x+2)^{h(x+1,\vec{x},\vec{\alpha})} \\ & =& h^{\downarrow }(x,\vec{x},\vec{\alpha})pr(x+2)^{g(h^{\downarrow }(x,\vec{x },\vec{\alpha}),x,\vec{x},\vec{\alpha})} \end{array} \)

lo cual nos dice que \(h^{\downarrow }=R(f_{1},g_{1})\) donde
\(\displaystyle \begin{array}{rcl} f_{1} & =& \lambda \vec{x}\vec{\alpha}\left[ 2^{f(\vec{x},\vec{\alpha})}\right] \\ g_{1} & =& \lambda Ax\vec{x}\vec{\alpha}\left[ Apr(x+2)^{g(A,x,\vec{x},\vec{ \alpha})}\right] \end{array} \)

O sea que \(h^{\downarrow }\) es \(\Sigma \)-p.r. ya que \(f_{1}\) y \(g_{1}\) lo son. Finalmente notese que
\(\displaystyle h=\lambda ix[(x)_{i}]\circ (Suc\circ p_{1}^{1+n,m},h^{\downarrow }) \)

lo cual nos dice que \(h\) es \(\Sigma \)-p.r.. \(\Box\)

\include{independence_of_the_alphabet}
\section{Sintaxis de $\mathcal{S}^{\Sigma}$}

\textbf{\underline{Lemma 52:}} Se tiene que:
(a) Si \(I_{1}...I_{n}=J_{1}...J_{m}\), con \( I_{1},...,I_{n},J_{1},...,J_{m}\in \mathrm{Ins}^{\Sigma }\), entonces \(n=m\) y \(I_{j}=J_{j}\) para cada \(j\geq 1\).
(b) Si \(\mathcal{P}\in \mathrm{Pro}^{\Sigma }\), entonces existe una unica sucesion de instrucciones \(I_{1},...,I_{n}\) tal que \(\mathcal{P} =I_{1}...I_{n}\)


\textbf{\underline{Proof:}} (a) Supongamos \(I_{n}\) es un tramo final propio de \(J_{m}.\) Notar que entonces \(n >1\). Es facil ver que entonces ya sea \(J_{m}=\mathrm{L}\bar{u} I_{n}\) para algun \(u\in \mathbf{N}\), o \(I_{n}\) es de la forma \(\mathrm{GOTO} \;\mathrm{L}\bar{n}\) y \(J_{m}\) es de la forma \(w\mathrm{IF}\;\mathrm{P}\bar{k }\;\mathrm{BEGINS}\;a\;\mathrm{GOTO}\;\mathrm{L}\bar{n}\) donde \(w\in \{ \mathrm{L}\bar{n}:n\in \mathbf{N}\}\cup \{\varepsilon \}\). El segundo caso no puede darse porque entonces el anteultimo simbolo de \(I_{n-1}\) deberia ser \(\mathrm{S}\) lo cual no sucede para ninguna instruccion. O sea que

\(\displaystyle I_{1}...I_{n}=J_{1}...J_{m-1}\mathrm{L}\bar{u}I_{n} \)

lo cual dice que
(*) \(I_{1}...I_{n-1}=J_{1}...J_{m-1}\mathrm{L}\bar{u}.\)
Es decir que \(\mathrm{L}\bar{u}\) es tramo final de \(I_{n-1}\) y por lo tanto \(\mathrm{GOTO}\;\mathrm{L}\bar{u}\) es tramo final de \(I_{n-1}.\) Por (*), \(\mathrm{GOTO}\) es tramo final de \(J_{1}...J_{m-1}\), lo cual es impossible. Hemos llegado a una contradiccion lo cual nos dice que \(I_{n}\) no es un tramo final propio de \(J_{m}.\) Por simetria tenemos que \( I_{n}=J_{m} \), lo cual usando un razonamiento inductivo nos dice que \(n=m\) y \(I_{j}=J_{j} \) para cada \(j\geq 1\).

(b) Es consecuencia directa de (a). \(\Box\)

\section{Funciones $\Sigma$-computables}


\textbf{\underline{Theorem 53:}} Si \(f\) es \(\Sigma \)-computable, entonces \(f\) es \(\Sigma \)-efectivamente computable.


\textbf{\underline{Proof:}} Supongamos por ejemplo que \(f:S\subseteq \omega ^{n}\times \Sigma ^{\ast m}\rightarrow \omega \) es computada por \(\mathcal{P}\in \mathrm{Pro}^{\Sigma }\). Es claro que el procedimiento que consiste en realizar las sucesivas instrucciones de \(\mathcal{P}\) (partiendo de \(((x_{1},...,x_{n},0,0,...),( \alpha _{1},...,\alpha _{m},\varepsilon ,\varepsilon ,...))\)) y eventualmente terminar en caso de que nos toque realizar la instruccion \(n( \mathcal{P})+1\), y dar como salida el contenido de la variable \(\mathrm{N}1\) , es un procedimiento efectivo que computa a \(f\). \(\Box\)


\textbf{\underline{Proposition 54:}}

(a) Sea \(f:S\subseteq \omega ^{n}\times \Sigma ^{\ast m}\rightarrow \omega \) una funcion \(\Sigma \)-computable. Entonces hay un macro
\(\displaystyle \left[ \mathrm{V}\overline{n+1}\leftarrow f(\mathrm{V}1,...,\mathrm{V}\bar{n} ,\mathrm{W}1,...,\mathrm{W}\bar{m})\right] \)
(b) Sea \(f:S\subseteq \omega ^{n}\times \Sigma ^{\ast m}\rightarrow \Sigma ^{\ast }\) una funcion \(\Sigma \)-computable. Entonces hay un macro
\(\displaystyle \left[ \mathrm{W}\overline{m+1}\leftarrow f(\mathrm{V}1,...,\mathrm{V}\bar{n} ,\mathrm{W}1,...,\mathrm{W}\bar{m})\right] \)

\textbf{\underline{Proof:}} (b) Sea \(\mathcal{P}\) un programa que compute a \(f\). Tomemos un \(k\) tal que \( k\geq n,m\) y tal que todas las variables y labels de \(\mathcal{P}\) estan en el conjunto

\(\displaystyle \{\mathrm{N}1,...,\mathrm{N}\bar{k},\mathrm{P}1,...,\mathrm{P}\bar{k}, \mathrm{L}1,...,\mathrm{L}\bar{k}\}\text{.} \)

Sea \(\mathcal{P}^{\prime }\) la palabra que resulta de reemplazar en \( \mathcal{P}\):
- la variable \(\mathrm{N}\overline{j}\) por \(\mathrm{V}\overline{n+j}\) , para cada \(j=1,...,k\)
- la variable \(\mathrm{P}\overline{j}\) por \(\mathrm{W}\overline{m+j}\) , para cada \(j=1,...,k\)
- el label \(\mathrm{L}\overline{j}\) por \(\mathrm{A}\overline{j}\), para cada \(j=1,...,k\)
Notese que

\(\displaystyle \begin{array}{l} \mathrm{V}\overline{n+1}\leftarrow \mathrm{V}1 \\ \ \ \ \ \ \ \ \ \ \vdots \\ \mathrm{V}\overline{n+n}\leftarrow \mathrm{V}\overline{n} \\ \mathrm{V}\overline{n+n+1}\leftarrow 0 \\ \ \ \ \ \ \ \ \ \ \vdots \\ \mathrm{V}\overline{n+k}\leftarrow 0 \\ \mathrm{W}\overline{m+1}\leftarrow \mathrm{W}1 \\ \ \ \ \ \ \ \ \ \ \vdots \\ \mathrm{W}\overline{m+m}\leftarrow \mathrm{W}\overline{m} \\ \mathrm{W}\overline{m+m+1}\leftarrow \varepsilon \\ \ \ \ \ \ \ \ \ \ \vdots \\ \mathrm{W}\overline{m+k}\leftarrow \varepsilon \\ \mathcal{P}^{\prime } \end{array} \)

es el macro buscado, el cual tendra sus variables auxiliares y labels en la lista
\(\displaystyle \mathrm{V}\overline{n+1},...,\mathrm{V}\overline{n+k},\mathrm{W}\overline{m+2 },...,\mathrm{V}\overline{m+k},\mathrm{A}1,...,\mathrm{A}\overline{k}. \)

\(\Box\)




\textbf{\underline{Proposition 55:}} Sea \(P:S\subseteq \omega ^{n}\times \Sigma ^{\ast m}\rightarrow \omega \) un predicado \(\Sigma \)-computable. Entonces hay un macro
\(\displaystyle \left[ \mathrm{IF}\;P(\mathrm{V}1,...,\mathrm{V}\bar{n},\mathrm{W}1,..., \mathrm{W}\bar{m})\;\mathrm{GOTO}\;\mathrm{A}1\right] \)



\textbf{\underline{Theorem 56:}} Si \(h\) es \(\Sigma \)-recursiva, entonces \(h\) es \(\Sigma \) -computable.

\textbf{\underline{Proof:}} Probaremos por induccion en \(k\) que

(*) Si \(h\in \mathrm{R}_{k}^{\Sigma }\), entonces \(h\) es \(\Sigma \) -computable.
El caso \(k=0\) es dejado al lector. Supongamos (*) vale para \(k\), veremos que vale para \(k+1\). Sea \(h\in \mathrm{R}_{k+1}^{\Sigma }-\mathrm{R} _{k}^{\Sigma }.\) Hay varios casos

Caso 1. Supongamos \(h=M(P)\), con \(P:\omega \times \omega ^{n}\times \Sigma ^{\ast m}\rightarrow \omega \), un predicado perteneciente a \(\mathrm{R} _{k}^{\Sigma }\). Por hipotesis inductiva, \(P\) es \(\Sigma \)- computable y por lo tanto tenemos un macro

\(\displaystyle \left[ \mathrm{IF}\;P(\mathrm{V}1,...,\mathrm{V}\overline{n+1},\mathrm{W} 1,...,\mathrm{W}\bar{m})\;\mathrm{GOTO}\;\mathrm{A}1\right] \)

lo cual nos permite realizar el siguiente programa
\(\displaystyle \begin{array}{ll} \mathrm{L}2 & \mathrm{IF}\;P(\mathrm{N}\overline{n+1},\mathrm{N}1,..., \mathrm{N}\bar{n},\mathrm{P}1,...,\mathrm{P}\bar{m})\text{\ }\mathrm{GOTO}\; \mathrm{L}1 \\ & \mathrm{N}\overline{n+1}\leftarrow \mathrm{N}\overline{n+1}+1 \\ & \mathrm{GOTO}\;\mathrm{L}2 \\ \mathrm{L}1 & \mathrm{N}1\leftarrow \mathrm{N}\overline{n+1} \end{array} \)

Es facil chequear que este programa computa \(h.\)
Caso 2. Supongamos \(h=R(f,\mathcal{G})\), con

\(\displaystyle \begin{array}{rcl} f & :& S_{1}\times ...\times S_{n}\times L_{1}\times ...\times L_{m}\rightarrow \Sigma ^{\ast } \\ \mathcal{G}_{a} & :& S_{1}\times ...\times S_{n}\times L_{1}\times ...\times L_{m}\times \Sigma ^{\ast }\times \Sigma ^{\ast }\rightarrow \Sigma ^{\ast } \text{, }a\in \Sigma \end{array} \)

elementos de \(\mathrm{R}_{k}^{\Sigma }\). Sea \(\Sigma =\{a_{1},...,a_{r}\}.\) Por hipotesis inductiva, las funciones \(f\), \(\mathcal{G}_{a}\), \(a\in \Sigma \) , son \(\Sigma \)-computables y por lo tanto podemos hacer el siguiente programa via el uso de macros
\(\displaystyle \begin{array}{rl} & \left[ \mathrm{P}\overline{m+3}\leftarrow f(\mathrm{N}1,...,\mathrm{N}\bar{ n},\mathrm{P}1,...,\mathrm{P}\bar{m})\right] \\ \mathrm{L}\overline{r+1} & \mathrm{IF}\;\mathrm{P}\overline{m+1}\ \text{ {B}}\mathrm{EGINS\ }a_{1}\text{ }\mathrm{GOTO}\;\mathrm{L}1 \\ & \ \ \ \ \ \ \ \ \ \ \ \ \vdots \\ & \mathrm{IF}\;\mathrm{P}\overline{m+1}\ \mathrm{BEGINS\ }a_{r}\text{ } \mathrm{GOTO}\;\mathrm{L}\bar{r} \\ & \mathrm{GOTO}\;\mathrm{L}\overline{r+2} \\ \mathrm{L}1 & \mathrm{P}\overline{m+1}\leftarrow \text{ }^{\curvearrowright } \mathrm{P}\overline{m+1} \\ & \left[ \mathrm{P}\overline{m+3}\leftarrow \mathcal{G}_{a_{1}}(\mathrm{N} 1,...,\mathrm{N}\bar{n},\mathrm{P}1,...,\mathrm{P}\bar{m},\mathrm{P} \overline{m+2},\mathrm{P}\overline{m+3})\right] \\ & \mathrm{P}\overline{m+2}\leftarrow \mathrm{P}\overline{m+2}a_{1} \\ & \mathrm{GOTO}\;\mathrm{L}\overline{r+1} \\ & \ \ \ \ \ \ \ \ \ \ \ \ \vdots \\ \mathrm{L}\bar{r} & \mathrm{P}\overline{m+1}\leftarrow \text{ } ^{\curvearrowright }\mathrm{P}\overline{m+1} \\ & \mathrm{P}\overline{m+3}\leftarrow \mathcal{G}_{a_{r}}(\mathrm{N}1,..., \mathrm{N}\bar{n},\mathrm{P}1,...,\mathrm{P}\bar{m},\mathrm{P}\overline{m+2}, \mathrm{P}\overline{m+3}) \\ & \mathrm{P}\overline{m+2}\leftarrow \mathrm{P}\overline{m+2}a_{r} \\ & \mathrm{GOTO}\;\mathrm{L}\overline{r+1} \\ \mathrm{L}\overline{r+2} & \mathrm{P}1\leftarrow \mathrm{P}\overline{m+3} \end{array} \)

Es facil chequear que este programa computa \(h.\)
El resto de los casos son dejados al lector. \(\Box\)

\section{Análisis de la recursividad de $\mathcal{S}^{\Sigma}$}


\textbf{\underline{Lemma 57:}} Sea \(\Sigma \) un alfabeto cualquiera. Las funciones \(S\) y \(\overline{\ \;}\) son \((\Sigma \cup \Sigma _{p})\)-p.r..

\textbf{\underline{Proof:}} Use el Teorema 51. \(\Box\)

\textbf{\underline{Lemma 58:}} Para cada \(n,x\in \omega \), tenemos que \( \left\vert \bar{n}\right\vert \leq x\) si y solo si \(n\leq 10^{x}-1\)

\textbf{\underline{Lemma 59:}} \(\mathrm{Ins}^{\Sigma }\) es un conjunto \((\Sigma \cup \Sigma _{p})\)-p.r..

\textbf{\underline{Proof:}} Para simplificar la Proof asumiremos que \(\Sigma =\{@,\& \}\). Ya que \( \mathrm{Ins}^{\Sigma }\) es union de los siguientes conjuntos

\(\displaystyle \begin{array}{rcl} L_{1} & =& \left\{ \mathrm{N}\bar{k}\leftarrow \mathrm{N}\bar{k}+1:k\in \mathbf{N}\right\} \\ L_{2} & =& \left\{ \mathrm{N}\bar{k}\leftarrow \mathrm{N}\bar{k}\dot{-}1:k\in \mathbf{N}\right\} \\ L_{3} & =& \left\{ \mathrm{N}\bar{k}\leftarrow \mathrm{N}\bar{n}:k,n\in \mathbf{N}\right\} \\ L_{4} & =& \left\{ \mathrm{N}\bar{k}\leftarrow 0:k\in \mathbf{N}\right\} \\ L_{5} & =& \left\{ \mathrm{IF}\;\mathrm{N}\bar{k}\neq 0\;\mathrm{GOTO}\; \mathrm{L}\bar{m}:k,m\in \mathbf{N}\right\} \\ L_{6} & =& \left\{ \mathrm{P}\bar{k}\leftarrow \mathrm{P}\bar{k}.@:k\in \mathbf{N}\right\} \\ L_{7} & =& \left\{ \mathrm{P}\bar{k}\leftarrow \mathrm{P}\bar{k}.\& :k\in \mathbf{N}\right\} \\ L_{8} & =& \left\{ \mathrm{P}\bar{k}\leftarrow \text{ }^{\curvearrowright } \mathrm{P}\bar{k}:k\in \mathbf{N}\right\} \\ L_{9} & =& \left\{ \mathrm{P}\bar{k}\leftarrow \mathrm{P}\bar{n}:k,n\in \mathbf{N}\right\} \\ L_{9} & =& \left\{ \mathrm{P}\bar{k}\leftarrow \varepsilon :k\in \mathbf{N} \right\} \\ L_{10} & =& \left\{ \mathrm{IF}\;\mathrm{P}\bar{k}\;\mathrm{BEGINS}\;@\; \mathrm{GOTO}\;\mathrm{L}\bar{m}:k,m\in \mathbf{N}\right\} \\ L_{11} & =& \left\{ \mathrm{IF}\;\mathrm{P}\bar{k}\;\mathrm{BEGINS}\;\& \; \mathrm{GOTO}\;\mathrm{L}\bar{m}:k,m\in \mathbf{N}\right\} \\ L_{12} & =& \left\{ \mathrm{GOTO}\;\mathrm{L}\bar{m}:m\in \mathbf{N}\right\} \\ L_{13} & =& \left\{ \mathrm{SKIP}\right\} \\ L_{14} & =& \left\{ \mathrm{L}\bar{k}\alpha :k\in \mathbf{N\;}\text{y }\alpha \in L_{1}\cup ...\cup L_{13}\right\} \end{array} \)

solo debemos probar que \(L_{1},...,L_{14}\) son \((\Sigma \cup \Sigma _{p})\) -p.r.. Veremos primero por ejemplo que
\(\displaystyle L_{10}=\left\{ \mathrm{IFP}\bar{k}\mathrm{BEGINS}@\mathrm{GOTOL}\bar{m} :k,m\in \mathbf{N}\right\} \)

es \((\Sigma \cup \Sigma _{p})\)-p.r.. Primero notese que \(\alpha \in L_{10}\) si y solo si existen \(k,m\in \mathbf{N}\) tales que
\(\displaystyle \alpha =\mathrm{IFP}\bar{k}\mathrm{BEGINS}@\mathrm{GOTOL}\bar{m} \)

Mas formalmente tenemos que \(\alpha \in L_{10}\) si y solo si
\(\displaystyle (\exists k\in \mathbf{N})(\exists m\in \mathbf{N})\;\alpha =\mathrm{IFP}\bar{ k}\mathrm{BEGINS}@\mathrm{GOTOL}\bar{m} \)

Ya que cuando existen tales \(k,m\) tenemos que \(\bar{k}\) y \(\bar{m}\) son subpalabras de \(\alpha \), el lema anterior nos dice que \(\alpha \in L_{10}\) si y solo si
\(\displaystyle (\exists k\in \mathbf{N})_{k\leq 10^{\left\vert \alpha \right\vert }}(\exists m\in \mathbf{N})_{m\leq 10^{\left\vert \alpha \right\vert }}\;\alpha =\mathrm{IFP}\bar{k}\mathrm{BEGINS}@\mathrm{GOTOL}\bar{m} \)

Sea
\(\displaystyle P=\lambda mk\alpha \left[ \alpha =\mathrm{IFP}\bar{k}\mathrm{BEGINS}@\mathrm{ GOTOL}\bar{m}\right] \)

Ya que \(D_{\lambda k\left[ \bar{k}\right] }=\omega \), tenemos que \( D_{P}=\omega \times (\Sigma \cup \Sigma _{p})^{\ast }\times (\Sigma \cup \Sigma _{p})^{\ast }\). Notese que
\(\displaystyle P=\lambda \alpha \beta \left[ \alpha =\beta \right] \circ \left( p_{3}^{2,1},f\right) \)

donde
\(\displaystyle f=\lambda \alpha _{1}\alpha _{2}\alpha _{3}\alpha _{4}\left[ \alpha _{1}\alpha _{2}\alpha _{3}\alpha _{4}\right] \circ \left( C_{\mathrm{IFP} }^{2,1},\lambda k\left[ \bar{k}\right] \circ p_{2}^{2,1},C_{\mathrm{BEGINS}@ \mathrm{GOTOL}}^{2,1},\lambda k\left[ \bar{k}\right] \circ p_{1}^{2,1}\right) \)

lo cual nos dice que \(P\) es \((\Sigma \cup \Sigma _{p})\)-p.r..
Notese que

\(\displaystyle \chi _{L_{10}}=\lambda \alpha \left[ (\exists k\in \mathbf{N})_{k\leq 10^{\left\vert \alpha \right\vert }}(\exists m\in \mathbf{N})_{m\leq 10^{\left\vert \alpha \right\vert }}\;P(m,k,\alpha )\right] \)

Esto nos dice que podemos usar dos veces el Lema 39 para ver que \(\chi _{L_{10}}\) es \((\Sigma \cup \Sigma _{p})\)-p.r.. Veamos como. Sea
\(\displaystyle Q=\lambda k\alpha \left[ (\exists m\in \mathbf{N})_{m\leq 10^{\left\vert \alpha \right\vert }}\;P(m,k,\alpha )\right] \)

Por el Lema 39 tenemos que
\(\displaystyle \lambda xk\alpha \left[ (\exists m\in \mathbf{N})_{m\leq x}\;P(m,k,\alpha ) \right] \)

es \((\Sigma \cup \Sigma _{p})\)-p.r. lo cual nos dice que
\(\displaystyle Q=\lambda xk\alpha \left[ (\exists m\in \mathbf{N})_{m\leq x}\;P(m,k,\alpha ) \right] \circ (\lambda \alpha \left[ 10^{\left\vert \alpha \right\vert } \right] \circ p_{2}^{1,1},p_{1}^{1,1},p_{2}^{1,1}) \)

lo es. Ya que
\(\displaystyle \chi _{L_{10}}=\lambda \alpha \left[ (\exists k\in \mathbf{N})_{k\leq 10^{\left\vert \alpha \right\vert }}\;Q(k,\alpha )\right] \)

podemos en forma similar aplicar el Lema 39 y obtener finalmente que \(\chi _{L_{10}}\) es \((\Sigma \cup \Sigma _{p})\)-p.r..
En forma similar podemos probar que \(L_{1},...,L_{13}\) son \((\Sigma \cup \Sigma _{p})\)-p.r.. Esto nos dice que \(L_{1}\cup ...\cup L_{13}\) es \((\Sigma \cup \Sigma _{p})\)-p.r.. Notese que \(L_{1}\cup ...\cup L_{13}\) es el conjunto de las instrucciones basicas de \(\mathcal{S}^{\Sigma }\). Llamemos \( \mathrm{InsBas}^{\Sigma }\) a dicho conjunto. Para ver que \(L_{14}\) es \( (\Sigma \cup \Sigma _{p})\)-p.r. notemos que

\(\displaystyle \chi _{L_{14}}=\lambda \alpha \left[ (\exists k\in \mathbf{N})_{k\leq 10^{\left\vert \alpha \right\vert }}(\exists \beta \in \mathrm{InsBas} ^{\Sigma })_{\left\vert \beta \right\vert \leq \left\vert \alpha \right\vert }\;\alpha =\mathrm{L}\bar{k}\beta \right] \)

lo cual nos dice que aplicando dos veces el Lema 39 obtenemos que \(\chi _{L_{14}}\) es \((\Sigma \cup \Sigma _{p})\)-p.r.. Ya que \( \mathrm{Ins}^{\Sigma }=\mathrm{InsBas}^{\Sigma }\cup L_{14}\) tenemos que \( \mathrm{Ins}^{\Sigma }\) es \((\Sigma \cup \Sigma _{p})\)-p.r.. \(\Box\)


\textbf{\underline{Lemma 60:}} \(Bas\) y \(Lab\) son funciones \((\Sigma \cup \Sigma _{p})\)-p.r.

\textbf{\underline{Proof:}} Sea \(< \) un orden total estricto sobre \(\Sigma \cup \Sigma _{p}\). Sea \(L=\{ \mathrm{L}\bar{k}:k\in \mathbf{N}\}\cup \{\varepsilon \}\). Dejamos al lector probar que \(L\) es un conjunto \((\Sigma \cup \Sigma _{p})\)-p.r.. Sea

\(\displaystyle P=\lambda I\alpha \left[ \alpha \in \mathrm{Ins}^{\Sigma }\wedge I\in \mathrm{Ins}^{\Sigma }\wedge \lbrack \alpha ]_{1}\neq \mathrm{L}\wedge (\exists \beta \in L)\ I=\beta \alpha \right] \)

Note que \(D_{P}=(\Sigma \cup \Sigma _{p})^{\ast 2}\). Dejamos al lector probar que \(P\) es \((\Sigma \cup \Sigma _{p})\)-p.r.. Notese ademas que cuando \(I\in \mathrm{Ins}^{\Sigma }\) tenemos que \(P(I,\alpha )=1\) sii \(\alpha =Bas(I)\). Dejamos al lector probar que \(Bas=M^{< }\left( P\right) \) por lo que para ver que \(Bas\) es \((\Sigma \cup \Sigma _{p})\)-p.r., solo nos falta ver que la funcion \(Bas\) es acotada por alguna funcion \((\Sigma \cup \Sigma _{p})\)-p.r. y \((\Sigma \cup \Sigma _{p})\)-total. Pero esto es trivial ya que \(\left\vert Bas(I)\right\vert \leq \left\vert I\right\vert =p_{1}^{0,1}(I)\) para cada \(I\in \mathrm{Ins}^{\Sigma }\).
Finalmente note que

\(\displaystyle Lab=M^{< }\left( \lambda I\alpha \left[ \alpha Bas(I)=I\right] \right) \)

lo cual nos dice que \(Lab\) es \((\Sigma \cup \Sigma _{p})\)-p.r.. \(\Box\)
Recordemos que dado un programa \(\mathcal{P}\) habiamos definido \(I_{i}^{ \mathcal{P}}=\varepsilon \), para \(i=0\) o \(i >n(\mathcal{P}).\) O sea que la funcion \((\Sigma \cup \Sigma _{p})\)-mixta \(\lambda i\mathcal{P}\left[ I_{i}^{ \mathcal{P}}\right] \) tiene dominio igual a \(\omega \times \mathrm{Pro} ^{\Sigma }\).

\textbf{\underline{Lemma 61:}}
(a) \(\mathrm{Pro}^{\Sigma }\) es un conjunto \((\Sigma \cup \Sigma _{p}) \)-p.r.
(b) \(\lambda \mathcal{P}\left[ n(\mathcal{P})\right] \) y \(\lambda i \mathcal{P}\left[ I_{i}^{\mathcal{P}}\right] \) son funciones \((\Sigma \cup \Sigma _{p})\)-p.r..

\textbf{\underline{Proof:}} Ya que \(\mathrm{Pro}^{\Sigma }=D_{\lambda \mathcal{P}\left[ n(\mathcal{P}) \right] }\) tenemos que (b) implica (a). Para probar (b) sea \(< \) un orden total estricto sobre \(\Sigma \cup \Sigma _{p}\). Sea \(P\) el siguiente predicado

\(\lambda x\left[ Lt(x) >0\wedge (\forall t\in \mathbf{N})_{t\leq Lt(x)}\;\ast ^{< }((x)_{t})\in \mathrm{Ins}^{\Sigma }\wedge \right. \)
\(\ \ \ \ \ \ \ \ \ \ \ \ \ \ \ \ \ \ \ \ \ \ \ \ \ \ \ \ \ (\forall t\in \mathbf{N})_{t\leq Lt(x)}(\forall m\in \mathbf{N})\;\lnot (\mathrm{L} \bar{m}\ \)t-final \(\ast ^{< }((x)_{t}))\vee \)
\(\ \ \ \ \ \ \ \ \ \ \ \ \ \ \ \ \ \ \ \ \ \ \ \ \ \ \ \ \ \ \ \ \ \ \ \ \ \ \ \ \ \ \ \ \ \ \ \ \ \ \ \ \ \ \ \ \ \ \ \left. (\exists j\in \mathbf{ N})_{j\leq Lt(x)}(\exists \alpha \in (\Sigma \cup \Sigma _{p})-Num)\;\mathrm{ L}\bar{m}\alpha \ \text{t-inicial}\ast ^{< }((x)_{j})\right] \)
Notese que \(D_{P}=\mathbf{N}\) y que \(P(x)=1\) sii \(Lt(x) >0\), \(\ast ^{< }((x)_{t})\in \mathrm{Ins}^{\Sigma }\), para cada \(t=1,...,Lt(x)\) y ademas \(\subset _{t=1}^{t=Lt(x)}\ast ^{< }((x)_{t})\in \mathrm{Pro}^{\Sigma }\). Para ver que \(P\) es \((\Sigma \cup \Sigma _{p})\)-p.r. solo nos falta acotar el cuantificador \((\forall m\in \mathbf{N})\) de la expresion lambda que define a \(P\). Ya que nos interesan los valores de \(m\) para los cuales \(\bar{m}\) es posiblemente una subpalabra de alguna de las palabras \(\ast ^{< }((x)_{j})\), el Lema 58 nos dice que una cota posible es \( 10^{\max \{\left\vert \ast ^{< }((x)_{j})\right\vert :1\leq j\leq Lt(x)\}}-1\) . Dejamos al lector los detalles de la Proof de que \(P\) es \((\Sigma \cup \Sigma _{p})\)-p.r.. Sea

\(\displaystyle Q=\lambda x\alpha \left[ P(x)\wedge \alpha =\subset _{t=1}^{t=Lt(x)}\ast ^{< }((x)_{t})\right] \text{.} \)

Note que \(D_{Q}=\mathbf{N}\times (\Sigma \cup \Sigma _{p})^{\ast }\). Claramente \(Q\) es \((\Sigma \cup \Sigma _{p})\)-p.r.. Ademas note que \( D_{M(Q)}=\mathrm{Pro}^{\Sigma }\). Notese que para \(\mathcal{P}\in \mathrm{Pro }^{\Sigma }\), tenemos que \(M(Q)(\mathcal{P})\) es aquel numero tal que pensado como infinitupla (via mirar su secuencia de exponentes) codifica la secuencia de instrucciones que forman a \(\mathcal{P}\). Es decir
\(\displaystyle M(Q)(\mathcal{P})=\left\langle \#^{< }(I_{1}^{\mathcal{P}}),\#^{< }(I_{2}^{ \mathcal{P}}),...,\#^{< }(I_{n(\mathcal{P})}^{\mathcal{P}}),0,0,...\right \rangle \)

Por (b) del Lema 43, \(M(Q)\) es \((\Sigma \cup \Sigma _{p})\) -p.r. ya que para cada \(\mathcal{P}\in \mathrm{Pro}^{\Sigma }\) tenemos que
\(\displaystyle \begin{array}{rcl} M(Q)(\mathcal{P}) & =& \left\langle \#^{< }(I_{1}^{\mathcal{P}}),\#^{< }(I_{2}^{ \mathcal{P}}),...,\#^{< }(I_{n(\mathcal{P})}^{\mathcal{P}}),0,0,...\right \rangle \\ & =& \underset{i=1}{\overset{n(\mathcal{P})}{\Pi }}pr(i)^{\#^{< }(I_{1}^{ \mathcal{P}})} \\ & \leq & \underset{i=1}{\overset{\left\vert \mathcal{P}\right\vert }{\Pi }} pr(i)^{\#^{< }(\mathcal{P})} \end{array} \)

Ademas tenemos que
\(\displaystyle \begin{array}{rcl} \lambda \mathcal{P}\left[ n(\mathcal{P})\right] & =& \lambda x\left[ Lt(x) \right] \circ M(Q) \\ \lambda i\mathcal{P}\left[ I_{i}^{\mathcal{P}}\right] & =& \ast ^{< }\circ g\circ \left( p_{1}^{1,1},M(Q)\circ p_{2}^{1,1}\right) \end{array} \)

donde \(g=C_{0}^{1,1}\mid _{\{0\}\times \omega }\cup \lambda ix\left[ (x)_{i} \right] \), lo cual dice que \(\lambda \mathcal{P}\left[ n(\mathcal{P})\right] \) y \(\lambda i\mathcal{P}\left[ I_{i}^{\mathcal{P}}\right] \) son funciones \( (\Sigma \cup \Sigma _{p})\)-p.r.. \(\Box\)


\textbf{\underline{Lemma 62:}} Dado un orden total estricto \( < \) sobre \(\Sigma \cup \Sigma _{p}\), las funciones \(s\), \(S_{\#}\) y \(S_{\ast } \) son \((\Sigma \cup \Sigma _{p})\)-p.r..

\textbf{\underline{Proof:}} Necesitaremos algunas funciones \((\Sigma \cup \Sigma _{p})\)-p.r.. Dada una instruccion \(I\) en la cual al menos ocurre una variable, usaremos \(\#Var1(I)\) para denotar el numero de la primer variable que ocurre en \(I\). Por ejemplo

\(\displaystyle \#Var1\left( \mathrm{L}\bar{n}\;\mathrm{IF\;N}\bar{k}\neq 0\;\mathrm{GOTO\;L} \bar{m}\right) =k \)

Notese que \(\lambda I[\#Var1(I)]\) tiene dominio igual a \(\mathrm{Ins} ^{\Sigma }-L\), donde \(L\) es la union de los siguientes conjuntos \begin{gather*} \{\mathrm{GOTO L}\bar{m}:m\in \mathbf{N\}\cup }\{\mathrm{L}\bar{k} \mathrm{ GOTO L}\bar{m}:k,m\in \mathbf{N\}}
\left\{ \mathrm{SKIP}\right\} \mathbf{\cup }\{\mathrm{L}\bar{k} \mathrm{SKIP }:k\in \mathbf{N\}} \end{gather*} Dada una instruccion \(I\) en la cual ocurren dos variables, usaremos \( \#Var2(I)\) para denotar el numero de la segunda variable que ocurre en \(I\). Por ejemplo
\(\displaystyle \#Var2\left( \mathrm{N}\bar{k}\leftarrow \mathrm{N}\bar{m}\right) =m \)

Notese que el dominio de \(\lambda I[\#Var2(I)]\) es igual a la union de los siguientes conjuntos
\(\displaystyle \begin{array}{rcl} \{\mathrm{N}\bar{k} & \leftarrow & \mathrm{N}\bar{m}:k,m\in \mathbf{N\}\cup }\{ \mathrm{L}\bar{j}\ \mathrm{N}\bar{k}\leftarrow \mathrm{N}\bar{m}:j,k,m\in \mathbf{N\}} \\ \{\mathrm{P}\bar{k} & \leftarrow & \mathrm{P}\bar{m}:k,m\in \mathbf{N\}\cup }\{ \mathrm{L}\bar{j}\ \mathrm{P}\bar{k}\leftarrow \mathrm{P}\bar{m}:j,k,m\in \mathbf{N\}} \end{array} \)

Ademas notese que para una instruccion \(I\) tenemos que
\(\displaystyle \begin{array}{rcl} \#Var1(I) & =& \min_{k}(\mathrm{N}\bar{k}\mathrm{\leftarrow }\text{ }\mathrm{ ocu}\text{ }I\vee \mathrm{N}\bar{k}\mathrm{\neq }\text{ }\mathrm{ocu}\text{ } I\vee \mathrm{P}\bar{k}\mathrm{\leftarrow }\text{ }\mathrm{ocu}\text{ }I\vee \mathrm{P}\bar{k}\mathrm{B}\;\mathrm{ocu}\text{ }I) \\ \#Var2(I) & =& \min_{k}(\mathrm{N}\bar{k}\ \text{t-final }I\vee \mathrm{N}\bar{ k}\mathrm{+}\text{ }\mathrm{ocu}\text{ }I\vee \mathrm{N}\bar{k}\mathrm{\dot{- }}\text{ }\mathrm{ocu}\text{ }I\vee \mathrm{P}\bar{k}\ \text{t-final }I\vee \mathrm{P}\bar{k}.\text{ }\mathrm{ocu}\text{ }I) \end{array} \)

Esto nos dice que si llamamos \(P\) al predicado
\(\displaystyle \lambda k\alpha \left[ \alpha \in \mathrm{Ins}^{\Sigma }\wedge (\mathrm{N} \bar{k}\mathrm{\leftarrow }\text{ }\mathrm{ocu}\text{ }\alpha \vee \mathrm{N} \bar{k}\mathrm{\neq }\text{ }\mathrm{ocu}\text{ }\alpha \vee \mathrm{P}\bar{k }\mathrm{\leftarrow }\text{ }\mathrm{ocu}\text{ }\alpha \vee \mathrm{P}\bar{k }\mathrm{B}\;\mathrm{ocu}\text{ }\alpha )\right] \)

entonces \(\lambda I[\#Var1(I)]=M(P)\) por lo cual \(\lambda I[\#Var1(I)]\) es \( (\Sigma \cup \Sigma _{p})\)-p.r. Similarmente se puede ver que \(\lambda I[\#Var2(I)]\) es \((\Sigma \cup \Sigma _{p})\)-p.r.. Sea
\(\displaystyle \begin{array}{rll} F_{\dot{-}}:\mathbf{N}\times \mathbf{N} & \rightarrow & \omega \\ (x,j) & \rightarrow & \left\langle (x)_{1},....,(x)_{j-1},(x)_{j}\dot{-} 1,(x)_{j+1},...\right\rangle \end{array} \)

Ya que
\(\displaystyle F_{\dot{-}}(x,j)=\left\{ \begin{array}{lll} Q(x,pr(j)) & & \text{si }pr(j)\text{ divide }x \\ x & & \text{caso contrario} \end{array} \right. \)

tenemos que \(F_{\dot{-}}\) es \((\Sigma \cup \Sigma _{p})\)-p.r.. Sea
\(\displaystyle \begin{array}{rll} F_{+}:\mathbf{N}\times \mathbf{N} & \rightarrow & \omega \\ (x,j) & \rightarrow & \left\langle (x)_{1},....,(x)_{j-1},(x)_{j}+1,(x)_{j+1},...\right\rangle \end{array} \)

Ya que \(F_{+}(x,j)=x.pr(j)\) tenemos que \(F_{+}\) es \((\Sigma \cup \Sigma _{p}) \)-p.r.. Sea
\(\displaystyle \begin{array}{rll} F_{\leftarrow }:\mathbf{N}\times \mathbf{N}\times \mathbf{N} & \rightarrow & \omega \\ (x,j,k) & \rightarrow & \left\langle (x)_{1},....,(x)_{j-1},(x)_{k},(x)_{j+1},...\right\rangle \end{array} \)

Ya que \(F_{\leftarrow }(x,j,k)=Q(x,pr(j)^{(x)_{j}}).pr(j)^{(x)_{k}}\) tenemos que \(F_{\leftarrow }\) es \((\Sigma \cup \Sigma _{p})\)-p.r.. Sea
\(\displaystyle \begin{array}{rll} F_{0}:\mathbf{N}\times \mathbf{N} & \rightarrow & \omega \\ (x,j) & \rightarrow & \left\langle (x)_{1},....,(x)_{j-1},0,(x)_{j+1},...\right\rangle \end{array} \)

Es facil ver que \(F_{0}\) es \((\Sigma \cup \Sigma _{p})\)-p.r.. Para cada \( a\in \Sigma \), sea
\(\displaystyle \begin{array}{rll} F_{a}:\mathbf{N}\times \mathbf{N} & \rightarrow & \omega \\ (x,j) & \rightarrow & \left\langle (x)_{1},....,(x)_{j-1},\#^{< }(\ast ^{< }((x)_{j})a),(x)_{j+1},...\right\rangle \end{array} \)

Es facil ver que \(F_{a}\) es \((\Sigma \cup \Sigma _{p})\)-p.r.. En forma similar puede ser probado que
\(\displaystyle \begin{array}{rll} F_{\curvearrowright }:\mathbf{N}\times \mathbf{N} & \rightarrow & \omega \\ (x,j) & \rightarrow & \left\langle (x)_{1},....,(x)_{j-1},\#^{< }(^{\curvearrowright }(\ast ^{< }((x)_{j}))),(x)_{j+1},...\right\rangle \end{array} \)

es \((\Sigma \cup \Sigma _{p})\)-p.r.
Dado \((i,x,y,\mathcal{P})\in \omega \times \mathbf{N}\times \mathbf{N}\times \mathrm{Pro}^{\Sigma }\), tenemos varios casos en los cuales los valores \( s(i,x,y,\mathcal{P}),S_{\#}(i,x,y,\mathcal{P})\) y \(S_{\ast }(i,x,y,\mathcal{P })\) pueden ser obtenidos usando las funciones antes definidas:

(1) CASO \(i=0\vee i >n(\mathcal{P})\). Entonces
\(\displaystyle \begin{array}{rcl} s(i,x,y,\mathcal{P}) & =& i \\ S_{\#}(i,x,y,\mathcal{P}) & =& x \\ S_{\ast }(i,x,y,\mathcal{P}) & =& y \end{array} \)

(2) CASO \((\exists j\in \omega )\;Bas(I_{i}^{\mathcal{P}})=\mathrm{N} \bar{j}\leftarrow \mathrm{N}\bar{j}+1\). Entonces
\(\displaystyle \begin{array}{rcl} s(i,x,y,\mathcal{P}) & =& i+1 \\ S_{\#}(i,x,y,\mathcal{P}) & =& F_{+}(x,\#Var1(I_{i}^{\mathcal{P}})) \\ S_{\ast }(i,x,y,\mathcal{P}) & =& y \end{array} \)

(3) CASO \((\exists j\in \omega )\;Bas(I_{i}^{\mathcal{P}})=\mathrm{N} \bar{j}\leftarrow \mathrm{N}\bar{j}\dot{-}1\). Entonces
\(\displaystyle \begin{array}{rcl} s(i,x,y,\mathcal{P}) & =& i+1 \\ S_{\#}(i,x,y,\mathcal{P}) & =& F_{\dot{-}}(x,\#Var1(I_{i}^{\mathcal{P}})) \\ S_{\ast }(i,x,y,\mathcal{P}) & =& y \end{array} \)

(4) CASO \((\exists j,k\in \omega )\;Bas(I_{i}^{\mathcal{P}})=\mathrm{N }\bar{j}\leftarrow \mathrm{N}\bar{k}\). Entonces
\(\displaystyle \begin{array}{rcl} s(i,x,y,\mathcal{P}) & =& i+1 \\ S_{\#}(i,x,y,\mathcal{P}) & =& F_{\leftarrow }(x,\#Var1(I_{i}^{\mathcal{P} }),\#Var2(I_{i}^{\mathcal{P}})) \\ S_{\ast }(i,x,y,\mathcal{P}) & =& y \end{array} \)

(5) CASO \((\exists j,k\in \omega )\;Bas(I_{i}^{\mathcal{P}})=\mathrm{N }\bar{j}\leftarrow 0\). Entonces
\(\displaystyle \begin{array}{rcl} s(i,x,y,\mathcal{P}) & =& i+1 \\ S_{\#}(i,x,y,\mathcal{P}) & =& F_{0}(x,\#Var1(I_{i}^{\mathcal{P}})) \\ S_{\ast }(i,x,y,\mathcal{P}) & =& y \end{array} \)

(6) CASO \((\exists j,m\in \omega )\;\left( Bas(I_{i}^{\mathcal{P}})= \mathrm{IF}\;\mathrm{N}\bar{j}\neq 0\;\mathrm{GOTO}\;\mathrm{L}\bar{m}\wedge (x)_{j}=0\right) \). Entonces
\(\displaystyle \begin{array}{rcl} s(i,x,y,\mathcal{P}) & =& i+1 \\ S_{\#}(i,x,y,\mathcal{P}) & =& x \\ S_{\ast }(i,x,y,\mathcal{P}) & =& y \end{array} \)

(7) CASO \((\exists j,m\in \omega )\;\left( Bas(I_{i}^{\mathcal{P}})= \mathrm{IF}\;\mathrm{N}\bar{j}\neq 0\;\mathrm{GOTO}\;\mathrm{L}\bar{m}\wedge (x)_{j}\neq 0\right) \). Entonces
\(\displaystyle \begin{array}{rcl} s(i,x,y,\mathcal{P}) & =& \min_{l}\left( Lab(I_{l}^{\mathcal{P}})\neq \varepsilon \wedge Lab(I_{l}^{\mathcal{P}})\text{ }\mathrm{t}\text{ { -final} }I_{i}^{\mathcal{P}}\right) \\ S_{\#}(i,x,y,\mathcal{P}) & =& x \\ S_{\ast }(i,x,y,\mathcal{P}) & =& y \end{array} \)

(8) CASO \((\exists j\in \omega )\;Bas(I_{i}^{\mathcal{P}})=\mathrm{P} \bar{j}\leftarrow \mathrm{P}\bar{j}.a\). Entonces
\(\displaystyle \begin{array}{rcl} s(i,x,y,\mathcal{P}) & =& i+1 \\ S_{\#}(i,x,y,\mathcal{P}) & =& x \\ S_{\ast }(i,x,y,\mathcal{P}) & =& F_{a}(y,\#Var1(I_{i}^{\mathcal{P}})) \end{array} \)

(9) CASO \((\exists j\in \omega )\;Bas(I_{i}^{\mathcal{P}})=\mathrm{P} \bar{j}\leftarrow \) \(^{\curvearrowright }\mathrm{P}\bar{j}\). Entonces
\(\displaystyle \begin{array}{rcl} s(i,x,y,\mathcal{P}) & =& i+1 \\ S_{\#}(i,x,y,\mathcal{P}) & =& x \\ S_{\ast }(i,x,y,\mathcal{P}) & =& F_{\curvearrowright }(y,\#Var1(I_{i}^{ \mathcal{P}})) \end{array} \)

(10) CASO \((\exists j,k\in \omega )\;Bas(I_{i}^{\mathcal{P}})=\mathrm{ P}\bar{j}\leftarrow \mathrm{P}\bar{k}\). Entonces
\(\displaystyle \begin{array}{rcl} s(i,x,y,\mathcal{P}) & =& i+1 \\ S_{\#}(i,x,y,\mathcal{P}) & =& x \\ S_{\ast }(i,x,y,\mathcal{P}) & =& F_{\leftarrow }(y,\#Var1(I_{i}^{\mathcal{P} }),\#Var2(I_{i}^{\mathcal{P}})) \end{array} \)

(11) CASO \((\exists j\in \omega )\;Bas(I_{i}^{\mathcal{P}})=\mathrm{P} \bar{j}\leftarrow \varepsilon \). Entonces
\(\displaystyle \begin{array}{rcl} s(i,x,y,\mathcal{P}) & =& i+1 \\ S_{\#}(i,x,y,\mathcal{P}) & =& x \\ S_{\ast }(i,x,y,\mathcal{P}) & =& F_{0}(y,\#Var1(I_{i}^{\mathcal{P}})) \end{array} \)

(12) CASO \((\exists j,m\in \omega )(\exists a\in \Sigma )\;\left( Bas(I_{i}^{\mathcal{P}})=\mathrm{IF}\;\mathrm{P}\bar{j}\;\mathrm{BEGINS}\;a\; \mathrm{GOTO}\;\mathrm{L}\bar{m}\wedge \lbrack \ast ^{< }((y)_{j})]_{1}\neq a\right) \). Entonces
\(\displaystyle \begin{array}{rcl} s(i,x,y,\mathcal{P}) & =& i+1 \\ S_{\#}(i,x,y,\mathcal{P}) & =& x \\ S_{\ast }(i,x,y,\mathcal{P}) & =& y \end{array} \)

(13) CASO \((\exists j,m\in \omega )(\exists a\in \Sigma )\;\left( Bas(I_{i}^{\mathcal{P}})=\mathrm{IF\;P}\bar{j}\;\mathrm{BEGINS\;}a\;\mathrm{ GOTO\;L}\bar{m}\wedge \lbrack \ast ^{< }((y)_{j})]_{1}=a\right) \). Entonces
\(\displaystyle \begin{array}{rcl} s(i,x,y,\mathcal{P}) & =& \min_{l}\left( Lab(I_{l}^{\mathcal{P}})\neq \varepsilon \wedge Lab(I_{l}^{\mathcal{P}})\text{ }\mathrm{t}\text{ { -final} }I_{i}^{\mathcal{P}}\right) \\ S_{\#}(i,x,y,\mathcal{P}) & =& x \\ S_{\ast }(i,x,y,\mathcal{P}) & =& y \end{array} \)

(14) CASO \((\exists j\in \omega )\;Bas(I_{i}^{\mathcal{P}})=\mathrm{ GOTO}\) \(\mathrm{L}\bar{j}\). Entonces
\(\displaystyle \begin{array}{rcl} s(i,x,y,\mathcal{P}) & =& \min_{l}\left( Lab(I_{l}^{\mathcal{P}})\neq \varepsilon \wedge Lab(I_{l}^{\mathcal{P}})\text{ }\mathrm{t}\text{ { -final} }I_{i}^{\mathcal{P}}\right) \\ S_{\#}(i,x,y,\mathcal{P}) & =& x \\ S_{\ast }(i,x,y,\mathcal{P}) & =& y \end{array} \)

(15) CASO \(Bas(I_{i}^{\mathcal{P}})=\mathrm{SKIP}\). Entonces
\(\displaystyle \begin{array}{rcl} s(i,x,y,\mathcal{P}) & =& k+1 \\ S_{\#}(i,x,y,\mathcal{P}) & =& x \\ S_{\ast }(i,x,y,\mathcal{P}) & =& y \end{array} \)

O sea que los casos anteriores nos permiten definir conjuntos \( S_{1},...,S_{15}\), los cuales son disjuntos de a pares y cuya union da el conjunto \(\omega \times \mathbf{N}\times \mathbf{N}\times \mathrm{Pro} ^{\Sigma }\), de manera que cada una de las funciones \(s,S_{\#}\) y \(S_{\ast }\) pueden escribirse como union disjunta de funciones \((\Sigma \cup \Sigma _{p}) \)-p.r. restrinjidas respectivamente a los conjuntos \(S_{1},...,S_{15}\) . Ya que los conjuntos \(S_{1},...,S_{15}\) son \((\Sigma \cup \Sigma _{p})\) -p.r. el Lema 35 nos dice que \(s,S_{\#}\) y \(S_{\ast }\) lo son. \(\Box\)


\textbf{\underline{Lemma 63:}} Sean: ... hacer!!

\textbf{\underline{Proof:}} Hacer

\textbf{\underline{Proposition 64:}} Sean $n, m \leq 0$, las funciones $i^{n, m}, E_{\#j}^{n, m}, j = 1, 2, ...$ son
  $\Sigma \cup \Sigma_{p}$-PR.

\textbf{\underline{Proof:}} Sea $<$ un orden total estricto sobre $\Sigma \cup \Sigma_{p}$ y sean $s, S_{\#}, S_{\*}$ las funciones previamente definidas en el Lemma 62, definamos:

\(\displaystyle \begin{array}{rcl} C_{\#}^{n,m} & =& \lambda t\vec{x}\vec{\alpha}\mathcal{P}\left[ \left\langle E_{\#1}^{n,m}(t,\vec{x},\vec{\alpha},\mathcal{P}),E_{\#2}^{n,m}(t,\vec{x}, \vec{\alpha},\mathcal{P}),...\right\rangle \right] \\ C_{\ast }^{n,m} & =& \lambda t\vec{x}\vec{\alpha}\mathcal{P}\left[ \left\langle \#^{< }(E_{\ast 1}^{n,m}(t,\vec{x},\vec{\alpha},\mathcal{P} )),\#^{< }(E_{\ast 2}^{n,m}(t,\vec{x},\vec{\alpha},\mathcal{P} )),...\right\rangle \right] \end{array} \)

Notese que
\(\displaystyle \begin{array}{rcl} i^{n,m}(0,\vec{x},\vec{\alpha},\mathcal{P}) & =& 1 \\ C_{\#}^{n,m}(0,\vec{x},\vec{\alpha},\mathcal{P}) & =& \left\langle x_{1},...,x_{n}\right\rangle \\ C_{\ast }^{n,m}(0,\vec{x},\vec{\alpha},\mathcal{P}) & =& \left\langle \#^{< }(\alpha _{1}),...,\#^{< }(\alpha _{m})\right\rangle \\ i^{n,m}(t+1,\vec{x},\vec{\alpha},\mathcal{P}) & =& s(i^{n,m}(t,\vec{x},\vec{ \alpha},\mathcal{P}),C_{\#}^{n,m}(t,\vec{x},\vec{\alpha},\mathcal{P} ),C_{\ast }^{n,m}(t,\vec{x},\vec{\alpha},\mathcal{P})) \\ C_{\#}^{n,m}(t+1,\vec{x},\vec{\alpha},\mathcal{P}) & =& S_{\#}(i^{n,m}(t,\vec{x },\vec{\alpha},\mathcal{P}),C_{\#}^{n,m}(t,\vec{x},\vec{\alpha},\mathcal{P} ),C_{\ast }^{n,m}(t,\vec{x},\vec{\alpha},\mathcal{P})) \\ C_{\ast }^{n,m}(t+1,\vec{x},\vec{\alpha},\mathcal{P}) & =& S_{\ast }(i^{n,m}(t, \vec{x},\vec{\alpha},\mathcal{P}),C_{\#}^{n,m}(t,\vec{x},\vec{\alpha}, \mathcal{P}),C_{\ast }^{n,m}(t,\vec{x},\vec{\alpha},\mathcal{P})) \end{array} \)

Por el Lema 63 tenemos que \(i^{n,m}\), \( C_{\#}^{n,m}\) y \(C_{\ast }^{n,m}\) son \((\Sigma \cup \Sigma _{p})\)-p.r.. Ademas notese que
\(\displaystyle \begin{array}{rcl} E_{\#j}^{n,m} & =& \lambda t\vec{x}\vec{\alpha}\mathcal{P}\left[ (C_{\#}^{n,m}(t,\vec{x},\vec{\alpha},\mathcal{P}))_{j}\right] \\ E_{\ast j}^{n,m} & =& \lambda t\vec{x}\vec{\alpha}\mathcal{P}\left[ \ast ^{< }((C_{\ast }^{n,m}(t,\vec{x},\vec{\alpha},\mathcal{P}))_{j})\right] \end{array} \)

por lo cual las funciones \(E_{\#j}^{n,m}\), \(E_{\ast j}^{n,m}\), \(j=1,2,...\), son \((\Sigma \cup \Sigma _{p})\)-p.r. \(\Box\)


\textbf{\underline{Theorem 65:}} Las funciones \(\Phi _{\#}^{n,m}\) y \(\Phi _{\ast }^{n,m}\) son \((\Sigma \cup \Sigma _{p})\)-recursivas.

\textbf{\underline{Proof:}} Veremos que \(\Phi _{\#}^{n,m}\) es \((\Sigma \cup \Sigma _{p})\)-recursiva. Sea \(H\) el predicado \((\Sigma \cup \Sigma _{p})\)-mixto

\(\displaystyle \lambda t\vec{x}\vec{\alpha}\mathcal{P}\left[ i^{n,m}(t,x_{1},...,x_{n}, \alpha _{1},...,\alpha _{m},\mathcal{P})=n(\mathcal{P})+1\right] \text{.} \)

Note que \(D_{H}=\omega ^{n+1}\times \Sigma ^{\ast m}\times \mathrm{Pro} ^{\Sigma }\). Ya que the functiones \(i^{n,m}\) y \(\lambda \mathcal{P}\left[ n( \mathcal{P})\right] \) son \((\Sigma \cup \Sigma _{p})\)-p.r., \(H\) lo es. Notar que \(D_{M(H)}=D_{\Phi _{\#}^{n,m}}\). Ademas para \((\vec{x},\vec{\alpha}, \mathcal{P})\in D_{M(H)}\), tenemos que \(M(H)(\vec{x},\vec{\alpha},\mathcal{P} )\) es la menor cantidad de pasos necesarios para que \(\mathcal{P}\) termine partiendo del estado \(((x_{1},...,x_{n},0,0,...),(\alpha _{1},...,\alpha _{m},\varepsilon ,\varepsilon ,...))\). Ya que \(H\) es \((\Sigma \cup \Sigma _{p})\)-p.r., tenemos que \(M(H)\) es \((\Sigma \cup \Sigma _{p})\)-r.. Notese que para \((\vec{x},\vec{\alpha},\mathcal{P})\in D_{M(H)}=D_{\Phi _{\#}^{n,m}} \) tenemos que
\(\displaystyle \Phi _{\#}^{n,m}(\vec{x},\vec{\alpha},\mathcal{P})=E_{\#1}^{n,m}\left( M(H)( \vec{x},\vec{\alpha},\mathcal{P}),\vec{x},\vec{\alpha},\mathcal{P}\right) \)

lo cual con un poco mas de trabajo nos permite probar que
\(\displaystyle \Phi _{\#}^{n,m}=E_{\#1}^{n,m}\circ \left( M(H),p_{1}^{n,m+1},...,p_{n+m+1}^{n,m+1}\right) \)

Ya que la funcion \(E_{\#1}^{n,m}\) es \((\Sigma \cup \Sigma _{p})\)-r., lo es \( \Phi _{\#}^{n,m}\). \(\Box\)



\textbf{\underline{Corollary 66:}} Si \(f:D_{f}\subseteq \omega ^{n}\times \Sigma ^{\ast m}\rightarrow O\) es \( \Sigma \)-computable, entonces \(f\) es \(\Sigma \)-recursiva.

\textbf{\underline{Proof:}} Haremos el caso \(O=\Sigma ^{\ast }\). Sea \(\mathcal{P}_{0}\) un programa que compute a \(f\). Primero veremos que \(f\) es \((\Sigma \cup \Sigma _{p})\) -recursiva. Note que

\(\displaystyle f=\Phi _{\ast }^{n,m}\circ \left( p_{1}^{n,m},...,p_{n+m}^{n,m},C_{\mathcal{P }_{0}}^{n,m}\right) \)

donde cabe destacar que \(p_{1}^{n,m},...,p_{n+m}^{n,m}\) son las proyecciones respecto del alfabeto \(\Sigma \cup \Sigma _{p}\), es decir que tienen dominio \(\omega ^{n}\times (\Sigma \cup \Sigma _{p})^{\ast m}\). Ya que \(\Phi _{\ast }^{n,m}\) es \((\Sigma \cup \Sigma _{p})\)-recursiva tenemos que \(f\) lo es. O sea que el Teorema 51 nos dice que \(f\) es \(\Sigma \) -recursiva. \(\Box\)


\textbf{\underline{Tesis de Church:}} Toda funcion \(\Sigma \)-efectivamente computable es \(\Sigma \)-recursiva.


\textbf{\underline{Corollary 67:}} Si \(f:D_{f}\subseteq \omega ^{n}\times \Sigma ^{\ast m}\rightarrow O\) es \( \Sigma \)-recursiva, entonces existe un predicado \(\Sigma \)-p.r. \(P:\mathbf{N} \times \omega ^{n}\times \Sigma ^{\ast m}\rightarrow \omega \) y una funcion \( \Sigma \)-p.r. \(g:\mathbf{N}\rightarrow O\) tales que \(f=g\circ M(P).\)

\textbf{\underline{Proof:}} Supongamos que \(O=\Sigma ^{\ast }\). Sea \(\mathcal{P}_{0}\) un programa el cual compute a \(f\). Sea \(< \) un orden total estricto sobre \(\Sigma \). Note que podemos tomar

\(\displaystyle \begin{array}{rcl} P & =& \lambda t\vec{x}\vec{\alpha}[i^{n,m}\left( (t)_{1},\vec{x},\vec{\alpha}, \mathcal{P}_{0}\right) =n(\mathcal{P}_{0})+1\wedge (t)_{2}=\#^{< }(E_{\ast 1}^{n,m}((t)_{1},\vec{x},\vec{\alpha},\mathcal{P}_{0}))] \\ g & =& \lambda t\left[ \ast ^{< }((t)_{2})\right] \text{.} \end{array} \)

(Justifique por que \(P\) es \(\Sigma \)-p.r..) \(\Box\)


\textbf{\underline{Lemma 68:}} Supongamos \(f_{i}:D_{f_{i}}\subseteq \omega ^{n}\times \Sigma ^{\ast m}\rightarrow O\), \(i=1,...,k\), son funciones \(\Sigma \)-recursivas tales que \(D_{f_{i}}\cap D_{f_{j}}=\varnothing \) para \(i\neq j\). Entonces la funcion \(f_{1}\cup ...\cup f_{k}\) es \(\Sigma \)-recursiva.

\textbf{\underline{Proof:}} Probaremos el caso \(k=2\) y \(O=\Sigma ^{\ast }\). Sean \(\mathcal{P}_{1}\) y \( \mathcal{P}_{2}\) programas que computen las funciones \(f_{1}\) y \(f_{2}\), respectivamente. Sean

\(\displaystyle \begin{array}{rcl} P_{1} & =& \lambda t\vec{x}\vec{\alpha}\left[ i^{n,m}(t,\vec{x},\vec{\alpha}, \mathcal{P}_{1})=n(\mathcal{P}_{1})+1\right] \\ P_{2} & =& \lambda t\vec{x}\vec{\alpha}\left[ i^{n,m}(t,\vec{x},\vec{\alpha}, \mathcal{P}_{2})=n(\mathcal{P}_{2})+1\right] \end{array} \)

Notese que \(D_{P_{1}}=D_{P_{2}}=\omega \times \omega ^{n}\times \Sigma ^{\ast m}\) y que \(P_{1}\) y \(P_{2}\) son \((\Sigma \cup \Sigma _{p})\)-p.r.. Ya que son \(\Sigma \)-mixtos, el Teorema 51 nos dice que son \( \Sigma \)-p.r.. Tambien notese que \(D_{M((P_{1}\vee P_{2}))}=D_{f_{1}}\cup D_{f_{2}}\). Definamos
\(\displaystyle \begin{array}{rcl} g_{1} & =& \lambda \vec{x}\vec{\alpha}\left[ E_{\ast 1}^{n,m}(M\left( (P_{1}\vee P_{2})\right) (\vec{x},\vec{\alpha}),\vec{x},\vec{\alpha}, \mathcal{P}_{1})^{P_{i}(M\left( (P_{1}\vee P_{2})\right) (\vec{x},\vec{\alpha }),\vec{x},\vec{\alpha})}\right] \\ g_{2} & =& \lambda \vec{x}\vec{\alpha}\left[ E_{\ast 1}^{n,m}(M\left( (P_{1}\vee P_{2})\right) (\vec{x},\vec{\alpha}),\vec{x},\vec{\alpha}, \mathcal{P}_{2})^{P_{i}(M\left( (P_{1}\vee P_{2})\right) (\vec{x},\vec{\alpha }),\vec{x},\vec{\alpha})}\right] \end{array} \)

Notese que \(g_{1}\) y \(g_{2}\) son \(\Sigma \)-recursivas y que \( D_{g_{1}}=D_{g_{2}}=D_{f_{1}}\cup D_{f_{2}}\), Ademas notese que
\(\displaystyle g_{1}(\vec{x},\vec{\alpha})=\left\{ \begin{array}{lll} f_{1}(\vec{x},\vec{\alpha}) & & \text{si }(\vec{x},\vec{\alpha})\in D_{f_{1}} \\ \varepsilon & & \text{caso contrario} \end{array} \right. \)

\(\displaystyle g_{2}(\vec{x},\vec{\alpha})=\left\{ \begin{array}{lll} f_{2}(\vec{x},\vec{\alpha}) & & \text{si }(\vec{x},\vec{\alpha})\in D_{f_{2}} \\ \varepsilon & & \text{caso contrario} \end{array} \right. \)

O sea que \(f_{1}\cup f_{2}=\lambda \alpha \beta \left[ \alpha \beta \right] \circ (g_{1},g_{2})\) es \(\Sigma \)-recursiva. \(\Box\)


\textbf{\underline{Lemma 69:}} Supongamos \(\Sigma \supseteq \Sigma _{p}\). Entonces \( Halt^{\Sigma }\) es no \(\Sigma \)-recursivo.

\textbf{\underline{Proof:}} Supongamos \(Halt^{\Sigma }\) es \(\Sigma \)-recursivo y por lo tanto \(\Sigma \) -computable. Por la proposicion de existencia de macros tenemos que hay un macro

\(\displaystyle \left[ \mathrm{IF}\;Halt^{\Sigma }(\mathrm{W}1)\;\mathrm{GOTO}\;\mathrm{A}1 \right] \)

Sea \(\mathcal{P}_{0}\) el siguiente programa de \(\mathcal{S}^{\Sigma }\)
\(\displaystyle \mathrm{L}1\;\left[ \mathrm{IF}\;Halt^{\Sigma }(\mathrm{P}1)\;\mathrm{GOTO}\; \mathrm{L}1\right] \)

Note que
- \(\mathcal{P}_{0}\) termina partiendo desde \(\left( (0,0,...),( \mathcal{P}_{0},\varepsilon ,\varepsilon ,...)\right) \) sii \(Halt^{\Sigma }( \mathcal{P}_{0})=0\),
lo cual produce una contradiccion si tomamos en (*) \(\mathcal{P}= \mathcal{P}_{0}\). \(\Box\)

\section{Conjuntos $\Sigma$-recursivamente enumerables}

\textbf{\underline{Theorem 70:}} Sea \(S\subseteq \omega ^{n}\times \Sigma ^{\ast m}\). Entonces \(S\) es \(\Sigma \)-efectivamente enumerable sii \(S\) es \(\Sigma \)-recursivamente enumerable


\textbf{\underline{Proof:}} (\(\Rightarrow \)) Use la Tesis de Church.

(\(\Leftarrow \)) Use el Theorem 42. \(\Box\)



\textbf{\underline{Theorem 71:}} Dado \(S\subseteq \omega ^{n}\times \Sigma ^{\ast m} \), son equivalentes
(1) \(S\) es \(\Sigma \)-recursivamente enumerable
(2) \(S=I_{F}\), para alguna \(F:D_{F}\subseteq \omega ^{k}\times \Sigma ^{\ast l}\rightarrow \omega ^{n}\times \Sigma ^{\ast m}\) tal que cada \(F_{i}\) es \(\Sigma \)-recursiva.
(3) \(S=D_{f}\), para alguna funcion \(\Sigma \)-recursiva \(f\)
(4) \(S=\varnothing \) o \(S=I_{F}\), para alguna \(F:\omega \rightarrow \omega ^{n}\times \Sigma ^{\ast m}\) tal que cada \(F_{i}\) es \(\Sigma \)-p.r.


\textbf{\underline{Proof:}} (2)\(\Rightarrow \)(3). Para \(i=1,...,n+m\), sea \(\mathcal{P}_{i}\) un programa el cual computa a \(F_{i}\) y sea \(< \) un orden total estricto sobre \(\Sigma \). Sea \(P:\mathbf{N}\times \omega ^{n}\times \Sigma ^{\ast m}\rightarrow \omega \) dado por \(P(t,\vec{x},\vec{\alpha})=1\) sii se cumplen las siguientes condiciones

\(\displaystyle \begin{array}{rcl} i^{k,l}(\left( (t)_{k+l+1},(t)_{1},...,(t)_{k},\ast ^{< }((t)_{k+1}),...,\ast ^{< }((t)_{k+l})),\mathcal{P}_{1}\right) & =& n(\mathcal{P}_{1})+1 \\ & & \vdots \\ i\left( (t)_{k+l+1},(t)_{1}...(t)_{k},\ast ^{< }((t)_{k+1})...\ast ^{< }((t)_{k+l})),\mathcal{P}_{n+m}\right) & =& n(\mathcal{P}_{n+m})+1 \\ E_{\#1}^{k,l}((t)_{k+l+1},(t)_{1},...,(t)_{k},\ast ^{< }((t)_{k+1}),...,\ast ^{< }((t)_{k+l})),\mathcal{P}_{1}) & =& x_{1} \\ & & \vdots \\ E_{\#1}^{k,l}((t)_{k+l+1},(t)_{1},...,(t)_{k},\ast ^{< }((t)_{k+1}),...,\ast ^{< }((t)_{k+l})),\mathcal{P}_{n}) & =& x_{n} \\ E_{\ast 1}^{k,l}((t)_{k+l+1},(t)_{1},...,(t)_{k},\ast ^{< }((t)_{k+1}),...,\ast ^{< }((t)_{k+l})),\mathcal{P}_{n+1}) & =& \alpha _{1} \\ & & \vdots \\ E_{\ast 1}^{k,l}((t)_{k+l+1},(t)_{1},...,(t)_{k},\ast ^{< }((t)_{k+1}),...,\ast ^{< }((t)_{k+l})),\mathcal{P}_{n+m}) & =& \alpha _{m} \end{array} \)

Note que \(P\) es \((\Sigma \cup \Sigma _{p})\)-p.r. y por lo tanto \(P\) es \( \Sigma \)-p.r.. Pero entonces \(M(P)\) es \(\Sigma \)-r. lo cual nos dice que se cumple (3) ya que \(D_{M(P)}=I_{F}=S\).
(3)\(\Rightarrow \)(4). Supongamos \(S\neq \varnothing \). Sea \( (z_{1},...,z_{n},\gamma _{1},...,\gamma _{m})\in S\) fijo. Sea \(\mathcal{P}\) un programa el cual compute a \(f\) y sea \(< \) un orden total estricto sobre \( \Sigma \). Sea \(P:\mathbf{N}\rightarrow \omega \) dado por \(P(x)=1\) sii

\(\displaystyle i^{n,m}\left( (x)_{n+m+1},(x)_{1},...,(x)_{n},\ast ^{< }((x)_{n+1}),...,\ast ^{< }((x)_{n+m})),\mathcal{P}\right) =n(\mathcal{P})+1 \)

Es facil ver que \(P\) es \((\Sigma \cup \Sigma _{p})\)-p.r. por lo cual es \( \Sigma \)-p.r.. Sea \(\bar{P}=P\cup C_{0}^{1,0}\mid _{\{0\}}\). Para \(i=1,...,n\) , definamos \(F_{i}:\omega \rightarrow \omega \) de la siguiente manera
\(\displaystyle F_{i}(x)=\left\{ \begin{array}{ccc} (x)_{i} & \text{si} & \bar{P}(x)=1 \\ z_{i} & \text{si} & \bar{P}(x)\neq 1 \end{array} \right. \)

Para \(i=n+1,...,n+m\), definamos \(F_{i}:\omega \rightarrow \Sigma ^{\ast }\) de la siguiente manera
\(\displaystyle F_{i}(x)=\left\{ \begin{array}{lll} \ast ^{< }((x)_{i}) & \text{si} & \bar{P}(x)=1 \\ \gamma _{i-n} & \text{si} & \bar{P}(x)\neq 1 \end{array} \right. \)

Por el lema de division por casos, cada \(F_{i}\) es \(\Sigma \)-p.r.. Es facil ver que \(F=(F_{1},...,F_{n+m})\) cumple (4). \(\Box\)


\textbf{\underline{Corollary 72:}} Supongamos \(f:D_{f}\subseteq \omega ^{n}\times \Sigma ^{\ast m}\rightarrow O\) es \(\Sigma \)-recursiva y \(S\subseteq D_{f}\) es \( \Sigma \)-r.e., entonces \(f\mid _{S}\) es \(\Sigma \)-recursiva.


\textbf{\underline{Proof:}} Supongamos \(O=\Sigma ^{\ast }.\) Por el Theorem anterior \(S=D_{g}\), para alguna funcion \(\Sigma \)-recursiva \(g.\) Notese que componiendo adecuadamente podemos suponer que \(I_{g}=\{\varepsilon \}.\) O sea que tenemos \(f\mid _{S}=\lambda \alpha \beta \left[ \alpha \beta \right] \circ (f,g)\). \(\Box\)


\textbf{\underline{Corollary 73:}} Supongamos \(f:D_{f}\subseteq \omega ^{n}\times \Sigma ^{\ast m}\rightarrow O\) es \(\Sigma \)-recursiva y \(S\subseteq I_{f}\) es \(\Sigma \)-r.e., entonces \( f^{-1}(S)=\{(\vec{x},\vec{\alpha}):f(\vec{x},\vec{\alpha})\in S\}\) es \( \Sigma \)-r.e..

\textbf{\underline{Proof:}} Por el Theorem anterior \(S=D_{g}\), para alguna funcion \(\Sigma \)-recursiva \( g \). O sea que \(f^{-1}(S)=D_{g\circ f}\) es \(\Sigma \)-r.e.. \(\Box\)


\textbf{\underline{Corollary 74:}} Supongamos \(S_{1},S_{2}\subseteq \omega ^{n}\times \Sigma ^{\ast m}\) son conjuntos \(\Sigma \)-r.e.. Entonces \(S_{1}\cap S_{2}\) es \(\Sigma \)-r.e..

\textbf{\underline{Proof:}} Por el Theorem anterior \(S_{i}=D_{g_{i}}\), con \(g_{1},g_{2}\) funciones \( \Sigma \)-recursivas\(.\) Notese que podemos suponer que \(I_{g_{1}},I_{g_{2}} \subseteq \omega \) por lo que \(S_{1}\cap S_{2}=D_{\lambda xy\left[ xy\right] \circ (g_{1},g_{2})}\) es \(\Sigma \)-r.e.\(.\) \(\Box\)


\textbf{\underline{Corollary 75:}} Supongamos \(S_{1},S_{2}\subseteq \omega ^{n}\times \Sigma ^{\ast m}\) son conjuntos \(\Sigma \)-r.e.. Entonces \(S_{1}\cup S_{2}\) es \(\Sigma \)-r.e.

\textbf{\underline{Proof:}} Supongamos \(S_{1}\neq \varnothing \neq S_{2}.\) Sean \(F,G:\omega \rightarrow \omega ^{n}\times \Sigma ^{\ast m}\) tales que \(I_{F}=S_{1}\), \(I_{G}=S_{2}\) y las funciones \(F_{i} {\acute{}} s\) y \(G_{i} {\acute{}} s\) son \(\Sigma \)-recursivas. Sean \(f=\lambda x\left[ Q(x,2)\right] \) y \( g=\lambda x\left[ Q(x\dot{-}1,2)\right] .\) Sea \(H:\omega \rightarrow \omega ^{n}\times \Sigma ^{\ast m}\) dada por

\(\displaystyle H_{i}=(F_{i}\circ f)\mathrm{\mid }_{\{x:x\mathrm{\ es\ par}\}}\cup (G_{i}\circ g)\mathrm{\mid }_{\{x:x\mathrm{\ es\ impar}\}} \)

Por el Corollary 72 y el Lema 68, cada \(H_{i}\) es \( \Sigma \)-recursiva. Ya que \(I_{H}=S_{1}\cup S_{2}\).tenemos que \(S_{1}\cup S_{2}\) es \(\Sigma \)-r.e. \(\Box\)


\textbf{\underline{Theorem 76:}} Sea \(S\subseteq \omega ^{n}\times \Sigma ^{\ast m}\). Entonces \(S\) es \(\Sigma \)-efectivamente computable sii \(S\) es \(\Sigma \)-recursivo
\textbf{\underline{Proof:}} (\(\Rightarrow \)) Use la Tesis de Church.

(\(\Leftarrow \)) Use el Teorema 42. \(\Box\)




\textbf{\underline{Theorem 77:}} Sea \(S\subseteq \omega ^{n}\times \Sigma ^{\ast m}.\) Son equivalentes
(a) \(S\) es \(\Sigma \)-recursivo
(b) \(S\) y \((\omega ^{n}\times \Sigma ^{\ast m})-S\) son \(\Sigma \) -recursivamente enumerables


\textbf{\underline{Proof:}} (a)\(\Rightarrow \)(b)\(.\) Note que \(S=D_{Pred\circ \chi _{S}}.\)

(b)\(\Rightarrow \)(a). Note que \(\chi _{S}=C_{1}^{n,m}\mathrm{\mid }_{S}\cup C_{0}^{n,m}\mathrm{\mid }_{\omega ^{n}\times \Sigma ^{\ast m}-S}\). \(\Box\)

\textbf{\underline{Lemma 78:}} Supongamos que \(\Sigma \supseteq \Sigma _{p}.\) Entonces
\(\displaystyle A=\left\{ \mathcal{P}\in \mathrm{Pro}^{\Sigma }:Halt^{\Sigma }(\mathcal{P} )\right\} \)

es \(\Sigma \)-r.e. y no es \(\Sigma \)-recursivo. Mas aun el conjunto
\(\displaystyle N=\left\{ \mathcal{P}\in \mathrm{Pro}^{\Sigma }:\lnot Halt^{\Sigma }( \mathcal{P})\right\} \)
no es \(\Sigma \)-r.e.


\textbf{\underline{Proof:}} Sea \(P=\lambda t\mathcal{P}\left[ i^{0,1}(t,\mathcal{P},\mathcal{P})=n( \mathcal{P})+1\right] \). Note que \(P\) es \(\Sigma \)-p.r. por lo que \(M(P)\) es \(\Sigma \)-r.. Ademas note que \(D_{M(P)}=A\), lo cual implica que \(A\) es \( \Sigma \)-r.e.. Ya que \(Halt^{\Sigma }\) es no \(\Sigma \)-recursivo (Lema 69) y

\(\displaystyle Halt^{\Sigma }=C_{1}^{0,1}\mid _{A}\cup C_{0}^{0,1}\mid _{N} \)

el Lema 68 nos dice que \(N\) no es \(\Sigma \)-r.e.. Finalmente supongamos \(A\) es \(\Sigma \)-recursivo. Entonces el conjunto
\(\displaystyle N=\left( \Sigma ^{\ast }-A\right) \cap \mathrm{Pro}^{\Sigma } \)

deberia serlo, lo cual es absurdo. \(\Box\)

\section{Maquinas de Turing}


\textbf{\underline{Lemma 79:}} Sea \(L\subseteq \Sigma ^{\ast }.\) entonces \(L=L(M)\) para alguna maquina de Turing \(M\) sii \(L=H(M)\) para alguna maquina de Turing \(M=(Q,\Sigma ,\Gamma ,\delta ,q_{0},B,F)\).

\textbf{\underline{Proof:}} (\(\Rightarrow \)) Dada una maquina \(M=(Q,\Sigma ,\Gamma ,\delta ,q_{0},B,F)\), costruiremos una maquina \(M_{1}=(Q_{1},\Sigma ,\Gamma _{1},\delta _{1}, \tilde{q}_{0},B,\varnothing )\) tal que \(L(M)=H(M_{1}).\) Tomaremos \(\Gamma _{1}=\Gamma \cup \{X\}\), con \(X\) un simbolo nuevo no perteneciente a \(\Gamma \). Para cada \(a\in \Sigma \), sea \(q_{a}\) un estado nuevo, no perteneciente a \(Q.\) Sean \(\tilde{q}_{0},q_{r},q_{d},q_{B}\) estados nuevos no pertenecientes a \(Q.\) Tomemos entonces

\(\displaystyle Q_{1}=Q\cup \{\tilde{q}_{0},q_{r},q_{d},q_{B}\}\cup \{q_{a}:a\in \Sigma \} \)

Finalmente definamos \(\delta _{1}\) de la siguiente manera:
\(\displaystyle \begin{array}{rcl} \delta _{1}(\tilde{q}_{0},B) & =& \{(q_{B},X,R)\} \\ \delta _{1}(q_{B},a) & =& \{(q_{a},B,R)\}\text{, para }a\in \Sigma \\ \delta _{1}(q_{B},B) & =& \{(q_{0},B,K)\} \\ \delta _{1}(q_{a},b) & =& \{(q_{b},a,R)\}\text{, para }a,b\in \Sigma \\ \delta _{1}(q_{a},B) & =& \{(q_{r},a,L)\}\text{, para }a\in \Sigma \\ \delta _{1}(q_{r},a) & =& \{(q_{r},a,L)\}\text{, para }a\in \Sigma \\ \delta _{1}(q_{r},B) & =& \{(q_{0},B,K)\} \\ \delta _{1}(q,X) & =& \{(q,X,K)\}\text{, para }q\in Q \\ \delta _{1}(q,\sigma ) & =& \delta (q,\sigma )\cup \{(q_{d},\sigma ,K)\}\text{ , para }q\in F\text{ y }\sigma \in \Gamma \\ \delta _{1}(q,\sigma ) & =& \delta (q,\sigma )\text{, para }q\in Q-F\text{ y } \sigma \in \Gamma \\ \delta _{1}(q_{d},\sigma ) & =& \varnothing \text{, para }\sigma \in \Gamma \end{array} \)

(\(\delta _{1}\) se define igual a vacio para los casos no contemplados arriba).
(\(\Leftarrow \)) Dada \(M=(Q,\Sigma ,\Gamma ,\delta ,q_{0},B,F)\), dejamos al lector la construccion de una maquina \(M_{1}=(Q_{1},\Sigma ,\Gamma _{1},\delta _{1},\tilde{q}_{0},B,\varnothing )\) tal que \(H(M)=L(M_{1})\). \(\Box\)


\textbf{\underline{Lemma 80:}} El predicado \(\lambda ndd^{\prime }\left[ d\vdash d^{\prime }\right] \) es \( (\Gamma \cup Q)\)-p.r..

\textbf{\underline{Proof:}} Note que \(D_{\lambda dd^{\prime }\left[ d\vdash d^{\prime }\right] }=Des\times Des\). Tambien notese que los predicados

\(\displaystyle \begin{array}{rcl} & & \lambda p\sigma q\gamma \left[ (p,\sigma ,L)\in \delta (q,\gamma )\right] \\ & & \lambda p\sigma q\gamma \left[ (p,\sigma ,R)\in \delta (q,\gamma )\right] \\ & & \lambda p\sigma q\gamma \left[ (p,\sigma ,K)\in \delta (q,\gamma )\right] \end{array} \)

son \((\Gamma \cup Q)\)-p.r. ya que los tres tienen dominio igual a \(Q\times \Gamma \times Q\times \Gamma \) el cual es finito (Corolario 36 ). Sea \(P_{R}:Des\times Des\times \Gamma \times \Gamma ^{\ast }\times \Gamma ^{\ast }\times Q\times Q\rightarrow \omega \) definido por \(P_{R}(d,d^{\prime },\sigma ,\alpha ,\beta ,p,q)=1\) sii
\(\displaystyle d=\alpha p\beta \wedge (q,\sigma ,R)\in \delta \left( p,\left[ \beta B\right] _{1}\right) \wedge d^{\prime }=\alpha \sigma q^{\curvearrowright }\beta \)

Sea \(P_{L}:Des\times Des\times \Gamma \times \Gamma ^{\ast }\times \Gamma ^{\ast }\times Q\times Q\rightarrow \omega \) definido por \(P_{L}(d,d^{\prime },\sigma ,\alpha ,\beta ,p,q)=1\) sii
\(\displaystyle d=\alpha p\beta \wedge (q,\sigma ,L)\in \delta \left( p,\left[ \beta B\right] _{1}\right) \wedge \alpha \neq \varepsilon \wedge d^{\prime }=\left\lfloor \alpha ^{\curvearrowleft }q\left[ \alpha \right] _{\left\vert \alpha \right\vert }\sigma ^{\curvearrowright }\beta \right\rfloor \)

Sea \(P_{K}:Des\times Des\times \Gamma \times \Gamma ^{\ast }\times \Gamma ^{\ast }\times Q\times Q\rightarrow \omega \) definido por \(P_{K}(d,d^{\prime },\sigma ,\alpha ,\beta ,p,q)=1\) sii
\(\displaystyle d=\alpha p\beta \wedge (q,\sigma ,K)\in \delta \left( p,\left[ \beta B\right] _{1}\right) \wedge d^{\prime }=\left\lfloor \alpha q\sigma ^{\curvearrowright }\beta \right\rfloor \)

Se deja al lector la verificacion de que estos predicados son \((\Gamma \cup Q)\)-p.r.. Notese que \(\lambda dd^{\prime }\left[ d\vdash d^{\prime }\right] \) es igual al predicado
\(\displaystyle \lambda dd^{\prime }\left[ (\exists \sigma \in \Gamma )(\exists \alpha ,\beta \in \Gamma ^{\ast })(\exists p,q\in Q)(P_{R}\vee P_{L}\vee P_{K})(d,d^{\prime },\sigma ,\alpha ,\beta ,p,q)\right] \)

lo cual por el Lema 39 nos dice que \(\lambda dd^{\prime } \left[ d\vdash d^{\prime }\right] \) es \((\Gamma \cup Q)\)-p.r. \(\Box\)


\textbf{\underline{Proposition 81:}} \(\lambda ndd^{\prime }\left[ d\overset{n}{\vdash }d^{\prime }\right] \) es \( (\Gamma \cup Q)\)-p.r..

\textbf{\underline{Proof:}} Sea \(Q=\lambda dd^{\prime }\left[ d\vdash d^{\prime }\right] \cup C_{0}^{0,2}\mid _{(\Gamma \cup Q)^{\ast 2}-Des^{2}}\) es decir \(Q\) es el resultado de extender con el valor \(0\) al predicado \(\lambda dd^{\prime } \left[ d\vdash d^{\prime }\right] \) de manera que este definido en todo \( (\Gamma \cup Q)^{\ast 2}\). Sea \(< \) un orden total estricto sobre \(\Gamma \cup Q\) y sea \(Q_{1}:\mathbf{N}\times Des\times Des\rightarrow \omega \) definido por \(Q_{1}(x,d,d^{\prime })=1\) sii

\(\left( (\forall i\in \mathbf{N})_{i\leq Lt(x)}\ast ^{< }((x)_{i})\in Des\right) \wedge \ast ^{< }((x)_{1})=d\wedge \)

\(\ \ \ \ \ \ \ \ \ \ \ \ \ \ \ \ \ \ \ \ \ \ \ast ^{< }((x)_{Lt(x)})=d^{\prime }\wedge \left( (\forall i\in \mathbf{N})_{i\leq Lt(x)\dot{-}1}\;Q(\ast ^{< }((x)_{i}),\ast ^{< }((x)_{i+1}))\right) \)

Notese que dicho rapidamente \(Q_{1}(x,d,d^{\prime })=1\) sii \(x\) codifica una computacion que parte de \(d\) y llega a \(d^{\prime }\). Se deja al lector la verificacion de que este predicado es \((\Gamma \cup Q)\)-p.r.. Notese que

\(\displaystyle \lambda ndd^{\prime }\left[ d\overset{n}{\vdash }d^{\prime }\right] =\lambda ndd^{\prime }\left[ \left( \exists x\in \mathbf{N}\right) \;Lt(x)=n+1\wedge Q_{1}(x,d,d^{\prime })\right] \)

Es decir que solo nos falta acotar el cuantificador existencial, para poder aplicar el lema de cuantificacion acotada. Ya que cuando \( d_{1},...,d_{n+1}\in Des\) son tales que \(d_{1}\vdash d_{2}\vdash ...\vdash d_{n+1}\) tenemos que
\(\displaystyle \left\vert d_{i}\right\vert \leq \left\vert d_{1}\right\vert +n\text{, para } i=1,...,n \)

una posible cota para dicho cuantificador es
\(\displaystyle \prod_{i=1}^{n+1}pr(i)^{\left\vert \Gamma \cup Q\right\vert ^{\left\vert d\right\vert +n}}\text{.} \)

O sea que, por el lema de cuantificacion acotada, tenemos que el predicado \( \lambda ndd^{\prime }\left[ d\overset{n}{\vdash }d^{\prime }\right] \) es \( (\Gamma \cup Q)\)-p.r. \(\Box\)

\textbf{\underline{Theorem 82:}} Sea \(M=\left( Q,\Sigma ,\Gamma ,\delta ,q_{0},B,F\right) \) una maquina de Turing. Entonces \(L(M)\) es \(\Sigma \)-recursivamente enumerable.


\textbf{\underline{Proof:}} Sea \(P\) el siguiente predicado \((\Gamma \cup Q)\)-mixto

\(\displaystyle \lambda n\alpha \left[ (\exists d\in Des)\;\left\lfloor q_{0}B\alpha \right\rfloor \overset{n}{\vdash }d\wedge St(d)\in F\right] \)

Notese que \(D_{P}=\omega \times \Gamma ^{\ast }\). Dejamos al lector probar que \(P\) es \((\Gamma \cup Q)\)-p.r.. Sea \(P^{\prime }=P\mid _{\omega \times \Sigma ^{\ast }}\). Notese que \(P^{\prime }(n,\alpha )=1\) sii \(\alpha \in L(M) \) atestiguado por una computacion de longitud \(n\). Ya que \(P^{\prime }\) es \((\Gamma \cup Q)\)-p.r. (por que?) y ademas es \(\Sigma \)-mixto, el Teorema 51 nos dice que \(P^{\prime }\) es \(\Sigma \)-p.r.. Ya que \( L(M)=D_{M(P^{\prime })}\), el Teorema 71 nos dice que \( L(M)\) es \(\Sigma \)-r.e.. \(\Box\)

\include{sigma_turing_computable_functions}

\begin{thebibliography}{X}
\bibitem{Baz} \textsc{Diego Vaggione},
<<Apunte de Clase, 2017>>,
\textit{FaMAF, UNC}.
\bibitem{Baz} \textsc{Agustín Curto},
<<Carpeta de Clase, 2017>>,
\textit{FaMAF, UNC}.
\end{thebibliography}

\vspace{\fill}
\begin{center}
Por favor, mejorá este documento en github
\includegraphics[width=1cm]{graphics/github.png} \\
https://github.com/acurto714/resumenLengForm
\end{center}
\end{document}
