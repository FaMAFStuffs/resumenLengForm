\documentclass[12pt,a4paper]{article}
\usepackage[utf8]{inputenc}
\usepackage{amsmath}
\usepackage{amsfonts}
\usepackage{amssymb}
\usepackage{lmodern}
\usepackage{amsmath}
\usepackage{amsthm}
\usepackage{enumerate}
\usepackage[left=2cm,right=2cm,top=2cm,bottom=2cm]{geometry}
\usepackage{graphicx}
\usepackage{mathtools}
\usepackage{stackrel}
\newcounter{neq}
\providecommand{\abs}[1]{\lvert#1\rvert}
\newcommand{\SIGMA}{\Sigma^{\ast}}
\newcommand{\PN}{\par\noindent}

\newtheorem{theorem}[equation]{Theorem}
\newtheorem{lemma}[equation]{Lemma}
\newtheorem{proposition}[equation]{Proposition}
\newtheorem{corollary}[equation]{Corollary}

\author{Agustín Curto, agucurto95@gmail.com \\
			 Francisco Nievas, frannievas@gmail.com}
\title{Resumen de teoremas para el final \\ de Lenguajes Formales y Computabilidad}
\date{2017}

\begin{document}
% \clearpage\maketitle
% \thispagestyle{empty}
% \tableofcontents
% \pagebreak
%
% \section{Notación y conceptos básicos}

  % Lemma 1
  \begin{lemma}
    \par Sea $S \subseteq \omega \times \SIGMA$, entonces $S$ es rectangular si y solo si se cumple la siguiente
    propiedad:

    \[
      \text{Si } (x, \alpha), (y, \beta) \in S \Rightarrow (x, \beta) \in S
    \]
  \end{lemma}

  % Lemma 2
  \begin{lemma}
    \par La relación $<$ es un orden total estricto sobre $\SIGMA$.
  \end{lemma}

  % Lemma 3
  \begin{lemma}
    \par La función $s^{<}: \SIGMA \rightarrow \SIGMA$, definida recursivamente de la siguiente manera:

    \begin{eqnarray}
  		\nonumber s^{<}(\varepsilon) &=& a_{1} \\
  		\nonumber s^{<}(\alpha a_{i}) &=& \alpha a_{i+1}, \;\; i < n \\
  		\nonumber s^{<}(\alpha a_{n}) &=& s^{<}(\alpha) a_{1}
    \end{eqnarray}

    \par tiene la siguiente propiedad:

    \[
      s^{<}(\alpha) = \min \{\beta \in \SIGMA: \alpha < \beta\}
    \]
  \end{lemma}

  % Corollary 4
  \begin{corollary}
    \par $s^{<}$ es inyectiva.
  \end{corollary}

  % Lemma 5
  \begin{lemma}
    \par Se tiene que:

    \begin{enumerate}
      \item $\varepsilon \neq s^{<}(\alpha)$, para cada $\alpha \in \SIGMA$.
      \item Si $\alpha \neq \varepsilon$, entonces $\alpha = s^{<}(\beta)$ para algún $\beta$.
      \item Si $S\subseteq \SIGMA \neq \emptyset \Rightarrow \exists \alpha \in S$ tal que $\alpha < \beta$, para cada
      $\beta \in S-\{\alpha\}$.
    \end{enumerate}
  \end{lemma}

  % Lemma 6
  \begin{lemma}
    \par Tenemos que:

    \[
      \SIGMA = \{\ast^{<}(0), \ast^{<}(1), \dotsc\}
    \]

    \par Mas aún la función $\ast^{<}$ es biyectiva.
  \end{lemma}

  % Lemma 7
  \begin{lemma}
    \par Sea $n \geq 1$ fijo, entonces cada $x \geq 1$ se escribe en forma única de la siguiente manera:

    \[
      x = i_{k} n^{k} + i_{k-1} n^{k-1} + \dotsc + i_{0} n^{0}
    \]

    \par con $k\geq 0$ y $1 \leq i_{k}, i_{k-1}, \dotsc, i_{0} \leq n$.
  \end{lemma}

  % Lemma 8
  \begin{lemma}
    \par La función $\#^{<}$ es biyectiva.
  \end{lemma}

  % Lemma 9
  \begin{lemma}
    \par Las funciones $\#^{<}$ y $\ast^{<}$ son una inversa de la otra.
  \end{lemma}

  % Lemma 10
  \begin{lemma}
    \par Si $p, p_{1}, \dotsc, p_{n}$ son números primos y $p$ divide a $p_{1} \dotsc p_{n}$, entonces $p = p_{i}$,
    para algún $i$.
  \end{lemma}

  % Theorem 11
  \begin{theorem}
    \par Para cada $x \in \mathbf{N}$, hay una única sucesión $(s_{1}, s_{2}, \dotsc) \in
    \omega^{\left[\mathbf{N}\right]}$ tal que:

    \[
      x = \underset{i=1}{\overset{\infty}{\Pi}} pr(i)^{s_{i}}
    \]

    \par Notese que $\underset{i=1}{\overset{\infty}{\Pi}} pr(i)^{s_{i}}$ tiene sentido ya que es un producto que solo
    tiene una cantidad finita de factores no iguales a $1$.
  \end{theorem}

  % Lemma 12
  \begin{lemma}
    \par Las funciones:

    \begin{eqnarray}
    	\nonumber \mathbf{N} &\rightarrow& \omega^{\left[\mathbf{N}\right]} \\
    	\nonumber x &\rightarrow& ((x)_{1}, (x)_{2}, \dotsc) \\
      \nonumber \\
      \nonumber \omega^{\left[\mathbf{N}\right]} &\rightarrow& \mathbf{N} \\
      \nonumber (s_{1}, s_{2}, \dotsc) &\rightarrow& \left\langle s_{1}, s_{2}, \dotsc \right\rangle
  	\end{eqnarray}

    \par son biyecciones una inversa de la otra.
  \end{lemma}

  % Lemma 13
  \begin{lemma}
    \par Para cada $x \in \mathbf{N}$:

    \begin{enumerate}
      \item $Lt(x) = 0 \Leftrightarrow x = 1$
      \item $x = \prod\nolimits_{i=1}^{Lt(x)} pr(i)^{(x)_{i}}$
    \end{enumerate}

    \par Cabe destacar entonces que la función $\lambda ix[(x)_{i}]$ tiene dominio igual a $\mathbf{N}^{2}$ y la
    función $\lambda ix[Lt(x)]$ tiene dominio igual a $\mathbf{N}$.
  \end{lemma}

%\section{Procedimientos efectivos}

  % Lemma 14
  \begin{lemma}
    \par Sean $S_{1}, S_{2} \subseteq \omega^{n} \times \SIGMA$ conjuntos $\Sigma$-efectivamente enumerables, entonces:

    \begin{enumerate}[a)]
      \item $S_{1} \cup S_{2}$ es $\Sigma$-efectivamente enumerable.
      \item $S_{1} \cap S_{2}$ es $\Sigma$-efectivamente enumerable.
    \end{enumerate}
  \end{lemma}
  \begin{proof}
    \par El caso en el que alguno de los conjuntos es vacío es trivial. Supongamos que $S_{1}, S_{2} \neq \emptyset$ y
    sean $\mathbb{P}_{1}$ y $\mathbb{P}_{2}$ procedimientos que enumeran a $S_{1}$ y $S_{2}$.

    \begin{enumerate}[a)]
      \item El siguiente procedimiento enumera al conjunto $S_{1} \cup S_{2}$:

        \textbf{Si $x$ es par:} realizar $\mathbb{P}_{1}$ partiendo de $x/2$ y dar el elemento de $S_{1}$ obtenido como
        salida.
        \textbf{Si $x$ es impar:} realizar $\mathbb{P}_{2}$ partiendo de $(x-1)/2$ y dar el elemento de $S_{2}$ obtenido
        como salida.

      \item Veamos ahora que $S_{1} \cap S_{2}$ es $\Sigma$-efectivamente enumerable:

        \textbf{Si $S_{1} \cap S_{2} = \emptyset$:} entonces no hay nada que probar.

        \textbf{Si $S_{1} \cap S_{2} \neq \emptyset$:} sea $z_{0}$ un elemento fijo de $S_{1} \cap S_{2}$. Sea
        $\mathbb{P}$ un procedimiento efectivo el cual enumere a $\omega \times \omega$.

        \vspace{3mm}
        \par El siguiente procedimiento enumera al conjunto $S_{1} \cap S_{2}$:

        \textbf{Etapa 1:}
        Realizar $\mathbb{P}$ con dato de entrada $x$, para obtener un par $(x_{1}, x_{2}) \in \omega \times \omega $.

        \textbf{Etapa 2:}
        Realizar $\mathbb{P}_{1}$ con dato de entrada $x_{1}$ para obtener un elemento $z_{1} \in S_{1}$.

        \textbf{Etapa 3:}
        Realizar $\mathbb{P}_{2}$ con dato de entrada $x_{2}$ para obtener un elemento $z_{2} \in S_{2}$.

        \textbf{Etapa 4:}
        Si $z_{1} = z_{2}$, entonces dar como dato de salida $z_{1}$. En caso contrario dar como dato de salida $z_{0}$.
    \end{enumerate}
  \end{proof}

  % Lemma 15
  \begin{lemma}
    \par Si $S \subseteq \omega^{n} \times \SIGMA$ es $\Sigma$-efectivamente computable entonces $S$ es
    $\Sigma$-efectivamente enumerable.
  \end{lemma}
  \begin{proof}
    \par El caso en el que S es vacío es trivial. Supongamos $S \neq \emptyset$. Sea $(\vec{z}, \gamma) \in S$, fijo.
    Recordemos que $\omega^{n} \times \Sigma^{\ast m}$ es $\Sigma$-efectivamente enumerable. Sean:

    \begin{itemize}
      \item $\mathbb{P}_{1}$ un procedimiento efectivo que enumere a $\omega^{n} \times \Sigma^{\ast m}$
      \item $\mathbb{P}_{2}$ un procedimiento efectivo que compute a $\chi_{S}$.
    \end{itemize}

    \par El siguiente procedimiento enumera a $S$:

    \vspace{3mm}
    \textbf{Etapa 1:}
    Realizar $\mathbb{P}_{1}$ con $x$ de entrada para obtener $(\vec{x}, \vec{\alpha}) \in \omega^{n}\times
    \Sigma^{\ast m}$.

    \textbf{Etapa 2:}
    Realizar $\mathbb{P}_{2}$ con $(\vec{x}, \vec{\alpha})$ de entrada para obtener el valor \textit{Booleano} $e$ de
    salida.

    \textbf{Etapa 3:}
    \textbf{Si $e=1$:} dar como dato de salida $(\vec{x}, \vec{\alpha})$.

    $\qquad\qquad\;\;$\textbf{Si $e=0$:} dar como dato de salida $(\vec{z}, \gamma)$.
  \end{proof}

  \pagebreak
  % Theorem 16
  \begin{theorem}
    \par Sea $S \subseteq \omega^{n}\times \SIGMA$. Son equivalentes:

    \begin{enumerate}[a)]
      \item $S$ es $\Sigma$-efectivamente computable.
      \item $S$ y $(\omega^{n}\times \SIGMA)-S$ son $\Sigma$-efectivamente enumerables.
    \end{enumerate}
  \end{theorem}
  \begin{proof}
    \begin{tabular}{|c|}\hline $(a) \Rightarrow (b)$\\\hline\end{tabular} Si $S$ es $\Sigma$-efectivamente computable,
    por el \textbf{Lemma 15} tenemos que $S$ es $\Sigma$-efectivamente enumerable. Notese además que, dado que $S$ es
    $\Sigma$-efectivamente computable, $(\omega^{n} \times \Sigma^{\ast m}) - S$ también lo es, es decir, que aplicando
    nuevamente el \textbf{Lemma 15} tenemos que $(\omega^{n} \times \Sigma^{\ast m})-S$ es
    $\Sigma$-efectivamente enumerable.

    \vspace{3mm}
    \begin{tabular}{|c|}\hline $(b) \Rightarrow (a)$\\\hline\end{tabular} Sean:

    \begin{itemize}
      \item $\mathbb{P}_{1}$ un procedimiento efectivo que enumere a $S$.
      \item $\mathbb{P}_{2}$ un procedimiento efectivo que enumere a $(\omega^{n}\times \Sigma^{\ast m}) - S$.
    \end{itemize}

    \par El siguiente procedimiento computa el predicado $\chi_{S}$:

    \vspace{3mm}
    \textbf{Etapa 1:}
    Darle a la variable $T$ el valor $0$.

    \textbf{Etapa 2:}
    Realizar $\mathbb{P}_{1}$ con el valor de $T$ como entrada para obtener de salida la upla $(\vec{y}, \vec{\beta})$.

    \textbf{Etapa 3:}
    Realizar $\mathbb{P}_{2}$ con el valor de $T$ como entrada para obtener de salida la upla $(\vec{z}, \vec{\gamma})$.

    \textbf{Etapa 4:}
    \textbf{Si $(\vec{y}, \vec{\beta}) = (\vec{x}, \vec{\alpha})$:} entonces detenerse y dar como dato de salida el
    valor $1$.

    $\qquad\qquad\;\;$\textbf{Si $(\vec{z}, \vec{\gamma}) = (\vec{x}, \vec{\alpha})$:} entonces detenerse y dar como
    dato de salida el valor $0$.

    $\qquad\qquad\;\;$\textbf{Si no sucede ninguna de las dos posibilidades:} aumentar en $1$ el valor de

    $\qquad\qquad\;\;$la variable $T$ y dirijirse a la Etapa 2.
  \end{proof}

  % Theorem 17
  \begin{theorem}
    \par Dado $S \subseteq \omega^{n} \times \SIGMA$, son equivalentes:

    \begin{enumerate}
      \item $S$ es $\Sigma$-efectivamente enumerable.
      \item $S = \emptyset$ ó $S = I_{F}$, para alguna $F: \omega \rightarrow \omega^{n} \times \SIGMA$ tal que cada
        $F_{i}$ es $\Sigma$-efectivamente computable.
      \item $S = I_{F}$, para alguna $F:D_{F} \subseteq \omega^{k} \times \Sigma^{\ast l} \rightarrow \omega^{n} \times
        \SIGMA$ tal que cada $F_{i}$ es $\Sigma$-efectivamente computable.
      \item $S = D_{f}$, para alguna función $f$ la cual es $\Sigma$-efectivamente computable.
    \end{enumerate}
  \end{theorem}

\section{Funciones $\Sigma$-recursivas}

  % % Lemma 18: Con prueba.
  % \begin{lemma}
  %   \PN Si $f, f_{1}, \dotsc, f_{n+m}$ son $\Sigma$-efectivamente computables, entonces $f \circ (f_{1}, \dotsc,
  %   f_{n+m})$ lo es.
  % \end{lemma}
  % \begin{proof}
  %   \PN Sean:
  %
  %   \begin{itemize}
  %     \item $\mathbb{P}$ un procedimiento efectivo que compute a $f$.
  %     \item $\mathbb{P}_{1}$ un procedimiento efectivo que compute a $f_{1}$.
  %     \item $\mathbb{P}_{2}$ un procedimiento efectivo que compute a $f_{2}$.
  %     \item $\vdotswithin{(n-1)m+1} \qquad \vdotswithin{(n-1)m+2} \qquad \vdotswithin{(n-1)m+r}$
  %     \item $\mathbb{P}_{n+m}$ un procedimiento efectivo que compute a $f_{n+m}$.
  %   \end{itemize}
  %
  %   \PN El siguiente procedimiento $\mathbb{P}$ computa $f \circ (f_{1}, \dotsc, f_{n+m})$:
  %
  %   \vspace{3mm}
  %   \textbf{Etapa 1:}
  %   Realizar $\mathbb{P}_{1}$ con dato de entrada $(\vec{x}, \vec{\alpha})$ para obtener de salida $o_{1}$.
  %
  %   $\qquad\qquad\;\;$Realizar $\mathbb{P}_{2}$ con dato de entrada $(\vec{x}, \vec{\alpha})$ para obtener de salida
  %   $o_{2}$.
  %
  %   $\qquad\qquad\;\;\vdotswithin{(n-1)m+1} \qquad \vdotswithin{(n-1)m+2} \qquad \vdotswithin{(n-1)m+r}$
  %
  %   $\qquad\qquad\;\;$Realizar $\mathbb{P}_{n+m}$ con dato de entrada $(\vec{x}, \vec{\alpha})$ para obtener de salida
  %   $o_{n+m}$.
  %
  %   \textbf{Etapa 2:}
  %   Dar como dato de salida el resultado de $\mathbb{P}$ con dato de entrada $(o_{1}, o_{2}, \dotsc, o_{n+m})$.
  % \end{proof}
  %
  % % Lemma 19: Con prueba.
  % \begin{lemma}
  %   \PN Si $f$ y $g$ son $\Sigma $-efectivamente computables, entonces $R(f,g)$ lo es.
  % \end{lemma}
  % \begin{proof}
  %   \PN Sean:
  %
  %   \begin{itemize}
  %     \item $\mathbb{P}_{1}$ un procedimiento efectivo que compute a $f$.
  %     \item $\mathbb{P}_{2}$ un procedimiento efectivo que compute a $g$.
  %   \end{itemize}
  %
  %   \PN El siguiente procedimiento computa la función $R(f,g)$:
  %
  %   \vspace{3mm}
  %   \textbf{Etapa 1:}
  %   Darle a la variable $T$ el valor $0$.
  %
  %   \textbf{Etapa 2:}
  %   Realizar $\mathbb{P}_{1}$ con los valores $(\vec{x}, \vec{\alpha})$ como entrada para obtener de salida $A$.
  %
  %   \textbf{Etapa 3:}
  %   \textbf{Si $T = t$:} entonces detenerse y dar como dato de salida el valor de $A$.
  %
  %   $\qquad\qquad\;\;\;$\textbf{Si $T \neq t$:} aumentar en $1$ el valor de la variable $T$.
  %
  %   \textbf{Etapa 4:}
  %   \textbf{Si $Im(f), Im(g) \subseteq \omega$:} Realizar $\mathbb{P}_{2}$ con los valores $(A, T, \vec{x},
  %   \vec{\alpha})$ y dirijirse
  %
  %   $\qquad\qquad\;\;\;$a la Etapa 3.
  %
  %   $\qquad\qquad\;\;$\textbf{Si $Im(f), Im(g) \subseteq \SIGMA$:} Realizar $\mathbb{P}_{2}$ con los valores $(T,
  %   \vec{x}, \vec{\alpha}, A)$ y dirijirse
  %
  %   $\qquad\qquad\;\;\;$a la Etapa 3.
  % \end{proof}
  %
  % % Lemma 20: Sin prueba.
  % \begin{lemma}
  %   \PN Si $f$ y cada $\mathcal{G}_{a}$ son $\Sigma$-efectivamente computables, entonces $R(f,\mathcal{G})$ lo es.
  % \end{lemma}
  %
  % % Theorem 21: Con prueba.
  % \begin{theorem}
  %   \PN Si $f \in \mathrm{PR}^{\Sigma}$, entonces $f$ es $\Sigma$-efectivamente computable.
  % \end{theorem}
  % \begin{proof}
  %   \PN Recordemos que $PR^{\Sigma} = \bigcup\limits_{k \geq 0} PR_{k}^{\Sigma}$. Supongamos que $f \in
  %   PR_{k}^{\Sigma}$, probaremos este teorema por inducción en $k$.
  %
  %   \vspace{3mm}
  %   \PN \underline{Caso Base:} \begin{tabular}{|c|} \hline $k = 0$ \\\hline \end{tabular}
  %
  %   \PN Luego $f \in PR_{0}^{\Sigma}$, es decir $f \in \{Suc, Pred, C_{0}^{0,0}, C_{\varepsilon}^{0,0}\} \cup \{d_{a}:
  %   a \in \Sigma\} \cup \{p_{j}^{n,m} : 1 \leq j \geq n+m\}$. Por lo tanto, $f$ es $\Sigma$-efectivamente computable.
  %
  %   \vspace{3mm}
	% 	\PN \underline{Caso Inductivo:} \begin{tabular}{|c|} \hline $k > 0$ \\\hline \end{tabular}
  %
  %   \PN Supongamos ahora que si $f \in \mathrm{PR}_{k}^{\Sigma} \Rightarrow f$ es
  %   $\Sigma$-efectivamente computable, veamos que $f \in \mathrm{PR}_{k+1}^{\Sigma} \Rightarrow f$ es
  %   $\Sigma$-efectivamente computable.
  %
  %   \PN Dado que las funciones de $PR_{k}^{\Sigma}$ son $\Sigma$-efectivamente computable por hipótesis inductiva, y
  %   que el conjunto $PR_{k+1}^{\Sigma}$ se contruye a partir de dichas funciones, a través de recursiones o
  %   composiciones, las cuales probamos son $\Sigma$-efectivamente computables en el \textbf{Lemma 18} y
  %   \textbf{Lemma 19}, concluimos entonces que $f$ es $\Sigma$-efectivamente computable.
  % \end{proof}
  %
  % % Lemma 22: Con prueba.
  % \begin{lemma}
  %   \begin{enumerate}[a)]
  %     \item $\emptyset \in \mathrm{PR}^{\emptyset}$.
  %     \item $\lambda xy \left[x+y\right] \in \mathrm{PR}^{\emptyset}$.
  %     \item $\lambda xy\left[x.y\right] \in \mathrm{PR}^{\emptyset}$.
  %     \item $\lambda x\left[x!\right] \in \mathrm{PR}^{\emptyset}$.
  %   \end{enumerate}
  % \end{lemma}
  % \begin{proof}
  %   \begin{enumerate}[a)]
  %     \item Notese que $\emptyset = Pred \circ C_{0}^{0,0} \in \mathrm{PR}_{1}^{\emptyset}$, entonces $\emptyset \in
  %       \mathrm{PR}^{\emptyset}$.
  %
  %     \item Notar que:
  %       \begin{eqnarray*}
  %         \lambda xy \left[x+y\right](0, x_{1}) &=& x_{1} = p_{1}^{1,0}(x_{1}) \\
  %         \lambda xy \left[x+y\right](t+1, x_{1}) &=& \lambda xy \left[x+y\right](t, x_{1}) + 1 \\
  %         & =& Suc \circ p_{1}^{3,0}
  %       \end{eqnarray*}
  %
  %       \PN lo cual implica que $\lambda xy \left[x+y\right] = R(p_{1}^{1,0}, Suc \circ p_{1}^{3,0}) \in
  %       \mathrm{PR}_{2}^{\emptyset}$, entonces $\lambda xy \left[x+y\right] \in \mathrm{PR}^{\emptyset}$.
  %
  %     \item Primero note que:
  %       \begin{eqnarray*}
  %         C_{0}^{1,0}(0) &=& C_{0}^{0,0}(\Diamond) \\
  %         C_{0}^{1,0}(t+1) &=& C_{0}^{1,0}(t)
  %       \end{eqnarray*}
  %
  %       \PN lo cual implica que $C_{0}^{1,0} = R(C_{0}^{0,0}, p_{1}^{2,0}) \in \mathrm{PR}_{1}^{\emptyset}$.
  %       \PN También note que:
  %
  %       \begin{eqnarray*}
  %         \lambda xy \left[x.y\right](0, x_{1}) &=& 0 = C_{0}^{1,0}(x_{1}) \\
  %         \lambda xy \left[x.y\right](t+1, x_{1}) &=& \lambda xy \left[x.y\right](t, x_{1}) + x_{1} \\
  %         &=& \lambda xy \left[x+y\right] \circ (p_{1}^{3,0}, p_{3}^{3,0})
  %       \end{eqnarray*}
  %
  %       \PN lo cual implica que $\lambda xy \left[x.y\right] = R(C_{0}^{1,0}, \lambda xy \left[x+y\right] \circ
  %       (p_{1}^{3,0}, p_{3}^{3,0}))$, lo cual por (1) implica que $\lambda xy \left[x.y\right] \in
  %       \mathrm{PR}_{4}^{\emptyset}$, entonces $\lambda xy \left[x.y\right] \in \mathrm{PR}^{\emptyset}$.
  %
  %     \item Notar que:
  %       \begin{eqnarray*}
  %         \lambda x \left[x!\right](0) &=& 1 = C_{1}^{0,0}(\Diamond) \\
  %         \lambda x \left[x!\right](t+1) &=& \lambda x \left[x!\right](t).(t+1) \\
  %         &=& \lambda \left[x.y\right] \circ (p_{1}^{2,0}, Suc \circ p_{2}^{2,0})
  %       \end{eqnarray*}
  %
  %       \PN lo cual implica que: $\lambda x \left[x!\right] = R(C_{1}^{0, 0}, \lambda xy \left[x.y\right] \circ
  %       (p_{1}^{2,0}, Suc \circ p_{2}^{2,0}))$. Ya que $C_{1}^{0,0} = Suc \circ C_{0}^{0,0}$, tenemos que $C_{1}^{0,0}
  %       \in \mathrm{PR}_{1}^{\emptyset}$. Por (2), tenemos que $\lambda xy \left[x.y\right] \circ (p_{1}^{2,0}, Suc
  %       \circ p_{2}^{2,0}) \in \mathrm{PR}_{5}^{\emptyset}$, obteniendo que $\lambda x \left[x!\right] \in
  %       \mathrm{PR}_{6}^{\emptyset}$, entonces $\lambda x \left[x!\right] \in \mathrm{PR}^{\emptyset}$.
  %   \end{enumerate}
  % \end{proof}
  %
  % % Lemma 23: Con prueba.
  % \begin{lemma}
  %   \PN Supongamos $\Sigma \neq \emptyset$, entonces:
  %
  %   \begin{enumerate}[a)]
  %     \item $\lambda \alpha \beta \left[\alpha\beta\right] \in \mathrm{PR}^{\Sigma}$.
  %     \item $\lambda \alpha \left[\lvert\alpha \rvert\right] \in \mathrm{PR}^{\Sigma}$.
  %   \end{enumerate}
  % \end{lemma}
  % \begin{proof}
  %   \begin{enumerate}[a)]
  %     \item Ya que:
  %       \begin{eqnarray*}
  %         \lambda \alpha\beta \left[\alpha \beta\right](\alpha_{1}, \varepsilon) &=& \alpha_{1} = p_{1}^{0,1}
  %           (\alpha_{1}) \\
  %         \lambda \alpha\beta \left[\alpha \beta\right](\alpha_{1}, \alpha a) &=& d_{a}(\lambda \alpha\beta \left[\alpha
  %           \beta\right](\alpha_{1}, \alpha)) \qquad \text{para a} \in \Sigma
  %       \end{eqnarray*}
  %
  %       \PN tenemos que $\lambda \alpha\beta \left[\alpha \beta\right] = R(p_{1}^{0, 1}, \mathcal{G})$, donde
  %       $\mathcal{G}_{a} = d_{a} \circ p_{3}^{0,3}$, para cada $a \in \Sigma$. Luego, $\lambda \alpha\beta
  %       \left[\alpha\beta\right] \in \mathrm{PR}^{\Sigma}$.
  %
  %     \item Ya que:
  %       \begin{eqnarray*}
  %         \lambda \alpha \left[\lvert\alpha \rvert\right](\varepsilon) &=& 0 = C_{0}^{0,0}(\Diamond) \\
  %         \lambda \alpha \left[\lvert\alpha \rvert\right](\alpha a) &=& \lambda \alpha \left[\lvert\alpha \rvert\right]
  %           (\alpha) + 1
  %       \end{eqnarray*}
  %
  %       \PN tenemos que $\lambda \alpha \left[\lvert\alpha \rvert\right] = R(C_{0}^{0, 0}, \mathcal{G})$, donde
  %       $\mathcal{G}_{a} = Suc \circ p_{1}^{1, 1}$, para cada $a \in \Sigma$. Luego, $\lambda \alpha
  %       \left[\lvert\alpha \rvert\right] \in \mathrm{PR}^{\Sigma}$.
  %   \end{enumerate}
  % \end{proof}
  % 
  % % Lemma 24: Sin prueba.
  % \begin{lemma}
  %   \begin{enumerate}[a)]
  %     \item $C_{k}^{n,m}, C_{\alpha}^{n,m} \in \mathrm{PR}^{\Sigma}$, para $n, m, k \geq 0$, y $\alpha \in \SIGMA$.
  %     \item $C_{k}^{n,0} \in \mathrm{PR}^{\emptyset}$, para $n, k \geq 0$.
  %   \end{enumerate}
  % \end{lemma}

  % Lemma 25: Con prueba.
  \begin{lemma}
    \begin{enumerate}
      \item $\lambda xy \left[x^{y}\right] \in \mathrm{PR}^{\emptyset}$.
      \item $\lambda t\alpha \left[\alpha^{t}\right] \in \mathrm{PR}^{\Sigma}$.
    \end{enumerate}
  \end{lemma}
  \begin{proof}
    \begin{enumerate}[a)]
      \item Notar que:

        \begin{eqnarray*}
          \lambda tx \left[x^{t}\right](0, x_{1}) &=& 0 = C_{0}^{1,0}(x_{1}) \\
          \lambda tx \left[x^{t}\right](t+1, x_{1}) &=& \lambda tx \left[x^{t}\right](t, x_{1}) . x_{1} \\
          &=& \lambda xy \left[x.y\right] \circ (p_{1}^{3,0}, p_{3}^{3,0})
        \end{eqnarray*}
        \PN Osea que $\lambda xy \left[x^{y}\right] = \lambda tx \left[x^{t}\right] \circ (p_{2}^{2, 0}, p_{1}^{2, 0})
        \in \mathrm{PR}^{\emptyset}$.

      \item Notar que:

      \begin{eqnarray*}
        \lambda t\alpha \left[\alpha^{t}\right](t, \varepsilon) &=& \varepsilon = C_{\varepsilon}^{0,1}(t) \\
        \lambda t\alpha \left[\alpha^{t}\right](t, \alpha a) &=& \lambda t\alpha \left[\alpha^{t}\right](t,
          \alpha) \alpha \\
        &=& \lambda \alpha\beta \left[\alpha\beta \right] \circ \left(p_{3}^{1,2}, p_{2}^{1,2}\right)
      \end{eqnarray*}

      \PN Por lo tanto, $\lambda t\alpha \left[\alpha^{t}\right] \in \mathrm{PR}^{\Sigma}$.
    \end{enumerate}
  \end{proof}

  % Lemma 26: Con prueba.
  \begin{lemma}
    \PN Si $<$ es un orden total estricto sobre un alfabeto no vacío $\Sigma$, entonces:

    \begin{enumerate}[a)]
      \item $s^{<} \in \mathrm{PR}^{\Sigma}$.
      \item $\#^{<} \in \mathrm{PR}^{\Sigma}$.
      \item $\ast^{<} \in \mathrm{PR}^{\Sigma}$.
    \end{enumerate}
  \end{lemma}
  \begin{proof}
    \PN Supongamos $\Sigma = \{a_{1}, \dotsc, a_{k}\}$ y $<$ dado por $a_{1} < \dotsc < a_{k}$.

    \begin{enumerate}[a)]
      \item Ya que:
        \begin{eqnarray*}
          s^{<}(\varepsilon) &=& a_{1} \\
          s^{<}(\alpha a_{i}) &=& \alpha a_{i+1} \text{, para } i < k \\
          s^{<}(\alpha a_{k}) &=& s^{<}(\alpha) a_{1}
        \end{eqnarray*}

        \PN tenemos que $s^{<} = R(C_{a_{1}}^{0, 0}, \mathcal{G})$, donde $\mathcal{G} = \{\left(a_{i}, d_{a_{i+1}}
        \circ p_{1}^{0,2} \right), \left(a_{k}, d_{a_{1}} \circ p_{2}^{0,2} \right)\}$. Luego, $s^{<} \in
        \mathrm{PR}^{\Sigma}$.

      \item Ya que:
        \begin{eqnarray*}
          \ast^{<}(0) &=& \varepsilon \\
          \ast^{<}(t+1) &=& s^{<}(\ast^{<}(t))
        \end{eqnarray*}

        \PN tenemos que $\ast^{<} = R(C_{\varepsilon}^{0,0}, s^{<} \circ p_{1}^{2,0})$. Luego, $\ast^{<} \in
        \mathrm{PR}^{\Sigma}$.

      \item Ya que:
        \begin{eqnarray*}
          \#^{<}(\varepsilon) &=& 0 \\
          \#^{<}(\alpha a_{i}) &=& \#^{<}(\alpha). k + i \\
          \text{para } i &=& 1, \dotsc, k
        \end{eqnarray*}

        \PN tenemos que $\#^{<} = R(C_{0}^{0, 0}, \mathcal{G})$, donde $\mathcal{G}_{a_{i}} = \lambda xy
        \left[x+y\right] \circ (\lambda xy \left[x.y\right] \circ (p_{1}^{1, 1}, C_{k}^{1, 1}), C_{i}^{1, 1})
        \text{, para } i = 1, \dotsc, k$. Luego, $\#^{<} \in \mathrm{PR}^{\Sigma}$.
    \end{enumerate}
  \end{proof}

  % Lemma 27: Con prueba.
  \begin{lemma}
    \begin{enumerate}[a)]
      \item $\lambda xy \left[x \dot{-}y\right] \in \mathrm{PR}^{\emptyset}$.
      \item $\lambda xy \left[\max (x,y)\right] \in \mathrm{PR}^{\emptyset}$.
      \item $\lambda xy \left[x=y\right] \in \mathrm{PR}^{\emptyset}$.
      \item $\lambda xy \left[x \leq y\right] \in \mathrm{PR}^{\emptyset}$.
      \item Si $\Sigma \neq \emptyset \Rightarrow \lambda \alpha \beta \left[\alpha = \beta\right] \in
        \mathrm{PR}^{\Sigma}$.
    \end{enumerate}
  \end{lemma}
  \begin{proof}
    \begin{enumerate}[a)]
      \item Primero notar que:

        \begin{eqnarray*}
          \lambda x \left[x \dot{-}1\right](0) &=& 0 = C_{0}^{0,0} \\
          \lambda x \left[x \dot{-}1\right](t+1) &=& t \\
          &=& p_{2}^{2,0}
        \end{eqnarray*}

        \PN es decir $\lambda x \left[x \dot{-}1\right] = R(C_{0}^{0,0}, p_{2}^{2,0}) \in \mathrm{PR}^{\emptyset}$.
        \PN También notar que:

        \begin{eqnarray*}
          \lambda tx \left[x \dot{-}t\right](0, x_{1}) &=& x_{1} = p_{1}^{1,0}(x_{1}) \\
          \lambda tx \left[x \dot{-}t\right](t+1, x_{1}) &=& \lambda tx \left[x \dot{-}t\right](t, x_{1})
            \dot{-} 1\\
          &=& \lambda x \left[x \dot{-}1\right] \circ p_{1}^{3,0}
        \end{eqnarray*}

        \PN es decir, $\lambda tx \left[x \dot{-}t\right] = R(p_{1}^{1, 0}, \lambda x \left[x \dot{-}1\right] \circ
        p_{1}^{3, 0}) \in \mathrm{PR}^{\emptyset}$. Por lo tanto, $\lambda xy \left[x \dot{-}y\right] = \lambda tx
        \left[x \dot{-}t\right] \circ (p_{2}^{2, 0}, p_{1}^{2, 0}) \in \mathrm{PR}^{\emptyset}$.

      \item Notar que:

        \begin{eqnarray*}
          \lambda xy \left[\max (x,y)\right] &=& \lambda xy \left[(x + (y \dot{-}x)\right] \\
          &=& \lambda xy \left[x+y\right] \circ \left(p_{1}^{2,0}, \lambda xy \left[x \dot{-}y\right] \circ
            (p_{2}^{2,0}, p_{1}^{2,0})\right)
        \end{eqnarray*}

        \PN Por lo tanto, $\lambda xy \left[\max (x,y)\right] \in \mathrm{PR}^{\emptyset}$.

      \item Note que:

        \begin{eqnarray*}
          \lambda xy \left[x=y\right] &=& \lambda xy \left[1 \dot{-}((x \dot{-} y) + (y \dot{-} x))\right] \\
          &=& \lambda xy \left[x \dot{-}y\right] \circ (C_{1}^{2,0}, \lambda xy \left[x+y\right] \circ
            (\lambda xy \left[x \dot{-}y\right] \circ p_{1}^{2,0}, p_{2}^{2,0}, \lambda xy \left[x \dot{-}y\right] \circ
            p_{2}^{2,0}, p_{1}^{2,0}))
        \end{eqnarray*}

        \PN Por lo tanto, $\lambda xy \left[x=y\right] \in \mathrm{PR}^{\emptyset}$.

      \item Note que:

        \begin{eqnarray*}
          \lambda xy \left[x \leq y\right] &=& \lambda xy\left[1 \dot{-}(x \dot{-}y)\right] \\
          &=& \lambda xy \left[x \dot{-}y\right] \circ (C_{1}^{2,0}, \lambda xy \left[x \dot{-}y\right] \circ
            p_{1}^{2,0}, p_{2}^{2,0}))
        \end{eqnarray*}

        \PN Por lo tanto, $\lambda xy \left[x \leq y\right] \in \mathrm{PR}^{\emptyset}$.

      \item Sea $<$ un orden total estricto sobre $\Sigma$. Ya que:

        \[
          \alpha = \beta \Leftrightarrow \#^{<}(\alpha) = \#^{<}(\beta)
        \]

        \PN tenemos que:

        \[
          \lambda \alpha\beta \left[\alpha=\beta\right] = \lambda xy \left[x=y\right] \circ (\#^{<} \circ p_{1}^{0,2},
          \#^{<} \circ p_{2}^{0,2})
        \]

        \PN Luego, utilizando el inciso (c) y el \textbf{Lemma 28} obtenemos que $\lambda \alpha\beta
        \left[\alpha=\beta\right] \in \mathrm{PR}^{\Sigma}$.
    \end{enumerate}
  \end{proof}

  % Lemma 28: Con prueba.
  \begin{lemma}
    \PN Si $P: S \subseteq \omega^{n} \times \Sigma^{\ast m} \rightarrow \omega$ y $Q: S \subseteq \omega^{n} \times
    \Sigma^{\ast m} \rightarrow \omega$ son predicados $\Sigma$-PR, entonces $(P \vee Q), (P \wedge Q)$ y $\neg P$ lo
    son también.
  \end{lemma}
  \begin{proof}
    \PN Notar que:

    \begin{eqnarray*}
      \neg P &=& \lambda xy \left[x \dot{-} y\right] \circ (C_{1}^{n, m}, P) \\
      (P \wedge Q) &=& \lambda xt \left[x.y\right] \circ (P, Q) \\
      (P \vee Q) &=& \neg (\neg P \wedge \neg Q)
    \end{eqnarray*}
  \end{proof}

  % Lemma 29: Con prueba.
  \begin{lemma}
    \PN Si $S_{1}, S_{2} \subseteq \omega^{n} \times \Sigma^{\ast m}$ son $\Sigma$-PR, entonces $S_{1} \cup S_{2}$,
    $S_{1} \cap S_{2}$ y $S_{1} - S_{2}$ lo son.
  \end{lemma}
  \begin{proof}
    \PN Notar que:

    \begin{eqnarray*}
      \chi_{S_{1} \cup S_{2}} &=& (\chi_{S_{1}} \vee \chi_{S_{2}}) \\
      \chi_{S_{1} \cap S_{2}} &=& (\chi_{S_{1}} \wedge \chi_{S_{2}}) \\
      \chi_{S_{1} - S_{2}} &=& \lambda \left[x \dot{-} y\right] \circ (\chi_{S_{1}}, \chi_{S_{2}})
    \end{eqnarray*}
  \end{proof}

  % Corollary 30: Con prueba. Solo el caso n = m = 1.
  \begin{corollary}
    \PN Si $S \subseteq \omega^{n} \times \Sigma^{\ast m}$ es finito, entonces $S$ es $\Sigma$-PR.
  \end{corollary}
  \begin{proof}
    \PN Se probará el caso $n = m = 1$, es decir, $S \subseteq \omega \times \SIGMA$. Supongamos, sin pérdida de
    generalidad, utilizando el \textbf{Lemma 29} que:

    \[
      S = \{(z, \gamma)\}
    \]

    \PN Notar que $\chi_{S}$ es el siguiente predicado:

    \[
      \left(\chi_{z} \circ p_{1}^{1,1} \wedge \chi_{\gamma} \circ p_{2}^{1,1}\right)
    \]

    \PN Ya que los predicados:

    \begin{eqnarray*}
      \chi_{z} = \lambda xy \left[x = y\right] \circ \left(p_{1}^{1,0}, C_{z}^{1,0}\right) \\
      \chi_{\gamma} = \lambda \alpha\beta \left[\alpha = \beta\right] \circ \left(p_{1}^{0,1},
      C_{\gamma}^{0,1}\right)
    \end{eqnarray*}

    \PN son $\Sigma$-PR, el \textbf{Lema 28} implica que $\chi_{S}$ es $\Sigma$-PR, por lo tanto $S$ es $\Sigma$-PR.
  \end{proof}

  % Lemma 31: Con prueba. Solo el caso n = m = 1.
  \begin{lemma}
    \PN Supongamos $S_{1}, \dotsc, S_{n} \subseteq \omega, L_{1}, \dotsc, L_{m} \subseteq \SIGMA$ son conjuntos no
    vacíos, entonces $S_{1} \times \dotsc \times S_{n} \times L_{1} \times \dotsc \times L_{m}$ es $\Sigma$-PR $\Leftrightarrow$
    $S_{1}, \dotsc, S_{n}, L_{1}, \dotsc, L_{m}$ son $\Sigma$-PR.
  \end{lemma}
  \begin{proof}
    \PN Se probará el caso $n = m = 1$, es decir, $S \subseteq \omega$, $L \subseteq \SIGMA$.

    \vspace{3mm}
    \begin{tabular}{|c|} \hline $\Rightarrow$ \\\hline \end{tabular} Veremos que $L_{1}, S_{1}$ es $\Sigma$-PR. Sea
    $(z_{1}, \gamma_{1})$ un elemento fijo de $S_{1} \times L_{1}$. Notar que:

    \begin{eqnarray*}
      x \in S_{1} \Leftrightarrow (x, \gamma_{1}) \in S_{1} \times L_{1} \\
      \alpha \in L_{1} \Leftrightarrow (z_{1}, \alpha) \in S_{1} \times L_{1}
    \end{eqnarray*}

    \PN lo cual implica que:

    \begin{eqnarray*}
      \chi_{S_{1}} = \chi_{S_{1} \times L_{1}} \circ \left(p_{1}^{1,0}, C_{\gamma_{1}}^{0,1}\right) \\
      \chi_{L_{1}} = \chi_{S_{1} \times L_{1}} \circ \left(C_{z_{1}}^{0,1}, p_{1}^{0,1}\right)
    \end{eqnarray*}

    \PN por lo tanto, $L_{1}, S_{1}$ es $\Sigma$-PR.

    \vspace{3mm}
    \begin{tabular}{|c|} \hline $\Leftarrow$\\\hline \end{tabular} Notar que:

    \[
      \chi_{S_{1} \times L_{1}} = \left(\chi_{S_{1}} \circ p_{1}^{1,1} \wedge \chi_{L_{1}} \circ p_{2}^{1,1} \right)
    \]

    \PN luego, por el \textbf{Lemma 28}, $S_{1} \times L_{1}$ son $\Sigma$-PR.
  \end{proof}

  % Lemma 32: Con prueba.
  \begin{lemma}
    \PN Supongamos $f: D_{f} \subseteq \omega^{n} \times \Sigma^{\ast m} \rightarrow O$ es $\Sigma$-PR, donde $O \in
    \{\omega, \SIGMA\}$. Si $S \subseteq D_{f}$ es $\Sigma$-PR, entonces $f \mid_{S}$ es $\Sigma$-PR.
  \end{lemma}
  \begin{proof}
    \begin{tabular}{|c|} \hline $O = \SIGMA$ \\\hline \end{tabular} Notar que:

    \[
      f \mid_{S} = \lambda x\alpha \left[\alpha^{x}\right] \circ (Suc \circ Pred \circ \chi_{S}, f)
    \]

    \PN luego $f$ es $\Sigma$-PR.

    \begin{tabular}{|c|} \hline $O = \omega$ \\\hline \end{tabular} Notar que:

    \[
      f \mid_{S} = \lambda xy \left[x^{y}\right] \circ (f, Suc \circ Pred \circ \chi_{S})
    \]

    \PN luego $f$ es $\Sigma$-PR.

    \vspace{3mm}
    \PN Notar que \begin{tabular}{|c|} \hline $Suc \circ Pred \circ \chi_{S}$ \\\hline \end{tabular} funciona como un
    interruptor que evalua $f$, si el elemento pertenece a $S$, y que no evalua en caso contrario.
  \end{proof}

  % Lemma 33: Sin prueba.
  \begin{lemma}
    \PN Si $f: D_{f} \subseteq \omega^{n} \times \SIGMA \rightarrow O$ es $\Sigma$-PR, entonces existe una función
    $\Sigma$-PR $\bar{f}: \omega^{n} \times \Sigma^{\ast m} \rightarrow O$, tal que $f = \bar{f} \mid_{D_{f}}$.
  \end{lemma}

  % Proposition 34: Con prueba.
  \begin{proposition}
    \PN Un conjunto $S$ es $\Sigma$-PR $\Leftrightarrow S$ es el dominio de una función $\Sigma$-PR.
  \end{proposition}
  \begin{proof}
    \begin{tabular}{|c|} \hline $\Rightarrow$ \\\hline \end{tabular} Notar que $S = D_{Pred \circ \chi_{S}}$.

    \begin{tabular}{|c|} \hline $\Leftarrow$ \\\hline \end{tabular} Probaremos por inducción en $k$ que $D_{F}$ es
    $\Sigma$-PR para cada $F \in PR_{k}^{\Sigma}$.

    \vspace{3mm}
    \underline{Caso Base:} \begin{tabular}{|c|} \hline $k = 0$ \\\hline \end{tabular} es decir, $F \in PR_{0}^{\Sigma}$.
    Luego:

    \begin{eqnarray*}
      F &\in& \{Suc, Pred, C_{0}^{0,0}, C_{\varepsilon}^{0,0}\} \cup \{d_{a}: a \in \Sigma\} \cup \\
      && \{p_{j}^{n,m}: 1 \leq j \geq n+m\} \\
      D_{F} &\in& \{\omega, \mathbb{N}\}
    \end{eqnarray*}

    \PN luego, $S$ es $\Sigma$-PR.

    \vspace{3mm}
    \underline{Caso Inductivo:} Supongamos el resultado vale para un $k$ fijo y supongamos $F \in
    \mathrm{PR}_{k+1}^{\Sigma}$, veremos entonces que $D_{F}$ es $\Sigma$-PR. Existen varios casos, analizaremos cada
    uno por separado.

    \begin{enumerate}
      \item \begin{tabular}{|c|} \hline $F = R(f, g)$ \\\hline \end{tabular}
        \begin{itemize}
          \item Recursión primitiva sobre variable numérica.
          \begin{enumerate}
            \item Caso 1:
            \begin{eqnarray*}
              f &:& S_{1} \times \dotsc \times S_{n} \times L_{1} \times \dotsc \times L_{m} \rightarrow \omega \\
              g &:& \omega \times \omega \times S_{1} \times \dotsc \times S_{n} \times L_{1} \times \dotsc \times L_{m}
                \rightarrow \omega \\
              F &=& \omega \times S_{1} \times \dotsc \times S_{n} \times L_{1} \times \dotsc \times L_{m} \rightarrow
                \omega
            \end{eqnarray*}
            \item Caso 2:
            \begin{eqnarray*}
              f &:& S_{1} \times \dotsc \times S_{n} \times L_{1} \times \dotsc \times L_{m} \rightarrow \SIGMA \\
              g &:& \omega \times S_{1} \times \dotsc \times S_{n} \times L_{1} \times \dotsc \times L_{m} \times \SIGMA
                \rightarrow \SIGMA \\
              F &=& \omega \times S_{1} \times \dotsc \times S_{n} \times L_{1} \times \dotsc \times L_{m} \rightarrow
                \SIGMA
            \end{eqnarray*}
          \end{enumerate}

          \item Recursión primitiva sobre variable alfabética.
          \begin{enumerate}
            \item Caso 1:
            \begin{eqnarray*}
              f &:& S_{1} \times \dotsc \times S_{n} \times L_{1} \times \dotsc \times L_{m} \rightarrow \omega \\
              \mathcal{G}_{a} &:& \omega \times S_{1} \times \dotsc \times S_{n} \times L_{1} \times \dotsc \times L_{m}
                \times \SIGMA \rightarrow \omega \\
              F &=& S_{1} \times \dotsc \times S_{n} \times L_{1} \times \dotsc \times L_{m} \times \SIGMA \rightarrow
                \omega
            \end{eqnarray*}
            \item Caso 2:
            \begin{eqnarray*}
              f &:& S_{1} \times \dotsc \times S_{n} \times L_{1} \times \dotsc \times L_{m} \rightarrow \SIGMA \\
              \mathcal{G}_{a} &:& S_{1} \times \dotsc \times S_{n} \times L_{1} \times \dotsc \times L_{m} \times \SIGMA
                \times \SIGMA \rightarrow \SIGMA \\
              F &=& S_{1} \times \dotsc \times S_{n} \times L_{1} \times \dotsc \times L_{m} \times \SIGMA \rightarrow
                \SIGMA
            \end{eqnarray*}
          \end{enumerate}
        \end{itemize}

        \PN con $S_{1}, \dotsc, S_{n} \subseteq \omega$ y $L_{1}, \dotsc, L_{m} \subseteq \SIGMA$ conjuntos no vacíos
        y $f, g \in \mathrm{PR}_{k}^{\Sigma}$, para todos los casos anteriores.

        \PN Por hipótesis inductiva tenemos que $D_{f} = S_{1} \times \dotsc \times S_{n} \times L_{1} \times \dotsc
        \times L_{m}$ es $\Sigma$-PR, lo cual por el \textbf{Lemma 31} nos dice que los conjuntos $S_{1}, \dotsc, S_{n},
        L_{1}, \dotsc, L_{m}$ son $\Sigma$-PR. Ya que $\omega$ es $\Sigma$-PR, el \textbf{Lemma 31} nos dice que $D_{F}$
        es $\Sigma$-PR.

      \item \begin{tabular}{|c|} \hline $F = g \circ (g_{1}, \dotsc, g_{n+m})$ \\\hline \end{tabular} donde:

        \begin{eqnarray*}
          g &:& D_{g} \subseteq \omega^{n} \times \Sigma^{\ast m} \rightarrow O \\
          g_{i} &:& D_{g_{i}} \subseteq \omega^{k} \times \Sigma^{\ast l} \rightarrow \omega \qquad \;\; i = 1, \dotsc,
            n \\
          g_{i} &:& D_{g_{i}} \subseteq \omega^{k} \times \Sigma^{\ast l} \rightarrow \SIGMA \qquad i = n + 1, \dotsc,
            n + m
        \end{eqnarray*}

        \PN están en $\mathrm{PR}_{k}^{\Sigma}$. Por \textbf{Lemma 33}, hay funciones $\Sigma$-PR $\bar{g}_{1}, \dotsc,
        \bar{g}_{n+m}$ las cuales son $\Sigma$-totales y cumplen:

        \begin{eqnarray*}
          g_{i} &=& \bar{g}_{i} \mid_{D_{g_{i}}} \\
          \text{para } i &=& 1, \dotsc, n + m
        \end{eqnarray*}

        \PN Por hipótesis inductiva, los conjuntos $D_{g}$, $D_{g_{i}}$, para $i = 1, \dotsc, n + m$, son $\Sigma$-PR y
        por lo tanto:

        \[
          S = \bigcap_{i=1}^{n+m} D_{g_{i}}
        \]

        \PN lo es. Notese además, que:

        \[
          \chi_{D_{F}} = \left((\chi_{D_{g}} \circ (\bar{g}_{1}, \dotsc, \bar{g}_{n+m})) \wedge \chi_{S}\right)
        \]

        \PN lo cual nos dice que $D_{F}$ es $\Sigma$-PR.
    \end{enumerate}
  \end{proof}

% 
  % Lemma 35
  \begin{lemma}
    \par Supongamos $f_{i}: D_{f_{i}} \subseteq \omega^{n} \times \Sigma^{\ast m} \rightarrow O, i = 1, \dotsc, k$, son
    funciones $\Sigma$-PR tales que $D_{f_{i}} \cap D_{f_{j}} = \emptyset$ para $i \neq j$, entonces $f_{1} \cup \dotsc
    \cup f_{k}$ es $\Sigma$-PR.
  \end{lemma}
  \begin{proof}
    \par Supongamos $O = \SIGMA$ y $k = 2$. Sean:

    \[
      \bar{f}_{i}: \omega^{n} \times \Sigma^{\ast m} \rightarrow \SIGMA \; i = 1, 2
    \]

    \par funciones $\Sigma$-PR tales que $\bar{f}_{i} \mid_{D_{f_{i}}} = f_{i}$, $i = 1, 2$ (por \textbf{Lemma 33}).
    \par Por \textbf{Lemma 34} los conjuntos $D_{f_{1}}$ y $D_{f_{2}}$ son $\Sigma$-PR y por lo tanto lo es $ D_{f_{1}}
    \cup D_{f_{2}}$. Ya que:

    \[
      f_{1} \cup f_{2} = \left(\lambda \alpha \beta \left[\alpha \beta\right] \circ (\lambda x\alpha
      \left[\alpha^{x}\right] \circ (\chi_{D_{f_{1}}}, \bar{f}_{1}), \lambda x\alpha \left[\alpha^{x}\right] \circ
      (\chi_{D_{f_{2}}}, \bar{f}_{2}))\right) \mid_{D_{f_{1}} \cup D_{f_{2}}}
    \]

    \par tenemos que $f_{1} \cup f_{2}$ es $\Sigma$-PR.
    \par El caso $k > 2$ puede probarse por inducción ya que:

    \[
      f_{1} \cup \dotsc \cup f_{k} = (f_{1} \cup \dotsc \cup f_{k-1}) \cup f_{k}
    \]
  \end{proof}

  % Corollary 36
  \begin{corollary}
    \par Supongamos $f$ es una función $\Sigma$-mixta cuyo dominio es finito, entonces $f$ es $\Sigma$-PR.
  \end{corollary}

  % Lemma 37
  \begin{lemma}
    \par $\lambda i\alpha \left[\lbrack \alpha]_{i}\right]$ es $\Sigma$-PR.
  \end{lemma}

  % Lemma 38
  \begin{lemma}
    \par Sean $n, m \geq 0$.

    \begin{enumerate}[a)]
      \item Si $f: \omega \times S_{1} \times \dotsc \times S_{n} \times L_{1} \times \dotsc \times L_{m} \rightarrow
        \omega$ es $\Sigma$-PR, con $ S_{1}, \dotsc, S_{n} \subseteq \omega$ y $L_{1}, \dotsc, L_{m} \subseteq \SIGMA$
        no vacíos, entonces lo son las funciones $\lambda xy\vec{x}\vec{\alpha} \left[\sum_{t=x}^{t=y} f(t, \vec{x},
        \vec{\alpha})\right]$ y $\lambda xy\vec{x}\vec{\alpha} \left[\prod_{t=x}^{t=y}f(t, \vec{x}, \vec{\alpha})
        \right]$.
      \item Si $f: \omega \times S_{1} \times \dotsc \times S_{n} \times L_{1} \times \dotsc \times L_{m} \rightarrow
        \SIGMA$ es $\Sigma$-PR, con $ S_{1}, \dotsc, S_{n} \subseteq \omega$ y $L_{1}, \dotsc, L_{m} \subseteq \SIGMA$
        no vacíos, entonces lo es la función $\lambda xy\vec{x}\vec{\alpha}\left[\subset_{t=x}^{t=y} f(t, \vec{x},
        \vec{\alpha})\right]$.
    \end{enumerate}
  \end{lemma}
  \begin{proof}
    \par (a): Sea $G = \lambda tx\vec{x}\vec{\alpha}\left[\sum_{i=x}^{i=t} f(i, \vec{x}, \vec{\alpha})\right]$. Ya que:

    \[
      \lambda xy\vec{x}\vec{\alpha}\left[\sum_{i=x}^{i=y} f(i, \vec{x}, \vec{\alpha})\right] = G \circ \left(
      p_{2}^{n+2,m}, p_{1}^{n+2,m}, p_{3}^{n+2,m}, \dotsc, p_{n+m+2}^{n+2,m}\right)
    \]

    \par solo tenemos que probar que $G$ es $\Sigma$-PR. Primero note que:

    \begin{eqnarray}
      \nonumber G(0,x,\vec{x},\vec{\alpha}) &=& \left\{
        \begin{array}{lll}
          0 && \text{si } x > 0 \\
          f(0,\vec{x},\vec{\alpha}) && \text{si } x = 0
        \end{array}\right. \\
      \nonumber G(t+1,x,\vec{x},\vec{\alpha}) &=& \left\{
        \begin{array}{lll}
          0 && \text{si } x > t+1 \\
          G(t,x,\vec{x},\vec{\alpha}) + f(t+1,\vec{x},\vec{\alpha}) && \text{si } x \leq t+1
        \end{array} \right.
    \end{eqnarray}

    \par Sean:

    \begin{eqnarray}
      \nonumber D_{1} &=& \left\{(x,\vec{x},\vec{\alpha}) \in \omega \times S_{1} \times \dotsc \times S_{n} \times
        L_{1} \times \dotsc \times L_{m}: x > 0 \right\} \\
      \nonumber D_{2} &=& \left\{(x,\vec{x},\vec{\alpha}) \in \omega \times S_{1} \times \dotsc \times S_{n} \times
        L_{1} \times \dotsc \times L_{m}: x = 0 \right\} \\
      \nonumber H_{1} &=& \left\{(z,t,x,\vec{x},\vec{\alpha}) \in \omega^{3} \times S_{1} \times \dotsc \times S_{n}
        \times L_{1} \times \dotsc \times L_{m}: x > t+1\right\} \\
      \nonumber H_{2} &=& \left\{(z,t,x,\vec{x},\vec{\alpha}) \in \omega^{3} \times S_{1} \times \dotsc \times S_{n}
        \times L_{1} \times \dotsc \times L_{m}: x \leq t+1\right\}
    \end{eqnarray}

    \par Es fácil de chequear que estos conjuntos son $\Sigma$-PR. Veamos que por ejemplo $H_{1}$ lo es. Es decir
    debemos ver que $\chi_{H_{1}}$ es $\Sigma$-PR. Ya que $f$ es $\Sigma$-PR tenemos que $D_{f} = \omega \times S_{1}
    \times \dotsc \times S_{n} \times L_{1} \times \dotsc \times L_{m}$ es $\Sigma$-PR, lo cual por el \textbf{Lemma 31}
    nos dice que los conjuntos $S_{1}, \dotsc, S_{n}, L_{1}, \dotsc, L_{m}$ son $\Sigma$-PR. Ya que $\omega$ es
    $\Sigma$-PR, el \textbf{Lemma 31} nos dice que $R = \omega^{3} \times S_{1} \times \dotsc \times S_{n} \times L_{1}
    \times \dotsc \times L_{m}$ es $\Sigma$-PR. Notese que $\chi_{H_{1}} = (\chi_{R} \wedge \lambda ztx\vec{x}
    \vec{\alpha}\left[x > t+1\right])$ por cual $\chi_{H_{1}}$ es $\Sigma$-PR ya que es la conjunción de dos predicados
    $\Sigma$-PR.

    \par Además note que $G = R(h,g)$, donde:

    \begin{eqnarray}
      \nonumber h &=& C_{0}^{n+1,m} \mid_{D_{1}} \cup \lambda x\vec{x}\vec{\alpha}\left[f(0,\vec{x},\vec{\alpha})\right]
      \mid_{D_{2}} \\
      \nonumber g &=& C_{0}^{n+3,m} \mid_{H_{1}} \cup \lambda ztx\vec{x}\vec{\alpha}\left[z+f(t+1,\vec{x},\vec{\alpha})
      \right]) \mid_{H_{2}}
    \end{eqnarray}

    \par O sea que el \textbf{Lemma 35} y el \textbf{Lemma 32} garantizan que $G$ es $\Sigma$-PR.
  \end{proof}

  % Lemma 39
  \begin{lemma}
    \par Sean $n, m \geq 0$.

    \begin{enumerate}[a)]
      \item Sea $P: S \times S_{1} \times \dotsc \times S_{n} \times L_{1} \times \dotsc \times L_{m} \rightarrow
        \omega$ un predicado $\Sigma$-PR y supongamos $\bar{S} \subseteq S$ es $\Sigma$-PR, entonces $\lambda
        x\vec{x}\vec{\alpha} \left[(\forall t\in \bar{S})_{t\leq x} \; P(t,\vec{x},\vec{\alpha})\right]$ y $\lambda
        x\vec{x}\vec{\alpha} \left[(\exists t\in \bar{S})_{t\leq x} \; P(t,\vec{x},\vec{\alpha})\right]$ son predicados
        $\Sigma$-PR. (Notar que el dominio de estos predicados es $\omega \times S_{1} \times \dotsc \times S_{n} \times
        L_{1} \times \dotsc \times L_{m}$).
      \item Sea $P: S_{1} \times \dotsc \times S_{n} \times L_{1} \times \dotsc \times L_{m} \times L \rightarrow
        \omega$ un predicado $\Sigma$-PR y supongamos $\bar{L} \subseteq L$ es $\Sigma$-PR, entonces $\lambda
        x\vec{x}\vec{\alpha} \left[(\forall \alpha \in \bar{L})_{\left\vert \alpha \right\vert \leq x} \;
        P(\vec{x},\vec{\alpha},\alpha)\right]$ y $\lambda x\vec{x}\vec{\alpha} \left[(\exists \alpha \in
        \bar{L})_{\left\vert \alpha \right\vert \leq x} \; P(\vec{x},\vec{\alpha},\alpha )\right]$ son predicados
        $\Sigma$-PR.
    \end{enumerate}
  \end{lemma}
  \begin{proof}
    \par Se probará solamente el inciso (a). Sea:

    \[
      \bar{P} = P \mid_{\bar{S} \times S_{1} \times \dotsc \times S_{n} \times L_{1} \times \dotsc \times L_{m}} \cup
      C_{1}^{1+n,m} \mid_{(\omega -\bar{S}) \times S_{1} \times \dotsc \times S_{n} \times L_{1} \times \dotsc \times
      L_{m}}
    \]

    \par Notese que $\bar{P}$ es $\Sigma$-PR. Ya que:

    \begin{eqnarray}
      \nonumber \lambda x\vec{x}\vec{\alpha}\left[(\forall t\in \bar{S})_{t\leq x} P(t,\vec{x},\vec{\alpha})\right] &=&
        \lambda x\vec{x}\vec{\alpha}\left[\prod\limits_{t=0}^{t=x}\bar{P}(t,\vec{x},\vec{\alpha})\right] \\
      \nonumber &=& \lambda xy\vec{x}\vec{\alpha}\left[\prod\limits_{t=x}^{t=y}\bar{P}(t,\vec{x},\vec{\alpha})\right]
        \circ \left(C_{0}^{1+n,m}, p_{1}^{1+n,m}, \dotsc, p_{1+n+m}^{1+n,m}\right)
    \end{eqnarray}

    \par el \textbf{Lemma 38} implica que $\lambda x\vec{x}\vec{\alpha}\left[(\forall t\in \bar{S})_{t\leq x} \;
    P(t,\vec{x},\vec{\alpha})\right]$ es $\Sigma$-PR.

    \par Finalmente note que:

    \[
      \lambda x\vec{x}\vec{\alpha}\left[(\exists t\in \bar{S})_{t\leq x} \; P(t,\vec{x},\vec{\alpha})\right] = \lnot
      \lambda x\vec{x}\vec{\alpha}\left[(\forall t\in \bar{S})_{t\leq x} \; \lnot P(t,\vec{x},\vec{\alpha})\right]
    \]

    \par es $\Sigma$-PR.
  \end{proof}

  % Lemma 40
  \begin{lemma}
    \begin{enumerate}[a)]
      \item El predicado $\lambda xy\left[x \text{ divide } y\right]$ es $\emptyset$-PR.
      \item El predicado $\lambda x\left[x \text{ es primo}\right]$ es $\emptyset$-PR.
      \item El predicado $\lambda \alpha\beta \left[\alpha \text{\ }\mathrm{ inicial}\ \beta \right]$ es $\Sigma$-PR.
    \end{enumerate}
  \end{lemma}
  \begin{proof}
    \begin{enumerate}[a)]
      \item Si tomamos $P = \lambda tx_{1}x_{2}\left[x_{2}=t.x_{1}\right] \in \mathrm{PR}^{\emptyset}$, tenemos que:

        \begin{eqnarray}
          \nonumber \lambda x_{1}x_{2} \left[x_{1}\text{ divide } x_{2}\right] &=& \lambda x_{1}x_{2}\left[(\exists t
            \in \omega)_{t\leq x_{2}} \; P(t,x_{1},x_{2}) \right] \\
          \nonumber &=& \lambda xx_{1}x_{2}\left[(\exists t \in \omega)_{t\leq x} \; P(t,x_{1},x_{2})\right] \circ
            \left(p_{2}^{2,0}, p_{1}^{2,0}, p_{2}^{2,0}\right)
        \end{eqnarray}

        \par por lo que podemos aplicar el \textbf{Lemma 39} anterior.

      \item Ya que:

        \[
          x \text{ es primo sii } x > 1 \wedge \left((\forall t \in \omega)_{t\leq x} \; t=1 \vee t=x \vee \lnot
          (t\text{ divide } x)\right)
        \]

        \par podemos usar un argumento similar al de la prueba de (a).
      \item es dejado al lector.
    \end{enumerate}
  \end{proof}

  % Lemma 41
  \begin{lemma}
    \par Si $P: D_{P} \subseteq \omega \times \omega^{n} \times \Sigma^{\ast m} \rightarrow \omega$ es un predicado
    $\Sigma$-EC y $ D_{P}$ es $\Sigma$-EC, entonces la función $M(P)$ es $\Sigma$-EC.
  \end{lemma}
  \begin{proof}
    \par Ejercicio.
  \end{proof}

  % Theorem 42:
  \begin{theorem}
    \par Si $f \in \mathrm{R}^{\Sigma}$, entonces $f$ es $\Sigma$-EC.
  \end{theorem}
  \begin{proof}
    \par Dejamos al lector la prueba por inducción en $k$ de que si $f\in \mathrm{R}_{k}^{\Sigma}$, entonces $f$ es
    $\Sigma$-EC.
  \end{proof}

  % Lemma 43
  \begin{lemma}
    \par Sean $n, m \geq 0$. Sea $P: D_{P} \subseteq \omega \times \omega^{n} \times \Sigma^{\ast m} \rightarrow \omega$
    un predicado $\Sigma$-PR, ntonces:

    \begin{enumerate}[a)]
      \item $M(P)$ es $\Sigma$-R.
      \item Si hay una función $\Sigma$-PR $f: \omega^{n} \times \Sigma^{\ast m} \rightarrow \omega$ tal que:

        \[
          M(P)(\vec{x},\vec{\alpha}) = \min_{t}P(t,\vec{x},\vec{\alpha}) \leq f(\vec{x},\vec{\alpha}),
          \text{ para cada }(\vec{x},\vec{\alpha}) \in D_{M(P)}
        \]

        \par entonces $M(P)$ es $\Sigma$-PR.
    \end{enumerate}
  \end{lemma}
  \begin{proof}
    \begin{enumerate}[a)]
      \item Sea $\bar{P} = P \mid_{D_{P}} \cup C_{0}^{n+1,m} \mid_{(\omega^{n+1} \times \Sigma^{\ast m})-D_{P}}$.
        Dejamos al lector verificar cuidadosamente que $M(P) = M(\bar{P})$. Veremos entonces que $M(\bar{P})$ es
        $\Sigma$-R. Note que $\bar{P}$ es $\Sigma$-PR. Sea $k$ tal que $\bar{P} \in \mathrm{PR}_{k}^{\Sigma}$. Ya que
        $\bar{P}$ es $\Sigma$-total y $\bar{P} \in \mathrm{PR}_{k}^{\Sigma} \subseteq \mathrm{R}_{k}^{\Sigma}$, tenemos
        que $M(\bar{P}) \in \mathrm{R}_{k+1}^{\Sigma}$ y por lo tanto $M(\bar{P}) \in \mathrm{R}^{\Sigma}$.

      \item Primero veremos que $D_{M(\bar{P})}$ es un conjunto $\Sigma$-PR. Notese que:

        \[
          \chi_{D_{M(\bar{P})}} = \lambda \vec{x}\vec{\alpha}\left[(\exists t\in \omega)_{t\leq f(\vec{x},\vec{\alpha})}
          \; \bar{P}(t,\vec{x},\vec{\alpha})\right]
        \]

        \par lo cual nos dice que:

        \[
          \chi_{D_{M(\bar{P})}} = \lambda x\vec{x}\vec{\alpha}\left[(\exists t\in \omega)_{t\leq x} \;
          \bar{P}(t,\vec{x},\vec{\alpha})\right] \circ (f,p_{1}^{n,m},\dotsc,p_{n+m}^{n,m})
        \]

        \par Pero el \textbf{Lemma 39} nos dice que $\lambda x\vec{x}\vec{\alpha} \left[(\exists t\in \omega)_{t\leq x}
        \; \bar{P}(t,\vec{x},\vec{\alpha})\right]$ es $\Sigma$-PR por lo cual tenemos que $\chi _{D_{M(\bar{P})}}$ lo es.

        \par Sea:

        \[
          P_{1} = \lambda t\vec{x}\vec{\alpha}\left[\bar{P}(t,\vec{x},\vec{\alpha}) \wedge (\forall j\in \omega)_{j\leq
          t} \; j=t \vee \lnot \bar{P}(j,\vec{x},\vec{\alpha})\right]
        \]

        \par Note que $P_{1}$ es $\Sigma$-total. Dejamos al lector usando Lemmas anteriores probar que $P_{1}$ es
        $\Sigma$-PR. Además notese que para cada $(\vec{x},\vec{\alpha}) \in \omega^{n} \times \Sigma^{\ast m}$ tenemos
        que:

        \[
          P_{1}(t,\vec{x},\vec{\alpha}) = 1 \text{ si y solo si } t = M(\bar{P})(\vec{x},\vec{\alpha})
        \]

        \par Esto nos dice que:

        \[
          M(\bar{P}) = \left(\lambda \vec{x}\vec{\alpha}\left[\prod_{t=0}^{f(\vec{x},\vec{\alpha})}t^{P_{1}(t,\vec{x},
          \vec{\alpha})}\right]\right) \mid_{D_{M(\bar{P})}}
        \]

        \par por lo cual para probar que $M(\bar{P})$ es $\Sigma$-PR solo nos resta probar que:

        \[
          F = \lambda \vec{x}\vec{\alpha}\left[\prod_{t=0}^{f(\vec{x},\vec{\alpha})}t^{P_{1}(t,\vec{x},\vec{\alpha})}
          \right]
        \]

        \par lo es. Pero:

        \[
          F = \lambda xy\vec{x}\vec{\alpha}\left[\prod_{t=x}^{y}t^{P_{1}(t,\vec{x},\vec{\alpha})}\right] \circ
          (C_{0}^{n,m}, f, p_{1}^{n,m}, \dotsc, p_{n+m}^{n,m})
        \]

        \par y por lo tanto el \textbf{Lemma 38} nos dice que $F$ es $\Sigma$-PR. De esta manera hemos probado que
        $M(\bar{P})$ es $\Sigma$-PR y por lo tanto $M(P)$ lo es.
    \end{enumerate}
  \end{proof}

  % Lemma 44
  \begin{lemma}
    \par Las siguientes funciones son $\emptyset$-PR:

    \begin{enumerate}[a)]
      \item
        $\begin{array}{rll}
          Q: \omega \times \mathbf{N} &\rightarrow& \omega \\
          (x,y) & \rightarrow & \text{cociente de la division de } x \text{ por } y
        \end{array}$
      \item
        $\begin{array}{rll}
          R: \omega \times \mathbf{N} &\rightarrow& \omega \\
          (x,y) &\rightarrow& \text{resto de la division de } x \text{ por } y
        \end{array}$
      \item
        $\begin{array}{rll}
          pr: \mathbf{N} &\rightarrow& \omega \\
          n & \rightarrow & n\text{-esimo numero primo}
        \end{array}$
    \end{enumerate}
  \end{lemma}
  \begin{proof}
    \begin{enumerate}[a)]
      \item Veamos primero veamos que $Q=M(P)$, donde $P=\lambda txy\left[(t+1).y > x\right]$. Notar que:

        \begin{eqnarray}
          \nonumber D_{M(P)} &=& \{(x,y): (\exists t \in \omega) \; P(t,x,y) = 1\} \\
          \nonumber &=& \{(x,y): (\exists t \in \omega) \; (t+1).y > x \} \\
          \nonumber &=& \omega \times \mathbf{N} \\
          \nonumber &=& D_{Q}
        \end{eqnarray}

        \par Dejamos al lector la fácil verificación de que para cada $(x,y) \in \omega \times \mathbf{N}$, se tiene que:

        \[
          Q(x,y) = M(P)(x,y) = \min_{t}(t+1).y > x
        \]

        \par Esto prueba que $Q=M(P)$. Ya que $P$ es $\emptyset$-PR y además:

        \[
          Q(x,y) \leq p_{1}^{2,0}(x,y), \text{para cada }(x,y) \in \omega \times \mathbf{N}
        \]

        \par el inciso (b) del \textbf{Lemma 43} implica que $Q \in \mathrm{PR}^{\emptyset}$.
      \item Notese que:

        \[
          R = \lambda xy\left[x \dot{-} Q(x,y).y\right]
        \]

        \par y por lo tanto $R \in \mathrm{PR}^{\emptyset}$.
      \item Para ver que $pr$ es $\emptyset$-PR, veremos que la extensión $ h: \omega \rightarrow \omega$, dada por
        $h(0)=0$ y $h(n)=pr(n)$, $n \geq 1$, es $\emptyset$-PR. Primero note que:

        \begin{eqnarray}
          \nonumber h(0) &=& 0 \\
          \nonumber h(x+1) &=& \min\nolimits_{t}\left(t \text{ es primo} \wedge t > h(x)\right)
        \end{eqnarray}

        \par Osea que $h = R \left(C_{0}^{0,0},M(P)\right)$, donde:

        \[
          P = \lambda tzx\left[t \text{ es primo} \wedge t > z\right]
        \]

        \par Es decir que solo nos resta ver que $M(P)$ es $\emptyset$-PR.
        \par Claramente $P$ es $\emptyset$-PR. Veamos que para cada $(z,x) \in \omega^{2}$, tenemos que:

        \[
          M(P)(z,x) = \min\nolimits_{t}\left(t \text{ es primo} \wedge t > z\right) \leq z! + 1
        \]

        \par Sea $p$ primo tal que $p$ divide a $z!+1$. Es fácil ver que entonces $p > z$. Pero esto claramente nos dice
        que:

        \[
          \min\nolimits_{t}\left(t \text{ es primo} \wedge t > z\right) \leq p \leq z! + 1
        \]

        \par Osea que el inciso (b) del \textbf{Lemma 43} implica que $M(P)$ es $\emptyset$-PR ya que podemos tomar
        $f = \lambda zx\left[z! + 1\right]$.
    \end{enumerate}
  \end{proof}

  % Lemma 45
  \begin{lemma}
    \par Las funciones $\lambda xi\left[(x)_{i}\right]$ y $\lambda x\left[Lt(x)\right]$ son $\emptyset$-PR.
  \end{lemma}

  % Lemma 46
  \begin{lemma}
    \par Este lemma no se evalua.
  \end{lemma}

  % Lemma 47
  \begin{lemma}
    \par Supongamos que $\Sigma \neq \emptyset$. Sea $<$ un orden total estricto sobre $\Sigma$, sean $n, m \geq 0$ y
    sea $P: D_{P} \subseteq \omega^{n} \times \Sigma^{\ast m} \times \SIGMA \rightarrow \omega$ un predicado
    $\Sigma$-PR, entonces:

    \begin{enumerate}[a)]
      \item $M^{<}(P)$ es $\Sigma$-R.
      \item Si existe una función $\Sigma$-PR $f: \omega^{n} \times \Sigma^{\ast m} \rightarrow \omega$ tal que:

        \[
          \left\vert M^{<}(P)(\vec{x},\vec{\alpha})\right\vert = \left\vert \min\nolimits_{\alpha}^{<} P(\vec{x},
          \vec{\alpha},\alpha)\right\vert \leq f(\vec{x},\vec{\alpha}) \text{, para cada } (\vec{x},\vec{\alpha}) \in
          D_{M^{< }(P)}
        \]

        \par entonces $M^{<}(P)$ es $\Sigma$-PR.
    \end{enumerate}
  \end{lemma}

  % Lemma 48
  \begin{lemma}
    \par Este lemma no se evalua.
  \end{lemma}

  % Lemma 49
  \begin{lemma}
    \par Este lemma no se evalua.
  \end{lemma}

  % Lemma 50
  \begin{lemma}
    \par Este lemma no se evalua.
  \end{lemma}

  % Theorem 51
  \begin{theorem}
    \par Sean $\Sigma$ y $\Gamma$ alfabetos cualesquiera.

    \begin{enumerate}[a)]
      \item Supongamos una función $f$ es $\Sigma$-mixta y $\Gamma$-mixta, entonces $f$ es $\Sigma$-R (respectivamente
        $\Sigma$-PR) sii $f$ es $\Gamma$-R (respectivamente $\Gamma$-PR).
      \item Supongamos un conjunto $S$ es $\Sigma$-mixto y $\Gamma$-mixto, entonces $S$ es $\Sigma$-PR sii $S$ es
        $\Gamma$-PR.
    \end{enumerate}
  \end{theorem}

% \section{El lenguaje ${S}^{\Sigma}$}

  % Lemma 52
  \begin{lemma} Se tiene que:
    \begin{itemize}
      \item[(a)]  Si $I_{1}...I_{n}=J_{1}...J_{m}$, con $ I_{1},...,I_{n},J_{1},...,J_{m}\in \mathrm{Ins}^{\Sigma }$,
                  entonces $n=m$ y $I_{j}=J_{j}$ para cada $j\geq 1$.
      \item[(b)]  Si $\mathcal{P}\in \mathrm{Pro}^{\Sigma }$, entonces existe una única sucesión de instrucciones
                  $I_{1},...,I_{n}$ tal que $\mathcal{P} =I_{1}...I_{n}$
    \end{itemize}
  \end{lemma}

  % Theorem 53
  \begin{theorem}
    Si $f$ es $\Sigma $-computable, entonces $f$ es $\Sigma $-efectivamente computable.
  \begin{proof}
    Supongamos por ejemplo que $f:S\subseteq \omega ^{n}\times \Sigma ^{\ast m}\rightarrow \omega $ es computada por
    $\mathcal{P}\in \mathrm{Pro}^{\Sigma }$. Es claro que el procedimiento que consiste en realizar las sucesivas
    instrucciones de $\mathcal{P}$ (partiendo de $((x_{1},...,x_{n},0,0,...),( \alpha _{1},...,\alpha _{m},
    \varepsilon ,\varepsilon ,...))$) y eventualmente terminar en caso de que nos toque realizar la instrucción
    $n( \mathcal{P})+1$, y dar como salida el contenido de la variable $\mathrm{N}1$ , es un procedimiento efectivo
    que computa a $f$.
  \end{proof}
  \end{theorem}

  % Proposition 54
  \begin{proposition}
    \begin{itemize}
      \item[(a)]  Sea $f:S\subseteq \omega ^{n}\times \Sigma ^{\ast m}\rightarrow \omega $ una función
                  $\Sigma $-computable. Entonces hay un macro
                  \[
                  \displaystyle \left[ \mathrm{V}\overline{n+1}\leftarrow f(\mathrm{V}1,...,\mathrm{V}\bar{n} ,
                  \mathrm{W}1,...,\mathrm{W}\bar{m})\right]
                  \]
      \item[(b)]  Sea $f:S\subseteq \omega ^{n}\times \Sigma ^{\ast m}\rightarrow \Sigma ^{\ast }$ una función
                  $\Sigma $-computable. Entonces hay un macro
                  \[
                  \displaystyle \left[ \mathrm{W}\overline{m+1}\leftarrow f(\mathrm{V}1,...,\mathrm{V}\bar{n} ,
                  \mathrm{W}1,...,\mathrm{W}\bar{m})\right]
                  \]
    \end{itemize}
  \end{proposition}

  % Proposition 55
  \begin{proposition}
    Sea $P:S\subseteq \omega ^{n}\times \Sigma ^{\ast m}\rightarrow \omega $ un predicado $\Sigma $-computable.
    Entonces hay un macro
    \[
    \displaystyle \left[ \mathrm{IF}\;P(\mathrm{V}1,...,\mathrm{V}\bar{n},\mathrm{W}1,...,
    \mathrm{W}\bar{m})\;\mathrm{GOTO}\;\mathrm{A}1\right]
    \]
  \end{proposition}

  % Theorem 56
  \begin{theorem}
    Si $h$ es $\Sigma $-recursiva, entonces $h$ es $\Sigma $ -computable.

  \begin{proof}
    Probaremos por induccion en $k$ que
    \begin{itemize}
      \item[(*)] Si $h\in \mathrm{R}_{k}^{\Sigma }$, entonces $h$ es $\Sigma $ -computable.
    \end{itemize}
    El caso $k=0$ es dejado al lector.
    Supongamos (*) vale para $k$, veremos que vale para $k+1$.
    Sea $h\in \mathrm{R}_{k+1}^{\Sigma }-\mathrm{R} _{k}^{\Sigma }.$ Hay varios casos
    Supongamos que $\Sigma = \{@, \$\}$

    \noindent Caso 2. Supongamos $h=R(f,\mathcal{G})$, con
    \[
      \displaystyle \begin{array}{rcl}
        f & :& S_{1}\times ...\times S_{n}\times L_{1}\times ...\times L_{m}\rightarrow \Sigma ^{\ast } \\
        \mathcal{G}_{@} & :& S_{1}\times ...\times S_{n}\times L_{1}\times ...\times L_{m}\times \Sigma ^{\ast }
          \times \Sigma ^{\ast }\rightarrow \Sigma ^{\ast } \\
        \mathcal{G}_{\$} & :& S_{1}\times ...\times S_{n}\times L_{1}\times ...\times L_{m}\times \Sigma ^{\ast }
          \times \Sigma ^{\ast }\rightarrow \Sigma ^{\ast }
      \end{array}
    \]
    elementos de $\mathrm{R}_{k}^{\Sigma }$.  Por hipotesis inductiva, las funciones $f$, $\mathcal{G}_{@}$,
    $\mathcal{G}_{\$}$ , son $\Sigma $-computables y por lo tanto podemos hacer el siguiente programa via el uso de
    macros
    \begin{eqnarray*}
      && \qquad\;\;       \left[ \mathrm{P}\overline{m+3} \leftarrow
                              f(\mathrm{N}1,...,\mathrm{N}\bar{ n},\mathrm{P}1,...,\mathrm{P}\bar{m})\right] \\
      && \mathrm{L}3:\;\; \mathrm{IF}\;\mathrm{P}\overline{m+1}\; \text{BEGINS}\;@\; \mathrm{GOTO}\;\mathrm{L}1 \\
      && \qquad\;\;       \mathrm{IF}\;\mathrm{P}\overline{m+1}\; \text{BEGINS}\;\ \$\; \mathrm{GOTO}\;\mathrm{L}1 \\
      && \qquad\;\;       \mathrm{GOTO}\; \mathrm{L}4\\
      && \mathrm{L}1:\;\; \mathrm{P}\overline{m+1} \leftarrow \text{ }^{\curvearrowright } \mathrm{P}\overline{m+1} \\
      && \qquad\;\;       \left[ \mathrm{P}\overline{m+3}\; \leftarrow\; \mathcal{G}_{@}
                                (\mathrm{N} 1,...,\mathrm{N}\bar{n},
                                \mathrm{P}1,...,\mathrm{P}\bar{m},\mathrm{P} \overline{m+2},\mathrm{P}\overline{m+3})
                          \right] \\
      && \qquad\;\;       \mathrm{P}\overline{m+2}\leftarrow \mathrm{P}\overline{m+2}.@  \\
      && \qquad\;\;       \mathrm{GOTO}\;\mathrm{L}3 \\
      && \mathrm{L}2:\;\; \mathrm{P}\overline{m+1} \leftarrow \text{ }^{\curvearrowright } \mathrm{P}\overline{m+1} \\
      && \qquad\;\;       \left[ \mathrm{P}\overline{m+3}\; \leftarrow\; \mathcal{G}_{\$}
                                (\mathrm{N} 1,...,\mathrm{N}\bar{n},
                                \mathrm{P}1,...,\mathrm{P}\bar{m},\mathrm{P} \overline{m+2},\mathrm{P}\overline{m+3})
                          \right] \\
      && \qquad\;\;       \mathrm{P}\overline{m+2}\leftarrow \mathrm{P}\overline{m+2}.\$ \\
      && \qquad\;\;       \mathrm{GOTO}\;\mathrm{L}3 \\
      && \mathrm{L}4:\;\; \mathrm{P}1\leftarrow \mathrm{P}\overline{m+2} \\
    \end{eqnarray*}

    Luego se tiene que el programa computa a $h$, entonces $h$ es $\Sigma$-computable
  \end{proof}
  \end{theorem}

  %Lemma 57
  \begin{lemma}
    Sea $\Sigma $ un alfabeto cualquiera. Las funciones $S$ y $\overline{\ \;}$ son
    $(\Sigma \cup \Sigma _{p})$-p.r..
  \end{lemma}

  % Lemma 58
  \begin{lemma}
    Para cada $n,x\in \omega $, tenemos que $ \left\vert \bar{n}\right\vert \leq x$ si y solo si $n\leq 10^{x}-1$
  \end{lemma}

  % Lemma 59
  \begin{lemma}
    $\mathrm{Ins}^{\Sigma }$ es un conjunto $(\Sigma \cup \Sigma _{p})$-p.r..
  \begin{proof}
    Para simplificar la prueba asumiremos que $\Sigma =\{@,\& \}$. Ya que $ \mathrm{Ins}^{\Sigma }$ es union de los
    siguientes conjuntos

    \[
      \displaystyle
      \begin{array}{rcl}
        L_{1} & =& \left\{ \mathrm{N}\bar{k}\leftarrow \mathrm{N}\bar{k}+1:k\in \mathbf{N}\right\} \\
        L_{2} & =& \left\{ \mathrm{N}\bar{k}\leftarrow \mathrm{N}\bar{k}\dot{-}1:k\in \mathbf{N}\right\} \\
        L_{3} & =& \left\{ \mathrm{N}\bar{k}\leftarrow \mathrm{N}\bar{n}:k,n\in \mathbf{N}\right\} \\
        L_{4} & =& \left\{ \mathrm{N}\bar{k}\leftarrow 0:k\in \mathbf{N}\right\} \\
        L_{5} & =& \left\{ \mathrm{IF}\;\mathrm{N}\bar{k}\neq 0\;\mathrm{GOTO}\;
                           \mathrm{L}\bar{m}:k,m\in \mathbf{N}\right\} \\
        L_{6} & =& \left\{ \mathrm{P}\bar{k}\leftarrow \mathrm{P}\bar{k}.@:k\in \mathbf{N}\right\} \\
        L_{7} & =& \left\{ \mathrm{P}\bar{k}\leftarrow \mathrm{P}\bar{k}.\& :k\in \mathbf{N}\right\} \\
        L_{8} & =& \left\{ \mathrm{P}\bar{k}\leftarrow \text{ }^{\curvearrowright }
                           \mathrm{P}\bar{k}:k\in \mathbf{N}\right\} \\
        L_{9} & =& \left\{ \mathrm{P}\bar{k}\leftarrow \mathrm{P}\bar{n}:k,n\in \mathbf{N}\right\} \\
        L_{10} & =& \left\{ \mathrm{P}\bar{k}\leftarrow \varepsilon :k\in \mathbf{N} \right\} \\
        L_{11} & =& \left\{ \mathrm{IF}\;\mathrm{P}\bar{k}\;\mathrm{BEGINS}\;@\; \mathrm{GOTO}\;
                            \mathrm{L}\bar{m}:k,m\in \mathbf{N}\right\} \\
        L_{12} & =& \left\{ \mathrm{IF}\;\mathrm{P}\bar{k}\;\mathrm{BEGINS}\;\& \; \mathrm{GOTO}\;
                            \mathrm{L}\bar{m}:k,m\in \mathbf{N}\right\} \\
        L_{13} & =& \left\{ \mathrm{GOTO}\;\mathrm{L}\bar{m}:m\in \mathbf{N}\right\} \\
        L_{14} & =& \left\{ \mathrm{SKIP}\right\} \\
        L_{15} & =& \left\{ \mathrm{L}\bar{k}\alpha :k\in \mathbf{N\;}
                            \text{y }\alpha \in L_{1}\cup ...\cup L_{14}\right\}
      \end{array}
    \]

    solo debemos probar que $L_{1},...,L_{15}$ son $(\Sigma \cup \Sigma _{p})$ -p.r.. Veremos primero por ejemplo que
    \[
      \displaystyle L_{11}=\left\{ \mathrm{IFP}\bar{k}\mathrm{BEGINS}@\mathrm{GOTOL}\bar{m} :k,m\in \mathbf{N}\right\}
    \]
    es $(\Sigma \cup \Sigma _{p})$-p.r.. Primero nótese que $\alpha \in L_{11}$ si y solo si existen
    $k,m\in \mathbf{N}$ tales que
    \[
      \displaystyle \alpha =\mathrm{IFP}\bar{k}\mathrm{BEGINS}@\mathrm{GOTOL}\bar{m}
    \]
    Mas formalmente tenemos que $\alpha \in L_{11}$ si y solo si
    \[
      \displaystyle (\exists k\in \mathbf{N})(\exists m\in \mathbf{N})\;\alpha =\mathrm{IFP}\bar{ k}\mathrm{BEGINS}@
      \mathrm{GOTOL}\bar{m}
    \]
    Ya que cuando existen tales $k,m$ tenemos que $\bar{k}$ y $\bar{m}$ son subpalabras de $\alpha $, el
    \textbf{Lemma 58} nos dice que $\alpha \in L_{10}$ si y solo si
    \[
      \displaystyle (\exists k\in \mathbf{N})_{k\leq 10^{\left\vert \alpha \right\vert }}(\exists m\in \mathbf{N})_{m
      \leq 10^{\left\vert \alpha \right\vert }}\;\alpha =\mathrm{IFP}\bar{k}\mathrm{BEGINS}@\mathrm{GOTOL}\bar{m}
    \]
    Sea
    \[
      \displaystyle P=\lambda mk\alpha \left[ \alpha =\mathrm{IFP}\bar{k}\mathrm{BEGINS}@\mathrm{ GOTOL}\bar{m}\right]
    \]
    Ya que $D_{\lambda k\left[ \bar{k}\right] }=\omega $, tenemos que $ D_{P}=\omega \times (\Sigma \cup \Sigma _{p})
    ^{\ast }\times (\Sigma \cup \Sigma _{p})^{\ast }$. Nótese que
    \[
      \displaystyle P=\lambda \alpha \beta \left[ \alpha =\beta \right] \circ \left( p_{3}^{2,1},f\right)
    \]
    donde
    \[
      \displaystyle f=\lambda \alpha _{1}\alpha _{2}\alpha _{3}\alpha _{4}\left[ \alpha _{1}\alpha _{2}\alpha _{3}\alpha
      _{4}\right] \circ \left( C_{\mathrm{IFP} }^{2,1},\lambda k\left[ \bar{k}\right] \circ p_{2}^{2,1},C_{
      \mathrm{BEGINS}@ \mathrm{GOTOL}}^{2,1},\lambda k\left[ \bar{k}\right] \circ p_{1}^{2,1}\right)
    \]
    lo cual nos dice que $P$ es $(\Sigma \cup \Sigma _{p})$-p.r..
    Nótese que
    \[
      \displaystyle \chi _{L_{11}}=\lambda \alpha \left[ (\exists k\in \mathbf{N})_{k\leq 10^{\left\vert \alpha
      \right\vert }}(\exists m\in \mathbf{N})_{m\leq 10^{\left\vert \alpha \right\vert }}\;P(m,k,\alpha )\right]
    \]
    Esto nos dice que podemos usar dos veces el \textbf{Lema 39} para ver que $\chi _{L_{11}}$ es
    $(\Sigma \cup \Sigma _{p})$-p.r.. Veamos como. Sea
    \[
      \displaystyle Q=\lambda k\alpha \left[ (\exists m\in \mathbf{N})_{m\leq 10^{\left\vert \alpha \right\vert }}\;
      P(m,k,\alpha )\right]
    \]
    Por el \textbf{Lema 39} tenemos que
    \[
      \displaystyle \lambda xk\alpha \left[ (\exists m\in \mathbf{N})_{m\leq x}\;P(m,k,\alpha ) \right]
    \]
    es $(\Sigma \cup \Sigma _{p})$-p.r. lo cual nos dice que
    \[
      \displaystyle Q=\lambda xk\alpha \left[ (\exists m\in \mathbf{N})_{m\leq x}\;P(m,k,\alpha ) \right] \circ
      (\lambda \alpha \left[ 10^{\left\vert \alpha \right\vert } \right] \circ p_{2}^{1,1},p_{1}^{1,1},p_{2}^{1,1})
    \]
    lo es. Ya que
    \[
      \displaystyle \chi _{L_{10}}=\lambda \alpha \left[ (\exists k\in \mathbf{N})_{k\leq 10^{\left\vert \alpha \right
      \vert }}\;Q(k,\alpha )\right]
    \]
    podemos en forma similar aplicar el \textbf{Lema 39} y obtener finalmente que $\chi _{L_{11}}$ es
    $(\Sigma \cup \Sigma _{p})$-p.r..
    En forma similar podemos probar que $L_{1},...,L_{14}$ son $(\Sigma \cup \Sigma _{p})$-p.r..
    Esto nos dice que $L_{1}\cup ...\cup L_{14}$ es $(\Sigma \cup \Sigma _{p})$-p.r..
    Nótese que $L_{1}\cup ...\cup L_{14}$ es el conjunto de las instrucciones básicas de $\mathcal{S}^{\Sigma }$.
    Llamemos $ \mathrm{InsBas}^{\Sigma }$ a dicho conjunto.
    Para ver que $L_{15}$ es $ (\Sigma \cup \Sigma _{p})$-p.r. notemos que
    \[
      \displaystyle \chi _{L_{15}}=\lambda \alpha \left[ (\exists k\in \mathbf{N})_{k\leq 10^{\left\vert \alpha \right
      \vert }}(\exists \beta \in \mathrm{InsBas} ^{\Sigma })_{\left\vert \beta \right\vert \leq \left\vert \alpha \right
      \vert }\;\alpha =\mathrm{L}\bar{k}\beta \right]
    \]
    lo cual nos dice que aplicando dos veces el \textbf{Lema 39} obtenemos que $\chi _{L_{15}}$ es $(\Sigma \cup \Sigma _{p})
    $-p.r.. Ya que $ \mathrm{Ins}^{\Sigma }=\mathrm{InsBas}^{\Sigma }\cup L_{15}$ tenemos que
    $ \mathrm{Ins}^{\Sigma }$ es $(\Sigma \cup \Sigma _{p})$-p.r..
  \end{proof}
  \end{lemma}

  % Lemma 60
  \begin{lemma}
    $Bas$ y $Lab$ son funciones $(\Sigma \cup \Sigma _{p})$-p.r.

  \begin{proof}
    Sea $< $ un orden total estricto sobre $\Sigma \cup \Sigma _{p}$. Sea $L=\{ \mathrm{L}\bar{k}:k\in \mathbf{N}\}
    \cup \{\varepsilon \}$. Veamos que $ L $ es $\Sigma \cup \Sigma _{p}$-p.r.
    Sea $\chi_L$
    \[
      \displaystyle \chi_L(\alpha)=\left\{\begin{array}{lll}
                                            1 & & \text{si }\alpha\in L\\
                                            0 & & \text{si }\alpha\notin L
                                          \end{array}\right.
    \]
    Primero nótese que tenemos que $\alpha \in L$ sii
    \[
      \begin{array}{lll}
        (\exists k \in \mathbf{N})\ \alpha = L\bar{k} \lor \alpha = \varepsilon
      \end{array}
    \]
    Ya que cuando existe $k$ tenemos que $\bar{k}$ es una subpalabra de $\alpha $, el \textbf{lema 58} nos dice
    que $\alpha \in L$ si y solo si
    \[
      \begin{array}{lll}
        (\exists k \in \mathbf{N})_{k\leq 10^{\ \left\vert \alpha \right \vert}}\ \alpha = L\bar{k} \lor \alpha = \varepsilon
      \end{array}
    \]
    Sea
    \[
      \begin{array}{lll}
        Q & = & \lambda \alpha\left[
                                \lbrack \alpha = \mathrm{L}\bar{k} \lor \alpha = \varepsilon
                              \right]
      \end{array}
    \]
    Ya que $D_{\lambda k\left[ \bar{k}\right] }=\omega $, tenemos que
    $ D_{Q}=\omega \times (\Sigma \cup \Sigma _{p})^{\ast } $.
    Notese que
    \[
      \begin{array}{lll}
        Q & = & Q_1 \land Q_2
      \end{array}
    \]
    donde
    \[
      \begin{array}{lll}
        \displaystyle Q_1 = \lambda \alpha \beta \left[ \alpha =\beta \right] \circ
        \left( p_{2}^{1,1},
        \lambda \alpha \beta \left[ \alpha\beta \right] \circ (C_{L}^{1,1},\lambda k\left[ \bar{k}\right]
        \circ p_{2}^{1,1})
        \right) \\
        Q_2 = \lambda \alpha \beta \left[ \alpha =\beta \right] \circ (p_{2}^{1,1}, C_{\varepsilon}^{1,1})
      \end{array}
    \]
    lo cual nos dice que $Q$ es $(\Sigma \cup \Sigma _{p})$-p.r..
    Nótese que
    \[
      \displaystyle \chi_{L}=\lambda \alpha \left[ (\exists k\in \mathbf{N})_{k\leq 10^{\left\vert \alpha
      \right\vert }}\;Q(k,\alpha )\right]
    \]
    Esto nos dice que podemos usar el \textbf{Lema 39} para ver que $\chi_L$ es $(\Sigma \cup \Sigma _{p})$-p.r..
    Por el \textbf{Lema 39} tenemos que
    \[
      \displaystyle \lambda x\alpha \left[ (\exists m\in \mathbf{N})_{m\leq x}\;Q(m,\alpha ) \right]
    \]
    es $(\Sigma \cup \Sigma _{p})$-p.r. lo cual nos dice que
    \[
      \displaystyle \chi_L=\lambda x\alpha \left[ (\exists m\in \mathbf{N})_{m\leq x}\;Q(m,\alpha ) \right] \circ
      (\lambda \alpha \left[ 10^{\left\vert \alpha \right\vert } \right] \circ p_{1}^{0,1},p_{1}^{0,1})
    \]
    lo es. Luego como $ \chi_L $ es $(\Sigma \cup \Sigma _{p})$-p.r entonces $L$ es $(\Sigma \cup \Sigma _{p})$-p.r

    \noindent Sea
    \[
      \displaystyle P=\lambda I\alpha \left[ \alpha \in \mathrm{Ins}^{\Sigma }\wedge I\in \mathrm{Ins}^{\Sigma }\wedge
      \lbrack \alpha ]_{1}\neq \mathrm{L}\wedge (\exists \beta \in L)\ I=\beta \alpha \right]
    \]
    Veamos que $P$ es $(\Sigma \cup \Sigma _{p})$-p.r.
    Note que $D_{P}=(\Sigma \cup \Sigma _{p})^{\ast 2}$ y ademas
    \[
      \displaystyle P = P_1 \land P_2 \land P_3 \land P_4
    \]
    donde
    \[
      \begin{array}{lll}
        \displaystyle
        \bigskip
        P_1 = \lambda I\alpha \left[ \alpha \in \mathrm{Ins}^{\Sigma} \right] \\
        \bigskip
        P_2 = \lambda I\alpha \left[ I \in \mathrm{Ins}^{\Sigma } \right] \\
        \bigskip
        P_3 = \neg  (\lambda \alpha\beta \left[ \alpha =\beta \right] \circ
                    (
                      \lambda i\alpha \left[ \left[\alpha\right]_i \right] \circ
                      (
                      p_{2}^{0,2},
                      C_{1}^{0,2}
                      ), C_{L}^{0,2}
                    )) \\
        \bigskip
        P_4 = \lambda I\alpha \left[ (\exists \beta \in L)  I\ = \beta\alpha \right]
      \end{array}
    \]
    Es fácil ver que $P_1$, $P_2$ y $P_3$ son $(\Sigma \cup \Sigma _{p})$-p.r.
    Veamos que $P_4$ es $(\Sigma \cup \Sigma _{p})$-p.r. Para ello definamos el siguiente predicado
    \[
      R = \lambda I\alpha\beta \left[ I = \beta\alpha\right]
    \]
    Tenemos que
    \[
      P_4 = \lambda I\alpha \left[ (\exists \beta \in L)\; R(I, \alpha, \beta) \right]
    \]
    Notar que, como $\beta$ es una subpalabra de $I$, tenemos que $|\beta| \leq |I|$
    \[
      P_4 = \lambda I\alpha \left[ (\exists \beta \in L)_{|\beta|\leq |I|} R(I, \alpha, \beta) \right]
    \]
    Por el \textbf{Lema 39} tenemos que
    \[
      \displaystyle \lambda xI\alpha \left[ (\exists \beta \in L)_{|\beta|\leq x}\;Q(m,\alpha ) \right]
    \]
    es $(\Sigma \cup \Sigma _{p})$-p.r. Lo cual nos dice que
    \[
      \displaystyle P_4 = \lambda xI\alpha \left[ (\exists \beta \in L)_{|\beta|\leq x}\;Q(m,\alpha ) \right] \circ
      (\lambda \alpha \left[ |\alpha| \right] \circ p_{1}^{0,2}, p_{1}^{0,2}, p_{1}^{0,2})
    \]
    lo es.
    Por lo tanto $P$ es $(\Sigma \cup \Sigma _{p})$-p.r.
    Nótese ademas que cuando $I\in \mathrm{Ins}^{\Sigma }$ tenemos que
    $P(I,\alpha )=1$ sii $\alpha =Bas(I)$.
    Dejamos al lector probar que $Bas=M^{< }\left( P\right) $ % TODO: DEMOSTRAR
    por lo que para ver que $Bas$ es $(\Sigma \cup \Sigma _{p})$-p.r., solo nos falta ver que la función $Bas$
    es acotada por alguna función $(\Sigma \cup \Sigma _{p})$-p.r. y $(\Sigma \cup \Sigma _{p})$-total.
    Pero esto es trivial ya que $\left\vert Bas(I)\right\vert \leq \left\vert I\right\vert =p_{1}^{0,1}(I)$
    para cada $I\in \mathrm{Ins}^{\Sigma }$.
    Finalmente note que

    \[
      \displaystyle Lab=M^{< }\left( \lambda I\alpha \left[ \alpha Bas(I)=I\right] \right)
    \]
    lo cual nos dice que $Lab$ es $(\Sigma \cup \Sigma _{p})$-p.r..
  \end{proof}
  \end{lemma}


  % Lemma 61
  \begin{lemma}
    \par
    \begin{itemize}
      \item[(a)]  $\mathrm{Pro}^{\Sigma }$ es un conjunto $(\Sigma \cup \Sigma _{p}) $-p.r.
      \item[(b)]  $\lambda \mathcal{P}\left[ n(\mathcal{P})\right] $ y $\lambda i \mathcal{P}\left[ I_{i}
                  ^{\mathcal{P}}\right] $ son funciones $(\Sigma \cup \Sigma _{p})$-p.r..
    \end{itemize}
  \end{lemma}

  % Lemma 62
  \begin{lemma}
    \par Este lemma no se evalua.
  \end{lemma}

  % Lemma 63
  \begin{lemma}
    \par Este lemma no se evalua.
  \end{lemma}

  % Proposition 64
  \begin{proposition}
    Sean $n, m \leq 0$, las funciones $i^{n, m}, E_{\#j}^{n, m}, j = 1, 2, ...$ son $\Sigma \cup \Sigma_{p}$-PR.
  \end{proposition}

  % Theorem 65
  \begin{theorem}
    Las funciones $\Phi _{\#}^{n,m}$ y $\Phi _{\ast }^{n,m}$ son $(\Sigma \cup \Sigma _{p})$-recursivas.

  \begin{proof}
    Veremos que $\Phi _{\#}^{n,m}$ es $(\Sigma \cup \Sigma _{p})$-recursiva. Sea $H$ el predicado
    $(\Sigma \cup \Sigma _{p})$-mixto

    \[
      \displaystyle \lambda t\vec{x}\vec{\alpha}\mathcal{P}\left[ i^{n,m}(t,x_{1},...,x_{n}, \alpha _{1},...,\alpha _{m},
      \mathcal{P})=n(\mathcal{P})+1\right] \text{.}
    \]
    Note que $D_{H}=\omega ^{n+1}\times \Sigma ^{\ast m}\times \mathrm{Pro} ^{\Sigma }$.
    Ya que las funciones $i^{n,m}$ y $\lambda \mathcal{P}\left[ n( \mathcal{P})\right] $ son
    $(\Sigma \cup \Sigma _{p})$-p.r., $H$ lo es. Notar que $D_{M(H)}=D_{\Phi _{\#}^{n,m}}$.
    Ademas para $(\vec{x},\vec{\alpha}, \mathcal{P})\in D_{M(H)}$, tenemos que $M(H)(\vec{x},\vec{\alpha},
    \mathcal{P} )$ es la menor cantidad de pasos necesarios para que $\mathcal{P}$ termine partiendo del estado
    $((x_{1},...,x_{n},0,0,...),(\alpha _{1},...,\alpha _{m},\varepsilon ,\varepsilon ,...))$.
    Ya que $H$ es $(\Sigma \cup \Sigma _{p})$-p.r., tenemos que $M(H)$ es $(\Sigma \cup \Sigma _{p})$-r..
    Nótese que para $(\vec{x},\vec{\alpha},\mathcal{P})\in D_{M(H)}=D_{\Phi _{\#}^{n,m}} $ tenemos que
    \[
      \displaystyle \Phi _{\#}^{n,m}(\vec{x},\vec{\alpha},\mathcal{P})=E_{\#1}^{n,m}\left( M(H)( \vec{x},\vec{\alpha},
      \mathcal{P}),\vec{x},\vec{\alpha},\mathcal{P}\right)
    \]
    lo cual con un poco mas de trabajo nos permite probar que % TODO: Hacer el poco mas de trabajo
    \[
      \displaystyle \Phi _{\#}^{n,m}=E_{\#1}^{n,m}\circ \left( M(H),p_{1}^{n,m+1},...,p_{n+m+1}^{n,m+1}\right)
    \]
    Ya que la función $E_{\#1}^{n,m}$ es $(\Sigma \cup \Sigma _{p})$-r., lo es $ \Phi _{\#}^{n,m}$.
  \end{proof}
  \end{theorem}


  % Corollary 66
  \begin{corollary}
    Si $f:D_{f}\subseteq \omega ^{n}\times \Sigma ^{\ast m}\rightarrow O$ es $ \Sigma $-computable,
    entonces $f$ es $\Sigma $-recursiva.
  \begin{proof}
    Haremos el caso $O=\Sigma ^{\ast }$. Sea $\mathcal{P}_{0}$ un programa que compute a $f$.
    Primero veremos que $f$ es $(\Sigma \cup \Sigma _{p})$-recursiva. Note que
    \[
      \displaystyle f=\Phi _{\ast }^{n,m}\circ \left( p_{1}^{n,m},...,p_{n+m}^{n,m},C_{\mathcal{P }_{0}}^{n,m}\right)
    \]
    donde cabe destacar que $p_{1}^{n,m},...,p_{n+m}^{n,m}$ son las proyecciones respecto del alfabeto
    $\Sigma \cup \Sigma _{p}$, es decir que tienen dominio $\omega ^{n}\times (\Sigma \cup \Sigma _{p})^{\ast m}$.
    Ya que $\Phi _{\ast }^{n,m}$ es $(\Sigma \cup \Sigma _{p})$-recursiva tenemos que $f$ es
    $(\Sigma \cup \Sigma _{p})$-recursiva.
    O sea que el \textbf{Teorema 51} (independencia del alfabeto) nos dice que $f$ es $\Sigma $-recursiva.
  \end{proof}
  \end{corollary}


  % Tesis de Church
  \noindent \textbf{\underline{Tesis de Church:}} Toda función $\Sigma $-efectivamente computable es $\Sigma $-recursiva.

  % Corollary 67
  \begin{corollary}
    \par Este corolario no se evalua.
  \end{corollary}

  % Lemma 68
  \begin{lemma}
    Supongamos $f_{i}:D_{f_{i}}\subseteq \omega ^{n}\times \Sigma ^{\ast m}\rightarrow O$, $i=1,...,k$,
    son funciones $\Sigma $-recursivas tales que $D_{f_{i}}\cap D_{f_{j}}=\emptyset $ para $i\neq j$.
    Entonces la función $f_{1}\cup ...\cup f_{k}$ es $\Sigma $-recursiva.
  \begin{proof}
    Probaremos el caso $k=2$ y $O=\Sigma ^{\ast }$. Sean $\mathcal{P}_{1}$ y $ \mathcal{P}_{2}$ programas que
    computen las funciones $f_{1}$ y $f_{2}$, respectivamente. Sean
    \[
      \displaystyle
      \begin{array}{rcl}
        P_{1} & =& \lambda t\vec{x}\vec{\alpha}\left[ i^{n,m}(t,\vec{x},\vec{\alpha},
          \mathcal{P}_{1})=n(\mathcal{P}_{1})+1\right] \\
        P_{2} & =& \lambda t\vec{x}\vec{\alpha}\left[ i^{n,m}(t,\vec{x},\vec{\alpha},
          \mathcal{P}_{2})=n(\mathcal{P}_{2})+1\right]
      \end{array}
    \]
    Nótese que $D_{P_{1}}=D_{P_{2}}=\omega \times \omega ^{n}\times \Sigma ^{\ast m}$ y que $P_{1}$ y $P_{2}$
    son $(\Sigma \cup \Sigma _{p})$-p.r.. Ya que son $\Sigma $-mixtos, el \textbf{Teorema 51}
    (independencia del alfabeto) nos dice que son
    $\Sigma $-p.r.. También nótese que $D_{M((P_{1}\vee P_{2}))}=D_{f_{1}}\cup D_{f_{2}}$. Definamos
    \[
      \displaystyle
        \begin{array}{rcl}
          g_{1} & =& \lambda \vec{x}\vec{\alpha}\left[ E_{\ast 1}^{n,m}(M\left( (P_{1}\vee P_{2})\right)
            (\vec{x},\vec{\alpha}),\vec{x},\vec{\alpha}, \mathcal{P}_{1})^{P_{1}(M\left( (P_{1}\vee P_{2})\right)
            (\vec{x},\vec{\alpha }),\vec{x},\vec{\alpha})}\right] \\
          g_{2} & =& \lambda \vec{x}\vec{\alpha}\left[ E_{\ast 1}^{n,m}(M\left( (P_{1}\vee P_{2})\right)
            (\vec{x},\vec{\alpha}),\vec{x},\vec{\alpha}, \mathcal{P}_{2})^{P_{2}(M\left( (P_{1}\vee P_{2})\right)
            (\vec{x},\vec{\alpha }),\vec{x},\vec{\alpha})}\right]
        \end{array}
    \]
    Nótese que $g_{1}$ y $g_{2}$ son $\Sigma $-recursivas y que $ D_{g_{1}}=D_{g_{2}}=D_{f_{1}}\cup D_{f_{2}}$,
    Ademas nótese que
    \[
      \displaystyle
        g_{1}(\vec{x},\vec{\alpha})=\left\{
          \begin{array}{lll}
            f_{1}(\vec{x},\vec{\alpha}) & & \text{si } (\vec{x},\vec{\alpha})\in D_{f_{1}} \\
            \varepsilon & & \text{caso contrario}
          \end{array} \right.
    \]

    \[
      \displaystyle
        g_{2}(\vec{x},\vec{\alpha})=\left\{
          \begin{array}{lll}
            f_{2}(\vec{x},\vec{\alpha}) & & \text{si }(\vec{x},\vec{\alpha})\in D_{f_{2}} \\
            \varepsilon & & \text{caso contrario}
          \end{array} \right.
    \]
    O sea que $f_{1}\cup f_{2}=\lambda \alpha \beta \left[ \alpha \beta \right] \circ (g_{1},g_{2})$ es
    $\Sigma $-recursiva.
  \end{proof}
  \end{lemma}

  % Lemma 69
  \begin{lemma}
    Supongamos $\Sigma \supseteq \Sigma _{p}$. Entonces $ Halt^{\Sigma }$ es no $\Sigma $-recursivo.
  \begin{proof}
    Supongamos $Halt^{\Sigma }$ es $\Sigma $-recursivo y por lo tanto $\Sigma $ -computable.
    Por la proposición de existencia de macros tenemos que hay un macro
    \[
      \displaystyle \left[ \mathrm{IF}\;Halt^{\Sigma }(\mathrm{W}1)\;\mathrm{GOTO}\;\mathrm{A}1 \right]
    \]
    Sea $\mathcal{P}_{0}$ el siguiente programa de $\mathcal{S}^{\Sigma }$
    \[
      \displaystyle \mathrm{L}1\;\left[ \mathrm{IF}\;Halt^{\Sigma }(\mathrm{P}1)\;\mathrm{GOTO}\; \mathrm{L}1\right]
    \]
    Note que

    $\mathcal{P}_{0}$ termina partiendo desde $\left( (0,0,...),( \mathcal{P}_{0},\varepsilon ,\varepsilon ,...)
    \right) $ sii $Halt^{\Sigma }( \mathcal{P}_{0})=0$,

    \noindent lo cual produce una contradicción si tomamos en (*) $\mathcal{P}= \mathcal{P}_{0}$.
  \end{proof}
  \end{lemma}


  % Theorem 70
  \begin{theorem} % TODO: Hacer bien la ida y la vuelta
    Sea $S\subseteq \omega ^{n}\times \Sigma ^{\ast m}$. Entonces $S$ es $\Sigma $-efectivamente enumerable
    sii $S$ es $\Sigma $-recursivamente enumerable
  \begin{proof}
    ($\Rightarrow $) Use la Tesis de Church.

    ($\Leftarrow $) Use el Theorem 42.
  \end{proof}
  \end{theorem}

  % Theorem 71
  \begin{theorem} Dado $S\subseteq \omega ^{n}\times \Sigma ^{\ast m} $, son equivalentes
    \begin{enumerate}
      \item $S$ es $\Sigma $-recursivamente enumerable
      \item $S=I_{F}$, para alguna $F:D_{F}\subseteq \omega ^{k}\times \Sigma ^{\ast l}\rightarrow \omega ^{n}\times
            \Sigma ^{\ast m}$ tal que cada $F_{i}$ es $\Sigma $-recursiva.
      \item $S=D_{f}$, para alguna función $\Sigma $-recursiva $f$
      \item $S=\varnothing $ o $S=I_{F}$, para alguna $F:\omega \rightarrow \omega ^{n}\times \Sigma ^{\ast m}$
            tal que cada $F_{i}$ es $\Sigma $-p.r.
    \end{enumerate}
  \end{theorem}


  % Corollary 72
  \begin{corollary}
    Supongamos $f:D_{f}\subseteq \omega ^{n}\times \Sigma ^{\ast m}\rightarrow O$ es $\Sigma $-recursiva y
    $S\subseteq D_{f}$ es $ \Sigma $-r.e., entonces $f\mid _{S}$ es $\Sigma $-recursiva.
  \begin{proof}
    Supongamos $O=\Sigma ^{\ast }.$ Por el \textbf{Teorema 71} $S=D_{g}$, para alguna función
    $\Sigma $-recursiva $g.$ Nótese que componiendo adecuadamente podemos suponer que $I_{g}=\{\varepsilon \}.$
    O sea que tenemos $f\mid _{S}=\lambda \alpha \beta \left[ \alpha \beta \right] \circ (f,g)$.
  \end{proof}
  \end{corollary}


  % Corollary 73
  \begin{corollary}
    \par Este corolario no se evalua.
  \end{corollary}

  % Corollary 74
  \begin{corollary}
    Supongamos $S_{1},S_{2}\subseteq \omega ^{n}\times \Sigma ^{\ast m}$ son conjuntos $\Sigma $-r.e..
    Entonces $S_{1}\cap S_{2}$ es $\Sigma $-r.e..
  \begin{proof}
    Por el \textbf{Teorema 71} $S_{i}=D_{g_{i}}$, con $g_{1},g_{2}$ funciones $ \Sigma $-recursivas$.$
    Nótese que podemos suponer que $I_{g_{1}},I_{g_{2}} \subseteq \Sigma^{\ast} $ por lo que
    $S_{1}\cap S_{2}=D_{\lambda \alpha\beta \left[ \alpha\beta\right] \circ (g_{1},g_{2})}$ es $\Sigma $-r.e.$.$
  \end{proof}
  \end{corollary}

  % Corollary 75
  \begin{corollary}
    Supongamos $S_{1},S_{2}\subseteq \omega ^{n}\times \Sigma ^{\ast m}$ son conjuntos $\Sigma $-r.e..
    Entonces $S_{1}\cup S_{2}$ es $\Sigma $-r.e.
  \begin{proof}
    Supongamos $S_{1}\neq \emptyset \neq S_{2}.$ Sean $F,G:\omega \rightarrow \omega ^{n}\times \Sigma ^{\ast m}$
    tales que $I_{F}=S_{1}$, $I_{G}=S_{2}$ y las funciones $F_{i} {\acute{}} s$ y $G_{i} {\acute{}} s$ son
    $\Sigma $-recursivas. Sean $f=\lambda x\left[ Q(x,2)\right] $ y $ g=\lambda x\left[ Q(x\dot{-}1,2)\right] .$
    Sea $H:\omega \rightarrow \omega ^{n}\times \Sigma ^{\ast m}$ dada por
    \[
      \displaystyle H_{i}=(F_{i}\circ f)\mathrm{\mid }_{\{x:x\mathrm{\ es\ par}\}}\cup (G_{i}\circ g)
      \mathrm{\mid }_{\{x:x\mathrm{\ es\ impar}\}}
    \]

    Por el \textbf{Corollary 72} (restriccion de una funcion) y el \textbf{Lema 68} (union de funciones),
    cada $H_{i}$ es $ \Sigma $-recursiva. Ya que $I_{H}=S_{1}\cup S_{2}$.
    tenemos que $S_{1}\cup S_{2}$ es $\Sigma $-r.e.
  \end{proof}
  \end{corollary}

  % Theorem 76
  \begin{theorem}
     Sea $S\subseteq \omega ^{n}\times \Sigma ^{\ast m}$. Entonces $S$ es $\Sigma $-efectivamente computable sii
     $S$ es $\Sigma $-recursivo
  \begin{proof}
    ($\Rightarrow $) Use la Tesis de Church.

    ($\Leftarrow $) Use el Teorema 42. $\Box$
  \end{proof}
  \end{theorem}

  % Theorem 77
  \begin{theorem} Sea $S\subseteq \omega ^{n}\times \Sigma ^{\ast m}.$ Son equivalentes
    \begin{itemize}
      \item[(a)] $S$ es $\Sigma $-recursivo
      \item[(b)] $S$ y $(\omega ^{n}\times \Sigma ^{\ast m})-S$ son $\Sigma $ -recursivamente enumerables
    \end{itemize}
  \begin{proof}
    (a)$\Rightarrow $(b)$.$ Note que $S=D_{Pred\ \circ\ \chi _{S}}.$
    Luego, por \textbf{Teorema 71} $S$ es $\Sigma $-recursivamente enumerable. De igual manera podemos ver que
    $(\omega ^{n}\times \Sigma ^{\ast m})-S=D_{Pred\ \circ\ \chi _{(\omega ^{n}\times \Sigma ^{\ast m})-S}}$
    es $\Sigma $-recursivamente enumerable.
    Donde $\chi_{(\omega ^{n}\times \Sigma ^{\ast m})-S} = \lambda xy \left[x \dot{-}y\right] \circ\ (C_{1}^{1,0},
    \chi_S)$

    (b)$\Rightarrow $(a). Note que $\chi _{S}=C_{1}^{n,m}\mathrm{\mid }_{S}\cup C_{0}^{n,m}\mathrm{\mid }_{\omega
    ^{n}\times \Sigma ^{\ast m}-S}$.
  \end{proof}
  \end{theorem}

  % Lemma 78
  \begin{lemma} Supongamos que $\Sigma \supseteq \Sigma _{p}.$ Entonces
    \[
      \displaystyle A=\left\{ \mathcal{P}\in \mathrm{Pro}^{\Sigma }:Halt^{\Sigma }(\mathcal{P} )\right\}
    \]
    es $\Sigma $-r.e. y no es $\Sigma $-recursivo. Mas aun el conjunto
    \[
      \displaystyle N=\left\{ \mathcal{P}\in \mathrm{Pro}^{\Sigma }:\lnot Halt^{\Sigma }( \mathcal{P})\right\}
    \]
    no es $\Sigma $-r.e.
  \begin{proof}
    Sea $P=\lambda t\mathcal{P}\left[ i^{0,1}(t,\mathcal{P},\mathcal{P})=n( \mathcal{P})+1\right] $.
    Note que $P$ es $\Sigma $-p.r. por lo que $M(P)$ es $\Sigma $-r.. Ademas note que $D_{M(P)}=A$,
    lo cual implica que $A$ es $ \Sigma $-r.e.. Ya que $Halt^{\Sigma }$ es no $\Sigma $-recursivo por \textbf{Lema 69} y
    \[
      \displaystyle Halt^{\Sigma }=C_{1}^{0,1}\mid _{A}\cup C_{0}^{0,1}\mid _{N}
    \]
    el Lema 68 nos dice que $N$ no es $\Sigma $-r.e.. Finalmente supongamos $A$ es $\Sigma $-recursivo. Entonces el conjunto
    \[
      \displaystyle N=\left( \Sigma ^{\ast }-A\right) \cap \mathrm{Pro}^{\Sigma }
    \]
    debería serlo, lo cual es absurdo.
  \end{proof}
  \end{lemma}

% \section{Máquinas de Turing}

  % Lemma 79: Nada.
  \begin{lemma}
    \PN Este lemma no se evalua.
  \end{lemma}

	% Lemma 80: Con prueba.
  \begin{lemma}
  	\PN El predicado $\lambda dd^{\prime} \left[d \vdash d^{\prime}\right]$ es $(\Gamma \cup Q)$-PR.
  \end{lemma}
	\begin{proof}
	  \PN Note que $D_{\lambda dd^{\prime}\left[d\ \vdash d^{\prime }\right] }=Des\times Des$. También nótese que los
    predicados
    \begin{eqnarray*}
      \lambda q\sigma p\gamma \left[ (q,\sigma ,L)\in \delta (p,\gamma )\right] \\
      \lambda q\sigma p\gamma \left[ (q,\sigma ,R)\in \delta (p,\gamma )\right] \\
      \lambda q\sigma p\gamma \left[ (q,\sigma ,K)\in \delta (p,\gamma )\right]
    \end{eqnarray*}

    \PN son $(\Gamma \cup Q)$-PR, ya que los tres tienen dominio igual a $Q \times \Gamma \times Q \times \Gamma$ el
    cual es finito por \textbf{Corollary 36}.

    \vspace{3mm}
    \PN Sean:
    \begin{eqnarray*}
      && P_{R}: Des \times Des \times \Gamma \times \Gamma^{\ast} \times \Gamma^{\ast} \times Q \times Q \rightarrow
        \omega \\
      && P_{R}(d,d^{\prime},\sigma,\alpha,\beta,p,q) = 1 \Leftrightarrow \left(d = \alpha p \beta\right) \wedge
        \left((q,\sigma,R) \in \delta \left(p,\left[\beta B\right]_{1}\right)\right) \wedge \left(d^{\prime} =
        \left\lfloor \alpha \sigma q^{\curvearrowright}\beta \right\rfloor\right) \\
      \\
      && P_{L}: Des \times Des \times \Gamma \times \Gamma^{\ast} \times \Gamma^{\ast} \times Q \times Q \rightarrow
        \omega \\
      && P_{L}(d,d^{\prime},\sigma,\alpha,\beta,p,q) = 1 \Leftrightarrow \left(d = \alpha p \beta \right) \wedge
        \left((q,\sigma,L) \in \delta \left(p,\left[\beta B\right]_{1}\right)\right) \wedge \left(\alpha \neq
        \varepsilon\right) \wedge \left(d^{\prime} = \left\lfloor \alpha^{\curvearrowleft} q \left[\alpha\right]_{\lvert
        \alpha \rvert} \sigma^{\curvearrowright} \beta \right\rfloor\right) \\
      \\
      && P_{K}: Des \times Des \times \Gamma \times \Gamma^{\ast} \times \Gamma^{\ast} \times Q \times Q \rightarrow
        \omega \\
      && P_{K}(d,d^{\prime},\sigma,\alpha,\beta,p,q) = 1 \Leftrightarrow \left(d = \alpha p \beta\right) \wedge
        \left((q,\sigma,K) \in \delta \left(p,\left[\beta B\right]_{1}\right)\right) \wedge \left(d^{\prime} =
        \left\lfloor \alpha q \sigma^{\curvearrowright} \beta \right\rfloor\right)
      \end{eqnarray*}
    \PN Veamos, por ejemplo, que $P_L$ es $(\Gamma \cup Q)$-PR. Notar que:
    \[
      P_L = P_1 \land P_2 \land P_3 \land P_4
    \]

    \PN donde $P_1, P_2,  P_3, P_4$ son los siguientes predicados:
    \begin{eqnarray*}
      P_1 &=& \lambda dd^{\prime}\sigma\alpha\beta pq \left[d = \alpha p \beta\right] \\
          &=& \lambda \alpha\beta\left[\alpha = \beta\right]
                \circ (
                  p_{1}^{0,7},
                  \lambda \alpha_1\alpha_2\alpha_3\left[\alpha_1\alpha_2\alpha_3\right]
                    \circ (
                      p_{4}^{0,7},
                      p_{6}^{0,7},
                      p_{5}^{0,7})) \\
      \\
      P_2 &=& \lambda dd^{\prime}\sigma\alpha\beta pq \left[(q,\sigma,L) \in \delta \left(p,\left[\beta B\right]_{1}
        \right)\right] \\
          &=& \lambda q\sigma p\gamma \left[(q,\sigma,L) \in \delta (p,\gamma)\right]
                \circ (
                  p_{7}^{0,7},
                  p_{3}^{0,7},
                  p_{6}^{0,7},
                  \lambda i\alpha \left[\left[\alpha\right]_i\right]
                  \circ (
                    C_{1}^{0,7},
                    \lambda \alpha \beta \left[ \alpha\beta \right] \circ (p_{5}^{0,7}, C_{B}^{0,7}))) \\
      \\
      P_3 &=& \lambda d d^{\prime}\sigma\alpha\beta p q \left[ \alpha \neq \varepsilon \right] \\
          &=& \lambda \alpha\beta \left[\alpha \neq \beta\right]
              \circ (
                p_{4}^{0,7},
                C_{\varepsilon}^{0,7}) \\
      \\
      P_4 &=& \lambda dd^{\prime}\sigma\alpha\beta pq \left[d^{\prime} = \left\lfloor \alpha^{\curvearrowleft}q
        \left[\alpha\right]_{\lvert \alpha \rvert } \sigma^{\curvearrowright} \beta \right\rfloor\right] \\
          &=& \lambda \alpha\beta\left[ \alpha=\beta \right]
                \circ (
                  p_{2}^{0,7},
                  \lambda \alpha \left[ \lfloor \alpha \rfloor \right] \circ f) \\
    \end{eqnarray*}
    \PN donde:
    \begin{align*}
      f =  \lambda\alpha_1\alpha_2\alpha_3\alpha_4\alpha_5 \left[ \alpha_1\alpha_2\alpha_3\alpha_4\alpha_5 \right]
            \circ (
              \lambda \alpha \left[ \alpha ^{\curvearrowleft } \right] \circ p_{4}^{0,7},
              p_{7}^{0,7},
              \lambda i\alpha \left[ \left[\alpha\right]_i \right]
                \circ (
                  \lambda \alpha \left[\lvert\alpha\rvert \right] \circ p_{4}^{0,7},
                  p_{4}^{0,7}
              ),
              p_{3}^{0,7},
              \lambda \alpha \left[ ^{\curvearrowright}\alpha \right] \circ p_{5}^{0,7}
            )
    \end{align*}

    \PN Luego, notar que $P_1, P_2 , P_3, P_4 $ son $(\Gamma \cup Q)$-PR, por lo tanto $P_L$ es $(\Gamma \cup Q)$-PR. De
    manera similar, podemos ver que $P_K$ y $P_R$ son $(\Gamma \cup Q)$-PR.

    \PN Tomemos el siguiente predicado:
    \[
      P = (P_{R}\vee P_{L}\vee P_{K})
    \]

    \PN Tenemos que $P$ es $(\Gamma \cup Q)$-PR. Nótese que $\lambda dd^{\prime} \left[d\vdash d^{\prime}\right]$ es
    igual al predicado:
    \begin{eqnarray*}
      \lambda dd^{\prime}\left[(\exists \sigma \in \Gamma)(\exists \alpha,\beta \in \Gamma^{\ast})(\exists p,q \in Q)
      \; P(d,d^{\prime},\sigma,\alpha,\beta,p,q)\right]
    \end{eqnarray*}

    \PN Luego, aplicando cinco veces el \textbf{Lemma 39} obtenemos que $\lambda dd^{\prime} \left[d\vdash d^{\prime}
    \right]$ es $(\Gamma \cup Q)$-PR. Veamos esto. Notar primero que como $\beta, \alpha, \sigma, p, q$ son subpalabras
    de $d$ y $d^{\prime}$ respectivamente, tenemos que $\lvert\beta\rvert, \lvert\alpha\rvert, \lvert\sigma\rvert,
    \lvert p \rvert, \lvert q \rvert \leq \lvert d \rvert + \lvert d^{\prime} \rvert$.
    \begin{align*}
      L_1 &= \lambda xdd^{\prime}\sigma\alpha\beta p \left[(\exists q \in Q)_{\lvert q \rvert \leq x} \; P(d,d^{\prime},
             \sigma,\alpha,\beta,p,q)\right] \\
      Q_1 &= \lambda dd^{\prime}\sigma\alpha\beta p \left[(\exists q \in Q)_{\lvert q \rvert \leq \lvert d \rvert +
             \lvert d^{\prime} \rvert} \; P(d,d^{\prime},\sigma,\alpha,\beta,p,q)\right] \\
          &= L_1 \circ (\lambda xy \left[x + y\right] \circ(\lambda \alpha \left[\lvert\alpha\rvert\right] \circ
             p_{1}^{0,6}, \lambda \alpha \left[\lvert\alpha\rvert\right] \circ p_{2}^{0,6}), p_{1}^{0,6}, p_{2}^{0,6},
             p_{3}^{0,6}, p_{4}^{0,6}, p_{5}^{0,6}, p_{6}^{0,6})
    \end{align*}

    \PN Por \textbf{Lemma 39} tenemos que $L_{1}$ es $(\Gamma \cup Q)$-PR y por ende $Q_{1}$ es $(\Gamma \cup Q)$-PR. De
    la misma manera podemos que el predicado $Q_{2}$ es $(\Gamma \cup Q)$-PR.
    \begin{align*}
      L_2 &= \lambda xdd^{\prime}\sigma\alpha\beta \left[(\exists p \in Q)_{\lvert p \rvert \leq x} \;
             Q_{1}(d,d^{\prime},\sigma,\alpha,\beta,p)\right] \\
      Q_2 &= \lambda dd^{\prime}\sigma\alpha\beta \left[(\exists p \in Q)_{\lvert p \rvert \leq \lvert d \rvert +
             \lvert d^{\prime} \rvert} \; Q_{1}(d,d^{\prime},\sigma,\alpha,\beta,p)\right] \\
          &= L_2 \circ (\lambda xy \left[x + y\right] \circ(\lambda \alpha \left[\lvert\alpha\rvert\right] \circ
             p_{1}^{0,5}, \lambda \alpha \left[\lvert\alpha\rvert\right] \circ p_{2}^{0,5}), p_{1}^{0,5}, p_{2}^{0,5},
             p_{3}^{0,5}, p_{4}^{0,5}, p_{5}^{0,5})
    \end{align*}
    \PN finalmente, tenemos que $Q_{3}, Q_{4}, Q_{5}$ son $(\Gamma \cup Q)$-PR.
    \begin{align*}
      L_3 &= \lambda xdd^{\prime}\sigma\alpha \left[(\exists \beta \in \Gamma^{\ast})_{\lvert\beta\rvert \leq x} \;
             Q_{2}(d,d^{\prime},\sigma,\alpha,\beta)\right] \\
      Q_3 &= \lambda dd^{\prime}\sigma\alpha \left[(\exists \beta \in \Gamma^{\ast})_{\lvert\beta\rvert \leq \lvert d
             \rvert + \lvert d^{\prime} \rvert} \; Q_{2}(d,d^{\prime},\sigma,\alpha,\beta)\right] \\
          &= L_3 \circ (\lambda xy \left[x + y\right] \circ(\lambda \alpha \left[\lvert\alpha\rvert\right] \circ
             p_{1}^{0,4}, \lambda \alpha \left[\lvert\alpha\rvert\right] \circ p_{2}^{0,4}), p_{1}^{0,4}, p_{2}^{0,4},
             p_{3}^{0,4}, p_{4}^{0,4}) \\[10pt]
      L_4 &= \lambda xdd^{\prime}\sigma \left[(\exists \alpha \in \Gamma^{\ast})_{\lvert\alpha\rvert \leq x} \;
             Q_{3}(d,d^{\prime},\sigma,\alpha)\right] \\
      Q_4 &= \lambda dd^{\prime}\sigma \left[(\exists \alpha \in \Gamma^{\ast})_{\lvert\alpha\rvert \leq \lvert d
             \rvert + \lvert d^{\prime} \rvert} \; Q_{3}(d,d^{\prime},\sigma,\alpha)\right] \\
          &= L_4 \circ (\lambda xy \left[x + y\right] \circ(\lambda \alpha \left[\lvert\alpha\rvert\right] \circ
             p_{1}^{0,3}, \lambda \alpha \left[\lvert\alpha\rvert\right] \circ p_{2}^{0,3}), p_{1}^{0,3}, p_{2}^{0,3},
             p_{3}^{0,3}) \\[10pt]
      L_5 &= \lambda xdd^{\prime} \left[(\exists \sigma \in \Gamma)_{\lvert\sigma\rvert \leq x} \; Q_{4}(d,d^{\prime},
             \sigma)\right] \\
      Q_5 &= \lambda dd^{\prime} \left[(\exists \sigma \in \Gamma)_{\lvert\sigma\rvert \leq \lvert d \rvert + \lvert
             d^{\prime} \rvert} \; Q_{4}(d,d^{\prime},\sigma)\right] \\
          &= L_5 \circ (\lambda xy \left[x + y\right] \circ(\lambda \alpha \left[\lvert\alpha\rvert\right] \circ
             p_{1}^{0,2}, \lambda \alpha \left[\lvert\alpha\rvert\right] \circ p_{2}^{0,2}), p_{1}^{0,2}, p_{2}^{0,2})
    \end{align*}
    Notar que $Q_5 = \lambda dd^{\prime } \left[ d\vdash d^{\prime }\right]$. Por lo tanto,
    $\lambda dd^{\prime } \left[ d\vdash d^{\prime }\right]$ es $(\Gamma \cup Q)$-PR.
  \end{proof}

	% Proposition 81: Sin prueba.
	\begin{proposition}
		\PN $\lambda ndd^{\prime} \left[d \overset{n}{\vdash} d^{\prime}\right]$ es $(\Gamma \cup Q)$-PR.
	\end{proposition}

  % Theorem 82: Con prueba.
  \begin{theorem}
  	\PN Sea $M = \left(Q, \Sigma, \Gamma, \delta, q_{0}, B, F\right)$ una máquina de Turing, entonces $L(M)$ es
    $\Sigma$-recursivamente enumerable.
  \end{theorem}
  \begin{proof}
    \PN Sea $P$ el siguiente predicado $(\Gamma \cup Q)$-mixto:
    \[
      P = \lambda n\alpha \left[(\exists d \in Des) \; \left\lfloor q_{0} B \alpha \right\rfloor \overset{n}{\vdash} d
      \wedge St(d) \in F\right]
    \]

    \PN Nótese que $D_{P} = \omega \times \Gamma^{\ast}$. Veamos que $P$ es $(\Gamma \cup Q)$-PR. Para ello tomemos:
    \[
      P = P_1 \land P_2
    \]

    \PN donde $P_1$ y $P_2$:
    \begin{eqnarray*}
      P_1 &=& \lambda n\alpha \left[(\exists d \in Des) \left\lfloor q_{0} B \alpha \right\rfloor \overset{n}{\vdash}
        d\right] \\
      P_2 &=& \lambda n\alpha \left[St(d) \in F\right]
    \end{eqnarray*}

    \PN Sabemos que el conjunto $F$ es finito, por \textbf{Corollary 30}, $F$ es $(\Gamma \cup Q)$-PR. Además $\chi_F$
    es $(\Gamma \cup Q)$-PR, por lo tanto el predicado $P_2$ es $(\Gamma \cup Q)$-PR.

    \PN Definimos:
    \begin{eqnarray*}
      Q &=& \lambda n\alpha d \left[ \left\lfloor q_{0} B \alpha \right\rfloor \overset{n}{\vdash} d\right] \\
        &=& \lambda ndd^{\prime} \left[d \overset{n}{\vdash} d\right] \circ (\lambda \alpha \left[\left\lfloor \alpha
        \right\rfloor\right] \circ(\lambda \alpha\beta \left[\alpha\beta\right] \circ (C_{q_0 B}^{1,2}, p_{2}^{1,2})),
        p_{3}^{1,2})
    \end{eqnarray*}

    \PN Por \textbf{Lemma 39} tenemos que:
    \[
      L = \lambda x n\alpha \left[ (\exists d\in Des)_{\lvert d \rvert \leq x}\; Q(n, \alpha, d)\right] \\[10pt]
    \]

    \PN es $(\Gamma \cup Q)$-PR, es decir, solo nos falta acotar el cuantificador existencial, para poder aplicar el
    \textbf{Lemma 39}. Dado que cuando $d_{1},\dotsc,d_{n+1} \in Des$ son tales que $d_{1} \vdash d_{2} \vdash \dotsc
    \vdash d_{n+1}$ tenemos que:
    \[
      \left\vert d_{i} \right\vert \leq \left\vert d_{1} \right\vert + n \text{, para } i=1, \dotsc, n
    \]
    \PN luego, una posible cota para dicho cuantificador es:
    \[
      \lvert d \rvert \leq \lvert\lfloor q_{0} B \alpha \rfloor\rvert + n
    \]

    \PN Por lo tanto, tenemos que el predicado $P_1$ es $(\Gamma \cup Q)$-PR. En definitiva $P$ es $(\Gamma \cup Q)$-PR.
    \PN Sea:
    \[
      P^{\prime }=P\mid _{\omega \times \SIGMA}
    \]

    \PN nótese que $P^{\prime}(n,\alpha) = 1 \Leftrightarrow \alpha \in L(M)$ atestiguado por una computación de
    longitud $n$.
    \PN Por \textbf{Corollary 72}, $P^{\prime}$ es $(\Gamma \cup Q)$-PR, y además es $\Sigma$-mixto. Utilizando el
    \textbf{Theorem 51} tenemos que $P^{\prime}$ es $\Sigma $-PR.

    \PN Dado que $L(M) = D_{M(P^{\prime})}$, el \textbf{Theorem 71} nos dice que $L(M)$ es $\Sigma$-recursivamente
    enumerable.
  \end{proof}

	% Theorem 83: Con prueba.
	\begin{theorem}
		\PN Supongamos $f: S \subseteq \omega^{n} \times \Sigma^{\ast m} \rightarrow O$ es $\Sigma$-Turing computable,
    entonces $f$ es $\Sigma$-recursiva.
  \end{theorem}
	\begin{proof}
    Supongamos $O=\SIGMA$. Sean:
    \begin{itemize}
      \item $M = \left(Q,\Sigma,\Gamma,\delta,q_{0},B,\shortmid,F\right)$ una máquina de Turing determinística con unit
        la cual compute a $f$.
      \item $<$ un orden total estricto sobre $\Gamma \cup Q$.
      \item $P: \mathbf{N} \times \omega^{n} \times \Sigma^{\ast m} \rightarrow \omega$ dado por:
        \[
          P = P_1 \wedge P_2
        \]
      \end{itemize}
    \PN donde:
    \begin{align*}
      P_1 &= \lambda x\vec{x}\vec{\alpha} \left[(\exists q \in Q) \ \left\lfloor q_{0} B \shortmid^{x_{1}} B \dotsc B
             \shortmid^{x_{n}} B \alpha_{1} B \dotsc B \alpha_{m} \right\rfloor \overset{(x)_{1}}{\vdash} \left\lfloor
             q B \ast^{<}((x)_{2}) \right\rfloor\right] \\
      P_2 &= \lambda x\vec{x}\vec{\alpha} \left[(\forall d \in Des)_{\left\vert d \right\vert \leq \left\vert
             \ast^{<}((x)_{2}) \right\vert + 2} \; \left\lfloor q B \ast^{<}((x)_{2}) \right\rfloor \nvdash d\right]
    \end{align*}

    \PN Si tomamos $Q_1$ y $Q_2$ como:
    \begin{align*}
      Q_1 &= \lambda x\vec{x}\vec{\alpha}q \left[\left\lfloor q_{0} B \shortmid^{x_{1}} B \dotsc B \shortmid^{x_{n}} B
             \alpha_{1} B \dotsc B \alpha_{m} \right\rfloor \overset{(x)_{1}}{\vdash} \left\lfloor q B \ast^{<}((x)_{2})
             \right\rfloor\right] \\
      Q_2 &= \lambda x\vec{x}\vec{\alpha}d \left[\left\lfloor q B \ast^{<}((x)_{2}) \right\rfloor \nvdash d\right]
    \end{align*}
    \PN Tenemos que:
    \begin{align*}
      P_1 &= \lambda x\vec{x}\vec{\alpha} \left[(\exists q \in Q) \ Q_1(x,\vec{x},\vec{\alpha},q)\right] \\
      P_2 &= \lambda x\vec{x}\vec{\alpha} \left[(\forall d \in Des)_{\left\vert d \right\vert \leq \left\vert \ast^{<}
             ((x)_{2}) \right\vert + 2} \; Q_2(x,\vec{x},\vec{\alpha},d)\right]
    \end{align*}

    \PN Es fácil ver que $Q_1$ y $Q_2$ son $(\Gamma \cup Q)$-PR. Luego, aplicando el \textbf{Lemma 39} tenemos que $P_1$
    y $P_2$ son $(\Gamma \cup Q)$-PR y por ende $P$ es $(\Gamma \cup Q)$-PR. Dado que $P$ es $\Sigma $-mixto, el
    \textbf{Theorem 51} nos dice que es $\Sigma $-PR. Nótese que:
    \[
      f = \lambda \vec{x}\vec{\alpha} \left[\left(\min_{x} \ P(x,\vec{x},\vec{\alpha})\right)_{2}\right]
    \]
    \PN lo cual nos dice que $f$ es $\Sigma$-recursiva.
  \end{proof}

	% Lema 84: Con prueba.
	\begin{lemma}
		\PN Sea $\mathcal{P} \in \mathrm{Pro}^{\Sigma}$ y sea $k$ tal que las variables que ocurren en $\mathcal{P}$ están
    todas en la lista $\mathrm{N}1, \dotsc, \mathrm{N}\bar{k}, \mathrm{P}1, \dotsc, \mathrm{P}\bar{k}$. Para cada $a \in
    \Sigma \cup \{\shortmid\}$, sean:

    \begin{itemize}
      \item $\tilde{a}$ un nuevo símbolo
      \item $\Gamma = \Sigma \cup \{B, \shortmid\} \cup \{\tilde{a}: a \in \Sigma \cup \{\shortmid\}\}$
    \end{itemize}

    \PN entonces existe una máquina de Turing determinística con unit $M = \left(Q, \Gamma, \Sigma, \delta, q_{0}, B,
    \shortmid, \{q_{f}\}\right)$ la cual satisface:

    \begin{enumerate}
      \item $\delta(q_{f},\sigma) = \emptyset$, para cada $\sigma \in \Gamma$.
      \item Cualesquiera sean $x_{1}, \dotsc, x_{k} \in \omega$ y $\alpha_{1}, \dotsc, \alpha_{k} \in \SIGMA$, el
        programa $\mathcal{P}$ se detiene partiendo del estado:
        \[
          \left((x_{1}, \dotsc, x_{k}, 0, \dotsc), (\alpha_{1}, \dotsc, \alpha_{k}, \varepsilon,\dotsc)\right)
        \]
        \PN si y sólo si $M$ se detiene partiendo de la descripción instantánea:
        \[
          \lfloor q_{0} B \shortmid^{x_{1}} B \dotsc B \shortmid^{x_{k}} B \alpha_{1} B \dotsc B \alpha_{k} B \rfloor
        \]

      \item Si $x_{1}, \dotsc, x_{k} \in \omega$ y $\alpha_{1}, \dotsc, \alpha_{k} \in \SIGMA$ son tales que
        $\mathcal{P}$ se detiene partiendo del estado:
        \[
          \left((x_{1}, \dotsc, x_{k}, 0, \dotsc), (\alpha_{1}, \dotsc, \alpha_{k}, \varepsilon, \dotsc)\right)
        \]
        \PN y llega al estado
        \[
          \left((y_{1}, \dotsc, y_{k}, 0, \dotsc), (\beta_{1}, \dotsc, \beta_{k}, \varepsilon,\dotsc)\right)
        \]
        \PN entonces
        \[
          \lfloor q_{0} B \shortmid^{x_{1}} B \dotsc B \shortmid^{x_{k}} B \alpha_{1} B \dotsc B \alpha_{k} B \rfloor
          \overset {\ast}{\vdash} \lfloor q_{f} B \shortmid^{y_{1}} B \dotsc B \shortmid^{y_{k}} B \beta_{1} B \dotsc B
          \beta_{k} B \rfloor
        \]
    \end{enumerate}
  \end{lemma}
  \begin{proof}
    \PN Dado un estado $((x_{1}, \dotsc, x_{k}, 0, \dotsc),(\alpha_{1}, \dotsc, \alpha_{k}, \varepsilon, \dotsc))$,
    dicho estado se representará en la cinta de la siguiente manera:
    \[
      B \shortmid^{x_{1}} B \dotsc B \shortmid^{x_{k}} B \alpha_{1} B \dotsc B \alpha_{k} BBBB \dotsc
    \]

    \PN A continuación se describirán una serie de máquinas, las cuales simularán, vía la representación anterior, las
    distintas clases de instrucciones que pueden ocurrir en $\mathcal{P}$. Todas las máquinas definidas tendrán:

    \begin{itemize}
      \item $\shortmid$ como unit
      \item $B$ como blanco
      \item $\Sigma$ como su alfabeto terminal
      \item su alfabeto mayor será $\Gamma = \Sigma \cup \{B, \shortmid\} \cup \{\tilde{a}: a \in \Sigma \cup
        \{\shortmid\}\}$
      \item uno o dos estados finales con la propiedad de que si $q$ es un estado final, entonces $\delta (q,\sigma) =
        \emptyset$, para cada $\sigma \in \Gamma$.
        \PN Esta propiedad es importante ya que permitirá concatenar pares de
        dichas máquinas identificando algún estado final de la primera con el inicial de la segunda. \\
    \end{itemize}

    \PN Para $1 \leq i \leq k$, sea $M_{i}^{+}$ una máquina tal que:
    \[
      \begin{array}{lcl}
        B \shortmid^{x_{1}} B \dotsc B \shortmid^{x_{k}} B \alpha_{1} B \dotsc B \alpha_{k} &\overset{\ast}{\vdash}& B
        \shortmid^{x_{1}} B \dotsc B \shortmid^{x_{i-1}} B \shortmid^{x_{i}+1} B \shortmid^{x_{i+1}} \dotsc B
        \shortmid^{x_{k}} B \alpha_{1} B \dotsc B \alpha_{k} \\
        \uparrow && \uparrow \\
        q_{0} & & q_{f}
      \end{array}
    \]

    \PN Para $1 \leq i \leq k$, sea $M_{i}^{\dot{-}}$ una máquina tal que:
    \[
      \begin{array}{lcl}
        B \shortmid^{x_{1}} B \dotsc B \shortmid^{x_{k}} B \alpha_{1} B \dotsc B \alpha_{k} &\overset{\ast}{\vdash}& B
          \shortmid^{x_{1}} B \dotsc B \shortmid^{x_{i-1}} B \shortmid^{x_{i} \dot{-} 1} B \shortmid^{x_{i+1}} \dotsc B
          \shortmid^{x_{k}} B \alpha_{1} B \dotsc B \alpha_{k} \\
        \uparrow && \uparrow \\
        q_{0} && q_{f}
      \end{array}
    \]

    \PN Para $1 \leq i \leq k$ y $a \in \Sigma $, sea $M_{i}^{a}$ una máquina tal que:
    \[
      \begin{array}{lcl}
        B \shortmid^{x_{1}} B \dotsc B \shortmid^{x_{k}} B \alpha_{1} B \dotsc B \alpha_{k} &\overset{\ast}{\vdash}& B
          \shortmid^{x_{1}} B \dotsc B \shortmid^{x_{k}} B \alpha_{1} B \dotsc B \alpha_{i-1} B \alpha_{i} a B
          \alpha_{i+1} B \dotsc B \alpha_{k} \\
        \uparrow && \uparrow \\
        q_{0} && q_{f}
      \end{array}
    \]

    \PN Para $1 \leq i \leq k$, sea $M_{i}^{\curvearrowright}$ una máquina tal que:
    \[
      \begin{array}{lcl}
        B \shortmid^{x_{1}} B \dotsc B \shortmid^{x_{k}} B \alpha_{1} B \dotsc B \alpha_{k} &\overset{\ast}{\vdash}& B
          \shortmid^{x_{1}} B \dotsc B \shortmid^{x_{k}} B \alpha_{1} B \dotsc B \alpha_{i-1} B^{\curvearrowright}
          \alpha_{i} B \alpha_{i+1} B \dotsc B \alpha_{k} \\
        \uparrow && \uparrow \\
        q_{0} && q_{f}
      \end{array}
    \]

    \PN Para $j=1, \dotsc, k$, y $a \in \Sigma$, sea $IF_{j}^{a}$ una máquina con dos estados finales $q_{si}$ y
    $q_{no}$ tal que:
    \PN Si $\alpha_{j}$ comienza con $a$, entonces:
    \[
      \begin{array}{lcl}
        B \shortmid^{x_{1}} B \dotsc B \shortmid^{x_{k}} B \alpha_{1} B \dotsc B \alpha_{k} &\overset{\ast}{\vdash}& B
          \shortmid^{x_{1}} B \dotsc B \shortmid^{x_{k}} B \alpha_{1} B \dotsc B \alpha_{k} \\
        \uparrow && \uparrow \\
        q_{0} && q_{si}
      \end{array}
    \]

    \PN Caso contrario:
    \[
      \begin{array}{lcl}
        B \shortmid^{x_{1}} B \dotsc B \shortmid^{x_{k}} B \alpha_{1} B \dotsc B \alpha_{k} &\overset{\ast}{\vdash}& B
          \shortmid^{x_{1}} B \dotsc B \shortmid^{x_{k}} B \alpha_{1} B \dotsc B \alpha_{k} \\
        \uparrow && \uparrow \\
        q_{0} && q_{no}
      \end{array}
    \]

    \PN Análogamente, para $j=1,\dotsc,k$, sea $IF_{j}$ una máquina tal que:
    \PN Si $x_{j} \neq 0$, entonces:
    \[
      \begin{array}{lcl}
        B \shortmid^{x_{1}} B \dotsc B \shortmid^{x_{k}} B \alpha_{1} B \dotsc B \alpha_{k} &\overset{\ast}{\vdash}& B
          \shortmid^{x_{1}} B \dotsc B \shortmid^{x_{k}} B \alpha_{1} B \dotsc B \alpha_{k} \\
        \uparrow && \uparrow \\
        q_{0} && q_{si}
      \end{array}
    \]

	  \PN Si $x_{j} = 0$, entonces:
    \[
      \begin{array}{lcl}
        B \shortmid^{x_{1}} B \dotsc B \shortmid^{x_{k}} B \alpha_{1} B \dotsc B \alpha_{k} &\overset{\ast}{\vdash}& B
          \shortmid^{x_{1}} B \dotsc B \shortmid^{x_{k}} B \alpha_{1} B \dotsc B \alpha_{k} \\
        \uparrow && \uparrow \\
        q_{0} && q_{no}
      \end{array}
    \]

	  \PN Para $1 \leq i,j \leq k$, sea $M_{i\leftarrow j}^{\ast}$ una máquina tal que:
    \[
      \begin{array}{lcl}
        B \shortmid^{x_{1}} B \dotsc B \shortmid^{x_{k}} B \alpha_{1} B \dotsc B \alpha_{k} &\overset{\ast}{\vdash}& B
          \shortmid^{x_{1}} B \dotsc B \shortmid^{x_{k}} B \alpha_{1} B \dotsc B \alpha_{i-1} B \alpha_{j} B
          \alpha_{i+1} B \dotsc B \alpha_{k} \\
        \uparrow && \uparrow \\
        q_{0} & & q_{f}
      \end{array}
    \]

    \PN Para $1 \leq i,j \leq k$, sea $M_{i\leftarrow j}^{\#}$ una máquina tal que:
		\[
      \begin{array}{lcl}
        B \shortmid^{x_{1}} B \dotsc B \shortmid^{x_{k}} B \alpha_{1} B \dotsc B \alpha_{k} &\overset{\ast}{\vdash}& B
          \shortmid^{x_{1}} B \dotsc B \shortmid^{x_{i-1}} B \shortmid^{x_{j}} B \shortmid^{x_{i+1}} B \dotsc B
          \shortmid^{x_{k}} B \alpha_{1} B \dotsc B \alpha_{k} \\
        \uparrow && \uparrow \\
        q_{0} & & q_{f}
      \end{array}
    \]

    \PN Para $1 \leq i \leq k$, sea $M_{i\leftarrow 0}$ una máquina tal que:
    \[
      \begin{array}{lcl}
        B \shortmid^{x_{1}} B \dotsc B \shortmid^{x_{k}} B \alpha_{1} B \dotsc B \alpha_{k} &\overset{\ast}{\vdash}& B
          \shortmid^{x_{1}} B \dotsc B \shortmid^{x_{i-1}} B B \shortmid^{x_{i+1}} B \dotsc B \shortmid^{x_{k}} B
          \alpha_{1} B \dotsc B \alpha_{k} \\
        \uparrow && \uparrow \\
        q_{0} & & q_{f}
      \end{array}
    \]

    \PN Para $1 \leq i \leq k$, sea $M_{i\leftarrow \varepsilon}$ una máquina tal que:
		\[
      \begin{array}{lcl}
        B \shortmid^{x_{1}} B \dotsc B \shortmid^{x_{k}} B \alpha_{1} B \dotsc B \alpha_{k} &\overset{\ast}{\vdash}& B
          \shortmid^{x_{1}} B \dotsc B \shortmid^{x_{k}} B \alpha_{1} B \dotsc B \alpha_{i-1} B B \alpha_{i+1} B \dotsc
          B \alpha_{k} \\
        \uparrow && \uparrow \\
        q_{0} & & q_{f}
      \end{array}
    \]

    \PN Sean:
    \[
      M_{\mathrm{SKIP}} = \left(\{q_{0},q_{f}\},\Gamma,\Sigma,\delta,q_{0},B,\shortmid,\{q_{f}\}\right)
    \]

		\PN con $\delta(q_{0},B) = \{(q_{f},B,K)\}$ y $\delta = \emptyset$ en cualquier otro caso.
		\[
      M_{\mathrm{GOTO}} = \left(\{q_{0},q_{si},q_{no}\},\Gamma,\Sigma,\delta,q_{0},B,\shortmid,\{q_{si},q_{no}\}\right)
    \]

    \PN con $\delta(q_{0},B) = \{(q_{si},B,K)\}$ y $\delta = \emptyset$ en cualquier otro caso.

    \vspace{5mm}
    \PN Para poder realizar concretamente las máquinas recién descriptas deberemos diseñar antes algunas máquinas
    auxiliares. Para cada $j \geq 1$, sean:

    \begin{itemize}
      \item $D_{j}$ la siguiente máquina:

        \begin{figure}[h]
          \centering
          \includegraphics[scale=0.4]{graphics/figure_1.png}
        \end{figure}

        \PN Notar que:
    		\[
    		  \begin{array}{lcr}
            \alpha B \beta_{1} B \beta_{2} B \dotsc B \beta_{j} B \gamma &\overset{\ast}{\vdash}& \alpha B \beta_{1} B
              \beta_{2} B \dotsc B \beta_{j} B \gamma \\
            \ \ \uparrow && \uparrow \ \ \\
            \ \ q_{0} && q_{f} \ \
          \end{array}
    		\]

        \PN siempre que $\alpha, \gamma \in \Gamma ^{\ast}, \beta_{1}, \dotsc, \beta_{j} \in (\Gamma -\{B\})^{\ast}$.

      \item $I_{j}$ una máquina tal que:
    		\[
    		  \begin{array}{lcr}
            \alpha B \beta_{1} B
            \beta_{2} B \dotsc B \beta_{j} B \gamma &\overset{\ast}{\vdash}&  \alpha B \beta_{1} B \beta_{2} B \dotsc B
            \beta_{j} B \gamma \\
            \qquad\qquad\qquad\qquad \ \uparrow && \uparrow \qquad\qquad\qquad\qquad \ \\
            \qquad\qquad\qquad\qquad \ q_{0} && q_{f} \qquad\qquad\qquad\qquad \
          \end{array}
    		\]

        \PN siempre que $\alpha, \gamma \in \Gamma ^{\ast}, \beta_{1}, \dotsc, \beta_{j} \in (\Gamma -\{B\})^{\ast}$.

      \item $TD_{j}$ una máquina con un solo estado final $q_{f}$ y tal que:
    		\[
          \begin{array}{ccc}
            \alpha B \gamma &\overset{\ast}{\vdash} &\alpha B B \gamma \\
            \uparrow && \uparrow \ \ \\
            q_{0} & & q_{f} \ \
          \end{array}
        \]

        \PN cada vez que $\alpha, \gamma \in \Gamma^{\ast}$ y $\gamma$ tiene exactamente $j$ ocurrencias de $B$, es
        decir, la máquina $TD_{j}$ corre un espacio a la derecha todo el bloque $\gamma$ y agrega un blanco en el
        espacio que se genera a la izquierda de dicho bloque. Por ejemplo, para el caso de $\Sigma =\{\&\}$ podemos
        tomar $TD_{3}$ igual a la siguiente máquina:

        \begin{figure}[h]
          \centering
          \includegraphics[scale=0.4]{graphics/figure_3.png}
        \end{figure}

      \pagebreak
      \item $TI_{j}$ una máquina tal que:
    		\[
          \begin{array}{ccc}
            \alpha B \sigma \gamma &\overset{\ast}{\vdash}& \alpha B \gamma \\
            \uparrow \ && \uparrow \\
            q_{0} \ \ && q_{f}
          \end{array}
    	  \]

        \PN cada vez que $\alpha \in \Gamma ^{\ast }$, $\sigma \in \Gamma $ y $\gamma $ tiene exactamente $j$
        ocurrencias de $B$, es decir la máquina $TI_{j}$ corre un espacio a la izquierda todo el bloque $\gamma $ (por
        lo cual en el lugar de $\sigma $ queda el primer símbolo de $\gamma $).
    \end{itemize}

    \vspace{5mm}
    \PN Teniendo las máquinas auxiliares antes definidas podemos combinarlas para obtener las máquinas simuladoras de
    instrucciones. Por ejemplo $M_{i}^{a}$ puede ser la siguiente máquina:

    \begin{figure}[h]
      \centering
      \includegraphics[scale=0.4]{graphics/figure_4.png}
    \end{figure}

    \PN En la siguiente máquina, tenemos una posible forma de diseñar la máquina $IF_{i}^{a}$.

    \begin{figure}[h]
      \centering
      \includegraphics[scale=0.33]{graphics/figure_2.png}
    \end{figure}

    \pagebreak
    \PN En la siguiente máquina tenemos una posible forma de diseñar la máquina $M_{i\leftarrow j}^{\ast}$ para el caso
    $\Sigma = \{a,b\}$ y $i < j$:

    \begin{figure}[h]
      \centering
      \includegraphics[scale=0.33]{graphics/figure_7.png}
    \end{figure}

    \pagebreak
    \PN Supongamos ahora que $\mathcal{P} = I_{1}, \dotsc, I_{n}$. Para cada $i = 1, \dotsc, n$, definiremos una máquina
    $M_{i}$ que simulará la instrucción $I_{i}$. Luego uniremos adecuadamente dichas máquinas para formar la máquina que
    simulará a $\mathcal{P}$.

		\begin{itemize}
			\item Si $Bas(I_{i})=\mathrm{N}\bar{j} \leftarrow \mathrm{N}\bar{j} + 1$ tomaremos $M_{i} = M_{j}^{+}$
	    \item Si $Bas(I_{i})=\mathrm{N}\bar{j} \leftarrow \mathrm{N}\bar{j} \dot{-} 1 $ tomaremos $M_{i} =
        M_{j}^{\dot{-}}$
	    \item Si $Bas(I_{i})=\mathrm{N}\bar{j} \leftarrow 0$ tomaremos $M_{i} = M_{j \leftarrow 0}$
	    \item Si $Bas(I_{i})=\mathrm{N}\bar{j} \leftarrow \mathrm{N}\bar{m}$ tomaremos $M_{i} = M_{j \leftarrow m}^{\#}$
	    \item Si $Bas(I_{i})=\mathrm{IF} \ \mathrm{N}\bar{j} \not = 0 \ \mathrm{GOTO} \ \mathrm{L}\bar{m}$ tomaremos
        $M_{i} = IF_{j}$
	    \item Si $Bas(I_{i})=\mathrm{P}\bar{j} \leftarrow \mathrm{P}\bar{j}.a$ tomaremos $M_{i}=M_{j}^{a}$
	    \item Si $Bas(I_{i})=\mathrm{P}\bar{j} \leftarrow \ ^{\curvearrowright} \mathrm{P}\bar{j}$ tomaremos $M_{i} =
        M_{j}^{\curvearrowright}$
	    \item Si $Bas(I_{i})=\mathrm{P}\bar{j} \leftarrow \varepsilon$ tomaremos $M_{i} = M_{j\leftarrow \varepsilon}$
	    \item Si $Bas(I_{i})=\mathrm{P}\bar{j} \leftarrow \mathrm{P}\bar{m}$ tomaremos $M_{i} = M_{j \leftarrow m}^{\ast}$
	    \item Si $Bas(I_{i})=\mathrm{IF} \ \mathrm{P}\bar{j} \ \mathrm{BEGINS} \ a \ \mathrm{GOTO} \ \mathrm{L}\bar{m}$
        tomaremos $M_{i} = IF_{j}^{a}$
	    \item Si $Bas(I_{i})=\mathrm{SKIP}$ tomaremos $M_{i} = M_{\mathrm{SKIP}}$.
	    \item Si $Bas(I_{i})=\mathrm{GOTO} \ \mathrm{L}\bar{m}$ tomaremos $M_{i} = M_{\mathrm{GOTO}}$
		\end{itemize}

    \PN Dado que la máquina $M_{i}$ puede tener uno o dos estados finales, se representará como se muestra en la
    siguiente figura:

    \begin{figure}[h]
      \centering
      \includegraphics[scale=0.4]{graphics/figure_5.png}
    \end{figure}

    \PN entendiendo que en el caso en que $M_{i}$ tiene un solo estado final, este esta representado por el circulo de
    abajo a la izquierda y en el caso en que $M_{i}$ tiene dos estados finales, el estado final corresponde al estado
    $q_{si}$ y el otro al estado $q_{no}$.

    \PN Para armar la máquina que simulará a $\mathcal{P}$, primero unimos las máquinas $M_{1}, \dotsc, M_{n}$ como lo
    muestra la siguiente figura:

    \begin{figure}[h]
      \centering
      \includegraphics[scale=0.5]{graphics/figure_6.png}
    \end{figure}

    \pagebreak
    \PN Luego para cada $i$ tal que $Bas(I_{i})$ es de la forma $\alpha \ \mathrm{GOTO} \ \mathrm{L}\bar{m}$, ligamos
    con una flecha de la forma:
		\[
      \underrightarrow{\qquad B,B,K \qquad}
		\]

    \vspace{5mm}
    \PN el estado final $q_{si}$ de la $M_{i}$ con el estado inicial de la $M_{h}$, donde $h$ es tal que $I_{h}$ es la
    primer instrucción que tiene label $\mathrm{L}\bar{m}$. Es intuitivamente claro que la máquina así obtenida cumple
    con lo requerido aunque una prueba formal de esto puede resultar extremadamente tediosa.
	\end{proof}

  \pagebreak
	% Theorem 85: Con prueba.
	\begin{theorem}
		\PN Si $f: D_{f} \subseteq \omega^{n} \times \Sigma^{\ast m} \rightarrow O$ es $\Sigma$-recursiva, entonces $f$ es
    $\Sigma$-Turing computable.
  \end{theorem}
  \begin{proof}
    \PN Dado que $f$ es $\Sigma$-computable, existe $\mathcal{P} \in \mathrm{Pro}^{\Sigma}$ el cual computa $f$. Se
    probará solamente el caso $O = \SIGMA$. Notar que cuando $\mathcal{P}$ termina, en el estado alcanzado, las
    variables numéricas tienen todas el valor $0$ y las alfabéticas distintas de $\mathrm{P}1$ todas el valor
    $\varepsilon$.

    \PN Sean:
    \begin{itemize}
      \item $M$ la máquina de Turing con unit dada por el \textbf{Lemma 84}, donde elegimos el número $k$ con la
        propiedad adicional de ser mayor que $n$ y $m$.

      \item $M_{1}$ una máquina tal que para cada $(\vec{x},\vec{\alpha}) \in \omega^{n} \times \Sigma^{\ast m}$
      	\[
          \left\lfloor q_{0} B \shortmid^{x_{1}} B \dotsc B \shortmid^{x_{n}} B \alpha_{1} B \dotsc B \alpha_{m} B
          \right\rfloor \overset{\ast}{\vdash}\left\lfloor q B \shortmid^{x_{1}} B \dotsc B \shortmid^{x_{n}}
          B^{k-n} B \alpha_{1} B \dotsc B \alpha_{m} B \right\rfloor
      	\]

        \PN donde $q_{0}$ es el estado inicial de $M_{1}$ y $q$ es un estado tal que $\delta(q,\sigma) =
        \emptyset$, para cada $\sigma$.

      \item $M_{2}$ una máquina tal que para cada $\alpha \in \SIGMA$
      	\[
          \left\lfloor q_{0} B^{k+1} \alpha \right\rfloor \overset{\ast}{\vdash} \left\lfloor q B \alpha
          \right\rfloor
      	\]

        \PN donde $q_{0}$ es el estado inicial de $M_{2}$ y $q$ es un estado tal que $\delta(q,\sigma) =
        \emptyset$, para cada $\sigma$.
    \end{itemize}

    \PN Notar que la concatenación de $M_{1}$, $M$ y $M_{2}$, en ese orden, produce una máquina de Turing la cual
    computa $f$.
	\end{proof}

  % Theorem 86: Nada.
  \begin{theorem}
    \PN Este teorema no se evalua.
  \end{theorem}

%
% \begin{thebibliography}{X}
% \bibitem{Baz} \textsc{Diego Vaggione},
% <<Apunte de Clase, 2017>>,
% \textit{FaMAF, UNC}.
% \bibitem{Baz} \textsc{Agustín Curto},
% <<Carpeta de Clase, 2017>>,
% \textit{FaMAF, UNC}.
% \end{thebibliography}
%
% \vspace{\fill}
% \begin{center}
% Por favor, mejorá este documento en github
% \includegraphics[width=1cm]{graphics/github.png} \\
% https://github.com/acurto714/resumenLengForm
% \end{center}
\end{document}
