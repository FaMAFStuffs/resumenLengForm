Ahora consideraremos dos funciones las cuales son obtenidas naturalmente por recursion primitiva sobre variable alfabetica.

Lema 23 Supongamos \(\Sigma \) es no vacio.
(a) \(\lambda \alpha \beta \left[ \alpha \beta \right] \in \mathrm{PR} ^{\Sigma }\)
(b) \(\lambda \alpha \left[ \left\vert \alpha \right\vert \right] \in \mathrm{PR}^{\Sigma }\)
Prueba: (a) Ya que

\(\displaystyle \begin{array}{rcl} \lambda \alpha \beta \left[ \alpha \beta \right] (\alpha _{1},\varepsilon ) & =& \alpha _{1}=p_{1}^{0,1}(\alpha _{1}) \\ \lambda \alpha \beta \left[ \alpha \beta \right] (\alpha _{1},\alpha a) & =& d_{a}(\lambda \alpha \beta \left[ \alpha \beta \right] (\alpha _{1},\alpha )),a\in \Sigma \end{array} \)

tenemos que \(\lambda \alpha \beta \left[ \alpha \beta \right] =R\left( p_{1}^{0,1},\mathcal{G}\right) \), donde \(\mathcal{G}_{a}=d_{a}\circ p_{3}^{0,3}\), para cada \(a\in \Sigma \).
(b) Ya que

\(\displaystyle \begin{array}{rcl} \lambda \alpha \left[ \left\vert \alpha \right\vert \right] (\varepsilon ) & =& 0=C_{0}^{0,0}(\Diamond ) \\ \lambda \alpha \left[ \left\vert \alpha \right\vert \right] (\alpha a) & =& \lambda \alpha \left[ \left\vert \alpha \right\vert \right] (\alpha )+1 \end{array} \)

tenemos que \(\lambda \alpha \left[ \left\vert \alpha \right\vert \right] =R\left( C_{0}^{0,0},\mathcal{G}\right) \), donde \(\mathcal{G}_{a}=\) \( Suc\circ p_{1}^{1,1}\), para cada \(a\in \Sigma .\). \(\Box\)
Lema 24
(a) \(C_{k}^{n,m},C_{\alpha }^{n,m}\in \mathrm{PR}^{\Sigma }\), para \( n,m,k\geq 0\), \(\alpha \in \Sigma ^{\ast }\).
(b) \(C_{k}^{n,0}\in \mathrm{PR}^{\varnothing }\), para \(n,k\geq 0\).
Prueba: (a) Note que \(C_{k+1}^{0,0}=\) \(Suc\circ C_{k}^{0,0}\), lo cual implica \( C_{k}^{0,0}\in \mathrm{PR}_{k}^{\Sigma }\), para \(k\geq 0\). Tambien note que \( C_{\alpha a}^{0,0}=d_{a}\circ C_{\alpha }^{0,0}\), lo cual dice que \( C_{\alpha }^{0,0}\in \mathrm{PR}^{\Sigma }\), para \(\alpha \in \Sigma ^{\ast } \). Para ver que \(C_{k}^{0,1}\in \mathrm{PR}^{\Sigma }\) notar que

\(\displaystyle \begin{array}{rcl} C_{k}^{0,1}(\varepsilon ) & =& k=C_{k}^{0,0}(\Diamond ) \\ C_{k}^{0,1}(\alpha a) & =& C_{k}^{0,1}(\alpha )=p_{1}^{1,1}\left( C_{k}^{0,1}(\alpha ),\alpha \right) \end{array} \)

lo cual implica que \(C_{k}^{0,1}=R\left( C_{k}^{0,0},\mathcal{G}\right) \), con \(\mathcal{G}_{a}=p_{1}^{1,1}\), \(a\in \Sigma \). En forma similar podemos ver que \(C_{k}^{1,0},C_{\alpha }^{1,0},C_{\alpha }^{0,1}\in \mathrm{PR} ^{\Sigma }\). Supongamos ahora que \(m >0\). Entonces
\(\displaystyle \begin{array}{rcl} C_{k}^{n,m} & =& C_{k}^{0,1}\circ p_{n+1}^{n,m} \\ C_{\alpha }^{n,m} & =& C_{\alpha }^{0,1}\circ p_{n+1}^{n,m} \end{array} \)

de lo cual obtenemos que \(C_{k}^{n,m},C_{\alpha }^{n,m}\in \mathrm{PR} ^{\Sigma }\). El caso \(n >0\) es similar
(b) Use argumentos similares a los usados en la prueba de (a). \(\Box\)

Definamos \(0^{0}=1\). O sea que \(D_{\lambda xy\left[ x^{y}\right] }=\omega \times \omega .\) Tambien ya que \(\alpha ^{0}=\varepsilon \), tenemos que \( D_{\lambda x\alpha \left[ \alpha ^{x}\right] }=\omega \times \Sigma ^{\ast }\) .

Lema 25
(a) \(\lambda xy\left[ x^{y}\right] \in \mathrm{PR}^{\varnothing }\).
(b) \(\lambda t\alpha \left[ \alpha ^{t}\right] \in \mathrm{PR} ^{\Sigma }\).
Prueba: (a) Note que

\(\displaystyle \lambda tx\left[ x^{t}\right] =R\left( C_{1}^{1,0},\lambda xy\left[ x.y \right] \circ \left( p_{1}^{3,0},p_{3}^{3,0}\right) \right) \in \mathrm{PR} ^{\varnothing }. \)

O sea que \(\lambda xy\left[ x^{y}\right] =\lambda tx\left[ x^{t}\right] \circ \left( p_{2}^{2,0},p_{1}^{2,0}\right) \in \mathrm{PR}^{\varnothing }\).
(b) Note que

\(\displaystyle \lambda t\alpha \left[ \alpha ^{t}\right] =R\left( C_{\varepsilon }^{0,1},\lambda \alpha \beta \left[ \alpha \beta \right] \circ \left( p_{3}^{1,2},p_{2}^{1,2}\right) \right) \in \mathrm{PR}^{\Sigma }. \)

\(\Box\)
Ahora probaremos que si \(\Sigma \) es no vacio, entonces las biyeciones naturales entre \(\Sigma ^{\ast }\) y \(\omega \), dadas en el Lema 6, son \(\Sigma \)-p.r..

Lema 26 Si \(< \) es un orden total estricto sobre un alfabeto no vacio \( \Sigma \), entonces \(s^{< }\), \(\#^{< }\) y \(\ast ^{< }\) pertenecen a \(\mathrm{PR} ^{\Sigma }\)
Prueba: Supongamos \(\Sigma =\{a_{1},...,a_{k}\}\) y \(< \) dado por \(a_{1}< ...< a_{k}\). Ya que

\(\displaystyle \begin{array}{rcl} s^{< }(\varepsilon ) & =& a_{1} \\ s^{< }(\alpha a_{i}) & =& \alpha a_{i+1}\text{, para }i< k \\ s^{< }(\alpha a_{k}) & =& s^{< }(\alpha )a_{1} \end{array} \)

tenemos que \(s^{< }=R\left( C_{a_{1}}^{0,0},\mathcal{G}\right) \), donde \( \mathcal{G}_{a_{i}}=d_{a_{i+1}}\circ p_{1}^{0,2}\), para \(i=1,...,k-1\) y \( \mathcal{G}_{a_{k}}=d_{a_{1}}\circ p_{2}^{0,2}.\) O sea que \(s^{< }\in \mathrm{ PR}^{\Sigma }.\) Ya que
\(\displaystyle \begin{array}{rcl} \ast ^{< }(0) & =& \varepsilon \\ \ast ^{< }(t+1) & =& s^{< }(\ast ^{< }(t)) \end{array} \)

podemos ver que \(\ast ^{< }\in \mathrm{PR}^{\Sigma }.\) Ya que
\(\displaystyle \begin{array}{rcl} \#^{< }(\varepsilon ) & =& 0 \\ \#^{< }(\alpha a_{i}) & =& \#^{< }(\alpha ).k+i\text{, para }i=1,...,k, \end{array} \)

tenemos que \(\#^{< }=R\left( C_{0}^{0,0},\mathcal{G}\right) \), donde
\(\displaystyle \mathcal{G}_{a_{i}}=\lambda xy\left[ x+y\right] \circ \left( \lambda xy\left[ x.y\right] \circ \left( p_{1}^{1,1},C_{k}^{1,1}\right) ,C_{i}^{1,1}\right) \text{, para }i=1,...,k\text{.} \)

O sea que \(\#^{< }\in \mathrm{PR}^{\Sigma }\). \(\Box\)
Dados \(x,y\in \omega \), definamos

\(\displaystyle x\dot{-}y=\max (x-y,0). \)

Lema 27
(a) \(\lambda xy\left[ x\dot{-}y\right] \in \mathrm{PR}^{\varnothing }.\)
(b) \(\lambda xy\left[ \max (x,y)\right] \in \mathrm{PR}^{\varnothing }.\)
(c) \(\lambda xy\left[ x=y\right] \in \mathrm{PR}^{\varnothing }.\)
(d) \(\lambda xy\left[ x\leq y\right] \in \mathrm{PR}^{\varnothing }.\)
(e) Si \(\Sigma \) es no vacio, entonces \(\lambda \alpha \beta \left[ \alpha =\beta \right] \in \mathrm{PR}^{\Sigma }\)
Prueba: (a) Primero notar que \(\lambda x\left[ x\dot{-}1\right] =R\left( C_{0}^{0,0},p_{2}^{2,0}\right) \in \mathrm{PR}^{\varnothing }.\) Tambien note que

\(\displaystyle \lambda tx\left[ x\dot{-}t\right] =R\left( p_{1}^{1,0},\lambda x\left[ x\dot{ -}1\right] \circ p_{1}^{3,0}\right) \in \mathrm{PR}^{\varnothing }. \)

O sea que \(\lambda xy\left[ x\dot{-}y\right] =\lambda tx\left[ x\dot{-}t \right] \circ \left( p_{2}^{2,0},p_{1}^{2,0}\right) \in \mathrm{PR} ^{\varnothing }.\)
(b) Note que \(\lambda xy\left[ \max (x,y)\right] =\lambda xy\left[ (x+(y\dot{ -}x)\right] .\)

(c) Note que \(\lambda xy\left[ x=y\right] =\lambda xy\left[ 1\dot{-}((x\dot{- }y)+(y\dot{-}x))\right] .\)

(d) Note que \(\lambda xy\left[ x\leq y\right] =\lambda xy\left[ 1\dot{-}(x \dot{-}y)\right] .\)

(e) Sea \(< \) un orden total estricto sobre \(\Sigma .\) Ya que

\(\displaystyle \alpha =\beta \text{ sii }\#^{< }(\alpha )=\#^{< }(\beta ) \)

tenemos que
\(\displaystyle \lambda \alpha \beta \left[ \alpha =\beta \right] =\lambda xy\left[ x=y \right] \circ \left( \#^{< }\circ p_{1}^{0,2},\#^{< }\circ p_{2}^{0,2}\right) . \)

O sea que podemos aplicar (c) y Lema 28 implica que \(\chi _{S}\) es \( \Sigma \)-p.r.. \(\Box\)
El siguiente lema caracteriza cuando un conjunto rectangular es \(\Sigma \) -p.r..


%% =========== FALTA UNA PARTE ===========%%

Lema 31 Supongamos \(S_{1},...,S_{n}\subseteq \omega \), \( L_{1},...,L_{m}\subseteq \Sigma ^{\ast }\) son conjuntos no vacios. Entonces \( S_{1}\times ...\times S_{n}\times L_{1}\times ...\times L_{m}\) es \(\Sigma \) -p.r. sii \(S_{1},...,S_{n},L_{1},...,L_{m}\) son \(\Sigma \)-p.r.
Prueba: (\(\Rightarrow \)) Veremos por ejemplo que \(L_{1}\) es \(\Sigma \)-p.r.. Sea \( (z_{1},...,z_{n},\zeta _{1},...,\zeta _{m})\) un elemento fijo de \( S_{1}\times ...\times S_{n}\times L_{1}\times ...\times L_{m}.\) Note que

\(\displaystyle \alpha \in L_{1}\text{ sii }(z_{1},...,z_{n},\alpha ,\zeta _{2},...,\zeta _{m})\in S_{1}\times ...\times S_{n}\times L_{1}\times ...\times L_{m}, \)

lo cual implica que
\(\displaystyle \chi _{L_{1}}=\chi _{S_{1}\times ...\times S_{n}\times L_{1}\times ...\times L_{m}}\circ \left( C_{z_{1}}^{0,1},...,C_{z_{n}}^{0,1},p_{1}^{0,1},C_{\zeta _{2}}^{0,1},...,C_{\zeta _{m}}^{0,1}\right) . \)

(\(\Leftarrow \)) Note que \(\chi _{S_{1}\times ...\times S_{n}\times L_{1}\times ...\times L_{m}}\) es el predicado
\(\displaystyle \left( \chi _{S_{1}}\circ p_{1}^{n,m}\wedge ...\wedge \chi _{S_{n}}\circ p_{n}^{n,m}\wedge \chi _{L_{1}}\circ p_{n+1}^{n,m}\wedge ...\wedge \chi _{L_{m}}\circ p_{n+m}^{n,m}\right) . \)

\(\Box\)
Dada una funcion \(f\) y un conjunto \(S\subseteq D_{f}\), usaremos \(f\mid _{S}\) para denotar la restriccion de \(f\) al conjunto \(S\), i.e. \(f\mid _{S}=f\cap (S\times I_{f}).\)

Lema 32 Supongamos \(f:D_{f}\subseteq \omega ^{n}\times \Sigma ^{\ast m}\rightarrow O\) es \(\Sigma \)-p.r., donde \(O\in \{\omega ,\Sigma ^{\ast }\}.\) Si \(S\subseteq D_{f}\) es \(\Sigma \)-p.r., entonces \(f\mid _{S}\) es \(\Sigma \)-p.r..
Prueba: Supongamos \(O=\Sigma ^{\ast }\). Entonces

\(\displaystyle f\mid _{S}=\lambda x\alpha \left[ \alpha ^{x}\right] \circ \left( Suc\circ Pred\circ \chi _{S},f\right) \)

es \(\Sigma \)-p.r.. El caso \(O=\omega \) es similar usando \(\lambda xy\left[ x^{y}\right] \) en lugar de \(\lambda x\alpha \left[ \alpha ^{x}\right] \). \(\Box\)
Usando el lema anterior en combinacion con el Lema 28 podemos ver que muchos predicados usuales son \(\Sigma \)-p.r.. Por ejemplo sea

\(\displaystyle P=\lambda x\alpha \beta \gamma \left[ x=\left\vert \gamma \right\vert \wedge \alpha =\gamma ^{Pred(\left\vert \beta \right\vert )}\right] . \)

Notese que
\(\displaystyle D_{P}=\omega \times \Sigma ^{\ast }\times (\Sigma ^{\ast }-\{\varepsilon \})\times \Sigma ^{\ast } \)

es \(\Sigma \)-p.r. ya que
\(\displaystyle \chi _{D_{P}}=\lnot \lambda \alpha \beta \left[ \alpha =\beta \right] \circ \left( p_{3}^{1,3},C_{\varepsilon }^{1,3}\right) . \)

Tambien note que los predicados
\(\displaystyle \begin{array}{rcl} & & \lambda x\alpha \beta \gamma \left[ x=\left\vert \gamma \right\vert \right] \\ & & \lambda x\alpha \beta \gamma \left[ \alpha =\gamma ^{Pred(\left\vert \beta \right\vert )}\right] \end{array} \)

son \(\Sigma \)-p.r. ya que pueden obtenerse componiendo funciones \(\Sigma \) -p.r.. O sea que \(P\) es \(\Sigma \)-p.r. ya que
\(\displaystyle P=\left( \lambda x\alpha \beta \gamma \left[ x=\left\vert \gamma \right\vert \right] \mid _{D_{P}}\wedge \lambda x\alpha \beta \gamma \left[ \alpha =\gamma ^{Pred(\left\vert \beta \right\vert )}\right] \right) . \)
